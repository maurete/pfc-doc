%
%
\subsection{El clasificador lineal}
%
El clasificador lineal es una máquina de aprendizaje \cite{nilsson}
que sirve como fundamento de la máquina de vectores de soporte. De
hecho, esta última puede ser entendida como un tipo particular de
clasificador lineal.

En primer lugar, sea una función $\BPhi:X\rightarrow{}Z$ tal que
transforma el denominado \e{espacio de entrada} $X$ a otro espacio
vectorial inducido $Z$, denominado \e{espacio imagen}.  A partir del
conocimiento previo del problema: se pretende que la transformación
$\BPhi$ sea tal que los \e{vectores imagen} $\zz=\BPhi(\xx)$ se
ubiquen, según su clase $y$, en regiones del espacio $Z$ bien
diferenciadas entre sí.

Entonces, dado un patrón de entrada $\xx\in{}X$, el clasificador
lineal calcula primero el vector imagen $\zz=\BPhi(\xx)$, para luego
asignarle una clase de salida $\hat{y}=\pm{}1$ según sea el signo de
la función discriminante lineal $f=\ww^T\zz+b$.  Los valores de $\ww$
y $b$ se determinan en el entrenamiento del clasificador, cuyos detalles
se omiten de la presente descripción.

Por último, se observa que la ecuación $\ww^T\zz+b=0$ define un
hiperplano en el espacio inducido $Z$. Dicho hiperplano determina una
frontera de decisión entre las clases $\hat{y}=+1$ e
$\hat{y}=-1$. Para visualizarlo, obsérvese que, desde el punto de
vista geométrico, el signo de $\ww^T\zz+b$ determina de qué lado del
plano se encuentra el punto $\zz$.

%
\begin{quote}
  {\bfseries Notación.}\quad{}En la literatura, resulta común
  encontrar que el vector $\zz$ se denomina \e{vector de
    características} (\e{feature vector}) y el espacio vectorial $Z$
  como \e{espacio de las características} (\e{feature space}).  En
  este trabajo se prefiere denominarlos con los nombres alternativos
  \e{vector imagen} y \e{espacio vectorial imagen} para evitar
  confusión con el proceso de extracción de características, la etapa
  de pre-procesamiento de los datos que genera vectores aptos para su
  utilización en la máquina de aprendizaje.
\end{quote}
%
