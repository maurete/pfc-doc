%
\subsection{Búsqueda exhaustiva}
%
La búsqueda exhaustiva determina el número óptimo de neuronas ocultas
mediante una estrategia de prueba y error, entrenando modelos con
diferentes cantidades de neuronas en la capa oculta y seleccionando
aquel que obtiene la mayor tasa \GM{} promedio de validación cruzada.
Dado que no existe una regla general para determinar el número
recomendado de neuronas ocultas \cite{nnfaq3}, se establece un rango
de prueba de entre $0$ y $200$ neuronas.
Por razones de costo computacional, en lugar de efectuar la búsqueda
sobre todos los valores posibles, se prueban $20$ valores $H$ de
neuronas en la capa oculta, variando en una escala aproximadamente
logarítmica: $H\in\{0,1,2,3,4,5,7,9,11,14,19,24,32,41,54,70,91,118,154,200\}$.
Una vez entrenados y probados los diferentes modelos mediante validación
cruzada (\iflatexml{}Sección~\ref{s2:crossval}\else\autoref{s2:crossval}\fi{}),
se retorna como número óptimo de neuronas ocultas aquel del modelo que
maximiza la tasa \GM{} sobre los datos de validación.
