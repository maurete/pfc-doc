%
\subsubsection{Búsqueda exhaustiva}
%
La estrategia de búsqueda exhaustiva utiliza como función objetivo el
valor $G_m$ promedio de validación cruzada, y selecciona los
hiperparámetros óptimos mediante prueba y error dentro de un rango
preestablecido. Dado que la naturaleza de los hiperparámetros es
diferente según el clasificador, el algoritmo de búsqueda también
difiere en cada caso.

El hiperparámetro del perceptrón multicapa es el número de neuronas en
la capa oculta $h$, una variable discreta no negativa. La
inicialización aleatoria del perceptrón multicapa introduce
fluctuaciones aleatorias en la función objetivo $G_m$ de validación
cruzada, lo que implica que se deben efectuar una gran cantidad de
repeticiones de inicialización--entrenamiento--prueba para obtener
resultados fiables.  Por ello, en lugar de efectuar la búsqueda sobre
todos los valores posibles, se prueban 20 valores de $h$ entre 0 y 200
en una escala aproximadamente logarítmica:
%
\begin{align}
  \label{mlp-hidden-tries}
  h=0,1,2,3,4,5,7,9,11,14,19,24,32,41,54,70,91,118,154,200.
\end{align}
%
Por defecto, se efectúan 5 repeticiones de
inicialización--entrenamiento--prueba para cada valor de $h$. Se
selecciona aquel valor de $h$ para el cual se obtuvo el mayor valor de
$G_m$ de validación cruzada en promedio.
