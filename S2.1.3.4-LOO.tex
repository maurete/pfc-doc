%
\subsubsection{Validación cruzada dejando uno fuera}
%
En el método de validación cruzada ``clásico'' de $k$ iteraciones, se
tiene para valores pequeños de $k$ una reducción en la varianza y para
valores grandes una reducción del sesgo introducido por el particionado.
En el extremo, cuando se
establece $k$ al número de elementos en $D$, se está ante un método de
\e{validación cruzada dejando uno fuera}. La condición $k=\ell=\|D\|$
implica la reducción del sesgo a un valor mínimo, ya que se separa
sólo un elemento para validación. Sin embargo, la varianza obtenida
mediante este método es máxima.

Para obtener una estimación del error de generalización mediante este
método es necesario entrenar el clasificador $\ell$ veces, lo que deriva
en un costo computacional muy elevado. Sin embargo, para el caso de un
clasificador SVM existen técnicas para el cálculo eficiente del error
dejando uno fuera \cite{chapelle,lee-keerthi}.
