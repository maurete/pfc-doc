\section{Descripción general del método}

El método recibe como entrada archivos de secuencias en un formato
estándar denominado FASTA. Para cada archivo se debe especificar la
\e{clase} de los ejemplos contenidos: positiva (+1) para pre-miRNAs
verdaderos, negativa (-1) para ``pseudo'' pre-miRNAs.  Adicionalmente,
se debe especificar (para cada archivo) la proporción de elementos
a utilizar para entrenamiento y prueba.

Se distinguen tres tareas principales que describen la funcionalidad
del sistema: generación del problema de clasificación, obtención del
modelo óptimo del clasificador, y clasificación (predicción) de nuevos
ejemplos.

La generación del problema de clasificación abarca el procesamiento de
los datos de entrada, la extracción de características, la
normalización de las matrices de datos y la partición de en conjuntos
de entrenamiento, prueba, y validación cruzada.

La obtención del modelo del clasificador incluye las tareas de
optimización de los hiperparámetros y el entrenamiento con el conjunto
completo de entrenamiento.  El modelo obtenido en este proceso puede
ser utilizado para clasificar nuevos ejemplos.

La clasificación es la tarea de aplicar el modelo ya entrenado a
nuevos datos, a fin de obtener predicciones de clase.