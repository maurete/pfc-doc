%En la \iflatexml{}Tabla~\ref{tbl:cost-main}\else\autoref{tbl:cost-main}\fi{}

En el clasificador MLP, la estrategia de búsqueda exhaustiva efectuó
en todos los casos $201$ entrenamientos: $10$ modelos de validación
cruzada por $20$ pruebas con diferente número de neuronas en la capa
oculta, más el entrenamiento final del modelo.
Se observó que el tiempo insumido para la generación del modelo
aumentó según el número de ejemplos en el conjunto de entrenamiento.
Asimismo, se observó un aumento en el tiempo requerido para generar el
modelo con el conjunto de \caract{s} \dset{S}.

Al probar el clasificador SVM con núcleo lineal, se observó que el
conjunto de \caract{s} resultó un factor determinante del tiempo
requerido para generar el modelo del clasificador.
Mientras que el uso del conjunto de \caract{s} \dset{E} derivó en los
menores tiempos de entrenamiento, las \caract{s} de secuencia y de
tripletes requirieron los mayores tiempos de entrenamiento.
Este efecto se observó especialmente en la estrategia de búsqueda
exhaustiva, que aún efectuando el mismo número de entrenamientos en
todos los casos, presentó una variación considerable en el tiempo
insumido según las \caract{s} consideradas.

La generación del modelo de clasificador SVM con núcleo RBF requirió
en general un mayor número de entrenamientos, aunque el tiempo de
ejecución requerido resultó en muchos casos similar o incluso inferior
a aquel requerido por el clasificador SVM con núcleo lineal.
En comparación con el clasificador SVM-lineal, el tiempo de ejecución
resultó más uniforme entre los distintos grupos de \caract{s}.
En particular, se destaca la velocidad del método al aplicar la
estrategia de selección de \hparam{s} mediante minimización de la cota
radio-margen, que requiere muy pocos entrenamientos, con un tiempo de
ejecución muy inferior al de las demás estrategias.
