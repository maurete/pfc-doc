%
%
%
\section{Construcción del modelo del clasificador MLP}
%
El primer paso para la construcción de un modelo MLP es determinar el
número óptimo de neuronas en la capa oculta mediante una estrategia
de búsqueda exhaustiva que maximiza la tasa \GM promedio de
validación cruzada.
%% Para ello, se utiliza una estrategia de prueba y error, que prueba
%% distintos modelos mediante validación cruzada, y selecciona como
%% número óptimo de neuronas el de aquel modelo que obtiene la mejor tasa
%% $G_m$ en promedio sobre el conjunto de validación.
Alternativamente, se define una estrategia de selección trivial que
siempre selecciona 0 neuronas en la capa oculta, resultando en un
clasificador lineal.

Una vez determinado el número óptimo de neuronas en la capa oculta,
se procede al entrenamiento del modelo final utilizando el conjunto
de entrenamiento completo.

%% uego se procede a efectuar el entrenamiento propiamente dicho, con
%% regularización de corte prematuro cuando el error de validación
%% cruzada alcanza su valor mínimo.

%% La arquitectura de la red resultante contendrá una o ninguna capa
%% oculta.

%% En todo entrenamiento del clasificador, se promedian en realidad cinco
%% redes inicializadas aleatoriamente, y la salida del clasificador en
%% realidad es un voto mayoritario de la salida de estas cinco redes.

%% Adicionalmente, se acepta el valor especial de 0 neuronas ocultas para
%% representar un MLP sin capa oculta, capaz de efectuar clasificación lineal.

%% Las estrategias de selección del \hparam{} de la red (el número de
%% neuronas en la capa oculta) son
%% %
%% \begin{description}
%% \item[Estrategia trivial:] Consiste en seleccionar cero neuronas
%%   en la capa oculta, sin entrenamiento. Esto redunda en una red que es
%%   capaz de efectuar separación lineal, y se utiliza como criterio de base para
%%   establecer el rendimiento de la red.
%% \item[Estrategia de búsqueda exhausiva:] Consiste en determinar, mediante prueba y error,
%%   el número óptimo de neuronas que maximizan una tasa promedio de validación cruzada.
%% \end{description}
%% %
%% Una vez encontrado el valor óptimo, se procede al entrenamiento.
