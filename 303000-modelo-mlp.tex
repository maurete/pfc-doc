%
%
%
\section{Construcción del modelo del clasificador MLP}
%
La construcción del modelo del clasificador MLP genera una red
neuronal con propagación hacia adelante con una topología de una
capa oculta, siguiendo la recomendación de \cite{nnfaq}.
La capa de entrada de la red contiene $F$ nodos sensores, en donde $F$
es el número de características, y una única neurona en la capa de
salida.
En un primer paso se calcula el número $H$ de neuronas a disponer en
la capa oculta, mediante una estrategia de prueba y error que maximiza
la tasa de clasificación sobre el conjunto de validación cruzada.
Como alternativa a esta búsqueda exhaustiva se creó una ``estrategia
trivial'' que genera, sin efectuar entrenamiento, una red mínima
equivalente a un perceptrón simple (el caso $H=0$).
Una vez determinado el número óptimo $H$ de neuronas ocultas, se
inicializa aleatoriamente una red con esta topología ($F$--$H$--$1$),
y se ajustan sus pesos a partir de los datos del conjunto de
entrenamiento mediante un algoritmo de retropropagación.
