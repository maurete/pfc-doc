%
%
%
\section{Pruebas del clasificador SVM con núcleo RBF}
%
Los resultados de aplicar el clasificador SVM-RBF sobre los tres
problemas definidos se presentan en la
\iflatexml{}Tabla~\ref{tbl:rbf-results}\else\autoref{tbl:rbf-results}\fi.

Los resultados obtenidos para los tres problemas de clasificación
resultaron comparables a los obtenidos con el clasificador SVM con
núcleo lineal.
En el caso del problema \prob\tripletsvm{}, los resultados de las
pruebas arrojaron mejores tasas de clasificación, para los problemas
\prob\mipred{} y \prob\micropred{} las tasas obtenidas en general
resultaron inferiores que las obtenidas con el clasificador SVM-lineal.

La utilización del conjunto de \caract{s} \dset{S-E} obtuvo los mejores
resultados de clasificación para todas las pruebas efectuadas con este
clasificador.
Estos resultados superan además los obtenidos con los clasificadores
MLP y SVM-lineal.

La estrategia de selección de \hparam{s} mediante minimización de la
cota RM obtuvo muy buenos resultados de clasificación, especialmente
teniendo en cuenta su bajo coste computacional, que se analiza más
adelante.
Resulta prudente sospechar que el modelo generado con esta estrategia
presenta sobreajuste en el caso del problema \prob\tripletsvm{}.
Asimismo, se observa que la estrategia divirgió al utilizar el
conjunto de \caract{s} de secuencia \dset{S} para el problema
\prob\mipred{}.
