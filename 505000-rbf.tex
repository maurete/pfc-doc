%
%
%
\section{Pruebas del clasificador SVM con núcleo RBF}
%
En la
\iflatexml{}Tabla~\ref{tbl:rbf-results}\else\autoref{tbl:rbf-results}\fi
se observa que los resultados obtenidos para los tres problemas de
clasificación resultaron comparables a los obtenidos con el
clasificador SVM con núcleo lineal.
Mientras que para el problema \prob\tripletsvm{} las tasas de
clasificación superaron las obtenidas con el núcleo lineal, para los
problemas \prob\mipred{} y \prob\micropred{} se observaron tasas
generalmente inferiores.

Para todos los problemas, la utilización del conjunto de \caract{s}
\dset{S-E} resultó en las mayores tasas de clasificación, superando
además las tasas obtenidas con los clasificadores MLP y SVM con núcleo
lineal.

La estrategia de selección de \hparam{s} mediante minimización de la
cota RM obtuvo muy buenos resultados de clasificación, especialmente
teniendo en cuenta su bajo coste computacional, que se analiza más
adelante.
Resulta prudente sospechar que el modelo generado con esta estrategia
presenta sobreajuste en el caso del problema \prob\tripletsvm{}.
Asimismo, se observa que la estrategia presentó divergencia al
utilizar el conjunto de \caract{s} de secuencia \dset{S} para el
problema \prob\mipred{}.
