%
%
\subsection{Problema \tripletsvm}
%
El problema \prob{\tripletsvm} replica los datos utilizados
en \cite{xue} para la evaluación del método \work{\tripletsvm}.
Éste fue uno de los primeros métodos de predicción de \premirna{s}
mediante Máquinas de Vectores de Soporte, y su nombre deriva del hecho
que utiliza características de tripletes como entrada al clasificador.

Los conjuntos de entrenamiento y prueba que componen el problema
\prob{\tripletsvm} son idénticos a aquellos utilizados en \cite{xue}
para las pruebas del método \work{\tripletsvm}, ya que los autores
publicaron estos datos como material suplementario, incluyendo la
partición en entrenamiento y prueba.

Los ejemplos de clase positiva son tomados de la versión 5.0 (de
septiembre de 2004) de \work\mirbase.
De los 207 ejemplos de \premirna{s} de la especie humana (clase
positiva) presentes en esta versión, se eliminan aquellos que
contienen ramificaciones en la estructura secundaria, resultando en
193 ejemplos con estructura secundaria en forma de horquilla.
De estos ejemplos, se utilizan 163 para componer el conjunto de
entrenamiento y los 30 restantes para el conjunto de prueba.
Los ejemplos de clase negativa se obtienen de la base de datos
\dset{coding} \cite{xue}, utilizando 168 ejemplos para el conjunto de
entrenamiento y 1000 para el armado del conjunto de prueba.

La composición de los conjuntos de entrenamiento y prueba se resume en
la \iflatexml{}Tabla~\ref{tbl:mainxue}\else\autoref{tbl:mainxue}\fi.
