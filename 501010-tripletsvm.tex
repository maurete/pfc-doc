%
%
\subsection{Problema \tripletsvm}
%
El problema \prob{\tripletsvm} replica los datos utilizados en
\cite{xue} para la evaluación del método \work{\tripletsvm}.
Éste fue uno de los primeros métodos de predicción de \premirna{s}
mediante máquinas de vectores de soporte, y su nombre deriva del hecho
que utiliza características de tripletes como entrada al clasificador.

Los conjuntos de entrenamiento y prueba que componen el problema
\prob{\tripletsvm} son idénticos a aquellos utilizados en \cite{xue}
para las pruebas del método \work{\tripletsvm}, ya que los autores
publicaron estos datos como material suplementario, incluyendo la
partición en conjuntos de entrenamiento y prueba.
Los ejemplos de clase positiva se obtienen de la versión $5$.$0$
(de septiembre de $2004$) de la base de datos \dset\mirbase.
De los $207$ \premirna{s} de la especie humana presentes en esta
versión, se eliminan aquellos con ramificaciones en la estructura
secundaria, resultando en $193$ ejemplos con estructura secundaria en
forma de horquilla.
De estos ejemplos, se utilizan $163$ para componer el conjunto de
entrenamiento y los $30$ restantes para el conjunto de prueba.
Los ejemplos de clase negativa se obtienen del conjunto de datos
\dset{coding}, utilizando $168$ ejemplos para armar el conjunto de
entrenamiento y otros $1000$ para el armado del conjunto de prueba.
La composición de los conjuntos de entrenamiento y prueba de este
problema se resume en la Tabla~\ref{tbl:mainxue}.
