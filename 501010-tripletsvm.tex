%
%
\subsection{Problema Triplet-SVM}
%
Este problema replica los datos utilizador por los autores de
\e{Triplet-SVM} \cite{xue} para la evaluación del método.
Este método fue uno de los primeros métodos de predicción de
pre-miRNAs mediante máquinas de vectores de soporte.
En el trabajo original, los autores utilizan vectores con
características de tripletes como entrada al clasificador SVM.

Los ejemplos de clase positiva son tomados de la versión 5.0 (de
septiembre de 2004) de miRBase.
De los 207 ejemplos de \premirna{s} de la especie humana (clase
positiva) presentes en esta versión, se eliminan aquellos que
contienen ramificaciones en la estructura secundaria, resultando en
193 ejemplos con estructura secundaria en forma de horquilla.
De estos ejemplos, se utilizan 163 para componer el conjunto de
entrenamiento y los 30 restantes para el conjunto de prueba.
Los ejemplos de clase negativa se obtienen de la base de datos
\dataset{coding}, utilizando 168 ejemplos para el conjunto de
entrenamiento y 1000 para el armado del conjunto de prueba.

La composición de los conjuntos de entrenamiento y prueba se resumen
en la \autoref{tbl:mainxue}.
Los ejemplos que componen los conjuntos de entrenamiento y de prueba
son exactamente los mismos que aquellos utilizados en el trabajo
original, ya que los autores publicaron ambos conjuntos por separado
en el material suplementario.
