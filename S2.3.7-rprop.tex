%
\subsubsection{Rprop}
%
El método Rprop (\eng{Resilient backpropagation}, ``retropropagación
resiliente'') \cite{rprop} es un esquema local cuyo principio básico
es eliminar la influencia del tamaño del gradiente sobre el
tamaño de incremento del peso. Consecuentemente, sólo el signo de la
derivada es considerado para indicar la dirección de la actualización
de los pesos.

Al principio, todos los valores de actualización se establecen a
$\Delta_0$, que es uno de los dos parametros de Rprop. Ya que
$\Delta_0$ determina directamente el tamaño de la primera
actualización de los pesos, se debe elegir su valor de acuerdo al
tamaño de los mismos pesos. La elección de este valor es más bien no
crítica, ya que se va adaptando a medida que el aprendizaje procede.

Con el objeto de prevenir que los pesos se vuelvan demasiado grandes,
se define el segundo parámetro $\Delta_{T{máx}}$, que limita el tamaño
máximo de la actualización de los pesos. Comúnmente, la convergencia
es insensible también al valor de este parámetro. Sin embargo, para
algunos problemas puede resultar ventajoso establecer este valor más bajo
con el objetivo de permitir sólo actualizaciones de pesos cautas.

El tamaño del cambio de los pesos se determina por un
valor de actualización específico para cada peso $\Delta_{ij}$:
%
\begin{align}
\label{mlp20}
  \Delta{}w_{ij}(t) = 
  \begin{cases}
    -\Delta_{ij}(t), & \T{si } \dpar{E}{w_{ij}}{}(t) > 0 \\
    +\Delta_{ij}(t), & \T{si } \dpar{E}{w_{ij}}{}(t) < 0 \\
    0, & \T{en otro caso}
  \end{cases}
\end{align}
%
El segundo paso del método Rprop es determinar los valores nuevos
$\Delta_{ij}(t)$. Esto se basa en un proceso de adaptación dependiente
del signo
%
\begin{align}
  \Delta{}_{ij}^{(t)} = 
  \begin{cases}
    \eta^+\,\Delta_{ij}^{(t-1)}, & \T{si }\dpar{E}{w_{ij}}{}^{(t-1)}
      \dpar{E}{w_{ij}}{}^{(t)} > 0 \\
    \eta^-\,\Delta_{ij}^{(t-1)}, & \T{si }\dpar{E}{w_{ij}}{}^{(t-1)}
      \dpar{E}{w_{ij}}{}^{(t)} < 0 \\
    \Delta_{ij}^{(t-1)}, & \T{en otro caso},
  \end{cases}
  && 0<\eta^-<1<\eta^+.
\end{align}
%
Los factores de incremento y decremento se establecen en $\eta^+=1,2$
y $\eta^-=0.5$ respectivamente. Estos valores se basan en
consideraciones teóricas y evaluaciones empíricas \cite{riedmiller},
reduciendo el número de parámetros libres a 2: $\Delta_0$ y
$\Delta_{T{máx}}$.

Para resumir, el principio básico de Rprop es la adaptación directa de
los valores de actualización de los pesos $\Delta_{ij}$. Contrastando
con los algoritmos que ajustan la velocidad de aprendizaje, Rprop
modifica directamente el tamaño del paso introduciendo el concepto de
valores de actualización ``resilientes''. Como resultado, el esfuerzo
de adaptación no se ``desenfoca'' como resultado de un comportamiento
impredecible del gradiente. Dada la claridad y simplicidad de las
leyes de aprendizaje, se introduce un pequeño incremento en el tiempo de
entrenamiento, en comparación con la retropropagación ordinaria.

Rprop sufre de los mismos problemas que el resto de los algoritmos de
aprendizaje adaptativos. Ya que la adaptación se basa en una
estimación de la topología de la función de error, tanto la adaptación
como la actualización de los pesos se puede efectuar recién luego de
que toda la información de gradiente esté disponible, en otras
palabras, luego de que cada patrón haya sido presentado y el gradiente
de la suma de los errores de todos los patrones es conocida.

Consecuentemente, los procedimientos de aprendizaje adaptativos
típicamente se basan en el ``aprendizaje batch''. Esto posiblemente
reduce su eficacia en conjuntos de entrenamiento redundantes
comparados a un descenso del gradiente estocástico, y presenta
problemas con el uso de conjuntos de entrenamiento variables.

Más aún, a un esquema de adaptación restringido local inherentemente le
falta la visión del todo que las técnicas globales podrían llegar a
tener. Si por ejemplo la dirección óötima del mínimo se sitúa sobre
una diagonal, un esquema local tratará de reducir el error en cada
dimensión con pasos pequeños en lugar de incrementar el paso compuesto
sobre la diagonal, que sería el enfoque más apropiado en este caso.

Sin embargo, los resultados reportados en las pruebas
\cite{riedmiller} muestran las propiedades favorables de las
estrategias de adaptación locales en las aplicaciones prácticas.
