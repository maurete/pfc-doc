%
\begin{table}[h]
  \tableStyle
  \smaller
  \iflatexml%
  \begin{tabular}{lrrrrrrr}
  \else%
  \sisetup{
    table-format = 2.1(2),
    table-number-alignment = right,
    separate-uncertainty=true,
    %uncertainty-separator = \pm
  }
  \begin{tabular}{lS[table-format=2.0]
      S[table-format=4.0]SSSSS[table-format=2.1]}
  \fi%
    \toprule
    {Problema} & {Clase} & {Elems.} &
    {MLP-B}    & {SVM-LE}   & {SVM-RE}   & {SVM-RR}   & \cite{xue}\\
    \midrule
    updated           & +1 &   39 &
    94.4(11) & 94.9(00) & 93.3(14) & 94.9(00) & 92.3 \\
    cross-species     & +1 &  581 &
    90.7(30) & 96.0(01) & 92.3(33) & 79.2(00) & 90.9 \\
    conserved-hairpin & -1 & 2444 &
    91.8(31) & 91.6(00) & 93.4(02) & 95.7(00) & 89.0 \\
    \bottomrule
    \\
  \end{tabular}
  \caption{\captionStyle
    Tasas de clasificación obtenidas para los problemas basados en los
    conjuntos de prueba suplementarios de \work\tripletsvm{}.
    En la última columna, la tasa de clasificación reportada en
    el trabajo original \cite{xue}.
  }
  \label{tbl:suppl-xue}
%
\end{table}
%
