\section{Clasificación de pre-miRNAs}
Los métodos computacionales para clasificación de pre-miRNAs se basan
en técnicas Aprendizaje Automático, una de las principales disciplinas
de la Inteligencia Computacional.  En general, estas técnicas de
\e{aprendizaje supervisado} permiten generar un modelo o
representación interna que caracteriza los pre-miRNAs extrayendo
información a partir de un conjunto de datos de ejemplo denominado
\e{conjunto de entrenamiento}.  El modelo generado en la etapa de
entrenamiento puede ser aplicado a nuevos conjuntos de datos,
generando de esta manera un \emph{clasificador} con capacidad de
\e{generalización}, que es capaz de discriminar nuevos ejemplos en dos
clases: \e{positiva}, asociada a pre-miRNAs ``reales'', y
\e{negativa}, asociada a los ejemplos que no se corresponden con
pre-miRNAs.

En el presente trabajo se utilizan dos técnicas de aprendizaje
supervisado para la generación de un clasificador de pre-miRNAs: el
\e{Perceptrón Multicapa} \cite{mlp1}\cite{mlp2} y la \e{Máquina de
  Vectores de Soporte} \cite{svm}.

El Preceptrón Multicapa (\e{MLP}, del inglés \eng{Multilayer
  Perceptron}) es un tipo de red neuronal artificial con propagación
hacia adelante, en la que se disponen las neuronas (nodos
computadores) en \e{capas}. La salida de cada neurona se determina al
aplicar una \e{función de activación}, de tipo sigmoidea, a la suma
ponderada de las salidas en la capa anterior. Durante el entrenamiento
del MLP, se ajustan progresivamente los pesos (ponderaciones) de cada
entrada de cada neurona mediante un algoritmo de aprendizaje basado en
la \e{propagación hacia atrás} del error de clasificación, hasta
obtener una tasa de clasificación satisfactoria \cite{jain}.

La Máquina de Vectores de Soporte (\emph{SVM}, de su nombre en inglés
\eng{Support Vector Machine}) es un algoritmo de clasificación que se
basa en transformar, mediante una función llamada \e{núcleo}, el
espacio $N$-dimensional de los datos de entrada en otro espacio de
dimensión $M: M\gg N$, donde se espera que los datos sean linealmente
separables mediante un hiperplano. El entrenamiento consiste en
encontrar el hiperplano óptimo de separación para el conjunto de datos
presentado \cite{bottou}.

El objetivo del presente trabajo es desarrollar un método de
clasificación de pre-miRNAs completo, que permita la utilización de
clasificadores MLP y SVM, y que abarque desde el procesamiento de los
datos de entrada hasta la obtención de las predicciones de clase para
los elementos que se desean clasificar. Adicionalmente, se pretende
que el método desarrollado tenga un alto grado de automatización de
los aspectos que no sean especialidad del usuario final, brindando
facilidad de uso con buenos resultados de clasificación.
