\subsubsection{Cáclulo del gradiente $\nabla\rho$}
El cálculo de las derivadas de $\rho$ se basa en resultados de
análisis de perturbación de problemas de optimización, ya que tanto
$\rho_R$ como $\rho_M$ son soluciones a problemas de este tipo.  La
discusión en este apartado se basa en aquella presentada por
\citeauthor{chung} en \cite{chung}.

El Teorema 4.1 de \cite{bonnans-shapiro} establece:

\begin{quote}
  Considérese un problema de optimización en la forma
  $f_*(\vv)=\min_{\uu\in{}U}f(\uu,\vv)$ con vector $\uu$ de variables a
  optimizar, vector de parámetros $\vv$, y $U$ independiente de
  $\vv$. Entonces, si para cada $\vv$ existe un $\uu_*(\vv)\in{}U$
  tal que $f_*(\vv)=f(\uu_*(\vv),\vv)$, la derivada
  $\dpar{f_*}{\vv}{}(\vv)$ existe y $\dpar{f_*}{v_i}{}$ viene dado por
  $\dpar{f_*}{v_i}{}(\uu_*,\vv)$ para cualquier parámetro $v_i$.
\end{quote}

Como se ha visto anteriormente, exigiendo la no-repetición de ejemplos
en el conjunto de entrenamiento (\ref{cond-kmatrix-defpos}) se
garantiza la existencia de la solución $\uu_*$.
El problema (\ref{svm-oneclass}) cumple con los requisitos del teorema,
luego las derivadas de $R^2$ respecto de los hiperparámetros vienen
dadas por

\begin{align}
  \dpar{R^2}{C}{} &= \dpar{(1-\Bbeta^T\KK\Bbeta)}{C}{}
  = -\Bbeta^T \dpar{\KK}{C}{}\Bbeta = 0, \\
  \dpar{R^2}{\gamma}{} &= \dpar{(1-\Bbeta^T\KK\Bbeta)}{\gamma}{}
  = -\Bbeta^T \dpar{\KK}{\gamma}{}\Bbeta.
\end{align}
En el caso $\rho_M$ el teorema anterior no aplica, ya que

\begin{enumerate}
\item La restricción $\alpha_i\leq{}C$ depende de $C$, violando la condición
  necesaria ``$U$ independiente de $\vv$''.
\item El vector $\Bxi$ óptimo puede no ser único ya que el parámetro $b$ óptimo
  puede no ser único cuando el modelo no contiene vectores de soporte
  ``libres'' (esto es, cuando $u=\emptyset$ (\ref{unbounded-sv-set})).
\end{enumerate}
Para salvar estos inconvenientes, se aprovecha la dualidad del
problema en el punto óptimo $\Balpha_*$, y se efectúa el cambio de
variable $\bar{\Balpha}=\Balpha/C$ en el problema dual
(\ref{svmprob-dual-soft}) de modo que las restricciones sean
independientes del hiperparámetro $C$

\begin{align}
\begin{split}
    \max_{\bar{\Balpha}}\quad&
    f(\bar{\Balpha}) = C^2 \left( \frac{\B{1}^T\bar{\Balpha}}{C}
    -\frac{1}{2}\bar{\Balpha}^T\QQ\bar{\Balpha}\right)\\
    \T{sujeto a}\quad & \yy^T\bar{\Balpha} = 0, \\
    & 0\leq\bar{\alpha}_i\leq 1,
    \T{ para todo } i\in {1,\ldots,\ell }.
\end{split}\end{align}
Este problema cumple con las condiciones del teorema ya que
no depende de $\Bxi$ y cuenta con solución única
ya que $Q$ es definida positiva (\ref{cond-kmatrix-defpos}). Entonces,

\begin{align}
\label{eq:rmb-alpha-equiv}
  \B{1}^T\Balpha-\frac{1}{2}\Balpha^T\QQ\Balpha
  &= C^2\left(\frac{\B{1}^T\bar{\Balpha}}{C} -
  \frac{1}{2}\bar{\Balpha}^T\QQ\bar{\Balpha}\right).
\end{align}
Con estos resultados se está en condiciones de calcular las derivadas de
$\rho$ respecto de los hiperparámetros $C$ y $\gamma$:
Las derivadas respecto a $C$ vienen dadas por

\begin{align}
  \begin{aligned}
    \dpar{\rho_M}{C}{}
    &= \dpar{}{C}{}2\left(  \B{1}^T\Balpha-\frac{1}{2}\Balpha^T\QQ\Balpha \right) \\
    &= \dpar{}{C}{}\left( 2C^2\left(\frac{\B{1}^T\bar{\Balpha}}{C} -
    \frac{1}{2}\bar{\Balpha}^T\QQ\bar{\Balpha}\right)
    \right) \\
    &= 4C \left(\frac{\B{1}^T\bar{\Balpha}}{C} -
    \frac{1}{2}\bar{\Balpha}^T\QQ\bar{\Balpha}\right) - 2C^2 \left(\frac{\B{1}^T\bar{\Balpha}}{C^2} \right) \\
    &= 2\left(\B{1}^T\bar{\Balpha} - C \bar{\Balpha}^T\QQ\bar{\Balpha}\right) \\
    &= \frac{2}{C} \left(\B{1}^T\Balpha - \Balpha^T\QQ\Balpha\right)
  \end{aligned}
  \\
  \begin{aligned}
    \dpar{\rho_M}{C}{}
    &= \dpar{}{C}{} \left( R^2 + \frac{1}{C} \right) \\
    &= -\frac{1}{C^2}.
  \end{aligned}
\end{align}

Las derivadas respecto del hiperparámetro $\gamma$ vienen dadas por

\begin{align}
  \begin{aligned}
    \dpar{\rho_M}{\gamma}{}
    &= \dpar{}{\gamma}{} 2\left(  \B{1}^T\Balpha-\frac{1}{2}\Balpha^T\QQ\Balpha \right) \\
    &= - \Balpha^T \dpar{\QQ}{\gamma}{}\Balpha \\
    & = - \Balpha^T \left(\yy^T \dpar{\KK}{\gamma}{}\yy\right) \Balpha
    %% \\
    %% & = \sum_{i,j=1}^\ell \alpha_i\alpha_j y_i y_j \dpar{k(\xx_i,\xx_j)}{\gamma}{}, \\[0.2em]
  \end{aligned}
  \begin{aligned}
    \dpar{\rho_R}{\gamma}{} &= \dpar{R^2}{\gamma}{} \\
    = - \Bbeta^T \dpar{\KK}{\gamma}{} \Bbeta. % \\
%    &= \sum_{i,j=1}^\ell \beta_i\beta_j \dpar{k(\xx_i,\xx_j)}{\gamma}{}.
  \end{aligned}
\end{align}
Los elementos $\dpar{k_{ij}}{\gamma}{}$ de la matriz $\dpar{\KK}{\gamma}{}$
vienen ddados por

\begin{align}
  \dpar{k_{ij}}{\gamma}{}
  = \dpar{}{\gamma}{}k(\xx_i,\xx_j)
  = \dpar{}{\gamma}{} \left(e^{-\gamma\|\xx_i-\xx_j\|}\right)
  = -k_{ij}\|\xx_i-\xx_j\|
\end{align}
Con todo esto, las derivadas de la función $\rho$ respecto de los
hiperparámetros vienen dadas por

\begin{align*}
    \dpar{\rho}{C}{} &= \dpar{\rho_M}{C}{} \rho_R + \rho_M \dpar{\rho_R}{C}{} \\
    &= \frac{2}{C} \left(\B{1}^T\Balpha - \Balpha^T\QQ\Balpha\right) \left( R^2 + \frac{1}{C} \right)
    - 2\left(  \B{1}^T\Balpha-\frac{1}{2}\Balpha^T\QQ\Balpha \right)
    \left( \frac{1}{C^2} \right)\\
    &=
  \\[2em]
    \dpar{\rho}{\gamma}{} &= \dpar{\rho_M}{\gamma}{} \rho_R + \rho_M \dpar{\rho_R}{\gamma}{}\\
    &= \left( - \Balpha^T \left(\yy^T \dpar{\KK}{\gamma}{}\yy\right) \Balpha \right)
    \left( R^2 + \frac{1}{C} \right)
    - 2\left(  \B{1}^T\Balpha-\frac{1}{2}\Balpha^T\QQ\Balpha \right)
    \left( \Bbeta^T \dpar{\KK}{\gamma}{} \Bbeta \right)
\end{align*}
donde

\begin{align}
  k_{ij}&=k(x_i,x_j)=e^{-\gamma||x_i-x_j||},\,\, \T{y} \\
  \dpar{k_{ij}}{\gamma}{}&=-k_{ij}||x_i-x_j||
\end{align}
son la función kernel RBF y su derivada respecto del hiperparámetro $\gamma$,
respectivamente.

