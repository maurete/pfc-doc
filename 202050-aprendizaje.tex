%
%
\subsection{Entrenamiento}
%
El objetivo del entrenamiento en el perceptrón multicapa es ajustar
los pesos de la red de forma que ésta efectúe la transformación
expresada por el conjunto de entrenamiento $D=(\xx(n),\yy(n))$,
$n=1,\ldots,\ell$.
La diferencia entre las salidas deseadas y las salidas efectivas de la
red se mide con la función de \e{energía} del error $E$:
%
\begin{align}\label{e2:energy-function}
  E = \frac{1}{2}\sum_{j=1}^{\ell}\|\yy_j-\hat{\yy}_j\|,
\end{align}
%
en donde $\hat{\yy}_j$ es el vector formado con las salidas de cada
neurona de la capa de salida.
La función $E$ puede interpretarse como una medida de {aptitud} de los
pesos de la red al conjunto $D$, de modo que satisfacer el objetivo
del aprendizaje es equivalente a encontrar un mínimo global de $E$.
La discusión a continuación se basa en aquella presentada en
\cite[Capítulo~4]{haykin}.

Considerando una neurona $i$ en la capa de salida, sea $s_i(n)$ la
salida de la misma en respuesta al estímulo $\xx(n)$ aplicado en la
capa de entrada.
La \e{señal de error} producida en la salida de esta neurona viene
dada por
%
\begin{align}\label{e2:error-signal-neuron} %4.2
  e_i(n)=y_{k}(n)-s_{i}(n)
\end{align}
%
en donde $y_k(n)$ es la componente del vector de respuesta deseada
$\yy(n)$ correspondiente a la posición de la neurona $i$ en la capa de
salida.
%% en donde $y_{i}$ es el $i$-ésimo elemento del vector de respuesta
%% deseada $\yy$.
La \e{energía de error instantáneo} de la neurona $i$ se define según
%
\begin{align}\label{e2:error-energy-neuron} %4.3
  \C{E}_i(n)=\frac{1}{2}e^2_i(n).
\end{align}
%
Sumando las contribuciones error-energía de todas las neuronas en la
capa de salida, se expresa la \e{energía total de error instantáneo}
de la red como
%
\begin{align}\label{e2:error-energy-net} %4.4
  \C{E}(n)&=\sum_{i\in C}\C{E}_i(n) \\
  &=\frac{1}{2}\sum_{i\in C}e^2_i(n).
\end{align}
%
Aquí, el conjunto $C$ contiene los índices de las neuronas en la capa
de salida.
La energía de error promedio sobre el conjunto de entrenamiento viene
dada por
%
\begin{align}\label{e2:average-energy-net} %4.5
  \C{E}_{\T{av}}&=\frac{1}{\ell}\sum_{n=1}^\ell\C{E}(n) \\
  \tab=\frac{1}{2\ell} \sum_{n=1}^\ell \sum_{i\in C}e^2_i(n).
\end{align}
%
Naturalmente, tanto la energía de error instantáneo como la energía de
error promedio son funciones de todos los pesos sinápticos de la red.
Esta dependencia funcional no fue incluida de forma explícita en la
definición de $\C{E}(n)$ y $\C{E}_{\T{av}}$ en pos de simplificar la
notación.
