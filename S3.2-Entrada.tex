%
%
%
\section{Formato de entrada}
%
Los datos de entrada vienen dados en un formato estándar denominado
FASTA, originado por el software de igual nombre publicado
inicialmente en 1985 por \citeauthor{fasta} \cite{fasta}.  Para cada
archivo, se especifica
%
\begin{enumerate}
\item nombre del archivo,
\item clase de los ejemplos contenidos, y
\item proporción de elementos a utilizar para entrenamiento y prueba.
\end{enumerate}
%
Cuando el archivo contiene ejemplos de pre-miRNAs verdaderos, la clase
es ``positiva'' ($+1$), cuando contiene ejemplos de ``pseudo''
pre-miRNAs la clase se establece a ``negativa'' ($-1$).  No se permite
la lectura de archivos con elementos de clase mixta.  La proporción de
elementos a utilizar para entrenamiento y prueba determina la cantidad
de ejemplos de cada archivo que pasarán a integrar el conjunto de
entrenamiento, utilizado para la generación del modelo, y el de
prueba, utilizado para clasificación.

Los archivos FASTA son archivos de texto que contienen secuencias
\e{anotadas}, con líneas de descripción intercaladas entre fragmentos
de secuencia.
La línea de descripción comienza con el carácter \mono{>} seguido por
un nombre identificador (``número de acceso'') de la secuencia.  La
línea continúa con una descripción del ejemplo, que
puede contener, por ejemplo, el nombre científico de la especie, o la
posición de la secuencia en el genoma de la especie. Tanto el nombre
identificador como la descripción son opcionales.

Las líneas de secuencia siguen a la de descripción, y consisten en
carácteres contiguos, en el que cada carácter representa un
nucleótido. En el ácido ribonucleico (RNA), un nucleótido puede estar
compuesto por una de las cuatro bases siguientes:
%
\begin{enumerate}
\item Adenina (\mono{A}),
\item Citosina (\mono{C}),
\item Guanina (\mono{G}),
\item Uracilo (\mono{U}).
\end{enumerate}
%
Las líneas de secuencia se componen entonces por caracteres \mono{A},
\mono{C}, \mono{G}, y \mono{U}.  La secuencia termina cuando se
encuentra una línea que comience con el carácter \mono{>}, dando lugar
a un nuevo ejemplo, o bien, el fin de archivo.
En la \iflatexml{}Figura~\ref{fastafmt}\else\autoref{fastafmt}\fi se
muestra un fragmento de archivo FASTA
con 3 ejemplos de pre-miRNAs del organismo modelo \e{C. elegans}.
