%
\subsubsection{Validación cruzada de $k$ iteraciones}
\label{s2:crossval}
%
La validación cruzada es una técnica que permite estimar el error de
generalización probando (``validando'') el modelo sobre todos los
ejemplos del conjunto de entrenamiento.
En un procedimiento de \e{validación cruzada de $k$ iteraciones}
\cite{crossval}, el conjunto de entrenamiento $D$ se subdivide en $k$
conjuntos disjuntos $V_1,\ldots,V_k$ de tamaños similares, denominados
\e{particiones de validación}.
Para cada $i=1,\ldots,k$, se efectúa el entrenamiento sobre un
conjunto $D_i=D\setminus{}V_i$ que excluye a $V_i$, y se calcula la
tasa de error del modelo resultante sobre la partición $V_i$.
Una vez completado el proceso, se promedian las $k$ medidas de error
para obtener un estimador del error sobre el conjunto completo $D$.
Esta medida es un estimador más preciso del error de generalización
que la obtenida con el método de retención.
%% El costo computacional de la validación cruzada es $k$ veces
%% superior al del método de retención, y en general se justifica su
%% utilización
