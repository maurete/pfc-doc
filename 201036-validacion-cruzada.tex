%
\subsubsection{Validación cruzada de $k$ iteraciones}
%
La validación cruzada consiste en entrenar y evaluar 
múltiples modelos de forma tal que todos los ejemplos del conjunto de
entrenamiento sean utilizados para validación.
En un procedimiento de \e{validación cruzada de $k$ iteraciones}
\cite{crossval}, el conjunto de entrenamiento $D$ se subdivide en $k$
conjuntos disjuntos $V_1,\ldots,V_k$ de tamaños similares, denominados
\e{particiones de validación}.
Para cada $i=1,\ldots,k$, se efectúa el entrenamiento sobre un
conjunto $D_i=D\setminus{}V_i$ excluyendo a $V_i$, y se evalúa el
modelo obtenido con los datos de $V_i$.
Una vez completado el proceso se habrán validado todos los ejemplos en
el conjunto $D$ con modelos diferentes, aunque entrenados con conjuntos
de estimación similares.
Tal como en el caso de la validación ``simple'', la validación cruzada
se utiliza comúnmente para ajustar los hiperparámetros, antes de
entrenar el modelo final sobre el conjunto completo $D$.
