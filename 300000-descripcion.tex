%
%
%
%
\chapter{Descripción del software}
%
El software desarrollado es un sistema completo de reconocimiento de
\premirna{s} cuya funcionalidad puede describirse en torno a tres
tareas principales:
%
\begin{enumerate}
\item \e{Preprocesamiento}: abarca la funcionalidad necesaria para
  transformar los datos de entrada a un formato utilizable por la
  máquina de aprendizaje.
\item \e{Generación del modelo del clasificador}: efectúa el
  entrenamiento de la máquina de aprendizaje seleccionando
  aquellos \hparam{s} que maximicen el desempeño del clasificador.
\item \e{Clasificación}: aplica el modelo sobre datos
  nuevos para obtener predicciones de clase.
\end{enumerate}
%
Como punto de partida se consideraron los trabajos previos
\cite{xue,ng,batuwita,sheng,sewer,ding}, donde se presentan
herramientas de software con propósito similar al del método
desarrollado en el presente trabajo. En particular, se utilizaron
conjuntos de datos y técnicas de extracción de \caract{s} de
\cite{xue,ng,batuwita} debido a la disponibilidad de los datos
suplementarios.

La implementación se efectuó en lenguaje Matlab versión R2012b y hace
uso de los módulos adicionales ``Neural Network Toolbox'' y
``Bioinformatics Toolbox'' del mismo software. Además, integra
funcionalidad con las herramientas ``libSVM'' \cite{libsvm} y RNAFold
del paquete ``Vienna RNA'' \cite{vienna}. Se utilizó el software
``Web-demo Builder''\cite{webdemobuilder} para la creación de la
interfaz web.

En adelante, se describen la tres tareas principales de
preprocesamiento, generación del modelo del clasificador y
clasificación, precedidas por una descripción general del
funcionamiento del software.
