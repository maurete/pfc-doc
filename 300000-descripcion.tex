%
%
%
%
\setcounter{chapter}{2}
%
\chapter{Descripción del clasificador desarrollado}
%
El método desarrollado abarca un sistema completo de reconocimiento de
\premirna{s}, con funcionalidad para la generación de un modelo de
clasificador y para su posterior aplicación sobre nuevos datos.

Como punto de partida se consideraron los trabajos
\cite{xue,ng,batuwita,sheng,sewer,ding}, en los que se presentan
clasificadores con propósito similar al del presente desarrollo.
En particular, debido a la disponibilidad de datos suplementarios, se
utilizaron los conjuntos de datos y se aplicaron las técnicas de
extracción de \caract{s} de los trabajos \cite{xue,ng,batuwita}.

La implementación se efectuó en lenguaje Matlab versión R2012b,
utilizando los módulos adicionales ``Neural Network Toolbox''
y ``Bioinformatics Toolbox''.
Se decidió utilizar el lenguaje Matlab siguiendo un criterio práctico,
ya que el mismo es ampliamente utilizado en la disciplina para la
generación de modelos matemáticos y brinda funcionalidades específicas
en forma de módulos adicionales.
Se integraron asimismo la biblioteca \work{libSVM} \cite{libsvm} para
las máquinas de vectores de soporte y la herramienta \work{RNAFold}
\cite{vienna} para el cálculo de la estructura secundaria.
Se utilizó el software \work\webdemo{} \cite{webdemobuilder} para
la creación de una interfaz web de demostración.
