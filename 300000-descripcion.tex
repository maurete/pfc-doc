%
%
%
%
\setcounter{chapter}{2}
%
\chapter{Descripción del clasificador desarrollado}
%
%% El desarrollo abarca un sistema completo de reconocimiento de
%% \premirna{s}, con funcionalidad para la generación de un modelo de
%% clasificador y para su posterior aplicación sobre nuevos datos.
%
Como punto de partida se consideraron los trabajos
\cite{xue,ng,batuwita,sheng,sewer,ding}, en los que se presentan
clasificadores con propósito similar al del presente desarrollo.
En particular, debido a la disponibilidad de datos suplementarios, se
utilizaron los conjuntos de datos y se aplicaron las técnicas de
extracción de \caract{s} de los trabajos \cite{xue,ng,batuwita}.

Se optó por utilizar el lenguaje Matlab, dado que el mismo es
ampliamente utilizado en la disciplina para la generación de modelos
matemáticos y brinda funcionalidades específicas en forma de módulos
adicionales.
La codificación se efectuó sobre un entorno Matlab versión R2012b,
utilizando los módulos adicionales ``Neural Network Toolbox'' y
``Bioinformatics Toolbox''.
Se integró asimismo funcionalidad de la biblioteca \work{libSVM}
\cite{libsvm} para las máquinas de vectores de soporte y de la
herramienta \work{RNAFold} \cite{vienna} para el cálculo de la
estructura secundaria.
Se utilizó el software \work\webdemo{} \cite{webdemobuilder} para
la creación de una interfaz web de demostración.
