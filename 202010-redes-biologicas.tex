%
%
\subsection{Justificación: las redes neuronales biológicas}
%
Las redes neuronales naturales, entre las que contamos el cerebro
humano, son computadoras biológicas con características deseables que
no se encuentran en las computadoras clásicas, tales como un alto
grado de paralelismo, cómputo distribuido, capacidad de aprendizaje y
de generalización, adaptabilidad y tolerancia a fallos.  Las redes
neuronales artificiales (\eng{Artificial Neural Networks, ANNs}) son
un paradigma computacional inspirado en las redes neuronales
biológicas. Junto con un algoritmo de entrenamiento adecuado, las
ANNs pueden funcionar como máquinas de aprendizaje, incorporando
las características deseables de las redes neuronales naturales.

Biológicamente, una neurona (o célula nerviosa) es un tipo especial de
célula encargada de procesar información.  Se compone de un cuerpo
celular, denominado \e{soma}, y dos tipos de estructuras ramificadas:
el axón y las dendritas. Una neurona recibe señales (impulsos
eléctricos) provenientes de otras neuronas a través de sus dendritas
(receptores), y transmite señales generadas por su cuerpo celular a
través del axón, que se ramifica en ``hebras''.

Entre la terminal de la hebra proveniente de una neurona y la
dendrita de otra, se encuentra una estructura elemental denominada
sinapsis.  Cuando un impulso eléctrico alcanza la terminal,
se liberan sustancias denominadas neurotransmisores. Estos
neurotransmisores se dispersan a través de la sinapsis, aumentando o
disminuyendo, dependiendo del tipo de sinapsis, la tendencia de la
neurona receptora de emitir un impulso eléctrico propio.

En el tiempo, la efectividad de las sinapsis se ajusta según las
señales que han pasado por ella, de modo que la propia sinapsis
\e{aprende} de las actividades en las que participa. Esta dependencia
de las señales históricas actúa como una \e{memoria}, y es
posiblemente la dimensión física de la memoria humana, relacionada
asimismo a la \e{capacidad de aprendizaje} del cerebro humano.
