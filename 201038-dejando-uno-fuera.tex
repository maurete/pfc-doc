%
\subsubsection{Validación cruzada dejando uno fuera}
%
La \e{validación cruzada dejando uno fuera}
es el caso especial del procedimiento de validación cruzada 
cuando se establece el número de particiones $k$ igual al
número de elementos $\ell$ del conjunto de entrenamiento $D$.
Aplicando esta estrategia, el modelo obtenido en cada iteración
es muy similar a uno entrenado con el conjunto completo $D$.

La importancia de este procedimiento reside en el hecho que es el
mejor estimador del error de generalización de la máquina de aprendizaje.
En la práctica, no es muy utilizado debido a su elevado costo
computacional, ya que se requieren $\ell$ entrenamientos para su
cálculo.
Sin embargo, existen técnicas para el cálculo eficiente
para ciertas máquinas de aprendizaje específicas
\cite{chapelle,lee-keerthi}.
