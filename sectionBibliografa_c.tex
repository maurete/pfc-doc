\section{Bibliografía consultada}
\label{bibliografia}

A continuación se enumera la bibliografía consultada, referente a la
perspectiva biológica de los pre-miRNA, las técnicas de clasificación
SVM y MLP, y diferentes implementaciones de clasificadores pre-miRNA
utilizando estas técnicas de Inteligencia Computacional. Se presenta
una breve reseña orientativa junto a cada publicación.

\subsection{D. P. Bartel \cite{bartel116}}
Este trabajo brinda una revisión de los desarrollos en la
investigación de los microRNAs y las conclusiones que surgen de éstos
acerca de la biogénesis y función de los miRNAs.  Ofrece una
explicación didáctica del proceso de maduración de un pre-miRNA en
miRNA.  Publicado en 2004, es anterior a la aparición de los métodos
computacionales no-comparativos que permitieron un gran aumento en el
número de nuevos miRNAs descubiertos.

\subsection{L. Li et al. \cite{lili}}
En este trabajo, publicado en 2010, se ofrece un panorama de diversos
métodos para la identificación de miRNAs junto a una revisión de las
características biológicas de los mismos. Se presentan métodos tanto
de tipo comparativo como aquellos basados en aprendizaje de máquina,
así como métodos que trabajan sobre los datos experimentales de
técnicas de secuenciamiento en gran escala o \eng{deep sequencing}.
También se presenta el estado del arte en la predicción de dianas
(\eng{targets}) y regulación de la expresión de los miRNAs.

\subsection{D. Yue et al. \cite{yue}}
Se presenta en este trabajo un resumen de las técnicas disponibles
para la predicción de dianas y de métodos de predicción de las
funciones de los miRNAs, con una exposición clara de los conceptos
relevantes en este campo de estudio.

\subsection{L. Bottou y C.-J. Lin \cite{bottou}}
Este trabajo presenta una descripción del problema general de
clasificación mediante SVM \cite{svm}, detallando el problema de
optimización a resolver y los parámetros a considerar en la
utilización e implementación de las máquinas de vectores de soporte.

\subsection{C. Xue et al. \cite{xue}}
En este trabajo se presenta un clasificador de secuencias de
pre-microRNA mediante SVM \cite{svm}.  En el clasificador se utiliza
un conjunto de características de tipo estructura-secuencia,
utilizando como entrada al clasificador un vector de frecuencia de 32
``triplets'' que combinan el nucleótido de la secuencia con la
estructura secundaria del entorno donde éste se presenta.

El método aquí descripto es notable en que obtiene una buena tasa de
clasificación utilizando una combinación simple de características de
la secuencia y la estructura secundaria.

\subsection{S. K. L. Ng y S. K. Mishra \cite{ng}}
En este trabajo se presenta un clasificador de secuencias de
pre-miRNA mediante SVM.

El conjunto de características utilizado consiste en 29 medidas
``globales e intrínsecas'' a la secuencia y su estructura secundaria:
(1) 17 medidas de composición de la secuencia: frecuencia de
ocurrencia de dinucleótidos
$\mono{XY}:(\mono{X},\mono{Y})\in\{\mono{A},\mono{G},\mono{C},\mono{U}\}$
y frecuencia agregada de ocurrencia de los nucleótidos \mono{G} y
\mono{C}; (2) 6 medidas de plegado basadas en la mínima energía libre
y la distribución de los pares de bases; (3) un descriptor
topológico; y (4) 5 variantes normalizadas de estas características.

El método además es probado para detectar pre-miRNAs en genes de
virus, recorriendo éste mediante una ventana deslizante de 95nt y
clasificando las secuencias extraídas.

\subsection{R. Batuwita y V. Palade \cite{batuwita}}
En esta publicación se describe un método de clasificación de
pre-miRNAs que elabora sobre \cite{ng} incorporando características
basadas en la mínima energía libre, otras características relacionadas
el programa RNAfold, y características basadas en los pares de bases.

Se discute acerca de la relevancia de cada una de las características
y se presenta luego un conjunto reducido de éstas para mejorar la
performance del clasificador, y también se discute acerca del problema
de \emph{desbalance de clases}, siempre presente en la
clasificación de pre-miRNAs.

\subsection{S. Sewer et al. \cite{sewer}}
Este trabajo describe un método de predicción de microRNAs en el
genoma homano, de rata y de ratón. El método tiene un enfoque
comparativo para la selección de las regiones candidatas del genoma,
así como un enfoque no-comparativo para la predicción de los microRNAs
en estas regiones. Para la clasificación se utilizan Máquinas de
Vector de Soporte.

\subsection{J. Hertel y P. F. Stadler \cite{hertel}}
Este trabajo presenta un método de clasificación de pre-miRNAs basado
en características de la secuencia y de la estructura secundaria, con
un clasificador basado en SVM. Como paso inicial se utiliza una
técnica de ``alineación múltiple'' de la secuencia, ajustando una
ventana para determinar la posición exacta del pre-miRNA candidato en
la región.  El resultado es una elevada especificidad del 99\% (muy
pocos falsos positivos), con una sensibilidad del 80\%.

\subsection{Y. Xu et al. \cite{xu}}
En este trabajo se presenta un método no-comparativo que en lugar de
utilizar un método de aprendizaje supervisado como SVM, utiliza un
algoritmo de \eng{ranking} basado en \eng{random walks}.  Este método
se caracteriza por no requerir de ejemplos negativos para el
entrenamiento. Finalmente el método es aplicado para la identificación
de nuevos pre-miRNAs en \eng{Anopheles gambiae}, un mosquito que es el
principal vector de la malaria en África.

\subsection{J. Ding et al. \cite{ding}}
Este trabajo presenta un método de clasificación mediante SVM con un
enfoque específicamente diseñado para resolver el problema de
desbalance de ejemplos positivos y negativos, y que a la vez no asume
características estructurales de los pre-miRNAs.

Se describe la utilización conjunta de diferentes clasificadores SVM
para reducir los problemas del desbalance de clases. Por otro lado
incorpora características de pre-miRNAs tipo multi-loop y realiza una
selección del conjunto de características finales a considerar
mediante la técnica \eng{F-score}, que permite seleccionar aquellas
características de entrada más ``relevantes'' para el clasificador.