%
%
%
\subsection{Normalización de los vectores de características}
%
%% Los elementos que componen el vector de características representan
%% cantidades diversas con rangos numéricos diferentes, según su
%% naturaleza. Esta disparidad numérica entre los elementos del
%% mismo vector implica que, al efectuar operaciones algebraicas
%% algunas \caract{s} de valor absoluto mayor tendrán más ``peso''
%% que otras de menor magnitud.

La normalización de los vectores de características consiste en
modificar el rango de cada variable a un intervalo determinado, por
ejemplo $[0,1]$ o $[-1,+1]$. Esta normalización genera problemas mejor
condicionados, al tiempo que incrementa la velocidad de convergencia
del entrenamiento \cite{nnfaq2}.

Para la normalización se considera un conjunto de datos representativo
$I$, tal como aquel formado con el primer achivo leído a la entrada.
Este conjunto de $\ell$ vectores de $F$ características, conforma una
matriz $M^I{}_{(\ell\times{}F)}$, en la que cada fila $i$ corresponde
a un ejemplo y cada columna $j$ representa una variable
(\caract{}). Para cada columna de $M^I$, se calcula un desplazamiento
$d_j(I)$ y un factor de escalado $s_j(I)$ que permiten transformar el
rango de la variable $x_j$ al intervalo especificado. Por ejemplo,
para llevar las variables al rango $[0,1]$ se tiene
%
\begin{align}
  d_j(I) &= - \min_i m_{ij}, &
  s_j(I) &= \frac{1}{\max_i m_{ij} - \min_i m_{ij}}.
\end{align}
%
Similarmente, para llevar las variables al intervalo simétrico
$[-1,+1]$,
%
\begin{align}
  d_j(I) &= -\frac{1}{2}\left(\max_i m_{ij} + \min_i m_{ij}\right), &
  s_j(I) &= \frac{2}{\max_i m_{ij} - \min_i m_{ij}}.
\end{align}
%
El desplazamiento $\B{d}(I)$ y el factor de escala $\B{s}(I)$
calculados a partir de $I$ se aplican a todos los datos
de entrada al clasificador, manteniendo las diferencias relativas
entre los diferentes conjuntos de datos. La normalización de un
vector de características $\xx=(x_{1},x_{2},\ldots,x_{F})$ se
efectúa haciendo
%
\begin{align}
  x_j^{*} = ( x_j + d_j(I) ) s_j(I), \quad j=1,\ldots,F.
\end{align}
%
%donde $\C{S}$ es la matriz diagonal tal que $\C{S}_{jj}=s_j$.

La normalización aplicada a los conjuntos de datos durante el
entrenamiento tiene un impacto directo sobre el modelo generado, ya
que el mismo aprende a clasificar vectores de características
\e{normalizados}. Por ello, la información de normalización se guarda
como parte del mismo modelo y es aplicada automáticamente a cada nuevo
ejemplo a clasificar.
