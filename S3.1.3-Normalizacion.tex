%
%
%
\subsection{Normalización de los vectores de características}
%
Las distintas variables que componen el vector de \caract{s}
poseen rangos numéricos diferentes, dada su naturaleza diversa.
Esto implica que algunos componentes del vector de \caract{s}
tendrán magnitudes mayores que otros, generando una ponderación
implícita de algunas variables por sobre otras.
La normalización de los vectores de características permite evitar
este problema modificando el rango de cada variable (posición dentro
del vector) a un intervalo preestablecido, por ejemplo $[0,1]$ o
$[-1,+1]$.
La normalización entonces redunda en una ponderación ``equitativa'' de
todas las \caract{s}, generando problemas mejor condicionados, e
incrementando la velocidad de convergencia en el entrenamiento
\cite{nnfaq2}.

El método de normalización procede del siguiente modo: en
primer lugar, se considera un conjunto de datos representativo
$I$, tal como aquel formado con el primer achivo leído a la entrada.
Este conjunto de ${N}$ vectores de $F$ características, conforma una
matriz $M^I{}_{({N}\times{}F)}$, donde cada fila $i$ corresponde
a un ejemplo y cada columna $j$ representa una variable
(\caract{}). Para cada columna de $M^I$, se calcula un desplazamiento
$d_j(I)$ y un factor de escalado $s_j(I)$ que permiten transformar el
rango de la variable $x_j$ al intervalo especificado.

Para llevar las variables al rango $[0,1]$, $d_j(I)$ y $s_j(I)$ se
calculan según
%
\begin{align}
  d_j(I) &= - \min_i m_{ij}, &
  s_j(I) &= \frac{1}{\max_i m_{ij} - \min_i m_{ij}}.
\end{align}
%
Similarmente, para llevar las variables al intervalo simétrico
$[-1,+1]$,
%
\begin{align}
  d_j(I) &= -\frac{1}{2}\left(\max_i m_{ij} + \min_i m_{ij}\right), &
  s_j(I) &= \frac{2}{\max_i m_{ij} - \min_i m_{ij}}.
\end{align}
%
El desplazamiento $\B{d}(I)$ y el factor de escala $\B{s}(I)$
calculados a partir de $I$ se aplican a todos los datos
de entrada al clasificador, manteniendo las diferencias relativas
entre los diferentes conjuntos de datos. La normalización de un
vector de características $\xx=(x_{1},x_{2},\ldots,x_{F})$ se
efectúa haciendo
%
\begin{align}
  x_j^{*} = ( x_j + d_j(I) ) s_j(I), \quad j=1,\ldots,F.
\end{align}
%

El hecho de aplicar normalización a los conjuntos de datos durante el
entrenamiento tiene un impacto directo sobre el modelo generado, ya
que el mismo aprende a clasificar vectores de características
\e{normalizados}. Por ello, la información de normalización se guarda
como parte del mismo modelo y se aplica automáticamente a cada nuevo
ejemplo a clasificar.
