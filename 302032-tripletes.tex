%
%
\subsubsection{Características de tripletes}
%
Las \caract{s} de tripletes, propuestas en el método \e{Triplet-SVM}
\cite{xue}, se basan en la idea de que la propiedad distintiva de los
\premirna{s} es una estructura secundaria en forma de horquilla con
una alta complementariedad entre sus ``brazos''.

En el trabajo original, los autores proponen un método para clasificar
secuencias con estructura secundaria en forma de horquilla entrenando
con \premirna{s} ``reales'' y otras secuencias con la misma estructura
secundaria denominadas ``pseudo \premirna{s}''.
Las \caract{s} de tripletes se justifican en la observación empírica
que la distribución (frecuencia de aparición) de sub-estructuras
locales en la región del tallo de la horquilla difiere
significativamente entre ambas clases.

%% A partir de estas observaciones, proponen generar un conjunto de
%% \caract{s} que representa la distribución de las sub-estructuras
%% locales dentro del ``tallo'' de la horquilla.

Un ``triplete'' es una cadena de caracteres que relaciona la base de
un nucleótido en la posición $i$ (\ntA, \ntC, \ntG, o \ntU) con su
estructura secundaria local en $i-1,i,i+1$, representada en forma
binaria como ``acoplado'' \pairL (paréntesis) y ``no acoplado''
\noPair (punto).

El conjunto de \caract{s} de tripletes incluye medidas que cuentan
el número de ocurrencias de las 32 combinaciones posibles de tripletes
en la región del tallo junto con 4 medidas auxiliares derivadas
del cálculo de los tripletes:
%
\begin{itemize}
\item longitud del tallo,
\item número de pares de bases,
\item complementariedad de ambos brazos de la horquilla,
\item proporción de bases \ntG y \ntC en la zona del tallo.
\end{itemize}
%

En la \iflatexml{}Figura~\ref{triplet}\else\autoref{triplet}\fi{} se
ilustra el proceso de extracción de tripletes para el \premirna{}
\sbs{cel-lsy-66} correspondiente al organismo modelo \e{C. elegans}
\cite{mirbase1}, con anotaciones que indican las distintas partes que
componen la horquilla: tallo, bucle central, brazos, extremos, bases
sueltas y acopladas, y pares de bases.

La principal limitación de las características de tripletes reside en
que su cálculo resulta posible únicamente cuando la estructura
secundaria tiene forma de horquilla, ya que de otro modo la definición
del ``tallo'' pierde sentido. Por ello, estas \caract{s} no se
calculan cuando el ejemplo contiene bucles múltiples en su estrucutra
secundaria.

En la tabla a continuación se detallan las 36 características de
tripletes y su posición en el vector de características.
