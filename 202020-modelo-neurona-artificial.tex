%
%
\subsection{Modelo computacional de una neurona}
%
El modelo computacional básico de una neurona \cite{mlp1} consiste en
una \e{función de activación} binaria que calcula su valor \e{de
  salida} a partir de una suma ponderada de las variables \e{de
  entrada}:
si el valor de la suma alcanza un valor mínimo (umbral) $U$, entonces
el valor de salida se ``activa'' al valor $1$.
Matemáticamente, este modelo de la neurona se escribe
%
\begin{align}\label{e2:neuron-basic}
  y \tab = H\left( \sum_{j=1}^{N} w_j x_j - U\right),
\end{align}
%
en donde $H$ es la función escalón unitario, y los \e{pesos} $w_j$ son
las ponderaciones aplicadas a las diferentes entradas.
Los pesos $w_j$ suelen denominarse \e{pesos sinápticos}, ya que
cumplen una función análoga a las sinapsis biológicas, amplificando o
reduciendo la importancia de las diferentes señales de entrada según
sea su valor.

%% Imaginando que esta neurona es parte de una red (un nodo en un grafo),
%% en donde cada peso $w_j$ multiplica la señal proveniente de otra
%% neurona a través de una arista del grafo (conexión), resulta posible
%% efectuar una analogía entre este modelo artificial y una neurona
%% biológica: el peso $w_j$ representa la sinapsis, las conexiones entre
%% neuronas representan las hebras y dendritas, y la función de
%% activación representa la actividad neuronal que acontece en el soma.

Una forma general del modelo de una neurona $i$
(\iflatexml{}Figura~\ref{fig:neurona}\else\autoref{fig:neurona}\fi) es
%
\begin{align}\label{e2:neuron-general}
  s_i \tab = f(v_i), \tabs v_i \tab = \sum_{j=0}^N w_j x_j,
\end{align}
%
en donde $v_i$ se denomina el \e{campo local inducido} de la neurona
$i$, y $f$ es una función de activación no lineal que acota el rango
de la salida $s_i$ a un intervalo conocido. Algunas funciones de
activación típicas son la función arcotangente o alguna variante de la
función sigmoidea, tal como
%
\begin{align}\label{e2:sigmoid-symmetric}
  f(v_i) = \frac{2}{1+e^{-v_j}}-1,
\end{align}
%
la cual posee un rango de salida simétrico en $(-1,1)$. Estas
funciones monótonas crecientes y continuamente derivables presentan un
equilibrio entre un comportamiento lineal cerca del origen y no lineal
para magnitudes mayores de $v_i$, con un comportamiento asintótico a
la función escalón. El umbral $b_i$ de la neurona, equivalente al
valor $U$ en la
\iflatexml{}Ecuación~\ref{e2:neuron-basic}\else\autoref{e2:neuron-basic}\fi{}
se representa en este modelo como el peso de una entrada constante
(ficticia) $s_0=1$, de modo que
%
\begin{align}
  b_i\tab=-w_{i0}.
\end{align}
%
