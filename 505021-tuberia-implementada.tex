La tubería implementada es similar a la tubería teórica, aunque con
menos tareas de testing. Al igual que aquella, un push del código al
servidor GitLab dispara la ejecución de la tubería. Una tarea fallida
aborta la ejecución de la tubería completa.
%
\subsubsection{Etapa de construcción del código}
%
Entorno de ejecución: contenedor Docker del servicio CI/CD.
%
\begin{enumerate}
\item \sbs{Verificar que la versión no ha sido publicada}.
\item \sbs{Compilar la aplicación}.
\item \sbs{Ejecutar tests unitarios}.
\item \sbs{Ejecutar tests de integración}.
\item \sbs{Empaquetar la aplicación}. Se genera el artefacto para
  el despliegue y se conserva el mismo en una ubicación temporal.
\end{enumerate}
%
\subsubsection{Etapa de verificación}
%
Entorno de ejecución: contenedor Docker del servicio CI/CD, con
servidor de aplicación integrado y acceso a la base de datos.
%
\begin{enumerate}
\item \sbs{Configurar el entorno de testing local}.
\item \sbs{Desplegar la aplicación}.
\item \sbs{Tests de integración de sistemas}.
\end{enumerate}
%
\subsubsection{Publicación}
%
En el caso que el disparador de la tubería haya sido un cambio sobre
la rama principal (\e{master}), se publica el paquete de la
aplicación al repositorio de artefactos y se genera la etiqueta de
versión correspondiente en el repositorio Git.
%
\subsubsection{Despliegue en el entorno para tests de integración de sistemas}

Esta acción requiere aprobación manual, típicamente de un
desarrollador. Permite actualizar la aplicación en el entorno de
\e{integración} para que otros sistemas puedan validar sus
integraciones.
%
\begin{enumerate}
\item \sbs{Configurar entorno}.
\item \sbs{Despliegue de aplicación}.
\end{enumerate}
%
\subsubsection{Despliegue en el entorno para tests de aceptación de usuario}
%
Esta acción requiere aprobación manual. Es idéntica a la de la tubería
teórica, pero excluye las pruebas de humo.
%
\begin{enumerate}
\item \sbs{Configurar entorno}.
\item \sbs{Despliegue de aplicación}.
\end{enumerate}
%
\subsubsection{Despliegue en el entorno de producción}
%
También se trata de una acción de aprobación manual. Aplican las
mismas consideraciones que en la tubería teórica, siendo la única
diferencia que no se ejecutan pruebas de humo.
%
\begin{enumerate}
\item \sbs{Configurar entorno}.
\item \sbs{Despliegue de aplicación}.
\end{enumerate}
%
%
\section{Dificultades encontradas en la implementación}
%
La implementación de las tuberías en los servicios resultó una tarea
de una complejidad mucho mayor a la prevista en el plan de
trabajo. Estas dificultades impactaron retrasando el desarrollo del
Proyecto. A continuación se presenta una revisión de los problemas
encontrados a la hora de implementar tuberías de integración y entrega
continuas en los servicios.
%
\subsection{Generación de imágenes Docker}
%
Las imágenes de Docker son entornos de ejecución aislados y
reproducibles, generados a partir de una especificación de
configuración como código basada en un lenguaje de dominio
específico. Cuando una imagen de Docker es instanciada para su
ejecución, se crea un contenedor que incluye el contenido de la imagen
asociado a recursos específicos tales como la memoria asignada y los
recursos de red. Las imágenes de Docker pueden construirse en base a
otras imágenes ya existentes, lo cual ofrece una gran flexibilidad de
adaptación a los requerimientos de cada aplicación.

Dicho esto, también resulta cierto que los entornos Docker son
limitados por diseño. Por ejemplo, dentro de un contenedor se permite
la ejecución de un único proceso, a contramano de lo que sucede en las
instancias virtualizadas, que ejecutan un sistema operativo completo.

A la hora de configurar los entornos de Docker para la ejecución de
las tareas de CI/CD, se optó por generar entornos que se adaptaran a
los requerimientos del proceso de construcción del código y las
pruebas, en lugar de adaptar los procesos a un entorno limitado como
Docker. Esta decisión significó realizar un gran esfuerzo en la
generación de imágenes de Docker personalizadas.

La creación de cada imagen de Docker implicó la creación de un
repositorio específico para el código de la misma. Estos repositorios
incorporan a su vez sus propias pipelines de integración y entrega
continuas, encargadas de generar las imágenes y publicarlas en el
repositorio público Docker Hub. Las imágenes de Docker creadas fueron
las siguientes:
%
\begin{itemize}
\item \mono{diptunl/debian}: sistema operativo Debian de base con
  herramientas y configuraciones estándares requeridas por los
  servicios de la DIPT. Incluye configuración de locales, hora y
  fecha, y certificados SSL.
\item \mono{diptunl/java}: basada en diptunl/debian, incorpora la máquina
  virtual OpenJDK configurada con opciones de ejecución adaptadas a
  los servicios de la DIPT e incorpora los certificados SSL al
  depósito de claves de Java.
\item \mono{diptunl/gradle}: se basa en diptunl/java y contiene la
  herramienta Gradle.
\item \mono{diptunl/maven}: se basa en diptunl/java y contiene la herramienta
  Maven.
\item \mono{diptunl/wildfly}: basada en diptunl/java, diptunl/gradle y
  diptunl/maven, posee un servidor Wildfly que incluye un driver para
  conectarse a PostgreSQL y está parametrizado para los servicios Java
  de la DIPT.
\item \mono{diptunl/postgres}: basada en diptunl/debian, posee un servidor
  PostgreSQL que soporta el esquema de nombres y permisos utilizados
  en la DIPT.
\end{itemize}
%
Esta arquitectura de imágenes de Docker fue evolucionando conforme se
avanzó con la implementación de las tuberías de cada servicio.
%
\subsection{Falta de documentación}
%
Otro de los grandes problemas encontrados a la hora de la
implementación de las tuberías fue la falta de documentación de los
procesos, configuraciones y comandos necesarios para poder compilar
las aplicaciones y ejecutar los tests. En particular los tests de
integración de sistemas, los cuales fueron diseñados únicamente para
soportar el caso de uso de la ejecución local en la PC del
desarrollador. Estos inconvenientes impactaron en la implementación de
las tuberías generando retrasos debidos a las situaciones de prueba y
error, y de ingeniería inversa a partir de la inspección del código
fuente.

Como un aspecto positivo, debido a estos problemas se incorporaron
modificaciones al código fuente de la aplicación, incluyendo
documentación e incorporando soporte de parametrizaciones para que los
procesos de testing fueran más flexibles.
