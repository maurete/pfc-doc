\begin{abstract}
La propuesta DevOps consiste en aplicar prácticas y principios ágiles
del mundo de la industria manufacturera a las Tecnologías de la
Información (TI). Una de sus propuestas es integrar el trabajo de las
diferentes áreas involucradas en la gestión de servicios de software
para generar una cultura en común. También postula aprovechar las
posibilidades de la automatización para eliminar trabajo innecesario
de las personas. También destaca la importancia de un proceso de
mejora continua, el cual debe estar basado en objetivos medibles y
verificables.

En el presente Proyecto se implementaron prácticas y herramientas
tecnológicas en busca de integrar el trabajo de desarrollo y
operaciones de acuerdo a la propuesta DevOps. El trabajo incluyó la
automatización de tareas de infraestructura, la implementación de
tuberías de integración y entrega continuas y la unificación de los
repositorios de código fuente y configuración. Asimismo se
implementaron herramientas para brindar autoservicio de la
infraestructura, visibilidad de métricas y registros y
comunicación. El ámbito de trabajo fue la Dirección de Informatización
y Planificación Tecnológica de la Universidad Nacional del Litoral
(DIPT-UNL).
\end{abstract}
