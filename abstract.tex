%
\begin{abstract}
%
DevOps es un conjunto de prácticas y principios utilizados para
agilizar la producción del software. Busca generar una cultura en
común dentro de las organizaciones integrando el trabajo de las
diferentes áreas involucradas en la producción y entrega del
software. Propone aprovechar la automatización para eliminar trabajo
innecesario de las personas, y destaca la importancia de establecer un
proceso de mejora continua basado en objetivos medibles y
verificables.
  
En este Proyecto se implementaron prácticas y herramientas
tecnológicas en busca de integrar el trabajo de desarrollo y
operaciones de acuerdo a la propuesta DevOps. El trabajo incluyó la
automatización de tareas de infraestructura, la implementación de
tuberías de integración y entrega continuas y la unificación de los
repositorios de código fuente y configuración. Asimismo, se
implementaron herramientas para brindar autoservicio de la
infraestructura, visibilidad de métricas y registros y
comunicación. El ámbito de trabajo fue la Dirección de Informatización
y Planificación Tecnológica de la Universidad Nacional del Litoral
(DIPT-UNL).
%
\end{abstract}
%
