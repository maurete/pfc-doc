%
\subsubsection{Función de pérdida}
%
Para un ejemplo $(\xx,\yy)\in{}X\times{}Y$, la función de pérdida mide
la cantidad de error del modelo $\hat{\yy}=h(\xx)$ mediante la
transformación $L(\xx,\yy,\hat{\yy})$.
Si la predicción resultante
del modelo $\hat{\yy}$ es correcta, el error es nulo.
En la práctica,
la mayoría de las funciones de pérdida no dependen de la entrada
$\xx$, sino sólo de la predicción $\hat{\yy}$ y la clase $\yy$.
Algunas de las funciones de pérdida más comunes son:
%
\begin{align}
  \T{Pérdida 0-1:} \tab\tabs
    L(\yy,\hat{\yy})=\begin{cases}0,\quad{}\yy=\hat{\yy}\\
      1,\quad\T{en otro caso}\end{cases} \\
  \T{Pérdida ``bisagra'':} \tab\tabs
    L(\yy,\hat{\yy})=\max\{0,\yy-\hat{\yy}\}\\
  \T{Error absoluto:} \tab\tabs
    L(\yy,\hat{\yy})=|\yy-\hat{\yy}|\\
  \T{Error cuadrático:} \tab\tabs
    L(\yy,\hat{\yy})=(\yy-\hat{\yy})^2.
\end{align}
%
