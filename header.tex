%
%
%%%%%%%%%%%%%%%%%%%%%%%%%%%%%%%%%%%%%%%%%%%%%%%%%%%%%%%%%%%%%%%%%%%%%%%%%%%%%%%
% ../pfc/doc/conf/preconfig.tex
%%%%%%%%%%%%%%%%%%%%%%%%%%%%%%%%%%%%%%%%%%%%%%%%%%%%%%%%%%%%%%%%%%%%%%%%%%%%%%%
%
% define  \ifpdf {...} \fi to test for compilation with pdflatex.
\newif\ifpdf
\ifnum\pdfoutput<0
\pdffalse\fi
\ifnum\pdfoutput=0
\pdffalse\fi
\ifnum\pdfoutput>0
\pdftrue\fi
%
%%%%%%%%%%%%%%%%%%%%%%%%%%%%%%%%%%%%%%%%%%%%%%%%%%%%%%%%%%%%%%%%%%%%%%%%%%%%%%%
% ../pfc/doc/conf/fuentes.tex
%%%%%%%%%%%%%%%%%%%%%%%%%%%%%%%%%%%%%%%%%%%%%%%%%%%%%%%%%%%%%%%%%%%%%%%%%%%%%%%
%
\iflatexml\else
\usepackage[T1]{fontenc}

% Linux Libertine
\usepackage{libertine}

% Linux Libertine
\usepackage{libertinust1math}

% TeX Gyre Heros
%\usepackage[scale=0.96]{tgheros}

% Inconsolata
\usepackage[scaled=0.9]{zi4}

% Para evitar clash de euler con koma-script
\renewcommand{\tt}{\mathtt}

% Source Sans Pro
% \usepackage{sourcesanspro}

% Roboto
\usepackage{roboto}

% Bitstream charter en modo math
% \usepackage[bitstream-charter]{mathdesign}

\fi
%
%%%%%%%%%%%%%%%%%%%%%%%%%%%%%%%%%%%%%%%%%%%%%%%%%%%%%%%%%%%%%%%%%%%%%%%%%%%%%%%
% ../pfc/doc/conf/packages.tex
%%%%%%%%%%%%%%%%%%%%%%%%%%%%%%%%%%%%%%%%%%%%%%%%%%%%%%%%%%%%%%%%%%%%%%%%%%%%%%%
% ===
% === I18n / L10n
% ===
%
% babel me da separación de sílabas para palabras en el idioma que le paso como
%       argumento opcional.
\usepackage[spanish,es-tabla,english]{babel}
%
%
% inputenc define la codificación de caracteres del código fuente, acá utf8.
\usepackage[utf8]{inputenc}
%
% ===
% === Layout
% ===
%
% relsize provee comandos para utilizar tamaños de fuente relativos
\usepackage{relsize}
%
% ===
% === Gráficos
% ===
%
% pst-pdf me permite usar PSTricks con pdflatex. Necesito cargarlo sólo si está
%         definida la variable pdf, por eso está entre \ifpdf ... \fi
%\ifpdf\usepackage{pst-pdf}\fi
%
% color me permite usar colores en el documento.
\usepackage[dvipsnames,svgnames,tables]{xcolor}
%
% graphicx me da el comando \includegraphics para insertar imágenes (?)
\usepackage{graphicx}
%
% pstricks es un conjunto de macros basadas en PostScript para TeX, en
%          castellano: me da un entorno pstricks y comandos que uso dentro de
%          éste, que me sirven para dibujar figuras/diagramas/etc de manera
%          relativamente simple.
%\usepackage{pstricks}
%
% pst-circ me da macros para pstricks que me dibujan elementos de circuitos
%\usepackage{pst-circ}
%
% pst-plot me provee de funciones de ploteo para pstricks
%\usepackage{pst-plot}
%
% pst-2dplot me sirve para plotear en pstricks, entorno pstaxes
%\usepackage{pst-2dplot}
%
% url es un verbatim para escribir URL's que permite linebreaks dentro de ésta.
%     para usarlo, \url{<URL>}
\usepackage{url}

%
% ===
% === Matemáticas
% ===
%
\usepackage{amsmath}		%Para enornos matemáticos mas flexibles
\usepackage{bbold}		%Fuente bb para modo math: \mathbb{R} = reales
\iflatexml
\newcommand*{\mathds}[1]{\mathbb{#1}}
\else
\usepackage{dsfont}		%Fuente ds para modo math: \mathds{R} = reales
\fi
%
%
\usepackage{multirow}		%Para "combinar" celdas en tablas
\usepackage{float}		%Para mejorar cuadros, figuras, etc
%
%
%

% better citation handling
\iflatexml
\newcommand*{\citeauthor}[1]{\textcolor{RoyalBlue}{(autor de \cite{#1})}}
\else
%%\usepackage[square,comma,numbers,sort&compress]{natbib}
\usepackage[backend=biber,sorting=none,style=ieee,eprint=false,url=false]{biblatex} %% style=ieee
%% requiere texlive-bibtex-extra en debian
\fi


%% % para eliminar espacio entre items de una lista
\usepackage{enumitem}
%% \setlist{noitemsep}
%% \setlist[description]{noitemsep}
%% \setlist[enumerate]{noitemsep}
%% \setlist[itemize]{noitemsep}

\usepackage{tikz}
\iflatexml
\else
\usetikzlibrary{external}
\tikzexternalize[mode=no graphics] % generar archivos de imagenes con las figs
\fi
%% \usepackage{pgfkeys}
%% \usepackage{pgfgantt}
%% \usetikzlibrary{babel}
%% \usetikzlibrary{arrows}
%% \usetikzlibrary{shapes.multipart}

% typearea: uso con koma-script para ajustar márgenes de página.
% vars globales a setear en la clase koma-script: DIV=12, BCOR=margen de ``binding'' para double side
%% \usepackage{typearea}

% para poder usar footnotes p.ej, adentro de un tabular
%% \usepackage{footnote}
%% \makesavenoteenv{tabular}

% para tabulars mas lindos/legibles
\usepackage{booktabs}
% tablas que se extienden por varias págs.
\usepackage{longtable}

% para alinear numeros en el punto decimal (dentro de un tabular)
\iflatexml
\else
\usepackage[detect-all,alsoload=abbr]{siunitx}
\fi

%\usepackage{glossaries}
%% \usepackage[spanish]{algorithm2e}

%% Verbatims con fomato lindo
\usepackage{fancyvrb}
% interpretar \ como un comando dentro del verbatim
\iflatexml
\else
\fvset{fontsize=\footnotesize,commandchars=\\\{\}}
\fi

% para teoremas etc
\usepackage{amsthm}

%Permitir que los entornos equation, align, etc permitan saltos de página
%\allowdisplaybreaks[1]

%Colores
\definecolor{negro}	{cmyk}{0,0,0,1}
\definecolor{marron}	{cmyk}{0,.5,1,.41}
\definecolor{rojo}	{cmyk}{0,1,1,0}
\definecolor{naranja}	{cmyk}{0,.35,1,0}
\definecolor{amarillo}	{cmyk}{0,0,1,0}
\definecolor{verde}	{cmyk}{1,0,1,0}
\definecolor{azul}	{cmyk}{1,1,0,0}
\definecolor{violeta}	{cmyk}{.45,1,0,0}
\definecolor{gris}	{cmyk}{0,0,0,.5}
\definecolor{blanco}	{cmyk}{0,0,0,0}
\definecolor{dorado}	{cmyk}{0,.16,1,0}
\definecolor{plateado}	{cmyk}{0,0,0,.25}


%%%%%%%%%%%%%%%%%%%%%%%%%%%%%%%%%%%%%%%%%%%%%%%%%%%%%%%%%%%%%%%%%%%%%%%%%%%%%%%
% header.tex
%%%%%%%%%%%%%%%%%%%%%%%%%%%%%%%%%%%%%%%%%%%%%%%%%%%%%%%%%%%%%%%%%%%%%%%%%%%%%%%

\hyphenation{micro-RNA}
\hyphenation{micro-RNAs}
\hyphenation{mi-RNA}
\hyphenation{mi-RNAs}

\newcommand{\e}{\emph}
\newcommand{\fr}[1]{(\iflatexml{Ecuacion \ref{#1}}\else\autoref{#1}\fi)}
\newcommand{\iid}{independiente e idénticamente distribuido}
\newtheorem{definicion}{Definición}[chapter]
\newtheorem{lema}[definicion]{Lema}
\newcommand{\ntA}{\textrm{\mono{A}}}
\newcommand{\ntC}{\textrm{\mono{C}}}
\newcommand{\ntG}{\textrm{\mono{G}}}
\newcommand{\ntU}{\textrm{\mono{U}}}
\newcommand{\pairL}{\textrm{\mono{(}}}
\newcommand{\pairR}{\textrm{\mono{)}}}
\newcommand{\noPair}{\textrm{\mono{.}}}
\newcommand{\dn}[1]{\textrm{\mono{#1}}}
\newcommand{\bp}[2]{\textrm{\mono{#1-#2}}}
\newcommand{\grad}[1]{\nabla{#1}}
\newcommand{\tableStyle}{
\iflatexml\else\ifpdf\relsize{-0.5}\center\sffamily\fi\fi
}
\newcommand{\captionStyle}{
\iflatexml\else\ifpdf\relsize{-0.5}\fi\fi
}
\newcommand{\figureStyle}{\tableStyle}
\newcommand{\tE}{&{\smaller $\bar{x}$}&}
\newcommand{\tF}[1]{\multirow{2}{*}{#1}\tE}
\newcommand{\tZ}{&{\smaller $\sigma$}&}
\newcommand{\tA}{\addlinespace[4pt]}
\newcommand{\rowMEAN}[1]{&\multirow{2}{*}{#1}&{\smaller $\bar{x}$}}
\newcommand{\rowSTD}{&&{\smaller $\sigma$}}
\newcommand{\rowSKIP}{\addlinespace[2pt]}
\newcommand{\ti}{\textsmaller}
\newcommand{\func}[1]{{\texttt{#1}}}
\newcommand{\dataset}[1]{{\texttt{#1}}}
\newcommand{\Ft}[1]{\mrow{2}{*}{#1}}
\newcommand{\tbmean}{\smaller{2} $\bar{x}$}
\newcommand{\tbstd}{\smaller $\sigma$}
\newcommand{\mirbase}{{miRBase}}
\newcommand{\rfam}{{Rfam}}
\newcommand{\tripletsvm}{{Triplet-SVM}}
\newcommand{\mipred}{{miPred}}
\newcommand{\micropred}{{microPred}}
\newcommand{\deltamirbase}{{\ensuremath{\mathbf{\Delta}}miRBase}}
\newcommand{\premirna}[1]{pre-miRNA#1}
\newcommand{\webdemo}{Web-demo builder}
\newcommand{\ncrna}[1]{ncRNA#1}
\newcommand{\caract}[1]{característica#1}
\newcommand{\hparam}[1]{hiperparámetro#1}
\newcommand{\MVS}[1]{máquina#1 de vectores de soporte}
\newcommand{\rowMEANN}[2]{&
  \multirow{2}{*}{\tablenum[table-format=1.0,table-number-alignment=right]{#1}}&
  \multirow{2}{*}{\tablenum[table-format=3.0,table-number-alignment=right]{#2}}&
           {\smaller $\bar{x}$}}
\newcommand{\rowSTDN}{&&&{\smaller $\sigma$}}

\newcommand{\VP}{\T{\slshape VP}}
\newcommand{\VN}{\T{\slshape VN}}
\newcommand{\FP}{\T{\slshape FP}}
\newcommand{\FN}{\T{\slshape FN}}
\newcommand{\SE}{\T{\slshape SE}}
\newcommand{\SP}{\T{\slshape SP}}
\newcommand{\GM}{\ensuremath{\T{\slshape G}_m}}
\newcommand{\Gm}{\GM}
\newcommand{\PR}{\T{\slshape Pr}}
\newcommand{\AC}{\T{\slshape Ac}}
\newcommand{\avg}{\T{$\bar{x}$}}
\newcommand{\std}{\T{$\sigma$}}

\newcounter{FeatureCounter}
\newcommand{\headRow}{  \sfbf{\footnotesize Pos.} &
  \sfbf{\footnotesize Caract.} &
  \sfbf{\footnotesize Descripción}
}

\newcommand{\feedforward}{con propagación hacia adelante}
\newcommand{\funcheader}[3]{\begin{description}%
  \item[Función]{#1}\item[Entradas]{#2}\item[Salidas]{#3}%
\end{description}}
\newcommand{\funcentry}[4]{\begin{description}%
  \item[Función]{#1}
  \item[Entradas]{#2}
  \item[Salidas]{#3}
  \item[Descripción]{#4}
\end{description}}
\renewcommand{\bibfont}{\normalfont\footnotesize}
%
% ===
% === Comandos
% ===
%
% T: para escribir texto común cuando en modo math
%    uso: \T{texto que aparecerá en letra normal}
\newcommand{\T}{\textrm}
%
% aclaracion: dibuja un recuadrito aclaratorio, como <quote> en HTML.
%             uso: \aclaracion{Texto...}
\newcommand{\aclaracion}[1]{%
\smallpencil\-\begin{minipage}{0.9\textwidth}
%\vspace*{6pt}
{#1}\smallskip\end{minipage}}
%
% consigna: parecido a aclaración, pero con texto _slanted_
%           uso: \consigna{Consigna...}
\newcommand{\consigna}[1]{%
\leftpointright\ \parbox[t]{0.9\textwidth}{\textsl{#1}\vspace{8pt}}}
%
% pinterno: para representar el producto interno entre los dos argumentos
%           uso: \pinterno{X}{Y}
\newcommand{\pinterno}[2]{%
\left\langle #1 , #2 \right\rangle}
%
% pint: para representar el producto interno entre los dos argumentos
%           uso: \pinterno{X}{Y}
\newcommand{\pint}[2]{%
\left\langle\!#1,#2\!\right\rangle}
%
% pinT: para representar el producto interno (primero transpuesto)
%           uso: \pinT{X}{Y}
\newcommand{\pinT}[2]{%
#1^{T}#2}
%
% === Estilos de texto
%
% resalt: resaltado con fondo verde
%         uso: \resalt{texto resaltado}
\newcommand{\resalt}{\colorbox{yellow}}
%
% sfbf: texto en negrita + slanted
%       uso:
\newcommand{\sfbf}[1]{\textsf{\bfseries #1}}
%
% small bold sans-serif
\newcommand{\sbs}[1]{\textsf{\relscale{0.9}\bfseries #1}}
%
% eng: itálica (para palabras en inglés)
%      uso: \eng{some English text}
\newcommand{\eng}{\textit}
%
% mean: significado de una sigla - slanted
%       uso: (...) SNCF: \mean{Société Nationale des Chemins de Fer Francais} ...
\newcommand{\mean}{\textsl}
\newcommand{\desc}{\textsl}
%
% defin: pone en negrita el texto, útil para definiciones
%        uso: \defin{asshole}: vulgar slang for anus
\newcommand{\defin}{\textbf}
%
% R, N: cambia la tipografía en modo math, probar para ver cómo quedan
%       uso: \R{R} , \N{N}
\newcommand{\R}{\mathds}
\newcommand{\N}{\mathbf}
\newcommand{\C}{\mathcal}
\newcommand{\B}{\boldsymbol}
%
% dx: para escribir d2y/dx2, etc
\newcommand{\dx}[2]{\frac{d^{#2}\!#1}{d\!x^{#2}}}
%
% dp: para escribir derivadas parciales d2y/dx2, etc
\newcommand{\dpar}[3]{\frac{\partial^{#3}#1}{\partial{#2}^{#3}}}
%
% dvar: para escribir derivadas totales d2y/d(VAR)2, etc
\newcommand{\dvar}[3]{\frac{d^{#3}#1}{d{#2}^{#3}}}
%
% evalen: para escribir (loquesea)|_{evaluado_en}
\newcommand{\evalen}[2]{\left.{#2}\right|_{#1}}
%
% lil: para escribir texto pequeño. más cómodo que { \footnotesize texto pequeño... }
%      uso: \lil{texto pequeño... }
\newcommand{\lil}[1]{\footnotesize #1}  %Para texto pequeñooo
%
% mono: escribe el texto que le paso como parámetro con letra de ancho fijo
%       uso: \mono{texto monoespaciado}
\newcommand{\mono}[1]{{\texttt{#1}}}


%
% === Símbolos
%
\newcommand{\y}{\wedge}			%Y (Lógica)
\newcommand{\ve}{\vee}			%O (Lógica)
\newcommand{\ent}{\supset}		%Entonces (Lógica)
\newcommand{\dimp}{\leftrightarrow}	%Doble implicativo, equivalencia (Lógica)
\newcommand{\sii}{\leftrightarrow}	%Si y sólo si (Lógica)
\newcommand{\equi}{\equiv}		%Equivalencia (Lógica)
\newcommand{\portanto}{\vdash}		%Por lo tanto (Lógica)
\newcommand{\por}{\cdot}		%Producto punto
\newcommand{\RR}[1][1]{\mathds{R}}	%R de reales
\newcommand{\hfi}{\hat{\phi}}           %fi con gorrito arriba
\newcommand{\bfi}{\bar{\phi}}           %fi con raya arriba
\newcommand{\hpsi}{\hat{\psi}}          %Letra griega psi con gorrito arriba
\newcommand{\II}{\B{I}}                 %w negrita
\newcommand{\KK}{\B{K}}                 %w negrita
\newcommand{\QQ}{\B{Q}}                 %w negrita
\newcommand{\YY}{\B{Y}}                 %w negrita
\newcommand{\Bg}{\B{g}}                 %w negrita
\newcommand{\nn}{\B{n}}                 %w negrita
\newcommand{\uu}{\B{u}}                 %w negrita
\newcommand{\vv}{\B{v}}                 %w negrita
\newcommand{\ww}{\B{w}}                 %w negrita
\newcommand{\xx}{\B{x}}                 %x negrita
\newcommand{\yy}{\B{y}}                 %y negrita
\newcommand{\zz}{\B{z}}                 %z negrita
\newcommand{\BPhi}{\B{\Phi}}            %\Phi negrita
\newcommand{\Balpha}{\B{\alpha}}        %\alpha negrita
\newcommand{\Bbeta}{\B{\beta}}          %\beta negrita
\newcommand{\Btheta}{\B{\theta}}        %\theta negrita
\newcommand{\Bxi}{\B{\xi}}              %\xi negrita
%
%
% ===
% === Environments
% ===
%
% enunciado: un environment que básicamente tiene el mismo efecto que el
%            comando consigna.
%            uso: \begin{enunciado} ... contenido ... \end{enunciado}
\newenvironment{enunciado}
{\leftpointright\ \begin{varwidth}[t]{0.9\textwidth}\textsl}
{\end{varwidth}\vspace{8pt}}
%
% pvi: para tipear la definición de un problema de valor inicial/funciones
%      definidas de a trozos/etc directamente en el texto (sin necesidad de
%      cambiar a un modo matemático.
%      uso: \begin{pvi} linea1 \\ linea 2 \\ ... \end{pvi}
\newenvironment{pvi}{\begin{equation}\begin{cases}}
{\end{cases}\end{equation}}
%
% pvi*: comp pvi, pero sin número de ecuación
\newenvironment{pvi*}{\begin{equation*}\begin{cases}}
{\end{cases}\end{equation*}}
%
% verbatimsmall: un verbatim con letra más chica. usualmente queda bastante
%                mejor que el verbatim pelado.
%                uso: \begin{verbatimsmall} ........ \end{verbatimsmall}
\newenvironment{verbatimsmall}{\small\begin{verbatim*}}
{\end{verbatim*}}
%
% nota: escribe una aclaracion dentro del texto
\newenvironment{nota}{$$\left[\;\begin{minipage}{0.95\textwidth}\slshape}
{\end{minipage}\;\right]$$}
%
%

% multicolumn y multirow
\newcommand{\mcol}[3]{\multicolumn{#1}{#2}{#3}}
\newcommand{\mrow}[3]{\multirow{#1}{#2}{#3}}

% http://tex.stackexchange.com/questions/14377/how-can-i-test-for-the-current-font
\makeatletter
\newcommand{\showfont}{encoding: \f@encoding{},
  family: \f@family{},
  series: \f@series{},
  shape: \f@shape{},
  size: \f@size{}
}
\newcommand{\iffont}[3]{\ifthenelse{\equal{\f@family}{#1}}{#2}{#3}}
\newcommand{\ifseri}[3]{\ifthenelse{\equal{\f@series}{#1}}{#2}{#3}}
\makeatother
%
% sbsf: texto small, bold, sans-serif. Si ya es sans, no aplica small.
%       uso:
\newcommand{\sbsf}[1]{\iffont{qhv}{\textbf{#1}}{\textscale{0.9}{\textsf{\textbf{#1}}}}}
%
% ssf: texto small, sans-serif. Si ya es sans, no aplica small.
%       uso:
\newcommand{\ssf}[1]{\iffont{qhv}{#1}{\textscale{0.9}{\textsf{#1}}}}

%%%%%%%%%%%%%%%%%%%%%%%%%%%%%%%%%%%%%%%%%%%%%%%%%%%%%%%%%%%%%%%%%%%%%%%%%%%%%%%
% authorea/latexml workarounds
%%%%%%%%%%%%%%%%%%%%%%%%%%%%%%%%%%%%%%%%%%%%%%%%%%%%%%%%%%%%%%%%%%%%%%%%%%%%%%%
%
%

\iflatexml
\newcommand{\nt}{\textrm{nt}}
\else
\DeclareSIUnit\nt{nt}
\newcommand{\nt}{\si{nt}}
\fi

\iflatexml
\newcommand{\tab}{}
\newcommand{\tabs}{\quad}
\else
\newcommand{\tab}{&}
\newcommand{\tabs}{&}
\fi

% para highlight (comando \hl{})
\iflatexml
\newcommand*{\hl}[1]{\textcolor{BurntOrange}{\bfseries #1}}
\else
\usepackage{soulutf8}
\fi


\iflatexml
\newcommand{\refer}{\textcolor{OliveGreen}{Fig./Ec./etc. }\ref}
\else
\usepackage{hyperref}
\newcommand{\refer}{\autoref}
% lo siguiente hace que los enlaces generados por hyperref apunten a
% /la figura/, en lugar de /la caption/ de la figura
\usepackage[hypcap]{caption}
\hypersetup{
  colorlinks,
  citecolor=black,
  filecolor=black,
  linkcolor=black,
  urlcolor=black
}
\fi
%
% Koma-Script page headings
\iflatexml\else
\usepackage{scrlayer-scrpage}
\pagestyle{scrheadings}
\fi


%%%%%%%%%%%%%%%%%% estilos de ``caracter''

% dataset
\newcommand{\dset}[1]{%
\iflatexml{\bfseries #1}\else\ifpdf\upshape\ssf{#1}\fi\fi
}

% work
\newcommand{\work}[1]{%
\iflatexml\textsl{#1}\else\ifpdf\e{#1}\fi\fi
}

% problem
\newcommand{\prob}[1]{%
\iflatexml\textsl{#1}\else\ifpdf\sbsf{#1}\fi\fi
}

\usepackage{tabu}

\iflatexml
\newcommand{\pdfcomment}[2][]{}
\newcommand{\pdfmargincomment}[2][]{}
\else
\usepackage[author=Mauro Torrez]{pdfcomment}
\DTMsetregional % para que pdfcomment no me rompa las fechas
\fi
