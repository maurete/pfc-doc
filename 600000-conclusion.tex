%
%
%
%
\chapter{Conclusión}
%
En este trabajo se presentó el desarrollo de un clasificador de
secuencias de \premirna{s} utilizando técnicas de Inteligencia
Computacional.
Uno de los desafíos principales en un desarrollo de este tipo
recae sobre la extracción de características de los datos.
Si bien éste no ha sido el enfoque principal del presente trabajo, se
ha podido observar el efecto determinante de un buen conjunto de
características.
El otro aspecto fundamental sobre el que sí se trabajó fue la correcta
selección de hiperparametros del clasificador.
La incorporación de técnicas automáticas para la selección de
\hparam{s} en el clasificador SVM presenta una ventaja notable del
software desarrollado frente a otros métodos preexistentes.
Adicionalmente, esto se traduce en simplicidad de uso para el usuario
final, ahorrándole el tiempo requerido por la estrategia clásica de
búsqueda exhaustiva y reduciendo su procedimiento de ``prueba y
error'' a unas pocas combinaciones de clasificadores y estrategias
automáticas.
%
%
%
\section{Trabajos futuros}
%
Como ideas para trabajos futuros, se proponen
%
\begin{enumerate}
\item
  Modularizar/generalizar el proceso de extracción de \caract{s},
  ofreciendo mayor granularidad en la selección de las mismas y
  permitiendo la incorporar nuevas \caract{s} mediante llamados a
  funciones externas.
\item
  Incorporar algún método automático de selección de \caract{s},
  simplificando la experiencia del usuario en este aspecto.
\item
  Permitir la optimización específica de la sensibilidad (\SE) y la
  especificidad (\SP) por seperado, permitiendo la generación de modelos
  con requerimientos específicos.
  Este aspecto está actualmente soportado, aunque no ha sido validado
  con rigurosidad.
\end{enumerate}
%
