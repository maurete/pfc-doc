%
%
\subsection{Costo computacional}
%
En cada prueba efectuada, se midió en forma básica el ``costo
computacional'' mediante el tiempo y el número de entrenamientos
requeridos para generar el modelo del clasificador.
Si bien estas mediciones ignoran otros factores involucrados en el
costo computacional real, tales como la generación de modelos
auxiliares y la resolución de problemas de optimización, resultan
suficientes para brindar una idea global del comportamiento del
sistema.

Los resultados se resumen en la
\iflatexml{}Tabla~\ref{tbl:cost-main}\else\autoref{tbl:cost-main}\fi{},
discriminados según el clasificador y la estrategia de selección de
\hparam{s} probado, para cada problema de clasificación y conjunto de
\caract{s}.
Para una mayor claridad, se omite la estrategia trivial, que efectúa
en todos los casos un único entrenamiento.

En el caso del clasificador MLP, el número de entrenamientos
efectuados es fijo para las dos estrategias de selección del número de
neuronas ocultas.
El tiempo requerido para la generación del modelo aumenta
proporcionalmente al número de ejemplos en el conjunto de datos, lo
que se condice con el entrenamiento por épocas de la implementación.
Se observa asimismo un aumento del tiempo requerido para el
entrenamiento con el conjunto de \caract{s} \dset{S}.

Al probar el clasificador SVM con núcleo lineal, se observó que el
conjunto de \caract{s} resultó un factor determinante del tiempo
requerido para generar el modelo del clasificador.
Mientras que el uso del conjunto de \caract{s} \dset{E} derivó en los
menores tiempos de entrenamiento, las \caract{s} de secuencia y de
tripletes requirieron los mayores tiempos de entrenamiento.
Este efecto se observó especialmente en la estrategia de selección del
\hparam{} $C$ mediante búsqueda exhaustiva, la cual efectuó 41
entrenamientos en todos los casos.

La generación del modelo de clasificador SVM con núcleo RBF requirió
en general un mayor número de entrenamientos.
Sin embargo, el tiempo de ejecución resultó en muchos casos similar o
incluso inferior a aquel requerido por el clasificador SVM con núcleo
lineal.
En comparación con el clasificador SVM-lineal, el tiempo de ejecución
resultó más uniforme entre los distintos grupos de \caract{s}.
En particular, se destaca la velocidad del método al aplicar la
estrategia de selección de \hparam{s} mediante minimización de la cota
RM, requiriendo un tiempo muy inferior al de las demás estrategias no
triviales.
