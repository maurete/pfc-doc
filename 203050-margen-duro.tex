%
%
\subsection{SVM de margen duro}
%
La máquina de vectores de soporte más simple es aquella llamada ``de
margen duro'', y consiste en un clasificador de margen máximo que
incorpora el ``truco del núcleo''.
Dado un núcleo $k:X\times{}X\rightarrow\RR$, se asume que el conjunto
de datos transformado
$D=\left((\BPhi(\xx_1),y_1)),\ldots,(\BPhi(\xx_\ell),y_\ell)\right)$ es
linealmente separable en el espacio inducido $Z$.
El entrenamiento consiste en resolver el
Problema~\ref{e2:svm-problem-kernel}, que en notación vectorial se
escribe
%
\begin{equation}\label{e2:svm-hard-margin}
  \begin{aligned}
    \max_{\Balpha}\quad\tabs
      \B{1}^T\Balpha-\frac{1}{2}\Balpha^T\QQ\Balpha\\
    \T{sujeto a} \quad\tabs
      \yy^T\Balpha = 0, \\
      \tabs \alpha_i\geq 0,  i\in {1,\ldots,\ell }.
  \end{aligned}
\end{equation}
%
Aquí, la matriz $\QQ$ es semidefinida positiva con elementos
$Q_{ij}=y_iy_jk(\xx_i,\xx_j)$, y $\B{1}$ es un vector columna de
$\ell$ elementos iguales a $1$.
El entrenamiento de este problema se efectúa comúnmente mediante el
algoritmo denominado SMO (\eng{Sequential Minimal Optimization},
Optimización mínima secuencial) \cite{smo} específicamente diseñado
para la SVM, aunque puede ser resuelto mediante cualquier algoritmo
capaz de resolver un problema de optimización cuadrático.

El modelo de la máquina de vectores de soporte de margen
duro viene dado por
%
\begin{align}\label{e2:svm-model-hard0}
    h(\xx) &= \T{signo}\left(\langle\ww^*,\BPhi(\xx)\rangle+b^*\right).
\end{align}
%
Para el cálculo de $h(\xx)$, se utiliza la función núcleo $k$ y se
observa que
%
\begin{align}
  \langle\ww^*,\BPhi(\xx)\rangle \tab=
  \sum_{i=1}^\ell{}y_i\alpha^*_ik(\xx_i,\xx),\\
  b^* \tab= y_j - \sum_{i=1}^\ell{}y_i\alpha^*_ik(\xx_i,\xx_j).
  \label{e2:svm-model-b-kernel}
\end{align}
%
El subíndice $j$ de la
\iflatexml{}Ecuación~\ref{e2:svm-model-b-kernel}\else\autoref{e2:svm-model-b-kernel}\fi{}
es cualquiera que cumpla con la condición $\alpha_j>0$.
Con estos resultados, el modelo $h(\xx)$ se escribe
%
\begin{align}
  h(\xx) &=
  \T{signo}\left(\sum_{i=1}^\ell{}y_i\alpha^*_ik(\xx_i,\xx)+b^*\right).
\label{e2:svm-model-hard}
\end{align}
%
