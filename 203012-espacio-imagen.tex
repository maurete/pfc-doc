%
\subsubsection{Espacio imagen}
%
La primera adición de la SVM al clasificador lineal básico es una
transformación de los vectores de entrada a un \e{espacio imagen}.
Esto se logra aplicando una función $\BPhi:X\rightarrow{}Z$, que
transforma el \e{espacio de entrada} $X$ a otro espacio vectorial
inducido $Z$, llamado \e{espacio imagen}.
La elección de la función $\BPhi$ es tal que los datos transformados
sean separables en el espacio $Z$.
Dada una entrada $\xx\in{}X$, la máquina de vectores de soporte
calcula un vector imagen $\BPhi(\xx)$ y le asigna una clase de
salida $\hat{y}=\pm{}1$ según
%
\begin{align*}
  y = \T{signo}(\ww^T \BPhi(\xx)+b)
\end{align*}
%
Los valores de $\ww$ y $b$ se obtienen del algoritmo de entrenamiento,
que se explicará más adelante.
En este caso, la frontera de decisión se ubica en el espacio imagen
$Z$, y viene dada por el hiperplano con ecuación
$\ww^T\BPhi(\xx)+b=0$.
%
\begin{quote}
  {\bfseries Notación.}\quad{}En la literatura, resulta común
  encontrar que el vector $\BPhi(\xx)$ se denomina \e{vector de
    características} (\e{feature vector}) y el espacio vectorial $Z$
  como \e{espacio de las características} (\e{feature space}).  En
  este trabajo se prefiere denominarlos con los nombres alternativos
  \e{vector imagen} y \e{espacio vectorial imagen} para evitar
  confusión con el proceso previo de extracción de características.
\end{quote}
%
