%
\subsubsection{Método de retención}
\label{retencion}
%
Con frecuencia, en los problemas reales el conjunto de prueba $T$ no
viene dado como tal, y en cambio se genera un conjunto $T^*\subset{}D$
seleccionando ejemplos al azar del conjunto de entrenamiento $D$.  A
fin de que exista una cierta ``independencia''\footnote{Estrictamente,
  la independencia no existe en este caso, ya que la pertenencia de un
  ejemplo al conjunto de entrenamiento $D^*$ implica la no pertenecia
  al conjunto de prueba $T^*$.} de los datos en $T^*$ respecto del
conjunto de entrenamiento, se genera un nuevo conjunto de entrenamiento
$D^*\subset{}D:D^*\cap{}T^*=\emptyset$ que será usado para entrenar la
máquina de aprendizaje. El conjunto $D^*$ se denomina en ocasiones
\e{conjunto de estimación}.
Finalmente, el error de prueba puede ser estimado calculando el error
sobre el conjunto $T^*$. Este método es conocido como el \e{método de
  retención}.
