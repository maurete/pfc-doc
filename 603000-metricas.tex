%
%
\section{Servicio de métricas}
%
La necesidad de implementar un servicio de métricas surge como
respuesta a dos necesidades de la organización. La primera necesidad
es contar con una herramienta que permita visualizar y correlacionar
métricas de los servicios a partir de múltiples fuentes. Esto permite
visualizar, por ejemplo, un evento de despliegue en un servicio y la
cantidad de mensajes de error generados por éste. La otra necesidad se
relaciona con las limitaciones del servicio de monitoreo Zabbix.

En la Dirección se utiliza el software Zabbix como servicio de
monitoreo. El mismo ha demostrado ser una herramienta fiable y de buen
desempeño para monitorear las métricas básicas de los servicios tales
como la carga, el uso de memoria y la disponibilidad. Cuenta con un
módulo de alertas que notifica a los equipos responsables ante
cualquier eventualidad.

Las limitaciones de Zabbix se ponen en evidencia cuando se requiere
trabajar con métricas de más alto nivel. Si bien el mismo permite
utilizarlas, su configuración es muy compleja, impactando en la
mantenibilidad del servicio. Como ejemplo de este tipo de métricas
podemos citar: número de usuarios conectados a un servicio, cantidad
de transacciones registradas, códigos de respuesta HTTP de un servicio
o utilización de la memoria \e{heap} dentro de la máquina virtual
de Java.
%
\subsection{Arquitectura}
%
La solución implementada como servicio de métricas consta de tres
componentes. El primero es una base de datos de series de tiempo
(\eng{time-series database})\footnote{ Una base de datos de series de
  tiempo es un tipo de base de datos especializada en el
  almacenamiento de datos que evolucionan con el tiempo. } encargada
de recibir las mediciones, almacenarlas, y ofrecer una interfaz
programática (API) para su consulta. El segundo componente es una
interfaz de visualización de datos provenientes de múltiples
fuentes. El tercer componente de esta arquitectura es un \e{agente}
que se ejecuta en las instancias y se encarga de recopilar y enviar
las métricas hacia el servidor de base de datos.

Tal como fue implementado, el servicio de métricas se limita a
funcionar como complemento de los servicios de observabilidad
existentes, permitiendo generar tableros de visualización con métricas
de diversos orígenes. La arquitectura resultante tiene ciertas
funcionalidades duplicadas, pero se deja como trabajo a futuro la
racionalización de la misma. En la
\iflatexml{}Figura~\ref{fig:metricas}\else\autoref{fig:metricas}\fi{}
se presenta un diagrama de la arquitectura del servicio de métricas.
