%
%
%
\section{Normalización de los vectores de características}
%
Las componentes del vector de características representan cantidades
diversas que toman valores dentro de rangos numéricos también
diferentes, según su naturaleza.  Al efectuar operaciones algebraicas
con vectores de características, la disparidad numérica de las
componentes implica que algunas de ellas tendrán más ``peso'' que
otras de menor magnitud.

La normalización de los vectores de características consiste en
modificar el rango de cada variable a un intervalo determinado, por
ejemplo $[0,1]$ o $[-1,+1]$. Esta normalización tiende a generar
problemas mejor condicionados, al tiempo que incrementa la velocidad
de convergencia del entrenamiento \cite{nnfaq2}.

Partiendo de un conjunto de datos representativo $I$ con $\ell$
vectores de $F$ características, se forma una matriz
$M^I{}_{(\ell\times{}F)}$ en la que cada fila $i$ corresponde a un
ejemplo y cada columna $j$ representa una variable.  Para cada columna
de $M^I$, se calcula un desplazamiento $d_j(I)$ y un factor de
escalado $s_j(I)$ que permiten transformar el rango de la variable
$x_j$ al intervalo especificado. Por ejemplo, para llevar las
variables al rango $[0,1]$ se tiene
%
\begin{align}
  d_j(I) &= - \min_i m_{ij}, &
  s_j(I) &= \frac{1}{\max_i m_{ij} - \min_i m_{ij}}.
\end{align}
%
Similarmente, para llevar las variables al intervalo simétrico
$[-1,+1]$,
%
\begin{align}
  d_j(I) &= -\frac{1}{2}\left(\max_i m_{ij} + \min_i m_{ij}\right), &
  s_j(I) &= \frac{2}{\max_i m_{ij} - \min_i m_{ij}}.
\end{align}
%
El desplazamiento $\B{d}(I)$ y el factor de escala $\B{s}(I)$ se
aplican a todos los conjuntos de datos de entrada al clasificador,
estén o no incluidos en el conjunto representativo $I$, de modo que se
mantienen las diferencias relativas entre los diferentes conjuntos de
datos. La normalización de cada vector de características
$\xx_i=(m_{i1},m_{i2},\ldots,m_{iF})$ se efectúa haciendo
%
\begin{align}
  \xx_i^{*} = (\xx_i + \B{d}(I)) \C{S}(I),\quad \xx_i\in D,
\end{align}
%
donde $\C{S}$ es la matriz diagonal tal que $\C{S}_{jj}=s_j$.

La normalización aplicada a los conjuntos de datos durante el
entrenamiento tiene un impacto directo sobre el modelo generado, ya
que el mismo aprende a clasificar vectores de características
\e{normalizados}.  Por ello, la información de normalización se guarda
como parte del mismo modelo y es aplicada automáticamente a cada nuevo
ejemplo a clasificar.
