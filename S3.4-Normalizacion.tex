%
%
%
\section{Normalización de los vectores de características}
%
Dado que las componentes dentro de un vector de características
representan cantidades de naturaleza diversa, el rango numérico de
cada una de ellas también se da en forma dispar.  Al tratar
algebraicamente los vectores de características, esta disparidad
numérica implica que algunas características (variables) tendrán más
ponderación que otras de menor magnitud.

La normalización de los vectores de características consiste en
modificar el rango de cada variable a un intervalo conocido, como por
ejemplo $[0,1]$ o $[-1,+1]$. Esta normalización tiende a generar
problemas mejor condicionados, al tiempo que incrementa la velocidad
de convergencia del entrenamiento y reduce la posibilidad de recaer
sobre un mínimo local \cite{nnfaq2}.

El primer paso para aplicar normalización consiste en calcular valores
de desplazamiento y escalado para cada variable (característica), a
partir de un conjunto de datos ``testigo'', comúnmente el primer
archivo provisto como entrada al método.

Para el conjunto testigo $I$ con $\ell$ elementos, se forma una matriz
$M^I{}_{(\ell\times{}F)}$ en la que cada fila $i$ representa un
ejemplo y cada columna $j$ representa una variable. Para cada columna
en $M^I$, se calcula un valor de desplazamiento $d_j(I)$ y un valor de
escala $s_j(I)$ que permiten transformar el rango de la variable $x_j$
al intervalo deseado, a partir del rango observado para la variable en
el conjunto $I$.

Como ejemplo, para llevar las variables al rango $[0,1]$ se tiene
%
\begin{align}
  d_j(I) &= - \min_i m_{ij}, & s_j(I) &= \frac{1}{\max_i m_{ij} - \min_i m_{ij}}.
\end{align}
%
Similarmente, para llevar las variables de la matriz testigo $I$
al intervalo simétrico respecto al origen $[-1,+1]$,
%
\begin{align}
  d_j(I) &= - \frac{1}{2} \left(\max_i m_{ij} + \min_i m_{ij}\right), &
  s_j(I) &= \frac{2}{\max_i m_{ij} - \min_i m_{ij}}.
\end{align}
%
Una vez calculados $\B{d}(I)$ y $\B{s}(I)$, se aplican los mismos
factores de escalado y desplazamiento sobre \e{todos} los conjuntos de
datos, de modo de conservar las variaciones relativas entre ellos.
Para cada conjunto $D$ (incluyendo $I$), la normalización se aplica
sobre cada vector de características $\xx_i = (m_{i1}, m_{i2}, \ldots,
m_{iF})$ según
%
\begin{align}
  \xx_i^{*} = (\xx_i + \B{d}(I)) \C{D}(\B{s}(I)),\quad \xx_i\in D,
\end{align}
%
donde $\C{D}(\B{s})$ es la matriz diagonal tal que $\C{D}_{jj}=s_j$.

Al efectuar entrenamiento, la información de normalización se guarda
en el modelo del clasificador, ya que será necesario aplicarla a cada
nuevo ejemplo de prueba no visto previamente por el clasificador.
