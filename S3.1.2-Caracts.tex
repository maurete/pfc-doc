%
%
\subsection{Extracción de características}
%
La extracción de características es el proceso de construir una
representación de los datos mediante un conjunto fijo de
variables numéricas y/o categóricas que describen cada
observación (ejemplo).  Una ``característica'' es un atributo o medida
que se toma sobre el dato original y se trata como variable aleatoria.
En otras palabras, la extracción de \caract{s} transforma la
representación original de cadenas de caracteres a un conjunto fijo de
variables numéricas.  Mediante esta transformación se pretende
alcanzar una mejor representación de los datos desde el punto de vista
de la máquina de aprendizaje, que sea al mismo tiempo más informativa
(menos redundante) y numéricamente tratable.

La construcción del conjunto de características conforma toda una
disciplina en sí misma, y excede el alcance del presente trabajo.  En
el presente desarrollo se adopta un enfoque simple, que consiste en
construir un vector de 66 \caract{s} tomando como referencia los
trabajos \cite{xue,ng,batuwita}.

Las 66 medidas que componen el vector de \caract{s} se distinguen,
según el tipo de dato que representan, en 36 \caract{s} de tripletes,
23 \caract{s} de la secuencia, y 7 de la estructura secundaria.  A
continuación, se describe la composición del vector de \caract{s}
según este orden.
