%
%
\subsection{Extracción de características}
%
La extracción de características es el proceso de construir una
representación de los datos mediante un conjunto fijo de
características numéricas y/o categóricas que describen cada
observación (ejemplo).  Una ``característica'' es un atributo o medida
que se toma sobre el dato original y se trata como variable aleatoria.

En otras palabras, la extracción de \caract{s} transforma la
representación original (cadenas de caracteres) a un conjunto fijo de
variables numéricas.  Mediante esta transformación se pretende
alcanzar una mejor representación de los datos desde el punto de vista
de la máquina de aprendizaje, que sea al mismo tiempo más informativa
(menos redundante) y numéricamente tratable.

La construcción del conjunto de características conforma toda una
disciplina en sí misma, y excede el alcance del presente trabajo.  En
el software \hl{propio} se adopta un enfoque simple, generando un
conjunto de características agregado a partir de los trabajos tomados
como referencia \cite{xue,ng,batuwita}.  Según el tipo de medida que
representan, se distinguen grupos de \caract{s} de tripletes (x36), de
secuencia (x23) y de estructura secundaria (x7).
