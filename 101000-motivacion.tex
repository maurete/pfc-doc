\section{Motivación}

El presente proyecto se origina dentro del área de infraestructura de
la Dirección de Informatización y Planificación Tecnológica (DIPT) de
la Universidad Nacional del Litoral (UNL). La Dirección se encarga de
brindar servicios de software a la comunidad universitaria. El área de
infraestructura está conformada por un equipo de cinco administradores
de sistemas que, entre otras tareas, se encargan de la operación de
los servicios de software ofrecidos por la Dirección.

Desde hace un tiempo, el equipo de infraestructura ha emprendido la
tarea de automatizar su trabajo escribiendo piezas de código que le
permitieran reducir el tiempo invertido en tareas repetitivas. En
forma simultánea, a partir de las demandas de los equipos de
desarrollo se implementaron formas de lograr entornos reproducibles
donde probar el software. Otro hito relevante para este proyecto puede
identificarse en el momento que el equipo de infraestructura codificó
scripts para permitir a los desarrolladores aplicar modificaciones en
las bases de datos y actualizar las versiones de los servicios. De
este modo, se logró ofrecer a los desarrolladores el “autoservicio” de
dos de las operaciones más solicitadas, sin requerir intervención de
los administradores.

Estas acciones pusieron en evidencia los beneficios de fomentar la
colaboración entre los equipos de desarrollo e infraestructura, ya que
las mismas derivaron en menos frustraciones a la espera de la
resolución de tickets, menos trabajo tedioso y canales de comunicación
productivos entre ambos equipos. El trabajo realizado en este proyecto
es la continuación lógica de estas acciones, y busca dar formalidad a
la colaboración y la búsqueda de una cultura en común entre los
diversos equipos de trabajo que forman parte de la Dirección.

La principal motivación para el desarrollo de este proyecto fue la
necesidad de aumentar la productividad de los equipos, particularmente
en las áreas de desarrollo y de infraestructura. Se necesitaba reducir
el nivel de ocupación de los individuos, ya que se había alcanzado un
punto de saturación, lo que limitaba la capacidad de respuesta a los
requerimientos y la implementación de mejoras.

Desde la gerencia se determinaron además una serie de necesidades
concretas a basadas en la política de gestión de la Dirección:

\begin{itemize}
\item Contar con un servicio de integración y entrega continuas.
\item Generar una cultura en común entre los equipos, con especial
  énfasis en la gestión del código fuente, los artefactos y las
  operaciones.
\item Disponer de herramientas que agilicen la comunicación interna y
  canalicen las alertas.
\end{itemize}
