%
%
%
\section{Clasificación}
%
A partir del proceso de entrenamiento se obtiene un modelo entrenado
$\hat{y} = h(\xx)$ que, para un patrón $\xx$, devuelve una predicción
de pertenencia de clase positiva o negativa $\hat{y}\in\{-1,+1\}$.

Dado el modelo entrenado $\hat{y} = h(\xx)$ y el conjunto de prueba
%
\begin{align*}
  P = \left( (\xx_1,y_1),\ldots,(\xx_\ell,y_\ell) \right),
\end{align*}
%
la clasificación consiste simplemente en aplicar el modelo a todos los
elementos de $P$, obteniendo una predicción de clase $\hat{y}_y$
para cada ejemplo $\xx_i$:
%
\begin{align*}
  \hat{y}_i = h(\xx_i).
\end{align*}
%
Se debe notar que las etiquetas $y_i,\,i=1,\ldots,\ell$ presentes en
el conjunto $P$ son ignoradas.

Para el caso del clasificador MLP, el modelo $h$ es en realidad un
arreglo $\B{h}$ con $n$ modelos $(h_1,\ldots,h_n)$, y la pertenencia
de clase para un ejemplo $\xx$ se establece según
%
\begin{align}
  y = \T{moda}( h_1(\xx), \ldots, h_n(\xx)).
\end{align}
%
