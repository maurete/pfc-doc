%
%
%
\section{Clasificación}
%
El proceso de clasificación recibe como entrada un modelo de
clasificador $h$, obtenido mediante alguna de las funciones de
entrenamiento, así como un conjunto de datos de prueba $T$ generado
mediante una invocación al módulo de preprocesamiento.

La clasificación consiste en invocar la función definida en el modelo,
pasándole como argumentos el mismo modelo del clasificador y el
conjunto de datos de prueba recibidos como argumento.
Cuando se trata de un modelo MLP, se invoca la función de
clasificación del \work{Neural Network Toolbox} de Matlab; cuando se
trata de un modelo SVM, se invoca la función correspondiente a la
herramienta utilizada para generar el modelo.

A la salida se retorna un vector $\hat{\yy}=(\hat{y}_1,\hat{y}_2,\ldots)^T$,
que contiene las predicciones de clase para cada ejemplo
correspondiente $\xx_i$ del conjunto de prueba $T$:
%
\begin{align*}
  \hat{y}_i = h(\xx_i).
\end{align*}
%
