%
\begin{table}[h]
  \tableStyle
  \smaller
  %% \begin{tabular}{lccrScScScS}
  %%   \toprule
  %%       {Problema} & {Clase} & {N. elem.} &&
  %%                       {MLP-B} && {SVM-LE} && {SVM-RE} && {SVM-RR} \\
  %%   \midrule
  %%   \mrow{2}{*}{IE-NC}
  %%   \rowMEANN{-1}{12387}&  53.9 &&     47.5 &&     64.8 &&     88.1 \\
  %%   \rowSTDN            &   2.8 &&      2.5 &&      8.2 &&      3.4 \\
  %%   \mrow{2}{*}{IE-NH}
  %%   \rowMEANN{+1}{1918} &  92.8 &&     94.8 &&     91.3 &&     75.3 \\
  %%   \rowSTDN            &   0.8 &&      0.9 &&      1.9 &&      1.4 \\
  %%   \bottomrule
  %%   \\
  %% \end{tabular}
  \iflatexml%
  \begin{tabular}{lrrrrrrr}
  \else%
  \sisetup{
    table-format = 2.1(2),
    table-number-alignment = right,
    separate-uncertainty=true,
  }
  \begin{tabular}{lS[table-format=2.0]
      S[table-format=4.0]SSSSS[table-format=2.1]}
  \fi%
    \toprule
    {Problema} & {Clase} & {Elems.} &
    {MLP-B}    & {SVM-LE}   & {SVM-RE}   & {SVM-RR} & \cite{ng} \\
    \midrule
    IE-NC & -1 & 12387 & 53.9(28) & 47.5(25) & 64.8(82) & 88.1(34) & 68.7 \\
    IE-NH & +1 &  1918 & 92.8(08) & 94.8(09) & 91.3(19) & 75.3(14) & 92.1 \\
    \bottomrule
    \\
  \end{tabular}
  \caption{\captionStyle Resultados de clasificación de los
    conjuntos de prueba complementarios del método \work{\mipred}.
    En la columna de la derecha se muestra la tasa obtenida por los
    autores \cite{ng}.
  }
  \label{tbl:suppl-ng}
\end{table}
%
