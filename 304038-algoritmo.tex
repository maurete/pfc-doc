%
\subsubsection{Algoritmo de optimización}
%
El algoritmo comienza la búsqueda estableciendo en $1$ el valor de
todos los \hparam{s}: $\Btheta^0=(1,1,\ldots)$.
En cada iteración $t$, se efectúan los siguientes pasos:
%
\begin{enumerate}
\item
  Aplicando validación cruzada, se entrena un modelo SVM para cada
  conjunto de estimación.
\item
  Se calculan las salidas $f_k$ (\ref{fk}) de los modelos para todos
  los ejemplos $\xx_k$ en las respectivas particiones de validación.
\item
  Se determinan los valores óptimos de los parámetros $A$ y $B$,
  minimizando el problema (\ref{abproblem}), partiendo de los valores
  iniciales $A_0=1$, $B_0=0$.
\item
  Se calculan las probabilidades estimadas $\hat{p}_k$ y con ellas, el
  error empírico de cada ejemplo $E_k$ y global $E$.
\item
  Se calculan las derivadas $\dpar{E}{\theta_j}{}$.
\end{enumerate}
%
La búsqueda procede en cada punto $\Btheta^t$ evaluando la función
objetivo $E$ y determinando un nuevo punto $\Btheta^{t+1}$ a partir de
la información disponible en el gradiente $\nabla{}E^t$, mediante el
algoritmo de optimización BFGS \cite{nocedal}.
