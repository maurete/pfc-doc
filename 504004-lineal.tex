%

Las siguientes observaciones derivan de los resultados de la
\iflatexml{}Tabla~\ref{tbl:linear-results}\else\autoref{tbl:linear-results}\fi{}:
%
\begin{itemize}
\item
  \e{Estrategias de selección de \hparam{s}}.
  Las tres estrategias probadas arrojaron resultados similares en
  prácticamente todos los casos.
\item
  \e{Conjuntos de \caract{s}}.
  Tal como se observó con el clasificador MLP, los conjuntos que
  incluyen las \caract{s} de la estructura secundaria (\dset{E} y
  \dset{S-E}) obtuvieron las mejores tasas de clasificación.
\item
  \e{Variabilidad}.
  Para el problema \prob{\tripletsvm} con la estrategia de selección
  de \hparam{s} trivial se obtuvo una desviación estándar nula: esto
  se debe a que la determinación del \hparam{C} no involucra ningún
  factor aleatorio, como es el particionado de validación cruzada.
  Por ello, también se concluye que la variabilidad observada con la
  estrategia de selección trivial en los problemas \prob\mipred{} y
  \prob\micropred{} es atribuible únicamente a la composición
  diferente de las particiones de entrenamiento y prueba según la
  semilla aleatoria utilizada.
\item
  \e{Problema \tripletsvm{}}.
  Con el conjunto de \caract{s} de tripletes (\dset{T}), tal como el
  utilizado en \cite{xue}, se obtuvieron resultados similares a los
  reportados por los autores, con una menor especificidad (\SP{}).
  Con el uso de los conjuntos de \caract{s} \dset{E} y \dset{S-E},
  sin embargo, las tasas \SE{} y \SP{} mejoraron significativamente.
\item
  \e{Problema \mipred{}}.
  Al comparar con los resultados reportados por los autores \cite{ng}
  de una $\SE=84.5\%$ y una $\SP=98.0\%$, los resultados obtenidos con
  este problema presentan una mayor sensibilidad, que ronda el $90\%$
  con los conjuntos de \caract{s} \dset{E} y \dset{S-E}, y una menor
  especificidad ($\approx{94\%}$).
\item
  \e{Problema \micropred{}}.
  Tal como en el clasificador MLP, se observa que este problema
  resultó el más difícil de clasificar correctamente con el
  clasificador SVM-lineal.
\item
  \e{Desbalance de clases}.
  Se observó el efecto del ``desbalance de clases'' en los datos de
  entrenamiento de los problemas \prob\mipred{} y \prob\micropred{},
  ya que en ambos casos se obtuvo una especificidad (\SP{})
  sensiblemente superior a la especificidad (\SE{}).
\item
  \e{Comparación con el clasificador MLP}.
  Mientras que los resultados obtenidos para el problema
  \prob\tripletsvm{} resultaron comparables a los del clasificador
  MLP, en los problemas \prob\mipred{} y \prob\micropred{} se observó
  una pequeña mejora.
  Asimismo, la diferencia entre la especificidad (\SP) y la
  sensibilidad (\SE) observada en estos problemas \prob\mipred{} fue
  menor a la obtenida con el clasificador MLP.
\end{itemize}
%
