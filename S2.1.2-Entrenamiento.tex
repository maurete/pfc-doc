%
%
\subsection{Entrenamiento}
%
El entrenamiento es el proceso de construir un modelo $h(\xx)$ a
partir de los datos del conjunto de entrenamiento
$D=\left((\xx_1,\yy_1),\ldots,(\xx_\ell,\yy_\ell)\right)$.
Este conjunto es una muestra (de probabilidad desconocida) de la
distribución generadora de datos $\nu$, y define una distribución
\e{empírica}
%
\begin{align}
  \hat{\nu}=\frac{1}{\ell}\sum_{i=1}^{\ell}\delta_{(\xx_i,\yy_i)}(\xx,\yy)
\end{align}
%
donde $\delta_{(\xx,\yy)}$ denota el delta de Dirac en
$(\xx,\yy)\in{}X\times{}Y$.
%
\subsubsection{Error de entrenamiento}
%
La distribución empírica $\hat{\nu}$ conduce a la definición del
\e{error de entrenamiento} como el error del modelo $h$ sobre los
datos de entrenamiento en $D$:
%
\begin{align}
  \hat{\C{R}}_D(h)=E_\T{entrenamiento}=\R{E}_{\hat{\nu}}[L(\xx,\yy,h(\xx))]
  =\frac{1}{\ell}\sum_{i=1}^\ell L(\xx_i, \yy_i, h(\xx_i)),
\end{align}
%
en donde $L$ es una función de pérdida.
%

%
Una forma de entrenamiento básica de la máquina de aprendizaje consiste en
seleccionar el modelo con el mínimo error de entrenamiento
%
\begin{align}
  \hat{h}:=\arg \min \left\{ \hat{\C{R}}_D(h)\middle|h \in H\right\}.
\end{align}
%
Este procedimiento se conoce también como la minimización del riesgo
empírico sobre el conjunto de entrenamiento.
