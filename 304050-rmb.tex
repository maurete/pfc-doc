%
%
\subsection{Minimización de la cota radio-margen ${\rho}$}
%
Esta estrategia es aplicable a \MVS{s} con núcleo gaussiano y consiste
en minimizar una función $\rho(C,\gamma)$, llamada ``cota
radio-margen'', la cual posee un mínimo global cerca del punto
$(C^*,\gamma^*)$ que minimiza el error de validación cruzada
dejando-uno-fuera.
La función $\rho$ \cite{chung} es una adaptación para las \MVS{s} con
regularización $L1$ de la cota $RM$, propuesta en \cite{vapnik} para
las \MVS{s} con regularización $L2$:
%
\begin{align}
  E_{\T{LOO}} \leq {RM}.
\end{align}
%
Aquí, $R$ es el radio de la hiperesfera que contiene a los vectores
de soporte, $M$ es la amplitud del margen del modelo, y $E_{\T{LOO}}$
es el error de validación cruzada dejando-uno-fuera.
%% % vapnik sec. 10.7, pag 441
%% \begin{align}
%%   \T{RM} = 4R^2 \|\ww\|^2.
%% \end{align}
%% %
Por esta propiedad, en las SVMs con regularización $L2$, al minimizar
el producto ${RM}$ se minimiza también el error $E_{\T{LOO}}$ del
modelo.
%% De hecho, en \cite{chapelle} se presenta un método para selección
%% automática de hiperparámetros basado en esta función.

En \cite{chung} se proponen modificaciones a $RM$ de modo que sea
aplicable a las SVMs con regularización $L1$, lo que deriva en la
función $\rho$.
Si bien $\rho$ no representa estrictamente una cota, mantiene la
propiedad importante de poseer un mínimo global cerca de punto que
minimiza el error $E_{\T{LOO}}$.
A continuación se detalla el cálculo de la función objetivo $\rho$ y
sus derivadas, y se describe el proceso de minimización que da lugar a
esta estrategia.
