%
%
\subsection{Núcleos}
%
Un núcleo, denominado también por su nombre en inglés \eng{kernel}, es
una función $k:X\times{}X\rightarrow{}\RR$ \cite{stewart} que
generaliza el concepto de métrica.  Dado un par de vectores
$\xx_1,\xx_2\in{}X$, un núcleo permite comparar la similaridad entre
los vectores $\BPhi(\xx_1),\BPhi(\xx_2)$ \e{transformados} a un
espacio vectorial inducido $Z$, sin necesidad de calcular la
transformación $\BPhi$ de forma explícita.
%
\begin{definicion}[Núcleo]
  Una función $k:X\times{}X\rightarrow{}\R{R}$ se dice un núcleo
  en el conjunto $X$ si cumple con las siguientes propiedades:
  %
  \begin{enumerate}
  \item $k(\xx_1,\xx_2)=k(\xx_2,\xx_1)$ para todo $\xx_1,\xx_2\in{}X$
  \item Para cada $n\in\R{N}$ y para todos los puntos
    $(\xx_1,\ldots,\xx_\ell)\in{}X^\ell$, la matriz de Gram
    $K\in\R{R}^{\ell\times\ell}$ definida como
    $K_{ij}=k(\xx_i,\xx_j),\,i,j\in\{1,\ldots,\ell\}$ es semidefinida
    positiva.
  \end{enumerate}
  %
\end{definicion}
%
Para un núcleo $k$, el teorema de Mercer \cite{mercer} asegura la
existencia de un espacio de Hilbert $Z$ y una transformación
$\BPhi:X\rightarrow{}Z$ tal que, para cualquier par de vectores
$\xx_i,\xx_j\in{}X$, el núcleo calcula el producto interno de los
mismos en el espacio imagen:
$k(\xx_i,\xx_j)=\langle\BPhi(\xx_i),\BPhi(\xx_j)\rangle$.
%% Es importante tener en cuenta que ni el espacio $F$ ni la
%% transformación $\BPhi$ son únicos.

El teorema de Mercer puede aplicarse directamente a la función
objetivo del problema dual de la SVM (\ref{e2:svm-problem-dual2}),
reemplazando los productos internos
$\left\langle\BPhi(\xx_i),\BPhi(\xx_j)\right\rangle$, por la función
núcleo $k(\xx_i,\xx_j)$. Esto se llama comúnmente como el ``truco del
núcleo'', del inglés \e{kernel trick}, y da como resultado el problema
%
\begin{align}
  \begin{split}
    \max_\alpha &\quad
    -\frac{1}{2}\sum_{i=1}^\ell\sum_{j=1}^\ell y_iy_j\alpha_i\alpha_j
    k(\xx_i,\xx_j) + \sum_{i=1}^{\ell} \alpha_i \\
    \T{sujeto a} &\quad \yy^T\Balpha = 0, \\
    &\quad \alpha_i \geq 0\T{ para todo } i\in\{1,\ldots,\ell\}.
  \end{split}
  \label{e2:svm-problem-kernel}
\end{align}
%
En este problema, las operaciones de producto interno en el espacio
imagen $Z$ se efectúan a través de la función núcleo $k$, sin
necesidad de calcular la transformación $\BPhi$ en forma explícita.

Desde un punto de vista puramente matemático, la utilización del truco
del núcleo no aporta ninguna ventaja, ya que los algoritmos son
equivalentes. Desde un punto de vista computacional, sin embargo, la
diferencia es decisiva: utilizando el truco del núcleo, se pueden
calcular productos internos en espacios vectoriales de alta
dimensionalidad --incluso infinita--, sin necesidad de efectuar la
transformación $\BPhi$, cuyo cálculo directo muchas veces resulta
prohibitivo o incluso imposible en un ordenador.

%% Es intuitivamente claro que la elección de un núcleo que represente
%% una métrica específica al problema puede mejorar el rendimiento de las
%% máquinas de aprendizaje. Por ejemplo, en una tarea de clasificación es
%% conveniente elegir una métrica que agrupa las diferentes clases.

A continuación, se presentan algunos núcleos comúnmente
utilizados con las máquinas de vectores de soporte.
%
\begin{description}
%
\item[Núcleo lineal]
  %
  \begin{align}
    k(\xx_1,\xx_2)=\langle \xx_1, \xx_2\rangle=\xx_1^T\xx_2
  \end{align}
  %
  Calcula simplemente el producto interno entre sus argumentos. Aquí,
  el espacio imagen $Z$ es el mismo que el espacio de entrada, $X$,
  correspondiente a la transformación identidad $\BPhi(\xx)=\xx$.
%
\item[Núcleo polinómico]
  %
  \begin{align}
    k(\xx_1,\xx_2)=\left(\langle{}\xx_1,\xx_2\rangle+\theta\right)^d,
  \end{align}
  %
  con grado $d\in\R{N}$ y desvío $\theta\in\RR$.  Este núcleo es una
  generalización del núcleo lineal (el caso $d=1$ y $\theta=0$). Su
  espacio imagen es el espacio de los polinomios de grado $d$ sobre el
  espacio $X$.  El cálculo del producto interno aplicando el núcleo
  requiere $\C{O}(\dim{}X)$ operaciones, contrastando con las
  $\C{O}\left((\dim{}X)^d\right)$ necesarias para el cálculo de la
  transformación $\BPhi$.
%
\item[Función de base radial (RBF)]
  %
  \begin{align}
    k(\xx_1,\xx_2)=\exp\left(-\frac{\|\xx_1-\xx_2\|^2}{2\sigma^2}\right)
    =\exp\left(-\gamma\|\xx_1-\xx_2\|^2\right)
  \end{align}
  %
  También llamado núcleo gaussiano, o \eng{Radial Fasis Function} en
  inglés, es el núcleo más utilizado.  En la primer forma, el
  hiperparámetro $\sigma$ se llama parámetro de amplitud. En la
  segunda forma el hiperparámetro $\gamma$ se llama de concentración.
  La transformación $\BPhi$ correspondiente a este núcleo genera un
  espacio imagen de dimensión infinita, imposible de calcular
  directamente. El producto interno mediante este núcleo se puede
  calcular en $\C{O}(\dim{}X)$ operaciones.
  %
\end{description}
%
