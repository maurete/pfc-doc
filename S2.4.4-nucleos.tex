%
%
\subsection{Núcleos}
%
\iflatexml

\vspace{2em}

\fi
%
Un núcleo, denominado también por su nombre en inglés \eng{kernel}, es
una función $k:X\times{}X\rightarrow{}\RR$ \cite{stewart} que
generaliza el concepto de métrica. En términos intuitivos, el núcleo
permite comparar la similaridad entre dos elementos
$\xx_1,\xx_2\in{}X$ en un espacio vectorial inducido $F$, denominado
\e{espacio imagen}, que usualmente tiene una dimensionalidad elevada.

\begin{definicion}
[Núcleo]
  Una función $k:X\times{}X\rightarrow{}\R{R}$ se dice un núcleo (de
  Mercer) en el conjunto $X$ si cumple con las siguientes propiedades:
  \begin{enumerate}
  \item $k(\xx_1,\xx_2)=k(\xx_2,\xx_1)$ para todo $\xx_1,\xx_2\in{}X$
  \item Para cada $n\in\R{N}$ y para todos los puntos
    $(x_1,\ldots,x_n)\in{}X^n$ la matriz de Gram
    $K\in\R{R}^{n\times{}n}$ definida como
    $K_{ij}=k(\xx_i,\xx_j),\,i,j\in\{1,\ldots,n\}$ es semidefinida
    positiva.
  \end{enumerate}
\end{definicion}
Para un núcleo $k$, el teorema de Mercer \cite{mercer} asegura la
existencia de un espacio de Hilbert $F$ y una transformación
$\BPhi:X\rightarrow{}F$ tal que, para cualquier par de vectores
$\xx_1,\xx_2\in{}X$, el núcleo calcula el producto interno de los
mismos en el espacio imagen:
$k(\xx_1,\xx_2)=\langle\BPhi(\xx_1),\BPhi(\xx_2)\rangle$. Es
importante tener en cuenta que ni el espacio $F$ ni la transformación
$\BPhi$ son únicos.

El siguiente procedimiento se llama el \emph{truco del kernel} o
\emph{kernelización} de un algoritmo lineal: dado un método lineal
formulado en términos de productos internos de los datos y un núcleo
$k$, reemplazar todos los productos internos por la función $k$.  El
teorema de Mercer asegura que esto es equivalente a la aplicación de
la transformación $\BPhi$ primero y el algoritmo
lineal después.

Cuando se tienen dos vectores $\vv,\ww\in{}F$ definidos en términos
de combinaciones lineales de vectores transformados al espacio imagen

\begin{align*}
  \vv=\sum_{i=1}^{n}\alpha_i\BPhi(\xx_i) \quad\T{ y }\quad
  \ww=\sum_{j=1}^{m}\beta_j\BPhi(\xx_j'),
\end{align*}
el producto interno $\langle\vv,\ww\rangle$ se puede calcular en
términos del núcleo según

\begin{align}
  \langle \vv,\ww\rangle&=\big\langle\sum_{i=1}^{n}\alpha_i\BPhi(\xx_i),
      \sum_{j=1}^{m}\beta_j\BPhi(\xx_j')\big\rangle \notag\\
    &=\sum_{i=1}^{n}\sum_{j=1}^{m}
      \alpha_i\beta_j\big\langle\BPhi(\xx_i),\Phi(\xx_j')\big\rangle\notag\\
    &=\sum_{i=1}^{n}\sum_{j=1}^{m} \alpha_i\beta_jk(\xx_i,\xx_j').
\end{align}
De este resultado se derivan las siguientes propiedades

\begin{align}
  \langle{}v,\Phi(x)\rangle &= \sum_{i=1}^{n}\alpha_ik(x_i,x) \\
  \|v\|^2 &= \langle{}v\rangle=\sum_{i,j=1}^n\alpha_i\alpha_jk(x_i,x_j)\\
  \|\Phi(x)\|^2 &=k(x,x).
\end{align}
De este modo, las operaciones de producto interno en el espacio imagen
$F$ se pueden efectuar a través de la función núcleo $k$, sin necesidad
de calcular la transformación $\BPhi$ en forma explícita.

Desde un punto de vista puramente matemático, la utilización del truco
del kernel no aporta ninguna ventaja ya que los algoritmos son
equivalentes. Desde un punto de vista computacional, sin embargo, la
diferencia es decisiva: mediante el truco del kernel se pueden
calcular productos internos en espacios vectoriales de alta
dimensionalidad --incluso infinita--, sin necesidad de aplicar la
transformación $\BPhi$, cuyo cálculo directo muchas veces resulta
prohibitivo o incluso imposible en un ordenador.

Es intuitivamente claro que la elección de un núcleo que represente
una métrica específica al problema puede mejorar el rendimiento de las
máquinas de aprendizaje. Por ejemplo, en una tarea de clasificación es
conveniente elegir una métrica que agrupa las diferentes clases.

El núcleo lineal $k(x,x')=\langle{}x,x'\rangle=x^Tx'$ simplemente
calcula el producto interno en el espacio de entrada, correspondiente
a la transformación identidad $\BPhi(\xx)=\xx$.

Como primer ejemplo no trivial se considera el núcleo polinómico
$k(x,x')=\left(\langle{}x,x'\rangle+\theta\right)^d$ con grado
$d\in\R{N}$ y desvío $\theta\in\RR$. Este núcleo es una generalización
del núcleo lineal, el cual se obtiene para $d=1$ y $\theta=0$. Su
espacio imagen es el espacio de los polinomios de grado
$d$ sobre el espacio $X$, con la notable propiedad de que
el cálculo del núcleo toma sólo $\C{O}(\dim{}X)$ operaciones,
mientras que el cálculo explícito de la transformación
lleva 
$\C{O}\left((\dim{}X)^d\right)$ operaciones.

El núcleo más utilizado es la función de base radial
(\eng{Radial basis function, RBF})

\begin{align}
  k(x,x')=\exp\left(-\frac{\|x-x'\|^2}{2\sigma^2}\right)
  =\exp\left(-\gamma\|x-x'\|^2\right)
\end{align}
con el parámetro de amplitud $\sigma$ en la primera forma, o con el
parámetro de concentración $\gamma$ en la segunda forma. Corresponde
a un espacio imagen de dimensión infinita, que no permite
una representación directa de la transformación $\BPhi$.
El kernel RBF se puede calcular
en $\C{O}(\dim{}X)$ operaciones.
