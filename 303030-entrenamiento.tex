%
\subsection{Entrenamiento}
%
El modelo generado por el entrenamiento consta de 5 redes con la
topología especificada, inicializadas aleatoriamente.
El ajuste de los pesos de las redes se efectúa mediante el método de
retropropagación Rprop con los datos del conjunto de entrenamiento.

La inicialización aleatoria del perceptrón multicapa condiciona el
algoritmo de retropropagación: en algunos casos, la solución obtenida
es subóptima.
Para contrarrestar este efecto, el método soporta el entrenamiento de
múltiples redes MLP en paralelo, con inicializaciones aleatorias
diferentes, que son guardadas en el modelo.
Cuando el modelo se utiliza para obtener predicciones de clase, la
salida global se calcula según la \e{moda} entre las salidas de todas
las redes contenidas dentro del mismo.

La regularización de la red se efectúa separando un 20\% de los
ejemplos seleccionados al azar, deteniento el entrenamiento cuando el
error de clasificación alcanza su valor mínimo.
