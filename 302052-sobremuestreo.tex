%
\subsubsection{Sobremuestreo de la clase minoritaria}
%
Algunos clasificadores, como el perceptrón multicapa, son sensibles
al \e{desbalance} de clases en el conjunto de entrenamiento.
Esto es, cuando el conjunto de entrenamiento posee muchos más ejemplos
de una clase que de la otra, el modelo del clasificador generado
tiende a cometer muchos errores al clasificar la clase minoritaria.

El sobremuestreo de la clase minoritaria consiste simplemente en
agregar ejemplos al conjunto de entrnamiento repitiendo ejemplos de la
clase minoritaria seleccionados al azar, hasta igualar en número a la
clase mayoritaria.
Con el uso de esta técnica, se evita el desbalance de clases que
pudiera provocar un \e{sesgo} en el clasificador MLP.
Debe tenerse en cuenta asimismo que la repetición de ejemplos deriva
en problemas mal condicionados para los clasificadores SVM, por lo que
no se recomienda aplicar sobremuestreo en estos casos.
