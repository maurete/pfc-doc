%
%
\subsection{Minimización del error empírico}
%
La estrategia de minimización del error empírico, aplicable a la
máquina de vectores de soporte, optimiza una función objetivo
denominada \e{error empírico} basada en una interpretación
probabilística de la salida del modelo del clasificador SVM.

La función error empírico fue propuesta por \citeauthor{ayat} en
\cite{ayat} para la selección automática de los parámetros del núcleo.
En el presente trabajo, se incorpora el cálculo del gradiente de la
función respecto del hiperparámetro de regularización $C$
\cite{keerthi,glasmachers} tal como se implementa en \cite{shark}.  De
este modo, la estrategia de minimización del error empírico soporta la
selección automática de todos los hiperparámetros de la máquina de
vectores de soporte, incluyendo la regularización $C$.
%
\begin{quote}
  \sbs{Terminología.}
  El lector experto encontrará ambigua la denominación de la función
  ``error empírico''.  En la disciplina este nombre se utiliza como
  equivalente de ``error de entrenamiento''.  En este trabajo, se
  mantiene la denominación de los autores \cite{ayat}, diferenciando
  entre ``error de entrenamiento'' como la tasa de error del modelo
  sobre el conjunto de entrenamiento, y ``error empírico'', que se
  trata de una interpretación probabilística del error de validación
  cruzada sobre el conjunto de entrenamiento.
\end{quote}
%
