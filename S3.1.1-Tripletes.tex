
\subsection{Características de tripletes}
Este grupo consta de 32 características que cuentan la ocurrencia, en
la zona del tallo del pre-miRNA, de las 32 combinaciones posibles de
``tripletes'' que relacionan secuencia y estructura secundaria:
\mono{A...}, \mono{A..(}, \mono{A.(.}, etcétera.

Para un nucleótido en la posición $i$ de la secuencia, el ``triplete''
consiste en una secuencia que contiene la base del nucleótido (\ntA,
\ntC, \ntG, o \ntU) seguida de la estructura secundaria local en las
posiciones $i-1,i,i+1$. La estructura secundaria se representa en los
tripletes por un valor binario ``apareado'' \pairL (paréntesis) y ``no
apareado'' \noPair (punto).

Originalmente propuestas por \citeauthor{xue} en su trabajo
\cite{xue}, la justificación para estas características se basa en los
siguientes postulados:

\begin{itemize}
\item Dentro del pre-miRNA, el miRNA ``maduro'' se encuentra en la
  zona del ``tallo''.
\item Los pre-miRNAs, al ser bien conservados, contienen un alto grado
  de complementariedad en las bases que conforman el tallo.
\end{itemize}
Las características de tripletes describen en un vector de 32
elementos la relación secuencia--estructura secundaria local de los
nucleótidos en el tallo, y es de esperar que la aparición de
determinados patrones en esta relación permite distinguir aquellos
pre-miRNAs ``reales'' de aquellos ``pseudo'' pre-miRNAs, que
simplemente tienen estructura secundaria en forma de horquilla sin
contener miRNAs maduros.

Se debe remarcar que las características de tripletes están definidas
únicamente para instancias con estructura secundaria en forma de
\e{horquilla}, con sólo un bucle central y un tallo.  Si bien la mayor
parte de pre-miRNAs del reino animal cumplen con esta condición,
existe una proporción considerable de pre-miRNAs con estructura
secundaria más compleja. Más aún, en el reino vegetal, la mayoría de
pre-miRNAs presentan estructura secundaria con múltiples bucles y
tallos.

El procedimiento de cálculo para estas características consiste en

\begin{enumerate}
\item Validar que la instancia tiene estructura secundaria con un
  único tallo y un único bucle central.
\item Determinar las posiciones dentro de la secuencia que delimitan
  la región del tallo.
\item Para cada nucleótido en la región del tallo, incrementar en uno
  la cuenta del triplete correspondiente.
\item Normalizar la cuenta de cada triplete dividiendo entre la
  cantidad total de tripletes encontrados.
\end{enumerate}
Si el ejemplo no cumple con la primer condición, el cálculo de
características de triplete simplemente no se lleva a cabo.
