\section{Clasificación}
A partir del proceso de entrenamiento se obtiene un modelo entrenado
$\hat{y} = h(\xx)$ que, para un patrón $\xx$, devuelve una predicción
de pertenencia de clase positiva o negativa $\hat{y}\in\{-1,+1\}$.

Dado el modelo entrenado $\hat{y} = h(\xx)$ y el conjunto de prueba

\begin{align*}
  P = \left( (\xx_1,y_1),\ldots,(\xx_\ell,y_\ell) \right),
\end{align*}
la clasificación consiste simplemente en aplicar el modelo a todos los
elementos de $P$ de modo que la predicción de clase del ejemplo
$\xx_i$ es

\begin{align*}
  \hat{y}_i = h(\xx_i).
\end{align*}
Se debe notar que, toda vez que el conjunto $P$ sea utilizado para
clasificación, el valor de las etiquetas $y_i,\,i=1,\ldots,\ell,$ es
ignorado.

En el caso particular del MLP, el modelo $h$ es un conjunto $\B{h}$
de 5 modelos $(h_1, h_2, h_3, h_4, h_5)^T$, y la pertenencia de clase para
un ejemplo $\xx$ se establece según

\begin{align}
  y = \T{moda}( h_1(\xx), h_2(\xx), h_3(\xx), h_4(\xx), h_5(\xx)).
\end{align}

\chapter{Implementación}
La implementación refiere al conjunto de tareas efectuadas para
obtener un software funcional a partir de la especificación. En líneas
generales, estas tareas son la codificación, la documentación y la
puesta en funcionamiento en un equipo determinado.

La codificación del método se efectuó en el lenguaje y entorno Matlab,
versión R2012b. Se generaron funciones que implementan las siguientes
funcionalidades

\begin{itemize}
\item Lectura de datos de entrada
\item Extracción de características
\item Entrenamiento de clasificadores SVM y MLP
\item Selección automática de hiperparámetros
\item Clasificación de datos
\item Interfaz de usuario de línea de comandos
\item Interfaz de usuario web
\end{itemize}