%
%\subsubsection{Funciones núcleo comunes}
%

Algunas de las funciones núcleo más utilizadas con las máquinas de
vectores de soporte son las siguientes:
%
\begin{itemize}% [font=\relscale{0.9}]
%
\item \e{Núcleo lineal}:
  %
  \begin{align}
    k(\xx_1,\xx_2)=\pint{\xx_1}{\xx_2}=\pinT{\xx_1}{\xx_2}
  \end{align}
  %
  Calcula simplemente el producto interno entre sus argumentos.
  Aquí, el espacio imagen $Z$ es el mismo que el espacio de entrada,
  $X$, correspondiente a la transformación identidad $\BPhi(\xx)=\xx$.
%
\item \e{Núcleo polinómico}:
  %
  \begin{align}
    k(\xx_1,\xx_2)=\left(\pint{\xx_1}{\xx_2}+\theta\right)^d,
  \end{align}
  %
  con grado $d\in\R{N}$ y desvío $\theta\in\RR$.
  Este núcleo es una generalización del núcleo lineal (el caso $d=1$ y
  $\theta=0$).
  Su espacio imagen es el espacio de los polinomios de grado $d$ sobre
  el espacio $X$.
  El cálculo del producto interno aplicando el núcleo requiere
  $\C{O}(\dim{}X)$ operaciones, contrastando con las
  $\C{O}\left((\dim{}X)^d\right)$ necesarias para el cálculo de la
  transformación $\BPhi$ \cite{glasmachers}.
%
\item \e{Núcleo gaussiano}:
  %
  \begin{align}
    k(\xx_1,\xx_2)=\exp\left(-\frac{\|\xx_1-\xx_2\|^2}{2\sigma^2}\right)
    =\exp\left(-\gamma\|\xx_1-\xx_2\|^2\right)
  \end{align}
  %
  Es el núcleo más utilizado.
  También llamado \e{función de base radial} (en inglés \eng{Radial
    Basis Function}, \e{RBF}).
  En la primer forma, el hiperparámetro $\sigma$ se llama de
  \e{amplitud}.
  En la segunda forma el hiperparámetro $\gamma$ se llama de
  \e{concentración}.
  La transformación $\BPhi$ correspondiente a este núcleo genera un
  espacio imagen de dimensión infinita, imposible de calcular
  directamente.
  El producto interno mediante este núcleo se puede calcular en
  $\C{O}(\dim{}X)$ operaciones \cite{glasmachers}.
  %
\end{itemize}
%
