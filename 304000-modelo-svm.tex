%
%
\section{Construcción del modelo del clasificador SVM}
%
El método permite la construcción de modelos de clasificador mediante
SVMs con regularización L1, y soporta la utilización de los núcleos
lineal y de función de base radial (también llamado gaussiano o RBF).
Antes de efectuar el entrenamiento, se optimizan los valores de los
\hparam{s} del clasificador a partir de los datos de entrenamiento,
aplicando alguna de las siguientes estrategias de selección de
\hparam{s}:
%
\begin{itemize}
\item
  \e{Selección trivial}: Retorna valores preestablecidos,
  sin efectuar entrenamiento.
\item
  \e{Búsqueda en la grilla}: Efectúa una búsqueda exhaustiva de los
  \hparam{s} óptimos, maximizando la tasa \GM{} de validación cruzada
  sobre el conjunto de entrenamiento.
\item
  \e{Criterio del error empírico}: Minimiza una función probabilística
  derivada del error de clasificación de validación cruzada, mediante
  descenso por gradiente en el espacio de los \hparam{s}.
\item
  \e{Minimización de la cota radio-margen}: Minimiza mediante descenso
  por gradiente una función de derivación teórica que relaciona
  las propiedades geométricas del modelo con el error de clasificación.
\end{itemize}
%

En la implementación se codificó una interfaz común para las funciones
de la biblioteca \work{libSVM} \cite{libsvm} y las funciones SVM
disponibles en el paquete \work{Bioinformatics Toolbox} de Matlab.
Esta interfaz funciona como una capa de abstracción que permite el uso
indistinto de estas bibliotecas en casi todos los casos.
A continuación se describen las diferentes estrategias de selección de
\hparam{s} y el entrenamiento del modelo del clasificador.
