%
%
\section{Construcción del modelo del clasificador SVM}
%
Tal como en el caso del perceptrón multicapa, la construcción del
modelo SVM es un proceso de dos etapas: en primer lugar se determinan
valores óptimos para los \hparam{s}, y luego se procede a entrenar el
modelo final con los datos de entrenamiento y los \hparam{s} óptimos
encontrados.
%% A diferencia del caso MLP, sin embargo, en una máquina de vectores de
%% soporte puede haber más de un \hparam{}, y en general éstos se trata
%% de variables continuas.

Las propiedades analíticas de la SVM permiten aplicar estrategias de
selección de hiperparámetros más complejas que en el caso del MLP, que
en general resultan computacionalmente más eficientes.
Las cuatro estrategias de selección de \hparam{s} codificadas para el
clasificador SVM son:
%
\begin{itemize}
\item
  \e{Selección trivial}: Retorna valores preestablecidos para los
  \hparam{s}, sin efectuar entrenamiento.
\item
  \e{Búsqueda en la grilla}: Efectúa una búsqueda exhaustiva de los
  hiperparámetros, maximizando la tasa \GM{} de validación cruzada.
\item
  \e{Minimización del error empírico}: Minimiza una función que
  representa el error de clasificación de validación cruzada mediante
  descenso por gradiente en el espacio de los \hparam{s}.
\item
  \e{Minimización de la cota radio-margen}: Minimiza mediante descenso
  por gradiente en el espacio de los \hparam{s} una función de
  derivación teórica que relaciona propiedades geométricas del modelo
  con el error de clasificación.
\end{itemize}
%
