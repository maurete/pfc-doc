%
%
\section{Construcción del modelo del clasificador SVM}
%
Tal como en el caso del perceptrón multicapa, la construcción del
modelo SVM es un proceso de dos etapas: en primer lugar se determinan
valores óptimos para los \hparam{s}, y una vez encontrados, se procede
a entrenar el modelo final con los datos de entrenamiento y los
\hparam{s} óptimos encontrados.
%% A diferencia del caso MLP, sin embargo, en una máquina de vectores de
%% soporte puede haber más de un \hparam{}, y en general éstos se trata
%% de variables continuas.

Las propiedades analíticas de la SVM permiten aplicar estrategias de
selección de hiperparámetros más complejas que en el caso del MLP, que
en general resultan computacionalmente más eficientes.
Las cuatro estrategias de selección de \hparam{s} aplicables al
clasificador SVM son:
%
\begin{itemize}
\item
  Selección trivial: Retorna valores preestablecidospara los
  \hparam{s}, sin efectuar entrenamiento alguno.
\item
  Búsqueda en la grilla: Efectúa una búsqueda exhaustiva de los
  hiperparámetros, maximizando la tasa de clasificación de validación
  cruzada.
\item
  Minimización del error empírico: Minimiza, mediante descenso por
  gradiente en el espacio de los \hparam{s}, una función que representa
  el error de clasificación de validación cruzada.
\item
  Minimización de la cota radio-margen: Minimiza mediante descenso por
  gradiente en el espacio de los \hparam{s} una función de derivación
  teórica que se relaciona con el error de clasificación.
\end{itemize}
%
