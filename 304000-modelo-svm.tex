%
%
\section{Construcción del modelo del clasificador SVM}
%
AL igual que para el caso del clasificador MLP, la construcción
del modelo SVM es un proceso de dos etapas: en primer lugar se determinan
valores óptimos para los hiperparámetros, y una vez encontrados, se procede a
entrenar el modelo ``hiperparametrizado''.

A diferencia del caso MLP, sin embargo, en una máquina de vectores de soporte
los \hparam{s} pueden ser más de uno, y se trata de variables continuas.

Esto junto a las propiedades analíticas de la SVM permite aplicar estrategias
de selección de hiperparámetros más complejas que aprovechan mejor la información
del modelo SVM. Incluso, se puede aplicar optimización mediante descenso por gradiente.
%
\begin{description}
\item[Estrategia trivial:] Consiste en seleccionar hiperparámetros
  preestablecidos, sin entrenamiento. Se utiliza como estrategia
  ``básica'' para comparación de los métodos.
\item[Estrategia de búsqueda en la grilla:] Efectúa una búsqueda
  exhaustiva de los hiperparámetros maximizando la tasa de
  clasificación de validación cruzada.
\item[Estrategia de minimización del error empírico:] Realiza una
  búsqueda por gradiente en el espacio de los hiperparámetros de la
  tasa de clasificación de validación cruzada.
\item[Estrategia de minimización de la cota RMB]: Minimiza una función de derivación teórica
  intentando maximizar la tasa de clasificación.
\end{description}
%
