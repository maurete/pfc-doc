\section{Descripción de los problemas de clasificación}
A partir de la bibliografía, se defineron tres \e{problemas} de
clasificación, denominados \sbs{Triplet-SVM}, \sbs{miPred} y
\sbs{microPred}, que reproducen en mayor o menor medida los datos de
entrenamiento y prueba utilizados por los autores de
\e{Triplet-SVM} \cite{xue}, \e{miPred} \cite{ng}, y
\e{microPred} \cite{batuwita} para las pruebas de los respectivos
métodos.

En los tres problemas definidos, los ejemplos de clase positiva
consisten en pre-miRNAs de la especie humana experimentalmente
validados, y provienen de diferentes versiones de la base de datos
\e{miRBase} \cite{mirbase1, mirbase2, mirbase3}.  Esta base de datos es el
repositorio de referencia de \mbox{(pre-)}miRNAs, y es mantenida por el
laboratorio Griffiths-Jones en la Universidad de Manchester, Reino
Unido. Diferentes versiones de miRBase se publican periódicamente con
nuevos pre-miRNAs descubiertos para diferentes organismos.

Asimismo, los ejemplos de clase negativa provienen principalmente de
un conjunto de datos artificial denominado \e{CODING}. Este conjunto,
creado por \citeauthor{xue} y publicado con el método \e{Triplet-SVM},
contiene 8494 ``pseudo'' pre-miRNAs (ejemplos negativos), generados a
partir de segmentos de secuencia extraídos de regiones genómicas para
las que no se ha reportado la presencia de ningún pre-miRNA.  El
armado del conjunto de datos \e{CODING} se detalla en \cite{xue}.

\subsection{Problema Triplet-SVM}
Este problema define la tarea de clasificación utilizada en \cite{xue}
como evaluación del método \e{Triplet-SVM}.  Este método fue uno de
los primeros métodos de predicción de pre-miRNAs mediante máquinas de
vectores de soporte.  En el trabajo original, los autores utilizan
vectores con características de tripletes como entrada al clasificador
SVM.

Los ejemplos de clase positiva son tomados de la versión 5.0 (de
septiembre de 2004) de miRBase. De las 207 entradas de pre-miRNA de la
especie humana presentes en esta versión, se eliminan aquellas con
ramificaciones en la estructura secundaria, resultando en 193 ejemplos
de pre-miRNAs con estructura secundaria en forma de horquilla.  De
éstos, se utilizan 163 ejemplos para componer el conjunto de
entrenamiento y los 30 restantes para el conjunto de prueba.  Los
ejemplos de clase negativa se obtienen de la base de datos \e{CODING},
utilizando 168 ejemplos para el conjunto de entrenamiento y 1000 para
el armado del conjunto de prueba.

La composición de los conjuntos de entrenamiento y prueba se resumen
en la \autoref{tbl:mainxue}. A diferencia de los otros problemas
definidos, los ejemplos contenidos en los conjuntos de entrenamiento y
de prueba son exactamente los mismos que aquellos utilizados en el
trabajo original, ya que los autores publicaron los conjuntos por
separado en el material suplementario.