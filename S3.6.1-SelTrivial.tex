%
%
\subsection{Selección trivial}
%
La estrategia de selección trivial consiste en seleccionar valores
preestablecidos (genéricos) para los hiperparámetros, apropiados para
la mayoría de los problemas.

En el caso del perceptrón multicapa, la selección trivial consiste en
utilizar $0$ neuronas en la capa oculta, por lo que el clasificador
resultante es equivalente a un perceptrón simple.  En el caso de las
máquinas de vectores de soporte, la estrategia selecciona el valor $1$
para el hiperparámetro de regularización $C$ \cite{libsvm}, y un valor
$\gamma=\frac{1}{2F}$ para el hiperparámetro de amplitud del núcleo
RBF, donde $F$ es el número de características consideradas
\cite{glasmachersigel}.

Los hiperparámetros resultantes de utilizar esta estrategia no
dependen del problema de clasificación, por lo que no se requiere
entrenar el clasificador. Por este motivo, en general los valores
obtenidos se consideran subóptimos frente a las demás estrategias.  La
utilización de la selección trivial resulta conveniente para la
comparación con el funcionamiento de las demás estrategias de
selección automática de hiperparámetros.
