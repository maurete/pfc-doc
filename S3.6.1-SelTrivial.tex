\subsection{Selección trivial}
La selección trivial consiste en seleccionar valores preestablecidos
``razonables'' para los hiperparámetros del clasificador, que resulten
apropiados para la mayoría de los problemas.

Para el caso del Perceptrón Multicapa, la estrategia de selección
trivial consiste en seleccionar $0$ neuronas en la capa oculta, de
modo que el MLP resultante es equivalente a un perceptrón simple.

Para una Máquina de Vectores de Soporte, la estrategia selecciona el
valor $1$ para el hiperparámetro de regularización $C$ \cite{libsvm}, y
un valor $\gamma=\frac{1}{2F}$ para el hiperparámetro de amplitud del
kernel RBF, en donde $F$ es el número de características consideradas
\cite{glasmachersigel}.

Los hiperparámetros encontrados por esta estrategia no dependen del
problema de clasificación, sino que son genéricos. Por ello, no se
requiere entrenar el clasificador, y en general los valores
encontrados deben considerarse subóptimos frente a las demás
estrategias.  Así y todo, resulta conveniente su utilización para el
análisis del comportamiento de los clasificadores frente a un problema
dado.
