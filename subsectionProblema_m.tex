\subsection{Problema miPred}
El problema \sbs{miPred} se basa en los datos utilizados para
entrenamiento y prueba del método miPred \cite{ng}, publicado en 2007
por \citeauthor{ng}.

Los ejemplos de clase positiva se obtienen de la versión 8.2 (julio de
2006) de la base de datos \e{miRBase}. En total, se encuentran 323
pre-miRNAs de la especie humana, incluyendo tanto ejemplos con
estructura secundaria tipo horquilla (bucle central único) así como
ramificada (bucles múltiples).  Los ejemplos de clase negativa se
obtienen de la base de datos \e{CODING}, al igual que en el problema
\tripletsvm{}.

Para el armado del conjunto de entrenamiento, se seleccionaron al azar
200 ejemplos de clase positiva y 400 del conjunto de datos negativos.
Asimismo, el conjunto de prueba se compone con los 123 ejemplos
positivos restantes, y de 246 ejemplos seleccionados al azar del
conjunto negativo, excluyendo los ya utilizados para entrenamiento.

Si bien los conjuntos de datos de entrenamiento y prueba mantienen las
mismas proporciones de ejemplos positivos y negativos y utilizan las
mismas fuentes de datos que en el trabajo de referencia, resulta
imposible replicar la composición exacta de ambos conjuntos de datos,
ya que los autores no publicaron los conjuntos de datos por separado.

La composición de los conjuntos de entrenamiento y prueba se detallan
en \autoref{tbl:pruebasng}.