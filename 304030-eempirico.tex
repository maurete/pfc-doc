%
%
\subsection{Criterio del error empírico}
%
El \e{criterio del error empírico} es una estrategia que optimiza los
\hparam{s} del clasificador minimizando una función objetivo
denominada \e{error empírico}\,\footnote{
  El lector experto encontrará ambigua la denominación de la función
  ``error empírico'', ya que en la disciplina este nombre se utiliza
  como equivalente de ``error de entrenamiento''.
  En este trabajo, se mantiene la denominación de los autores
  \cite{ayat}, definiendo ``error empírico'' como la función que
  calcula un error probabilístico de validación cruzada sobre el
  conjunto de entrenamiento.
}, basada en la propuesta presentada en \cite{ayat}.
La función de error empírico mide el error de validación cruzada sobre
el conjunto de entrenamiento acoplando un modelo probabilístico a la
salida del clasificador, y permite su optimización mediante descenso
por gradiente ya que sus derivadas respecto de los \hparam{s} son
calculables.

En la propuesta original \cite{ayat}, la función {error empírico} se
optimiza únicamente respecto a los \hparam{s} del núcleo.
En la estrategia implementada, se agregó soporte para la optimización
del hiperparámetro $C$, siguiendo el método del cálculo de la derivada
respecto de $C$ propuesto en \cite{keerthi} y \cite{glasmachers}, tal
como se implementa en \cite{shark}.
