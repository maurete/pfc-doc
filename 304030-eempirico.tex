%
%
\subsection{Minimización del error empírico}
%
La función ``error empírico'' es una función que mide el error de
validación cruzada sobre el conjunto de entrenamiento adaptando un
modelo probabilístico a la salida del clasificador. De este modo,
resulta posible calcular las derivadas del modelo SVM respecto de sus
hiperparámetros, lo que deriva en la estrategia de minimización del
error empírico mediante descenso por gradiente.

La implementación se basa en la propuesta de \cite{ayat}, incorporando
el cálculo del gradiente del modelo SVM respecto del hiperparámetro de
regularización $C$ \cite{keerthi,glasmachers}, tal como se implementa
en \cite{shark}.
%
\begin{quote}
  \sbs{Terminología.}
  El lector experto encontrará ambigua la denominación de la función
  ``error empírico''.  En la disciplina este nombre se utiliza como
  equivalente de ``error de entrenamiento''.  En este trabajo, se
  mantiene la denominación de los autores \cite{ayat}, diferenciando
  entre ``error de entrenamiento'' como la tasa de error del modelo
  sobre el conjunto de entrenamiento, y ``error empírico'', que se
  trata de una interpretación probabilística del error de validación
  cruzada sobre el conjunto de entrenamiento.
\end{quote}
%
