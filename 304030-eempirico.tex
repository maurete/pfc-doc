%
%
\subsection{Minimización del error empírico}
%
La función \e{error empírico} mide el error de validación cruzada
sobre el conjunto de entrenamiento aplicando un modelo probabilístico
que se acopla a la salida del clasificador.
Este modelo probabilístico permite calcular las derivadas del modelo
respecto de los hiperparámetros, permitiendo la aplicación de una
estrategia de minimización del error mediante descenso por gradiente.

La estrategia de minimización del error empírico se basa en la
propuesta presentada en \cite{ayat}, e incorpora una técnica para el
cálculo del gradiente del modelo SVM respecto del hiperparámetro $C$
tal como es explicada en los trabajos \cite{keerthi} y
\cite{glasmachers} e implementada en \cite{shark}.
%
\begin{quote}
  \sbs{Terminología.}
  El lector experto encontrará ambigua la denominación de la función
  ``error empírico'', ya que en la disciplina este nombre se utiliza
  como equivalente de ``error de entrenamiento''.
  En este trabajo, se mantiene la denominación de los autores
  \cite{ayat}, diferenciando entre ``error de entrenamiento'' como la
  tasa de error del modelo sobre el conjunto de entrenamiento, y
  ``error empírico'', que se trata de una interpretación
  probabilística del error de validación cruzada sobre el conjunto de
  entrenamiento.
\end{quote}
%
