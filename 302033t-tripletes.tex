\newcommand{\tripletRow}[1]{
  \stepcounter{FeatureCounter}\theFeatureCounter &
  \dset{T} & $N_{\T{\mono{#1}}}$ &
  Número de ocurrencias del triplete \mono{#1} en la región del
  tallo. \iflatexml{}\\\fi
  }

\begin{longtable}{@{}p{0.07\textwidth}%
@{\hspace{0.01\textwidth}}p{0.07\textwidth}%
@{\hspace{0.01\textwidth}}p{0.13\textwidth}%
@{\hspace{0.01\textwidth}}p{0.70\textwidth}@{}}
  \headRow\endhead\iflatexml{}\\\fi
  \tripletRow{A...}\\
  \tripletRow{A..(}\\
  \tripletRow{A.(.}\\
  \tripletRow{A.((}\\
  \tripletRow{A(..}\\
  \tripletRow{A(.(}\\
  \tripletRow{A((.}\\
  \tripletRow{A(((}\\
  \tripletRow{G...}\\
  \tripletRow{G..(}\\
  \tripletRow{G.(.}\\
  \tripletRow{G.((}\\
  \tripletRow{G(..}\\
  \tripletRow{G(.(}\\
  \tripletRow{G((.}\\
  \tripletRow{G(((}\\
  \tripletRow{C...}\\
  \tripletRow{C..(}\\
  \tripletRow{C.(.}\\
  \tripletRow{C.((}\\
  \tripletRow{C(..}\\
  \tripletRow{C(.(}\\
  \tripletRow{C((.}\\
  \tripletRow{C(((}\\
  \tripletRow{U...}\\
  \tripletRow{U..(}\\
  \tripletRow{U.(.}\\
  \tripletRow{U.((}\\
  \tripletRow{U(..}\\
  \tripletRow{U(.(}\\
  \tripletRow{U((.}\\
  \tripletRow{U(((}\\
  33 & \dset{X} & $L_3$ &
  Longitud del tallo: cantidad de nucleótidos en el tallo de
  la estructura de horquilla. \\
  34 & \dset{X} & $P$ &
  Número de pares de bases en el \premirna{}. \\
  35 & \dset{X} & $L_3/P$ &
  Grado de complementariedad entre los dos brazos de la estructura de
  horquilla: para una complementariedad perfecta, se da el valor
  mínimo 2. Este valor aumenta conforme aumenta el número de
  bases ``sueltas'' (no acopladas) en el tallo. \\
  36 & \dset{X} & $GC=\frac{N_\ntC{}+N_\ntG{}}{L_3}$ &
  Proporción de bases \ntG{} y \ntC{} en el tallo.  Se calcula contando el
  número de bases \ntC{} y \ntG{} en el tallo y dividiendo por $L_3$.
\end{longtable}
