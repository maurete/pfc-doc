%
%
%
\section{Consideraciones generales}
%
Al analizar los resultados, se encontró que la elección del conjunto
de características resultó determinante para el desempeño del
clasificador.
La utilización del conjunto de características S-E resultó en las
mayores tasas de clasificación, mientras que el conjunto de
características E, compuesto únicamente por 7 características, obtuvo
el menor tiempo de ejecución con tasas de clasificación
satisfactorias.
Por el contrario, la utilización del conjunto de características S
resultó en las tasas de clasificación más bajas, además de un
incremento considerable en el tiempo de entrenamiento para todos los
métodos, incluso provocando la divergencia del método RMB en el
problema \mipred{}.

La recomendación al considerar los conjuntos de características es
utilizar el conjunto S-E, que obtiene los mejores valores de $G_m$ y
como segunda opción el conjunto E, que resulta en tiempos reducidos de
entrenamiento.
La utilización de estos conjuntos de características resulta además en
clasificadores más generales, sin restricciones en el número de bucles
en la estructura secundaria, restricciones que sí se presentan al
considerar las características T y X.
Se recomienda asimismo descartar de plano el conjunto de
características S.

Considerando los diferentes clasificadores, se obtuvieron resultados
comparables tanto para MLP como para SVM con kernel RBF y lineal.
Se destacaron en particular los resultados obtenidos al utilizar los
métodos de selección de parámetros de error empírico y RMB: la
estrategia de error empírico obtuvo los mejores resultados para los
problemas \mipred{} y \micropred{} (núcleo RBF), para el problema
\tripletsvm{} los mejores resultados se obtuvieron con la estrategia
RMB.

Respecto a los tiempos de ejecución de las diferentes estrategias no
triviales, se pudo observar que la a estrategia RMB resultó la más
rápida por un amplio margen.
Para la estrategia del criterio del error empírico se obtuvo un buen
compromiso entre la velocidad de ejecución y los resultados obtenidos:
aunque en varias ocasiones redunda en más entrenamientos que la
búsqueda exhaustiva, el entrenamiento resulta más rápido dado que se
entrena ``con valores más correctos'' que en una búsqueda exhaustiva.
