%
%
%
\section{Discusión}
%
Los resultados obtenidos en las pruebas muestran que la elección del
conjunto de \caract{s} resultó determinante para el desempeño de los
clasificadores.
La utilización del conjunto de \caract{s} \dset{S-E} obtuvo las
mejores tasas de clasificación, mientras que el conjunto de \caract{s}
\dset{E} obtuvo muy buenos resultados considerando que se trata de un
vector de sólo 7 \caract{s}.
Por el contrario, la utilización del conjunto de \caract{s} \dset{S}
resultó en las tasas de clasificación más bajas, incluso provocando la
divergencia de la estrategia de minimización de la cota RM en el
problema \prob\mipred{}.
La recomendación general es entonces utilizar los conjuntos de \caract{s}
\dset{S-E} o \dset{E}.
Se deberá evitar utilizar el conjunto \dset{S}, ya que obtiene tasas
de clasificación no satisfactorias y puede provocar inestabilidad.
Se recomienda asimismo no utilizar las \caract{s} de tripletes
\dset{T}, \dset{X} y \dset{T-X} para evitar potenciales problemas
cuando los ejemplos presenten bucles múltiples.

Respecto a los diferentes clasificadores, se obtuvieron resultados
comparables tanto para MLP como para SVM con núcleos RBF y lineal.
Se destacaron en particular los resultados obtenidos al utilizar las
estrategias de minimización del error empírico en el caso de los
problemas \prob\mipred{} y \prob\micropred{} y la cota radio-margen
para el problema \prob\micropred{}.
En general, se recomienda utilizar el clasificador SVM en alguna de
sus dos variantes, ya que en la práctica los resultados obtenidos con
el clasificador MLP varían en función de la inicialización aleatoria
de la red.
