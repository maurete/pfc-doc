%
%
%
\section{Análisis de los resultados}
%
Los resultados obtenidos en las pruebas muestran que la elección del
conjunto de \caract{s} resultó determinante para el desempeño de los
clasificadores.
El conjunto de \caract{s} \dset{S-E} obtuvo las mejores tasas de
clasificación, mientras que el conjunto de \caract{s} \dset{E} obtuvo
muy buenos resultados considerando que se trata de un vector de sólo 7
\caract{s}.
Por el contrario, la utilización del conjunto de \caract{s} \dset{S}
resultó en las tasas de clasificación más bajas, incluso provocando la
divergencia de la estrategia de minimización de la cota radio-margen
en el problema \prob\mipred{}.

Respecto a los clasificadores, se obtuvieron resultados comparables
con los tres clasificadores evaluados, con una pequeña ventaja de los
clasificadores SVM por sobre el perceptrón multicapa.
Se destacaron en particular los resultados obtenidos al utilizar las
estrategias de minimización del error empírico en el caso de los
problemas \prob\mipred{} y \prob\micropred{} y la cota radio-margen
para el problema \prob\tripletsvm{}.

A partir de estas observaciones se formulan las siguientes
recomendaciones generales:
%
\begin{itemize}
\item
  \e{Utilizar los conjuntos de \caract{s} \dset{S-E} o \dset{E}}.
  Mientras que el conjunto \dset{S} obtiene resultados no
  satisfactorios y puede provocar inestabilidades, las \caract{s} de
  tripletes \dset{T}, \dset{X} y \dset{T-X} no pueden clacularse
  cuando los ejemplos contienen bucles múltiples.
\item
  \e{Utilizar el clasificador SVM en alguna de sus dos variantes}.
  En la práctica, los resultados obtenidos con el clasificador MLP son
  dependientes de la inicialización aleatoria de la red, condicionando
  el modelo y agregando incertidumbre a la hora de clasificar nuevos
  datos.
\item
  \e{Probar la estrategia trivial con el clasificador SVM-lineal y la
    estrategia de minimización de la cota radio-margen con el
    clasificador SVM-RBF}.
  Ambas estrategias tienen un bajo coste computacional y obtiene
  buenos resultados con los respectivos clasificadores.
\end{itemize}
%
