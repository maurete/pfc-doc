%
%
\subsection{Problema microPred}
%
Este problema se basa en los datos utilizados en \cite{batuwita} para
entrenamiento y prueba del método \e{microPred}.
Los ejemplos de clase positiva se obtienen de la versión 12.0 de la
base de datos \dataset{miRBase}.
Para los ejemplos de clase negativa se utilizan dos fuentes, la base
de datos \dataset{coding} y una base de datos curada por los autores,
denominada \dataset{other human ncRNAs}, que contiene en total 754
ejemplos de ncRNAs (excluyendo pre-miRNAs) de la especie humana.

En el trabajo original, los autores no especifican una separación
en conjuntos de entrenamiento y prueba, sino que utilizan todos los
ejemplos contenidos en las bases de datos positivas y negativas para
entrenamiento, y reportan las medidas de rendimiento del método
a partir de los valores obtenidos en la validación cruzada.
%% Adicionalmente, aplican sobremuestreo de la clase minoritaria (positiva)
%% mediante la técnica SMOTE \cite{smote}.
Este esquema contrasta con la forma de efectuar las pruebas en el
presente trabajo, en donde los resultados se obtienen de clasificar un
conjunto de prueba especificado y separado de los datos de entrenamiento.

El problema \micropred{} define conjuntos de entrenamiento y prueba
a partir de los datos originales del siguiente modo:
se utiliza el 85\% (587/691) de los ejemplos positivos para
entrenamiento, reservando el 15\% restante (104 ejemplos) para el
conjunto de prueba.
Tal como en el problema \sbs{miPred}, se mantiene una proporcionalidad
de 1:2 entre ejemplos positivos y negativos, utilizando 1174 ejemplos
negativos en el conjunto de entrenamiento.
De ellos, 1078 provienen de la base de datos \dataset{coding} y 96 de
\dataset{other human ncrnas}, manteniendo la proporcionalidad entre
las mismas.
El conjunto de prueba incorpora todos los ejemplos de las bases de
datos positivas y negativas no utilizadas para entrenamiento:
104 ejemplos de \dataset{miRBase} (clase positiva), 7416 ejemplos de
\dataset{coding} (clase negativa), y 658 ejemplos de \dataset{other
  human ncRNAs} (clase negativa).
