%
%
\subsection{Problema microPred}
%
Este problema se basa en los datos utilizados en \cite{batuwita} para
entrenamiento y prueba del método \work{\micropred}.
Los ejemplos de clase positiva se obtienen de la versión 12.0 de la
base de datos \work{\mirbase}.
Para los ejemplos de clase negativa se utilizan dos fuentes: la base
de datos \dset{coding} y una base de datos curada por los autores,
denominada \dset{other human ncRNAs}, que contiene en total 754
ejemplos de \ncrna{s} (excluyendo \premirna{s}) de la especie humana.

En el trabajo original, los autores no especifican una separación
en conjuntos de entrenamiento y prueba, sino que utilizan todos los
ejemplos contenidos en las bases de datos positivas y negativas para
entrenamiento, y reportan las medidas de rendimiento del método
a partir de los resultados obtenidos en la validación cruzada.
Este esquema contrasta con la forma de efectuar las pruebas en el
presente trabajo, en donde los resultados se obtienen de clasificar un
conjunto de prueba previamente separado de los datos de
entrenamiento.

El armado de los conjuntos de datos de entrenamiento y prueba del
problema \prob\micropred{} es como sigue:
se utiliza el 85\% (587/691) de los ejemplos positivos en el conjunto
de entrenamiento, reservando el 15\% restante (104 ejemplos) para el
conjunto de prueba.
Tal como en el problema \prob{\mipred}, se mantiene una
proporcionalidad de 1:2 entre ejemplos positivos y negativos, por lo
que se incluyen 1174 ejemplos negativos en el conjunto de
entrenamiento.
De ellos, se toman 1078 de la base de datos \dset{coding} y 96
de \dset{human other \ncrna{s}}, respetando la proporcionalidad de
8494:754 entre ambas bases de datos.

El conjunto de prueba incorpora todos los ejemplos de las bases de
datos positivas y negativas no utilizados para entrenamiento: 104
ejemplos de \dset{miRBase} (clase positiva), 7416 ejemplos de
\dset{coding} (clase negativa), y 658 ejemplos de \dset{human other
  \ncrna{s}} (clase negativa).

En la \iflatexml{}Tabla~\ref{tbl:problembtw}\else\autoref{tbl:problembtw}\fi{}
se resume la composición de los conjuntos de entrenamiento y prueba
que integran el problema \prob\micropred.
