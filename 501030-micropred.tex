%
%
\subsection{Problema \micropred{}}
%
Este problema se basa en los datos utilizados en \cite{batuwita} para
entrenamiento y prueba del método \work{\micropred}.

Como ejemplos de clase positiva se utilizaron $691$ \premirna{s}
de la especie humana de la versión $12$.$0$ de
\dset{\mirbase}.
Los ejemplos de clase negativa provienen de dos fuentes: además de la
base de datos \dset{coding}, se utilizó una base de datos curada por
los autores, denominada \dset{other human ncRNAs}, que contiene en
total $754$ ejemplos de \ncrna{s} (excluyendo \premirna{s}) de la
especie humana \cite{batuwita}.

En el trabajo original, los autores no especifican una separación
en conjuntos de entrenamiento y prueba, sino que utilizan todos los
ejemplos contenidos en las bases de datos positivas y negativas para
entrenamiento, y reportan las medidas de rendimiento del método
a partir de los resultados obtenidos mediante validación cruzada.
Este esquema contrasta con la forma de efectuar las pruebas en el
presente trabajo, en donde los resultados se obtienen de clasificar un
conjunto de prueba previamente separado de los datos de entrenamiento,
aplicando el método de retención.

El armado de los conjuntos de datos del problema \prob\micropred{} se
efectuó según el procedimiento a continuación.
De los ejemplos de clase positiva, se utilizó el $85\%$ ($587/691$) de
los ejemplos, seleccionados aleatoriamente para generar el conjunto de
entrenamiento.
Tal como se propone en \cite{ng}, se mantuvo una proporcionalidad de
$1$ a $2$ entre ejemplos positivos y negativos, seleccionando al azar
$1174$ ejemplos negativos para entrenamiento.
De ellos, $1078$ se obtienen de la base de datos \dset{coding} y $96$
de \dset{human other \ncrna{s}}, respetando la proporcionalidad de
de $8494$ a $754$ entre ambas bases de datos.

El conjunto de prueba incorpora todos los ejemplos de las bases de
datos positivas y negativas no utilizados para entrenamiento: $104$
ejemplos de clase positiva de \dset{miRBase}, $7416$ ejemplos
negativos de \dset{coding}, y $658$ ejemplos de \dset{human other
  \ncrna{s}}, también de clase negativa.

En la \iflatexml{}Tabla~\ref{tbl:problembtw}\else\autoref{tbl:problembtw}\fi{}
se resume la composición de los conjuntos de entrenamiento y prueba
que integran el problema \prob\micropred{}.
