%
%
\subsection{La arquitectura del perceptrón multicapa}
%
El perceptrón multicapa tiene una arquitectura acíclica organizada en
\e{capas}.
La primer capa se denomina \e{capa de entrada} y contiene, en lugar de
neuronas, nodos sensores que ``leen'' el vector presentado como
entrada a la red.
Los valores de las salidas de esta capa equivalen a los componentes
del vector.
Las capas subsiguientes contienen neuronas, cada una de las cuales
recibe como entrada \e{todas} las salidas de la capa anterior.
Para el cálculo de las salidas de cada capa se requiere conocer los
valores de las salidas de la capa anterior, por lo que se dice que el
vector de entrada se \e{propaga hacia adelante} a través de la red.

La salida global de la red se compone por las salidas de las neuronas
en la última capa, denominada simplemente \e{capa de salida}.
Las capas que no son ni de entrada ni de salida se denominan \e{capas
  ocultas}.

En la \iflatexml{}Figura~\ref{fig:mlp}\else\autoref{fig:mlp}\fi{} se
representa un perceptrón multicapa de 2 capas, que lee a su entrada un
vector de 3 elementos.
Este vector es propagado a través de una capa oculta de 4 neuronas
hasta alcanzar las 2 neuronas en la capa de salida, que determinan la
salida de la red.
