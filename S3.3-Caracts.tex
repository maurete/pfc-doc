%
%
%
\subsection{Extracción de caracerísticas}
%
La extracción de características es el proceso de calcular, vectores
numéricos de longitud fija que representan cada ejemplo.  Estos
\e{vectores de características} son una representación más conveniente
de los ejemplos originalmente dados en forma de cadenas de caracteres
de longitud variable.

El vector de características incorpora medidas tomadas sobre la
secuencia, la estructura secundaria, y la mínima energía libre
resultante del cálculo de la estructura secundaria. Consta de 66 medidas
basadas en aquellas utilizadas en los trabajos previos de \cite{xue, ng,
  batuwita}. Según la naturaleza de estas características, se
distinguen cuatro ``grupos'':
%
\begin{itemize}
\item 32 características de tripletes,
\item 4 características auxiliares de tripletes,
\item 23 características de la secuencia, y
\item 7 características de la estructura secundaria.
\end{itemize}
%
El cálculo de las características (a excepción de las de la secuencia)
requiere el conocimiento de la estructura secundaria de cada
ejemplo. Por ello, el método incorpora el cálculo automático de la
estructura secundaria utilizando herramientas externas, cuando no se
provee la información de estructura secundaria.
