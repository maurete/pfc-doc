\section{Extracción de caracerísticas}
Los datos de entrada vienen dados en forma de cadenas de caracteres de
longitud variable. Si bien este tipo de datos no es estrictamente
intratable por los clasificadores SVM y MLP, resulta mucho más
conveniente convertir los ejemplos a vectores numéricos de longitud
fija. De ésto se trata el proceso de extracción de características.

Las ``características'' son valores numéricos representativos de cada
ejemplo, calculados a patir de la información dada en forma de texto.
Para cada ejemplo, se calcula un ``vector de características'' que
será la entrada a la máquina de aprendizaje durante el entrenamiento y
la clasificación.

El vector de características incorpora medidas tomadas sobre la
secuencia, la estructura secundaria, y la mínima energía libre
resultante del cálculo de la estructura secundaria.  Estas 66 medidas
replican aquellas utilizadas en los trabajos previos de \cite{xue, ng,
  batuwita}.  Según la naturaleza de las características, se
distinguen cuatro ``grupos'':

\begin{itemize}
\item 32 características de tripletes,
\item 4 características auxiliares de tripletes,
\item 23 características de la secuencia, y
\item 7 características de la estructura secundaria.
\end{itemize}
El cálculo de las características (excepto las de la secuencia)
requiere el conocimiento de la estructura secundaria de cada
ejemplo. Por ello, el método soporta la invocación automática de
herramientas externas para el cálculo de la estructura secundaria
cuando ésta no viene dada.
