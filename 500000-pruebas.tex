%
%
%
%
\setcounter{chapter}{3}
%
\chapter{Pruebas}
%
El método desarrollado se evaluó mediante casos de prueba
experimentales, llamados \e{problemas}, basados en la bibliografía.
Cada problema determina un conjunto de entrenamiento y un conjunto de
prueba, que en los experimentos se utilizan respectivamente para
generar el modelo del clasificador y para evaluar su desempeño.
La evaluación de cada modelo se efectuó mediante las medidas de
sensibilidad (\SE), especificidad (\SP) y media geométrica de la
sensibilidad y la especificidad (\GM).
Se probaron las diferentes variantes de los clasificadores,
estrategias de selección de \hparam{s} y conjuntos de \caract{s},
tabulando los resultados obtenidos para cada combinación.
Adicionalmente, se efectuaron pruebas menos exhaustivas con problemas
``complementarios'', que brindan una idea del funcionamiento ante
tipos de problemas comunes.

En adelante, se describe la composición de los problemas y la
configuración de las pruebas, seguida de un análisis de los resultados
obtenidos con los diferentes clasificadores.
Al final del Capítulo, se presentan los resultados de las pruebas
efectuadas con los problemas complementarios.
