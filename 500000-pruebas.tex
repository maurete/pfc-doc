%
%
%
%
\setcounter{chapter}{3}
%
\chapter{Pruebas}
%
La evaluación del software se efectuó mediante pruebas experimentales
basadas en la bibliografía.
Cada prueba determina un conjunto de entrenamiento, que se utiliza
para la generación del modelo, y un conjunto de prueba sobre el cual
se aplica el modelo obtenido.
Los resultados de clasificación del conjunto de prueba se evaluaron
con las métricas de sensibilidad (\SE), especificidad (\SP) y media
geométrica de la sensibilidad y la especificidad (\GM).
Se probaron las diferentes variantes de los clasificadores,
estrategias de selección de \hparam{s} y conjuntos de \caract{s},
tabulando los resultados obtenidos para cada combinación.
Adicionalmente, se realizaron pruebas complementarias, menos
exhausitivas, que permiten obtener una idea del funcionamiento del
software ante tipos de problemas comunes.

En la primer parte de este Capítulo se presenta una descripción de
los diferentes casos de prueba, seguida de un análisis de los
resultados obtenidos con cada clasificador.
