%
%
%
\section{Descripción de los problemas de clasificación}
%
A partir de la bibliografía, se generaron tres \e{problemas} de
clasificación, denominados \sbs{Triplet-SVM}, \sbs{miPred} y
\sbs{microPred}, que reproducen con mayor o menor exactitud los datos
de entrenamiento y prueba utilizados por los autores de
\e{Triplet-SVM} \cite{xue}, \e{miPred} \cite{ng}, y \e{microPred}
\cite{batuwita} para las pruebas de los respectivos métodos.

En los tres problemas definidos se utilizan pre-miRNAs de la especie
humana experimentalmente validados como ejemplos de clase positiva.
Estos \premirna{s} se obtienen de diferentes versiones de la base de
datos \dataset{miRBase} \cite{mirbase1, mirbase2, mirbase3} un repositorio
en línea de referencia que es mantenido por el laboratorio
Griffiths-Jones en la Universidad de Manchester, Reino Unido.
Diferentes versiones de miRBase se publican periódicamente con nuevos
pre-miRNAs descubiertos para diferentes organismos.

Los ejemplos de clase negativa provienen principalmente de un conjunto
de datos artificial denominado \dataset{coding}.
Este conjunto de ``pseudo \premirna{s}'' fue creado por los autores de
\e{Triplet-SVM}, y contiene 8494 ejemplos negativos generados a partir
de segmentos de secuencia extraídos de regiones genómicas para las que
no se ha reportado la presencia de ningún \premirna{}.
El armado del conjunto de datos \dataset{coding} se detalla en
\cite{xue}.
