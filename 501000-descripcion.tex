%
%
%
\section{Descripción de los casos de prueba}
%
Se utilizaron tres casos de prueba principales, denominados
\e{problemas} de clasificación, que se basan en los conjuntos de
datos utilizados por los autores de \work{\tripletsvm} \cite{xue},
\work{\mipred} \cite{ng}, y \work{\micropred} \cite{batuwita} para
evaluar sus respectivos métodos.
Los problemas se denominaron \prob\tripletsvm{}, \prob{\mipred} y
\prob\micropred{}, tal como los métodos para los que han sido
creados.

En los tres problemas seleccionados, los ejemplos de clase positiva
son \premirna{s} de la especie humana, provenientes de diferentes
versiones de la base de datos \work{\mirbase}, un repositorio de
referencia de \premirna{s} experimentalmente validados mantenido por
el laboratorio Griffiths-Jones de la Universidad de Manchester, Reino
Unido \cite{mirbase1, mirbase2, mirbase3}.
Diferentes versiones de \work\mirbase{} se publican regularmente con
los nuevos \premirna{s} descubiertos para diferentes organismos.
Asimismo, los ejemplos de clase negativa provienen principalmente de
un conjunto de datos artificial denominado \dset{coding}.
Este conjunto de ``pseudo \premirna{s}'' fue creado por los autores de
\work{\tripletsvm}, y contiene $8494$ ejemplos generados con fragmentos
del genoma humano que presentan una estructura secundaria de horquilla
similar a la de los \premirna{s}, pero sin embargo son extraídos de
regiones genómicas en las que no se ha reportado la presencia de
ningún \premirna{}.
El armado del conjunto de datos \dset{coding} se detalla en \cite{xue}.
