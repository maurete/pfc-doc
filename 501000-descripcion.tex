%
%
%
\section{Descripción de los problemas de clasificación}
%
Se generaron tres {problemas} de clasificación: \prob\tripletsvm{},
\prob{\mipred} y \prob\micropred{}, que reproducen con mayor o menor
exactitud los datos de entrenamiento y prueba utilizados por los
autores de \work{\tripletsvm} \cite{xue}, \work{\mipred} \cite{ng}, y
\work{\micropred} \cite{batuwita} para las pruebas de los respectivos
métodos.

En los tres problemas definidos, los ejemplos de clase positiva
son \premirna{s} de la especie humana experimentalmente validados.
Estos ejemplos se obtienen de diferentes versiones de la base de datos
\work{\mirbase}, el repositorio de referencia de \premirna{s}
experimentalmente validados, mantenido por el laboratorio
Griffiths-Jones de la Universidad de Manchester, Reino Unido
\cite{mirbase1, mirbase2, mirbase3}.
Diferentes versiones de \work\mirbase{} se publican regularmente con
los nuevos \premirna{s} descubiertos para diferentes organismos.

Los ejemplos de clase negativa provienen principalmente de un conjunto
de datos artificial denominado \dset{coding}.
Este conjunto de ``pseudo \premirna{s}'' fue creado por los autores de
\work{\tripletsvm}, y contiene 8494 ejemplos negativos generados a
partir de segmentos de secuencia con la estructura secundaria de
horquilla característica de los \premirna{s}, pero extraídos de
regiones genómicas para las que no se ha reportado la presencia de
ningún \premirna{}.
El armado del conjunto de datos \dset{coding} se detalla en
\cite{xue}.
