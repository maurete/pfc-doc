%
%
%
\section{Descripción de los problemas}
%
Tomando como base los trabajos \cite{xue,ng,batuwita}, se definieron
tres problemas principales a partir de los conjuntos de datos
utilizados respectivamente en las pruebas de los métodos
\work{\tripletsvm}, \work{\mipred}, y \work{\micropred}.
Los problemas se denominaron \prob\tripletsvm{}, \prob{\mipred} y
\prob\micropred{}, según el nombre del trabajo en el cual se basan.

Los tres problemas contienen \premirna{s} de la especie humana
extraídos de la base de datos \dset\mirbase{} como ejemplos de clase
positiva.
\dset{\mirbase} es un repositorio de referencia en línea que contiene
ejemplos de \premirna{s} experimentalmente validados, publicado y
actualizado por el laboratorio Griffiths-Jones de la Universidad de
Manchester, Reino Unido \cite{mirbase1, mirbase2, mirbase3}.
Diferentes versiones de \work\mirbase{} se publican regularmente, con
los nuevos \premirna{s} descubiertos para diferentes organismos.
Asimismo, los ejemplos de clase negativa de los tres problemas
provienen principalmente de la base de datos \dset{coding}, un
conjunto de datos artificial creado por los autores de
\work{\tripletsvm} que contiene $8494$ ejemplos de clase negativa
llamados ``pseudo \premirna{s}'' \cite{xue}.
Estos ejemplos, generados con fragmentos del genoma humano extraídos
de regiones genómicas en las que no se ha reportado la presencia de
ningún \premirna{}, presentan una estructura secundaria con forma de
horquilla, similar a la de los \premirna{s} reales.
