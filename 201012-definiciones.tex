%
\subsubsection{Definiciones formales del aprendizaje supervisado y la
  clasificación}
%
Sea $U$ una muestra de datos compuesta por \e{ejemplos}, expresados
como pares de vectores $(\xx,\yy)$, asociando una ``entrada''
(estímulo) $\xx\in{}X$ con una ``salida'' (respuesta) correspondiente
$\yy\in{}Y$.
Se supone que los ejemplos de la muestra $U$ fueron generados por un
sistema o fenómeno expresado como una distribución de probabilidades
fija y desconcida $\nu$ sobre la transformación $X\rightarrow Y$.
La distribución $\nu$ se denomina \e{distribución generadora de los
  datos}.
El \e{aprendizaje supervisado} trata la extracción de propiedades de
$\nu$ a partir de un número finito de variables aleatorias, que
conforman el \e{conjunto de datos de entrenamiento}
%
\begin{align}
  D = \left((\B{x}_1,\yy_1),\ldots,(\B{x}_\ell,\yy_\ell)\right)
  \subseteq U,
\end{align}
%
que se considera como una muestra independiente e idénticamente
distribuida de $\nu$.
La extracción de propiedades de $\nu$ se efectúa mediante una función
$h:X\rightarrow{}Y$ denominada \e{modelo} o \e{hipótesis},
que aproxima el fenómeno real que da origen a los datos relacionando
el espacio de entrada $X$ con el de salida $Y$.

Considerando a $\C{D}$ como el espacio de todos los conjuntos de
entrenamiento posibles y a $\C{H}=\{h:X\rightarrow Y\}$ como el
conjunto de todos los modelos posibles, una \e{máquina de aprendizaje}
es una transformación
%
\begin{align}
  A:\mathcal{D}\rightarrow\C{H}
  \label{eq:maqaprendizaje}
\end{align}
%
que relaciona un conjunto de datos a un modelo.
La máquina de aprendizaje $A$ no es necesariamente una transformación
en el sentido matemático, sino que en general se trata de un algoritmo
que efectúa el proceso computacional denominado \e{entrenamiento}.

Se dice que se está ante un problema de \e{clasificación} cuando el
espacio de las salidas $Y$ es un conjunto discreto y finito, donde
cada valor posible de $Y$ se corresponde con una categoría o
\emph{clase} \cite{russell}.
El caso más básico e importante es el de la \e{clasificación binaria},
en la que se consideran sólo dos valores en $Y$, comúnmente $+1$
(``positivo'') y $-1$ (``negativo'').
Un \e{clasificador} es, entonces, el caso especial de una máquina de
aprendizaje cuando el espacio de salida $Y$ es discreto y finito.
Naturalmente, siempre que se hable de máquinas de aprendizaje se
incluyen los clasificadores dentro de esta denominación.
