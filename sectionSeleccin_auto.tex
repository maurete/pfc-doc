\section{Selección automática de hiperparámetros}
La selección automática de hiperparámetros hace referencia a la
implementación de la optimización de los hiperparámetros $\Btheta$ del
clasificador para el problema dado.  Las diferentes ``estrategias''
descritas a continuación abarcan tanto una
\hyperref[funcobjetivo]{función objetivo} que representa la
\e{aptitud} del modelo obtenido, así como un algoritmo de optimización
de dicha función.

\subsection{Selección trivial}
La selección trivial consiste en seleccionar valores preestablecidos
``razonables'' para los hiperparámetros del clasificador, que resulten
apropiados para la mayoría de los problemas.

Para el caso del Perceptrón Multicapa, la estrategia de selección
trivial consiste en seleccionar $0$ neuronas en la capa oculta, de
modo que el MLP resultante es equivalente a un perceptrón simple.

Para una Máquina de Vectores de Soporte, la estrategia selecciona el
valor $1$ para el hiperparámetro de regularización $C$ \cite{libsvm}, y
un valor $\gamma=\frac{1}{2F}$ para el hiperparámetro de amplitud del
kernel RBF, en donde $F$ es el número de características consideradas
\cite{glasmachersigel}.

Los hiperparámetros encontrados por esta estrategia no dependen del
problema de clasificación, sino que son genéricos. Por ello, no se
requiere entrenar el clasificador, y en general los valores
encontrados deben considerarse subóptimos frente a las demás
estrategias.  Así y todo, resulta conveniente su utilización para el
análisis del comportamiento de los clasificadores frente a un problema
dado.

\subsection{Búsqueda exhaustiva}
La estrategia de búsqueda exhaustiva define como función objetivo el
valor $G_m$ promedio de validación cruzada, y selecciona
hiperparámetros óptimos probando distintos valores posibles de cada
hiperparámetro en un rango preestablecido.  Dado que la naturaleza de
los hiperparámetros es diferente según el clasificador, el algoritmo
de búsqueda también difiere en cada caso.

\subsubsection{Perceptrón multicapa}
El hiperparámetro a optimizar en el caso MLP es el número de neuronas
en la capa oculta, una variable discreta no negativa.  La búsqueda
consiste en entrenar 20 clasificadores con $h$ capas ocultas, variando
$h$ entre los valores

\begin{align}
  \label{mlp-hidden-tries}
  h=(0,1,2,3,4,5,7,9,11,14,19,24,32,41,54,70,91,118,154,200).
\end{align}
Estos valores representan una escala aproximadamente logarítmica entre
0 y 200. Una vez probados todos los clasificadores con los conjuntos
de validación cruzada, se selecciona el $h$ para el cual se obtuvo el
mayor valor de $G_m$ en promedio.

\subsubsection{Máquina de vectores de soporte}
En el caso SVM, los hiperparámetros a optimizar son las variables
reales no negativas $C$ y $\gamma$--sólo en caso de utilizar un núcleo
RBF. En la práctica, la búsqueda se efectúa en el espacio logarítmico
$\log{}C$, $\log\gamma$, eliminando el requerimiento de no negatividad
de las variables.

El algoritmo de búsqueda se basa en la técnica de \e{búsqueda en la
  grilla} propuesta por \citeauthor{hsu} en \cite{hsu}.  La idea
básica es considerar la combinación de hiperparámetros $(\log
C,\log\gamma)$ como puntos en el plano.  Seleccionando una serie de
puntos espaciados regularmente en el plano $\log-C\gamma$ (la
``grilla''), se entrena y evalúa el clasificador sobre cada punto,
interpolando la grilla en las cercanías de los puntos donde se obtiene
el mayor rendimiento para continuar la búsqueda.

En primer lugar, se definen los puntos de muestreo para los
hiperparámetros

\begin{align*}
  \log_2 C &= -5, -3, -1, 1, \ldots, 15, & \log_2\gamma &= -15,-13,
  -11, \ldots,3,
\end{align*}
que definen la grilla inicial. Para cada punto $(C_i,\gamma_j)$, se
entrena y prueba un clasificador SVM, obteniendo un ${G_m}_{ij}$
promedio de validación cruzada.

En etapas sucesivas, se interpola la grilla alrededor de los puntos
donde se obtuvieron los mayores valores $G_m$. Se definen tres algoritmos
heurísticos que determinan cuáles puntos interpolar:

\begin{itemize}
  %
\item \e{Zoom}: En un primer paso, se convoluciona la grilla con una
  ventana cuadrada uniforme de valor unitario, obteniendo una versión
  ``suavizada'' de la grilla. Se interpolan puntos en una región
  cuadrada centrada en el punto con mayor valor $G_m$ ``suavizado'',
  con precisión (refinamiento) y amplitud especificados por el
  usuario, por defecto, se utiliza una precisión de $0.5$ la precisión
  actual, y una amplitud de $0.5$ la amplitud actual en cada
  dimensión.
\item \e{Umbral}: A partir de un umbral $G_m$ definido por el usuario,
  por defecto el percentil 90, se interpola en cada dimensión al $0.5$
  de la precisión actual alrededor de los puntos por encima del
  umbral.
\item \e{$n$-mejores}: Similar al umbral, se seleccionan $n$ puntos
  con los mayores valores de $G_m$. Se interpola la grilla en ambas
  dimensiones al $0.5$ de la precisión actual alrededor de estos
  puntos.
\end{itemize}
Una vez interpolada la grilla, se calcula el valor $G_m$ promedio de
validación cruzada para los nuevos puntos. El procedimiento de
refinamiento se repite hasta un máximo de $N$ iteraciones o hasta
satisfacer un criterio de corte.

La búsqueda en la grilla tiene la ventaja de ser conceptualmente
simple, sin embargo, se torna inviable cuando el vector de parámetros
$\B{\theta}$ contiene más de 2 elementos. Dado que se trata de un
método de búsqueda exhaustiva, resulta generalmente lento.  Cuando se
trabaja con un clasificador SVM con núcleo lineal, la grilla en
cuestión tiene una única dimensión: la del hiperparámetro $\log C$.

\subsection{Minimización del error empírico}
La estrategia de minimización del error empírico es aplicable al
clasificador SVM, y optimiza mediante descenso de gradiente una
función objetivo denominada \e{error empírico} que se define a partir
de una interpretación probabilística del modelo SVM.

La función error empírico fue propuesta por \citeauthor{ayat} en
\cite{ayat} como una función objetivo continua y derivable respecto a
los hiperparámetros del núcleo.  La estrategia aquí presentada
incorpora además el cálculo del gradiente respecto del hiperparámetro
$C$ \cite{keerthi, glasmachers}, implementado en \cite{shark}, lo que
permite optimizar todos los hiperparámetros de la Máquina de Vectores
de Soporte, incluyendo la regularización $C$, siempre que el núcleo
sea derivable respecto de sus propios hiperparámetros.

\begin{quote}
  \sbs{Notación.}
  El lector experto encontrará que la denominación ``error empírico''
  resulta confusa, ya que en la disciplina este nombre se utiliza como
  equivalente de ``error de entrenamiento''. En el presente trabajo se
  ha decidido mantener esta denominación de los autores \cite{ayat},
  efectuando el mayor esfuerzo por distinguir entre ``error de
  entrenamiento'' y la función objetivo ``error empírico'', que se
  trata más precisamente de una interpretación probabilística del
  error de validación cruzada.
  %% Se advierte al lector experto en la materia que, en toda la
  %% extensión del presente trabajo, la denominación ``error empírico''
  %% hace referencia únicamente a la función objetivo aquí definida,
  %% utilizándose el término ``error de entrenamiento'' para referir al
  %% concepto de otro modo conocido como ``riesgo empírico'' en la
  %% especialidad.
\end{quote}
La función objetivo del error empírico se basa en la definición del
error del clasificador sobre un conjunto de prueba $T$, el cual es
un estimador
no sesgado del error de generalización.  Considérese la definición del
error de prueba sobre un conjunto $T=((\xx_k,y_k)),\,n=1,\ldots,N$ de
clase conocida (\autoref{eq:error-prueba}).  Este error puede
escribirse en la forma equivalente

\begin{align*}
\label{E^V}
  E^T = \frac{1}{N}\sum_{k=1}^{N} H(-{y}_k {c}_k))
\end{align*}
donde $c_k=h(\xx_k)$ es la predicción de clase del modelo SVM
entrenado, y $H(\cdot)$ es la función escalón de Heaviside.  Fijando
el tipo de núcleo utilizado y los conjuntos de datos de entrenamiento
y validación, se puede considerar

\begin{align*}
  E^T = E^T(\theta),
\end{align*}
y consecuentemente, minimizar $E^T(\Btheta)$ equivale a seleccionar
los hiperparámetros óptimos $\Btheta$. La función $E^T$ así definida
es una función objetivo válida para la comparación de modelos, sin
embargo, por definición es una función discontinua, lo que
imposibilita la optimización eficiente mediante descenso de gradiente.

Una interpretación alternativa del concepto de error, basada en la
teoría bayesiana, permite desarrollar una función objetivo con
características deseables de continuidad y derivabilidad.  Si $p_k$ es
la \e{probabilidad a posteriori} de que el ejemplo $\xx_k$ pertenezca
a la clase positiva

\begin{align}
  p_k = p(x_k) = P(c_k=+1|x_k), % = z_k
\end{align}
se puede caracterizar la \e{probabilidad de error} $E_k$ cometido al
clasificar el ejemplo $\xx_k$ según

\begin{align}
\label{Ek}
  E_k = P(c_k\neq y_k) = |t_k-{p}_k| =
  \begin{cases}
    {p}_k, & t_k=0\\ 1-{p}_k, & t_k = 1,
  \end{cases}
\end{align}
donde $t_k=\frac{y_k+1}{2}$ es un ``valor deseado'' calculado a partir
de la clase conocida $y_k$.  Entonces, la probabilidad de error para
el conjunto $T$ puede escribirse

\begin{align}
\label{Err1}
  E = \sum_{k=1}^{N} E_k.
\end{align}
La función $E$ especifica el denominado \e{error empírico} y es la
función objetivo a minimizar.  Para su cálculo, sin embargo, se
requiere el conocimiento de la probabilidad $p_k$, información que el
modelo SVM no brinda, al tratarse de una salida binaria.  Sin embargo,
resulta posible calcular una probabilidad estimada $\hat{p}_k$, que
permite su utilización en la práctica.

\subsubsection{Estimación de la probabilidad ${p}_k$}
El modelo $h$ de una máquina de vectores de soporte
(\autoref{eq:svm-model-hard}) puede escribirse

\begin{align*}
  h(\xx_k) = \T{signo}(f_k), \qquad\qquad
  f_k=\langle\ww,\BPhi(\xx_k)\rangle+b.
\end{align*}
Utilizando el valor de $f_k$, la salida del modelo
previa a aplicar la función signo, se puede calcular una probabilidad
estimada $\hat{p}_k$ según

\begin{align}
  \hat{p}_k=\frac{1}{1+e^{Af_k+B}} \qquad \approx p_k.
\end{align}
Los parámetros $A$ y $B$ se determinan optimizando el problema

\begin{align}
  \arg\min_{A,B} \quad & -\sum_{k=1}^{N} \log(E_k)
  \label{abproblem}
\end{align}
sobre el conjunto de prueba $T$. La optimización de este sub-problema
se realiza mediante una técnica de descenso de gradiente, ya que las
derivadas de $\hat{p}_k$ respecto de sus parámetros $A$ y $B$ son
fácilmente calculables:

\begin{align*}
  \dpar{\hat{p}_k}{A}{}&=-f_k e^{Af_k+B}\frac{1}{(1+e^{Af_k+B})^2}
  =-f_k\hat{p}_k(1-\hat{p}_k),&\\
  \dpar{\hat{p}_k}{B}{}&=    -e^{Af_k+B}\frac{1}{(1+e^{Af_k+B})^2}
  =-   \hat{p}_k(1-\hat{p}_k).
\end{align*}
Esta técnica para obtener un estimador $p_k$ de la probabiidad a
posteriori real fue presentada por \citeauthor{platt} en \cite{platt}.
La elección de una función logística como $\hat{p}_k$ se basa en la
presunción de que las salidas $f_k$ para las entradas $\{\xx_k\}$ de
clase positiva tienen una distribución gaussiana en $f$.

Una interpretación intuitiva de $\hat{p}_k$ es que, para una salida
$f_k$ de gran magnitud que se ubica lejos del hiperplano de separación
en el espacio inducido por el núcleo, el clasificador SVM tiene amplia
certeza de su decisión.  Asimismo, se tendrá que cuando el valor de
$f_k=0$, $x_k$ recae exactamente en el plano de separación y se
obtiene la peor predicción posible $\hat{p}_k=0.5$\footnote{La
  condición $\hat{p}_k=0.5$ cuando $x_k$ se ubica sobre el plano de
  separación implica $B=0$, sin embargo, se considera igualmente $B$
  como un valor a optimizar, respetando el método original
  \cite{platt}.} y, similarmente, cuanto más ancho sea el margen de
separación, más suave deberá ser la pendiente de $\hat{p}_k$.

\subsubsection{Cálculo del gradiente $\grad{E}$}
Sustituyendo la probabilidad real $p_k$ por el estimador
$\hat{p}_k$, y calculando el gradiente, se tiene

\begin{align}
\label{gradE}
  \grad{E} = \sum_{k=1}^{N} \grad{E_k} =
  \sum_{ i:t_i=0  } \grad{\hat{p}_i}
  - \sum_{ j:t_j=1  } \grad{\hat{p}_j}
  = \sum_{k=1}^{N} y_k \grad{\hat{p}_k}
\end{align}
Por la regla de composición de las derivadas, la derivada
$\grad{\hat{p}_k}=\dpar{\hat{p}_k}{\theta_j}{}$ respecto de un
hiperparámetro ${\theta_j}$ es

\begin{align}
  \dpar{\hat{p}_k}{\theta_j}{} =
  \dpar{\hat{p}_k}{f_k}{}\dpar{f_k}{\theta_j}{} =
  -A\hat{p}_k(1-\hat{p}_k)\dpar{f_k}{\theta_j}{}.
  \label{eq:deriv-pk-thetak}
\end{align}
Para la determinación de $\dpar{f_k}{\theta_j}{}$ se sigue el
procedimiento propuesto en \cite{keerthi, glasmachers} e implementado
en \cite{shark}.
En primer lugar, se observa que $f_k$ puede calcularse según

\begin{align}
  f_k = \langle w,\Phi(x_k)\rangle+b = \sum_{i=1}^l \alpha_i y_i k(x_i,x_k) + b
  \label{fk}
\end{align}
donde $k(\cdot,\cdot)$ es la función núcleo, y $\alpha_i, b$ son los
parámetros del modelo SVM entrenado $h$.

Según sea el valor de $\alpha_i$ correspondiente al vector de
entrenamiento $\xx_i$, se definen los conjuntos de índices

\begin{align}
  \label{unbounded-sv-set}
  u &= \left\{i\in\{1,\ldots,l\}:0<y_i\alpha_i<C \right\}\\
  \label{bounded-sv-set}
  g &= \left\{i\in\{1,\ldots,l\}: y_i\alpha_i=C \right\}\\
  n &= \left\{i\in\{1,\ldots,l\}: \alpha_i=0 \right\}.
\end{align}
Mediante interpretación geométrica de la solución al problema de la
SVM, se sabe que para aquellos vectores $\xx_i$ que no son de soporte,
se cumple que $\alpha_i=0$, luego las derivadas $\alpha_n$ respecto de los
hiperparámetros son nulas. Para el caso $\alpha_i=\pm{}C$, se tiene

\begin{align}
  \dpar{\alpha_g}{C}{} = y_g, && \dpar{\alpha_g}{{\theta}^K_j}{} = 0,
\end{align}
ya que el valor de $\alpha_g$ viene limitado (en valor absoluto) por el
parámetro $C$. Para simplificar el cálculo de las derivadas de
$\alpha_u$ y $b$ se plantea un problema en forma matricial. Sea la
\emph{matriz del núcleo} $K$

\begin{align}
  K = \begin{pmatrix} k(x_1,x_1) & k(x_1,x_2) & \cdots & k(x_1,x_l)
    \\ k(x_2,x_1) & k(x_2,x_2) & \cdots & k(x_2,x_l) \\ \vdots &
    \vdots & \ddots & \vdots \\ k(x_l,x_1) & k(x_l,x_2) & \cdots &
    k(x_l,x_l)
  \end{pmatrix}
  =
  \begin{pmatrix}
    (K_{\uu\uu}) & (K_{\uu\Bg}) & (K_{\uu\nn}) \\
    (K_{\Bg\uu}) & (K_{\Bg\Bg}) & (K_{\Bg\nn}) \\
    (K_{\nn\uu}) & (K_{\nn\Bg}) & (K_{\nn\nn})
  \end{pmatrix}
\end{align}
donde en la matriz de la derecha los elementos de $K$ han sido
reordenados en submatrices $K_{uu},K_{ug},\ldots$ según los conjuntos
de índices $u, g, n$ definidos anteriormente.  Sea la matriz

\begin{align}
  H=\begin{pmatrix} K_{uu} & \B{1}_u \\ \B{1}_u^T & 0
  \end{pmatrix}
\end{align}
con $\B{1}_u$ un vector columna de $|u|$ elementos iguales a 1,
entonces la derivada de $(\B{\alpha}_u,b)^T$ respecto del
hiperparámetro $j$ del núcleo $\theta^K_j$ viene dada por

\begin{align}
  \dpar{}{\theta^K_j}{} \begin{pmatrix}\alpha_{u}\\b\end{pmatrix} &=
    -H^{-1} \left[
      \dpar{H}{\theta^K_j}{}
      \begin{pmatrix}\alpha_{u}\\b\end{pmatrix}
        +C \begin{pmatrix}\dpar{K_{gu}}{\theta^K_j}{}\\0\end{pmatrix}
          y_g
          \right], \\
  \dpar{}{C}{} \begin{pmatrix}\alpha_{u}\\b\end{pmatrix} &=
    -H^{-1} \left[
      \begin{pmatrix}{K_{gu}}\\\B{1_g^T}\end{pmatrix} y_g
      \right],
\end{align}
donde $\B{1}_g$ un vector columna de $|g|$ elementos iguales a 1 y
$\B{y}_g$ el vector de clases correspondientes a los elementos en
$g$. Para los detalles de este resultado se refiere al lector a
\cite{glasmachers} y \cite{keerthi}.

Una vez obtenidas las derivadas de $({\B{\alpha},b})^T$ respecto de
los hiperparámetros $\B{\theta}$, la derivada de $f_k$  respecto
de los mismos viene dada por

\begin{align}
  \dpar{f_k}{C}{} &=  \sum_{i=1}^l y_i\left[
    \dpar{\alpha_i}{C}{} k(x_i,x_k) \right]
  + \dpar{b}{C}{} ,\\
  \dpar{f_k}{\theta^K_j}{} &=  \sum_{i=1}^l y_i \left[
    \dpar{\alpha_i}{\theta^K_j}{} k(x_i,x_k) +
    \dpar{k(x_i,x_k)}{\theta^K_j}{} \alpha_i \right]
  + \dpar{b}{\theta^K_j}{}. 
\end{align}
Con este resultado, resulta posible el cálculo de la derivada de la
probabilidad estimada $\dpar{\hat{p}_k}{\theta_j}{}$
(\autoref{eq:deriv-px-thetak}), y con ella la del gradiente
$\nabla{}E$ (\autoref{gradE}).

\subsubsection{Algoritmo de optimización}
Dado el conjunto de entrenamiento $D$, en cada punto de evaluación
$\Btheta$ el error empírico $E(\Btheta)$ se calcula del siguiente modo

\begin{enumerate}
\item Aplicando validación cruzada, entrenar un modelo SVM para cada
  conjunto de estimación.
\item Calcular las salidas $f_k$ (\autoref{fk}) de los
  modelos para todos los ejemplos $\xx_k$ en las respectivas
  particiones de validación.
\item Encontrar valores óptimos de los parámetros $A$ y $B$, partiendo
  de valores iniciales $A_0=1$, $B_0=0$ (\autoref{abproblem}).
\item Calcular las probabilidades estimadas $\hat{p}_k$ y con ellas,
  el error empírico de cada ejemplo $E_k$ y global $E$.
\item Calcular el gradiente $\nabla{}E$.
\end{enumerate}
Partiendo de un punto inicial $\Btheta^0$, la búsqueda procede en cada
punto $\Btheta^i$ evaluando la función objetivo y determinando un
nuevo punto $\Btheta^{i+1}$ a partir del valor $E^i=E(\Btheta^i)$ y la
información disponible en el gradiente $\nabla{}E^i$ hasta satisfacer
algún criterio de corte, por ejemplo

\begin{align*}
  |E^i-E^{i-1}|&<\eta, & \|\nabla E^i\| < \nu,
\end{align*}
donde $\eta$ y $\nu$ son números pequeños.  Este esquema es
implementado en la estrategia de minimización del error empírico
mediante el algoritmo de optimización BFGS \cite{bfgs}.

\subsection{Minimización de la cota radio-margen}
La cota ``radio-margen'' es una función con la propiedad de ser un
límite superior al error de validación cruzada dejando uno fuera, y se
calcula directamente a partir de la información disponible en el
modelo de una máquina de vectores de soporte.  La función
``radio-margen'' propuesta originalmente por \citeauthor{vapnik}
\cite{vapnik} para un modelo SVM  con término de regularización $L2$
(\autoref{svm-l2}), viene dada por % vapnik sec. 10.7, pag 441

\begin{align}
  \T{RM} = 4R^2 \|w\|^2.
\end{align}
Aquí, $R$ es el radio de la hiperesfera que, en el espacio inducido
por el núcleo, engloba todos los vectores de soporte, y $\|\ww\|$ es
la norma del vector normal al hiperplano de separación, que define el
margen del modelo $h$ de la SVM.  Para esta cota, se cumple
desigualdad

\begin{align}
  E_{\T{LOO}} \leq \T{RM},
\end{align}
donde $E_{\T{LOO}}$ es el error de validación cruzada dejando uno
fuera.

El error de validación cruzada dejando uno fuera $E_{\T{LOO}}$ es uno
de los mejores estimadores del error de generalización disponibles.
Por ello, la idea de minimizar una cota de este error, que es a la vez
es continua, derivable y fácilmente calculable, resulta muy atractiva.
De hecho, en \cite{chapelle} se presenta un algoritmo de minimización
de la cota $\T{RM}$ mediante descenso por gradiente en el espacio de
los hiperparámetros.  Sin embargo, dado que la mayoría de SVMs
utilizadas en la práctica incorporan regularización $L1$ en lugar de
$L2$, la utilidad real de minimizar la cota $\T{RM}$ es limitada.

En \cite{chung} se proponen varias funciones ``alternativas'',
inspiradas en la cota radio-margen $\T{RM}$, y aplicables al caso
específico del núcleo gaussiano (RBF) y formulación $L1$. La
estrategia aquí propuesta utiliza una de estas cotas
alternativas, denotada $\rho$ y definida según

\begin{align}
  \rho = \rho_R \cdot \rho_M,
\end{align}
donde $\rho_R$, $\rho_M$ son los factores ``radio'' y ``margen''

\begin{align}
  \rho_R &= R^2+\frac{1}{C}, \\
  \rho_M &= \|\ww\|^2+2C\sum\xi_i.
\end{align}
A diferencia de $\T{RM}$, la función $\rho$ ajusta el valor
real de $E_{\T{LOO}}$ con demasiada holgura, con lo que su valor
pierde el significado original de representar una tasa de error. Sin
embargo, tiene la importante propiedad de poseer un mínimo global en
las cercanías de los hiperparámetros óptimos $(C,\gamma)$ que
minimizan la tasa de error $E_{\T{LOO}}$.

\subsubsection{Cálculo de $\rho$}
El valor $R^2$, necesario para el cálculo de $\rho_R$, viene dado por

\begin{align}
  R^2 = 1 - \Bbeta_*^T \KK \Bbeta_*,
\end{align}
donde $\Bbeta_*$ es la solución al problema de optimización conocido
como ``SVM de una clase'' \cite{scholkopf}

\begin{align}
\begin{split}
  \arg\min_{\Bbeta}\quad&\Bbeta^T \KK \Bbeta,\\
  \T{sujeto a}    \quad&0\leq\beta_i\leq{}1,\quad{}i=1,\ldots,\ell,\\
                       &\B{1}^T_\ell\,\Bbeta=1.
  \end{split}
  \label{svm-oneclass}
\end{align}
Aquí, $\B{1}_{\ell}$ es un vector columna con $\ell$ elementos iguales
a 1, y $\KK$ es la matriz del núcleo con elementos
$k_{ij}=k(\xx_i,\xx_j)$. La matriz $\KK$ es definida positiva siempre
que se cumpla la condición

\begin{align}
\label{cond-kmatrix-defpos}
  i\neq j \iff \xx_i\neq\xx_j,\quad \forall\,i,j\in{1,\ldots,\ell},
\end{align}
esto es, siempre que no haya ejemplos repetidos en el conjunto de
entrenamiento. Sin perder mucha generalidad, se considera que tal es
el caso, y con ello, se asegura que el problema \cite{svm-oneclass}
tiene solución única \cite{SOL-UNICA-DEFINIDA-POSITIVA}.

El valor ``margen'' $\rho_M$ es dos veces la solución al problema de
optimización de la SVM (\autoref{svm-primal-blando}).  Dada la
equivalencia primal-dual de la solución, si $\Balpha$ es solución a
la forma dual (\autoref{svmprob-dual-soft}), se tiene simplemente

\begin{align}
\label{prieqdual}
  \rho_M &= 2\left(\frac{1}{2}\|\ww\|^2+C\sum\xi_i \right)
  =2\left(\B{e}^T\Balpha-\frac{1}{2}\Balpha^T\B{Q}\Balpha\right),
\end{align}
donde $\QQ$ es la matriz con elementos $Q_{ij}=y_iy_jk(\xx_i,\xx_j)$.

\subsubsection{Cáclulo del gradiente $\nabla\rho$}
El cálculo de las derivadas de $\rho$ se basa en resultados de
análisis de perturbación de problemas de optimización, ya que tanto
$\rho_R$ como $\rho_M$ son soluciones a problemas de este tipo.  La
discusión en este apartado se basa en aquella presentada por
\citeauthor{chung} en \cite{chung}.

El Teorema 4.1 de \cite{bonnans-shapiro} establece:

\begin{quote}
  Considérese un problema de optimización en la forma
  $f_*(\vv)=\min_{\uu\in{}U}f(\uu,\vv)$ con vector $\uu$ de variables a
  optimizar, vector de parámetros $\vv$, y $U$ independiente de
  $\vv$. Entonces, si para cada $\vv$ existe un $\uu_*(\vv)\in{}U$
  tal que $f_*(\vv)=f(\uu_*(\vv),\vv)$, la derivada
  $\dpar{f_*}{\vv}{}(\vv)$ existe y $\dpar{f_*}{v_i}{}$ viene dado por
  $\dpar{f_*}{v_i}{}(\uu_*,\vv)$ para cualquier parámetro $v_i$.
\end{quote}

Como se ha visto anteriormente, exigiendo la no-repetición de ejemplos
en el conjunto de entrenamiento (\ref{cond-kmatrix-defpos}) se
garantiza la existencia de la solución $\uu_*$.
El problema (\ref{svm-oneclass}) cumple con los requisitos del teorema,
luego las derivadas de $R^2$ respecto de los hiperparámetros vienen
dadas por

\begin{align}
  \dpar{R^2}{C}{} &= \dpar{(1-\Bbeta^T\KK\Bbeta)}{C}{}
  = -\Bbeta^T \dpar{\KK}{C}{}\Bbeta = 0, \\
  \dpar{R^2}{\gamma}{} &= \dpar{(1-\Bbeta^T\KK\Bbeta)}{\gamma}{}
  = -\Bbeta^T \dpar{\KK}{\gamma}{}\Bbeta.
\end{align}
En el caso $\rho_M$ el teorema anterior no aplica, ya que

\begin{enumerate}
\item La restricción $\alpha_i\leq{}C$ depende de $C$, violando la condición
  necesaria ``$U$ independiente de $\vv$''.
\item El vector $\Bxi$ óptimo puede no ser único ya que el parámetro $b$ óptimo
  puede no ser único cuando el modelo no contiene vectores de soporte
  ``libres'' (esto es, cuando $u=\emptyset$ (\ref{unbounded-sv-set})).
\end{enumerate}
Para salvar estos inconvenientes, se aprovecha la dualidad del
problema en el punto óptimo $\Balpha_*$, y se efectúa el cambio de
variable $\bar{\Balpha}=\Balpha/C$ en el problema dual
(\ref{svmprob-dual-soft}) de modo que las restricciones sean
independientes del hiperparámetro $C$

\begin{align}
\begin{split}
    \max_{\bar{\Balpha}}\quad&
    f(\bar{\Balpha}) = C^2 \left( \frac{\B{1}^T\bar{\Balpha}}{C}
    -\frac{1}{2}\bar{\Balpha}^T\QQ\bar{\Balpha}\right)\\
    \T{sujeto a}\quad & \yy^T\bar{\Balpha} = 0, \\
    & 0\leq\bar{\alpha}_i\leq 1,
    \T{ para todo } i\in {1,\ldots,\ell }.
\end{split}\end{align}
Este problema cumple con las condiciones del teorema ya que
no depende de $\Bxi$ y cuenta con solución única
ya que $Q$ es definida positiva (\ref{cond-kmatrix-defpos}). Entonces,

\begin{align}
\label{eq:rmb-alpha-equiv}
  \B{1}^T\Balpha-\frac{1}{2}\Balpha^T\QQ\Balpha
  &= C^2\left(\frac{\B{1}^T\bar{\Balpha}}{C} -
  \frac{1}{2}\bar{\Balpha}^T\QQ\bar{\Balpha}\right).
\end{align}
Con estos resultados se está en condiciones de calcular las derivadas de
$\rho$ respecto de los hiperparámetros $C$ y $\gamma$:
Las derivadas respecto a $C$ vienen dadas por

\begin{align}
  \begin{aligned}
    \dpar{\rho_M}{C}{}
    &= \dpar{}{C}{}2\left(  \B{1}^T\Balpha-\frac{1}{2}\Balpha^T\QQ\Balpha \right) \\
    &= \dpar{}{C}{}\left( 2C^2\left(\frac{\B{1}^T\bar{\Balpha}}{C} -
    \frac{1}{2}\bar{\Balpha}^T\QQ\bar{\Balpha}\right)
    \right) \\
    &= 4C \left(\frac{\B{1}^T\bar{\Balpha}}{C} -
    \frac{1}{2}\bar{\Balpha}^T\QQ\bar{\Balpha}\right) - 2C^2 \left(\frac{\B{1}^T\bar{\Balpha}}{C^2} \right) \\
    &= 2\left(\B{1}^T\bar{\Balpha} - C \bar{\Balpha}^T\QQ\bar{\Balpha}\right) \\
    &= \frac{2}{C} \left(\B{1}^T\Balpha - \Balpha^T\QQ\Balpha\right)
  \end{aligned}
  \\
  \begin{aligned}
    \dpar{\rho_M}{C}{}
    &= \dpar{}{C}{} \left( R^2 + \frac{1}{C} \right) \\
    &= -\frac{1}{C^2}.
  \end{aligned}
\end{align}

Las derivadas respecto del hiperparámetro $\gamma$ vienen dadas por

\begin{align}
  \begin{aligned}
    \dpar{\rho_M}{\gamma}{}
    &= \dpar{}{\gamma}{} 2\left(  \B{1}^T\Balpha-\frac{1}{2}\Balpha^T\QQ\Balpha \right) \\
    &= - \Balpha^T \dpar{\QQ}{\gamma}{}\Balpha \\
    & = - \Balpha^T \left(\yy^T \dpar{\KK}{\gamma}{}\yy\right) \Balpha
    %% \\
    %% & = \sum_{i,j=1}^\ell \alpha_i\alpha_j y_i y_j \dpar{k(\xx_i,\xx_j)}{\gamma}{}, \\[0.2em]
  \end{aligned}
  \begin{aligned}
    \dpar{\rho_R}{\gamma}{} &= \dpar{R^2}{\gamma}{} \\
    = - \Bbeta^T \dpar{\KK}{\gamma}{} \Bbeta. % \\
%    &= \sum_{i,j=1}^\ell \beta_i\beta_j \dpar{k(\xx_i,\xx_j)}{\gamma}{}.
  \end{aligned}
\end{align}
Los elementos $\dpar{k_{ij}}{\gamma}{}$ de la matriz $\dpar{\KK}{\gamma}{}$ 
vienen ddados por

\begin{align}
  \dpar{k_{ij}}{\gamma}{}
  = \dpar{}{\gamma}{}k(\xx_i,\xx_j)
  = \dpar{}{\gamma}{} \left(e^{-\gamma\|\xx_i-\xx_j\|}\right)
  = -k_{ij}\|\xx_i-\xx_j\|
\end{align}
Con todo esto, las derivadas de la función $\rho$ respecto de los
hiperparámetros vienen dadas por

\begin{align*}
    \dpar{\rho}{C}{} &= \dpar{\rho_M}{C}{} \rho_R + \rho_M \dpar{\rho_R}{C}{} \\
    &= \frac{2}{C} \left(\B{1}^T\Balpha - \Balpha^T\QQ\Balpha\right) \left( R^2 + \frac{1}{C} \right)
    - 2\left(  \B{1}^T\Balpha-\frac{1}{2}\Balpha^T\QQ\Balpha \right)
    \left( \frac{1}{C^2} \right)\\
    &=
  \\[2em]
    \dpar{\rho}{\gamma}{} &= \dpar{\rho_M}{\gamma}{} \rho_R + \rho_M \dpar{\rho_R}{\gamma}{}\\
    &= \left( - \Balpha^T \left(\yy^T \dpar{\KK}{\gamma}{}\yy\right) \Balpha \right)
    \left( R^2 + \frac{1}{C} \right)
    - 2\left(  \B{1}^T\Balpha-\frac{1}{2}\Balpha^T\QQ\Balpha \right)
    \left( \Bbeta^T \dpar{\KK}{\gamma}{} \Bbeta \right)
\end{align*}
donde

\begin{align}
  k_{ij}&=k(x_i,x_j)=e^{-\gamma||x_i-x_j||},\,\, \T{y} \\
  \dpar{k_{ij}}{\gamma}{}&=-k_{ij}||x_i-x_j||
\end{align}
son la función kernel RBF y su derivada respecto del hiperparámetro $\gamma$,
respectivamente.

\subsubsection{Algoritmo de optimización}
Partiendo de un punto inicial $(C^0,\gamma^0)$ la búsqueda procede
en cada punto $(C^k,\gamma^k)$ evaluando $\rho^k$ y determinando
un nuevo punto $(C^{k+1},\gamma^{k+1})$ en la dirección del
negativo del gradiente $\nabla\rho^k$ tal que $\rho^{k+1}<\rho^k$
La optimización de la función $\rho$ se efectúa mediante el algoritmo
BFGS \cite{bfgs} en el espacio logarítmico de los hiperparámetros
$(\ln(C),\ln(\gamma))$. Este cambio de coordenadas se traduce en un
incremento de la estabilidad numérica y evita tener que verificar en
cada iteración la no-negatividad de $C$ y $\gamma$.

\begin{align}
  \dpar{\rho}{\ln C}{}= C \dpar{\rho}{C}{}, &&
  \dpar{\rho}{\ln \gamma}{}= \gamma \dpar{\rho}{\gamma}{}
\end{align}
Para un conjunto de entrenamiento $D=((\xx_i,y_i),i=1,\ldots,\ell)$
cada evaluación de $\rho(\Btheta_k)$ consta de los siguientes pasos

\begin{enumerate}
\item Entrenar una máquina de vectores de soporte con núcleo RBF
  con hiperparámetros $\Btheta_k=(C_k,\gamma_k)$ sobre el conjunto
  de entrenamiento completo $D$
\item Calcular $\Bbeta_*$ óptimo para el problema (\ref{svm-oneclass})
\item Calcular el valor de $\rho$ como el producto de
  $\rho_M$ (\ref{rho_m}) y $\rho_r$ según (\ref{rho_r})
\item Calcular el gradiente $\nabla\rho$.
\end{enumerate}