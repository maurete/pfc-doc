%
\subsubsection{Cálculo de $\rho$}
%
El valor $R^2$, necesario para el cálculo de $\rho_R$, viene dado por
%
\begin{align}
  R^2 = 1 - \Bbeta_*^T \KK \Bbeta_*,
\end{align}
%
donde $\Bbeta_*$ es la solución al problema de optimización conocido
como ``SVM de una clase'' \cite{scholkopf}:
%
\begin{align}
\begin{split}
  \arg\min_{\Bbeta}\quad&\Bbeta^T \KK \Bbeta,\\
  \T{sujeto a}    \quad&0\leq\beta_i\leq{}1,\quad{}i=1,\ldots,\ell,\\
                       &\B{1}^T_\ell\,\Bbeta=1.
  \end{split}
  \label{svm-oneclass}
\end{align}
%
Aquí, $\B{1}_{\ell}$ es un vector columna con $\ell$ elementos iguales
a 1, y $\KK$ es la matriz del núcleo con elementos
$k_{ij}=k(\xx_i,\xx_j)$. El Problema (\ref{svm-oneclass}) se resuelve
mediante un algoritmo de optimización estándar, y su solución es única
siempre que no haya elementos repetidos en el conjunto de entrenamiento
\cite{chung}.

El valor ``margen'' $\rho_M$ es dos veces la solución al problema de
optimización de la SVM (\autoref{svm-primal-blando}). La
equivalencia primal-dual de la solución establece que, si $\Balpha$ es solución a
la forma dual (\autoref{svmprob-dual-soft}), se tiene simplemente
%
\begin{align}
\label{prieqdual}
  \rho_M &= 2\left(\frac{1}{2}\|\ww\|^2+C\sum\xi_i \right)
  =2\left(\B{e}^T\Balpha-\frac{1}{2}\Balpha^T\B{Q}\Balpha\right),
\end{align}
%
donde $\QQ$ es la matriz con elementos $Q_{ij}=y_iy_jk(\xx_i,\xx_j)$.
