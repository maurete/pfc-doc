%
\subsubsection{Parámetros del modelo y entrenamiento}
%
Comúnmente, el modelo $h$ deriva de una función más general $H(\xx,\B{p})$, en donde
$\B{p}$ es un conjunto de \e{parámetros} que modifican la respuesta (salida)
de $H$ ante determinados estímulos (entrada) $\xx$.

Se llama \e{entrenamiento} al proceso de selección de los parámetros $\B{p}$
a partir del conjunto de datos $D$, obteniendo un modelo $h(\xx)=H(\xx,\B{p})$
que ``ajusta'' los datos minimizando una medida de error $E=f(\xx,h(\xx))$.
