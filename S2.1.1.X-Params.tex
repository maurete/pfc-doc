%
\subsubsection{Parámetros del modelo}
%
Comúnmente, el modelo $h$ deriva de una función más general
$H(\xx,\B{p})$, en donde $\B{p}$ es un conjunto de \e{parámetros} que
modifican la respuesta (salida) de $H$ ante determinados estímulos
(entrada) $\xx$.
Entendido desde esta perspectiva, el {entrenamiento} es el proceso de
selección de los parámetros $\B{p}$ a partir del conjunto de datos
$D$, obteniendo un modelo $h(\xx)=H(\xx,\B{p})$ que ``ajusta'' los
datos en $D$ minimizando una medida de error $E=f(\xx,h(\xx))$.
