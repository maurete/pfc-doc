\section{Actualización de los repositorios}

Como parte de las tareas de adecuación de la infraestructura, se
actualizaron el repositorio de código GitLab y el repositorio de
artefactos Nexus. Ninguno de los dos servicios contaba con la
configuración escrita como código Ansible, ya que su implementación
fue anterior a la adopción de esta herramienta.

\subsection{Repositorio de código GitLab}

Se decidió actualizar el repositorio de código GitLab para corregir
vulnerabilidades y aprovechar las nuevas características ofrecidas por
sus desarrolladores. Como parte de estas actualización se implementó
además el código Ansible necesario para gestionar la infraestructura
del servicio. Este código permitió la configuración del servicio de
\e{runners} que es utilizado para la ejecución de los procesos
automáticos de integración y entrega continuas.

Se actualizó el software desde la versión 9.4 hasta la 13.1. Para
aplicar las actualizaciones se utilizaron las herramientas provistas
por el fabricante. El proceso se realizó de manera incremental,
migrando entre versiones mayores consecutivas hasta alcanzar la
versión actual. Durante las actualizaciones el servicio estuvo fuera
de línea.

\subsection{Repositorio de artefactos Nexus}

El repositorio de artefactos Sonatype Nexus funcionaba en su versión
2, la cual no recibe más soporte por parte del desarrollador. El
proceso fue más complejo que en el caso de GitLab, ya que la migración
de datos entre las versiones 2 y 3 no es automática. La secuencia de
pasos a continuación describen el proceso efectuado.

\begin{itemize}
\item El primer paso fue actualizar el servicio a la última versión de
  la serie 2. Para ello se escribió código Ansible, el cual fue
  probado en un entorno de test. Una vez ajustado el procedimiento, se
  aplicaron los cambios en el entorno de producción.
\item El segundo paso fue escribir código Ansible para la
  configuración de la versión 3. Se aplicó este código de modo que
  ambas versiones funcionaran en la instancia en forma simultánea.
\item Se realizó la migración entre versiones siguiendo el proceso
  explicado la documentación del desarrollador.
\item Se eliminó la instalación correspondiente a la versión 2 y se
  ajustó el código Ansible para que utilice la vesión 3.
\end{itemize}
Una vez actualizado el repositorio, se descubrió que la nueva versión
introdujo cambios incompatibles en su interfaz API. Por ello fue
necesario corregir los scripts de operaciones que utilizan
funcionalidades de esta interfaz.
