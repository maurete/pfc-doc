%
\subsection{Estrategia trivial}
%
La estrategia trivial retorna simplemente el valor $H=0$ como cantidad
``óptima'' de neuronas en la capa oculta, sin efectuar entrenamiento.
El modelo resultante es siempre equivalente a un perceptrón simple, y
permite clasificar conjuntos de datos linealmente separables.
Si bien esta estrategia no efectúa optimización alguna, resulta útil
en la generación de un modelo ``básico'' para ser comparado frente a
otros modelos generados con estrategias de mayor complejidad.
Incluso, se sabe que este modelo básico resultará óptimo en
determinados casos \cite{nnfaq3}.
%% \pdfmargincomment[color=Ivory]{
%%   De la fuente citada:
%%   You may not need any hidden layers at all. Linear and generalized
%%   linear models are useful in a wide variety of applications (McCullagh
%%   and Nelder 1989). And even if the function you want to learn is mildly
%%   nonlinear, you may get better generalization with a simple linear
%%   model than with a complicated nonlinear model if there is too little
%%   data or too much noise to estimate the nonlinearities accurately.
%% }
