
En líneas generales, se observa que la aplicación de la estrategia de
búsqueda exhaustiva para determinar el número de neuronas en la capa
oculta obtuvo una pequeña mejora en los resultados respecto a los
obtenidos mediante la estrategia trivial.

Con respecto a los conjuntos de \caract{s}, los mejores resultados se
obtuvieron con aquellos que incluyen las características de estructura
secundaria \dset{E} y \dset{S-E}.
Asimismo, el uso exclusivo del conjunto de \caract{s} de la secuencia
derivó en tasas de clasificación subóptimas en todos los casos.

La variabilidad observada en los resultados proviene de dos fuentes:
la inicialización aleatoria de la red y la partición pseudoaleatoria de
los datos en la generación del problema, regulada por una ``semilla''.
En el problema \prob\tripletsvm{}, las particiones de entrenamiento y
prueba son fijas e independientes de la semilla pseudoaleatoria,
razón por la cual se observa una menor desviación estándar en los
resultados obtenidos para este problema.

%% En el problema \prob\tripletsvm{} se observa una mayor paridad entre
%% los resultados obtenidos por las estrategias trivial y de búsqueda
%% exhaustiva para determinar el número de neuronas en la capa oculta.
%% Una interpretación de este resultado podría ser que, para este
%% problema, los datos de entrenamiento son linealmente separables.
En \cite{xue}, los autores reportan una $\SE=93.3\%$ y una
$\SP=88.1\%$ para el método \work\tripletsvm{} sobre el mismo conjunto
de prueba de \prob\tripletsvm{}, utilizando el conjunto de \caract{s}
de tripletes (\dset{T}).
Estos resultados son superiores a los obtenidos con el clasificador
MLP.
Sin embargo, la utilización de las \caract{s} de la estructura
secundaria (conjuntos \dset{E} y \dset{S-E}) mejoró significativamente
estos resultados, superando el 96\% tanto en la \SE{} como el la \SP.

En el caso de los problemas \prob\mipred{} y \prob\micropred{}, no se
probaron las \caract{s} de tripletes dado que ambos presentan ejemplos
con bucles múltiples en su estructura secundaria.
En ambos casos se observó que la especificidad (\SP) resultó
sensiblemente mayor a la sensibilidad (\SE).
Esto se atribuye al ``desbalance de clases'' presente en los
respectivos conjuntos de entrenamiento, que incluyen 2 ejemplos de
clase negativa por cada ejemplo de clase positiva.

Según \cite{ng} el método \work{\mipred} obtiene una $\SE=84.5\%$ y
una $\SP=98.0\%$ sobre un conjunto de prueba similar al del problema
\prob\mipred{}.
Este resultado está en línea con el obtenido por el clasificador MLP
utilizando \caract{s} \dset{S-E}.

