%

A partir de los resultados de la
\iflatexml{}Tabla~\ref{tbl:mlp-results}\else\autoref{tbl:mlp-results}\fi{}
se efectuaron las siguientes observaciones:
%
\begin{itemize}
\item
  \e{Estrategias de selección de \hparam{s}}.
  La estrategia de búsqueda exhaustiva obtuvo una pequeña mejora en
  los resultados respecto a aquellos obtenidos mediante la estrategia
  trivial.
\item
  \e{Conjuntos de \caract{s}}.
  Los mejores resultados se obtuvieron con aquellos conjuntos que
  incluyen las \caract{s} de estructura secundaria: \dset{E} y
  \dset{S-E}.
  Asimismo, al probar el conjunto de \caract{s} de la secuencia se
  obtuvieron tasas de clasificación subóptimas en todos los casos.
\item
  \e{Variabilidad}.
  La variabilidad observada en los resultados proviene de dos fuentes:
  la inicialización aleatoria de la red y la partición pseudoaleatoria
  de los datos en la generación del problema.
  Dado que en el problema \prob\tripletsvm{} las particiones son fijas
  e independientes de la semilla pseudoaleatoria, se observó en este
  caso una menor desviación estándar en los resultados obtenidos.
\item
  \e{Problema \tripletsvm{}}.
  Los resultados obtenidos para este problema pueden compararse
  directamente con aquellos presentados por los autores del método
  \work\tripletsvm{} \cite{xue}, ya que los datos utilizados en ambos
  casos son idénticos.
  En \cite{xue} se reporta una $\SE=93.3\%$ y una $\SP=88.1\%$ parael
  conjunto de \caract{s} de tripletes (\dset{T}).
  El clasificador MLP obtuvo resultados inferiores para este mismo
  conjunto de \caract{s}, mientras que con el conjunto de \caract{s}
  que incluyen la estructura secundaria (\dset{E} y \dset{S-E})
  los resultados obtenidos superaron al método \work\tripletsvm{}.
\item
  \e{Problema \mipred{}}.
  Según \cite{ng} el método \work{\mipred} obtuvo una $\SE=84.5\%$ y
  una $\SP=98.0\%$ sobre un conjunto de prueba similar al del problema
  \prob\mipred{}.
  Este resultado está en línea con el obtenido por el clasificador MLP
  utilizando el \caract{s} \dset{S-E}.
\item
  \e{Problema \micropred{}}.
  Con el problema \prob\micropred{} se obtuvo el menor rendimiento del
  clasificador, lo cual se explica por la mayor complejidad del
  conjunto de datos.
\item
  \e{Desbalance de clases}.
  En los resultados obtenidos para los problemas \prob\mipred{} y
  \prob\micropred{} se observa que la especificidad (\SP) fue
  sensiblemente mayor a la sensibilidad (\SE).
  Esto se atribuye al desbalance de clases en los respectivos
  conjuntos de entrenamiento, que contienen $2$ ejemplos de clase
  negativa por cada ejemplo de clase positiva.
\end{itemize}
%
