\section{Lectura de datos de entrada y extracción de características}

Para la lectura de los datos de entrada a partir de archivos FASTA, se
invoca la función \func{load\_fasta}, que carga los datos
en una estructura en memoria.
Con la información cargada en memoria, las funciones
\func{feats\_sequence}, \func{feats\_structure}, \func{feats\_triplet}
y \func{feats\_extra}, se encargan de la extracción de
características de secuencia, de estructura secundaria, de tripletes y
de tripletes auxiliares, respectivamente, retornando un vector
numérico de características para cada entrada leída.
\funcheader{load\_fasta}{Nombre de archivo a leer}%
{Representacion de los datos leídos como un arreglo de datos en memoria}
La función \func{load\_fasta} efectúa lectura del archivo fasta
indicado como entrada, y aplicando reglas de correspondencia de
expresiones regulares genera un arreglo de ``ejemplos'', los cuales
contienen los siguientes campos:

\begin{itemize}
\item Número de línea de comienzo y de fin del ejemplo
\item Línea de descripción
\item Secuencia
\item Estructura secundaria (si está presente)
\item Mínima energía libre (número al final de la estructura
  secundaria)
\item Identificador (primera ``palabra'' de la línea de descripción)
\end{itemize}
Para los ejemplos donde no haya información de estructura secundaria,
la función \func{load\_fasta} intenta efectuar el ``plegado'' mediante
las funciones propias del ``Bioinformatics Toolbox'' de Matlab.

\funcheader{feats\_sequence}{Salida de load\_fasta}%
{Matriz de YYYYYY filas con 23 características}
La función \func{feats\_sequence} procesa la popiedad ``Secuencia''
presente en cada ejemplo del arreglo de entrada y devuelve una matriz
con $N$ (número de ejemplos) filas y 23 columnas con las características
de la secuencia: longitud, cuenta de nucleótidos y dinucleótidos.

\funcheader{feats\_triplet}{Salida de load\_fasta}%
{Matriz de YYYYYY filas con 32 características}
La función \func{feats\_triplet} utiliza la información de secuencia y
estructura secundaria en cada ejemplo para calcular la cuenta de
``tripletes'' en la zona del tallo del pre-miRNA.  Si se procesa un
elemento con múltiples bucles, la fila correspondiente se completa con
32 ``NaN''s (Not A Number).  La implementación de esta función se
efectúa a partir de la especificación en \cite{xue} y se valida con
los datos de información suplementaria provistos por los autores.

\funcheader{feats\_extra}{Salida de load\_fasta}%
{Matriz de YYYYYY filas con 4 características}
La función \func{feats\_extra} lee la información de estructura
secundaria y calcula las 4 características: longitud del tallo, número
de pares de bases, longitud del tallo dividido pares de bases, y
cuenta de bases G y C en el tallo. Tal como \func{feats\_triplet}, en
las filas correspondientes a ejemplos con múltiples bucles retorna
NaN.
\funcheader{feats\_structure}{Salida de load\_fasta}%
{Matriz de YYYYYY filas con 7 características}
La función \func{feats\_structure} procesa la estructura secundaria de
cada ejemplo y la mínima energía libre, calculando las características
basadas en la misma, además de las cuentas de pares A-U, G-C y G-U.