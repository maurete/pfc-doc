%
%
\subsection{El algoritmo de retropropagación básico}
%
El algoritmo de retropropagación es un algoritmo específico para el
perceptrón multicapa que efectúa entrenamiento ajustando los pesos
$w_{ij}$ de la red, minimizando la función de error
$\C{E}(n)$ a partir de la información del gradiente
$\nabla\C{E}(n)=\dpar{\C{E}(n)}{w_{ij}}{}$.

Con el objetivo de simplificar la descripción, se describe en primer
lugar el algoritmo de retropropagación para un entrenamiento ``en
línea''.
La idea básica es aplicar a cada peso sináptico $w_{ij}$ de la red una
corrección $\Delta{}w_{ij}(n)$ proporcional a la derivada parcial
$\dpar{\C{E}(n)}{w_{ij}}{}$.
Según la regla de la cadena, esta derivada parcial puede escribirse
%
\begin{align}\label{e2:deriv-E-wrt-w-expanded}
  \dpar{\C{E}(n)}{w_{ij}}{} = \dpar{\C{E}(n)}{s_i(n)}{}
  \dpar{s_i(n)}{v_i(n)}{} \dpar{v_i(n)}{w_{ij}}{}.
\end{align}
%
Diferenciando ambos lados de la
\iflatexml{}Ecuación~\ref{e2:error-energy-neuron}\else\autoref{e2:error-energy-neuron}\fi{}
respecto de $s_i(n)$, se tiene
%
\begin{align}\label{e2:deriv-E-wrt-s}
  \dpar{\C{E}(n)}{s_i(n)}{} = -e_i(n).
\end{align}
%
Por la \iflatexml{}Ecuación~\ref{e2:neuron-general}\else\autoref{e2:neuron-general}\fi, se tiene
%
\begin{align}\label{e2:deriv-s-wrt-v}
  \dpar{s_i(n)}{v_i(n)}{} \tab = f'(v_i(n)),\tabs
  \dpar{v_i(n)}{w_{ij}}{} \tab = s_j(n).
\end{align}
%
Reemplazando las Ecuaciones
\ref{e2:deriv-E-wrt-s}--\ref{e2:deriv-s-wrt-v} en la
\iflatexml{}Ecuación~\ref{e2:deriv-E-wrt-w-expanded}
\else\autoref{e2:deriv-E-wrt-w-expanded}\fi, se tiene
%
\begin{align}\label{e2:deriv-E-wrt-w}
  \dpar{\C{E}(n)}{w_{ij}}{} = -e_i(n) f'(v_i(n)) s_j(n).
\end{align}
%
Cuando la neurona $i$ está en la capa de salida, esta fórmula se
utiliza directamente para el cálculo de $\dpar{\C{E}(n)}{w_{ij}}{}$,
ya que la señal de error $e_i(n)$ de la neurona viene dada por la
diferencia entre el valor deseado $y_k(n)$ y la salida de la neurona
$s_i(n)$ según la según la
\iflatexml{}Ecuación~\ref{e2:error-energy-neuron}
\else\autoref{e2:error-energy-neuron}\fi.

Sin embargo, las neuronas en las capas ocultas también deberán ajustar
sus pesos, ya que su valor de salida tiene influencia en la salida
global de la red, aunque resulta imposible determinar una ``respuesta
deseada'' correspondiente $y_k(n)$.  Por ello, cuando la neurona $i$
se ubica en una capa oculta, el cálculo de $\dpar{\C{E}(n)}{w_{ij}}{}$
resulta más complejo.  En particular, se desea obtener el valor de
$\dpar{\C{E}(n)}{s_i}{}$ correspondiente a una neurona en la capa
oculta.

El razonamiento para el cálculo de $\dpar{\C{E}(n)}{s_{i}}{}$, cuando
la neurona $i$ es oculta, es como sigue. Considérese en primer lugar
que el valor de $\dpar{\C{E}(n)}{s_k}{}$ es conocido, con
$k\in{}\C{S}(i)$, siendo $\C{S}(i)$ el conjunto de índices de todas
las neuronas en la capa siguiente a la capa de $i$.  Entonces, se
puede ver que
%
\begin{align}
  \dpar{\C{E}}{s_i}{} \tab= \sum_{k\in\C{S}(i)}
      \dpar{\C{E}}{s_k}{}\dpar{s_k}{s_i}{} \notag\\
    \tab= \sum_{k\in\C{S}(i)}
      \dpar{\C{E}}{s_k}{}\dpar{s_k}{v_k}{}\dpar{v_k}{s_i}{} \notag\\
    \tab= \sum_{k\in\C{S}(i)} \dpar{\C{E}}{s_k}{} s'(v_k) w_{ki}.
    \label{e2:deriv-E-wrt-s-hid}
\end{align}
%
Reemplazando la
\iflatexml{}Ecuación~\ref{e2:deriv-E-wrt-s-hid}\else\autoref{e2:deriv-E-wrt-s-hid}\fi{}
y la \iflatexml{}Ecuación~\ref{e2:deriv-s-wrt-v}
\else\autoref{e2:deriv-s-wrt-v}\fi{} en la
\iflatexml{}Ecuación~\ref{e2:deriv-E-wrt-w-expanded}\else\autoref{e2:deriv-E-wrt-w-expanded}\fi,
se tiene entonces
%
\begin{align}\label{e2:deriv-E-wrt-w-hid}
  \dpar{\C{E}(n)}{w_{ij}}{} =
  \sum_{k\in\C{S}(i)} \dpar{\C{E}(n)}{s_k(n)}{} s'(v_k(n)) w_{ki}
  f'(v_i(n)) s_j(n).
\end{align}
%
Esta fórmula es la clave del algoritmo de retropropagación: por la
arquitectura del MLP, la salida $s_i(n)$ de una neurona oculta ejerce
una influencia en la salida de todas las neuronas en las capas
subsiguientes.  Del mismo modo, se tiene que el \e{gradiente local} de
la señal de error de la neurona viene determinado por los gradientes
de error de todas las neuronas en las capas subsiguientes.

Comenzando el cálculo de $\dpar{\C{E}(n)}{w_{ij}}{}$ por la capa de
salida
(\iflatexml{}Ecuación~\ref{e2:deriv-E-wrt-w}\else\autoref{e2:deriv-E-wrt-w}\fi)
y continuando con las capas ocultas en orden inverso
(\iflatexml{}Ecuación~\ref{e2:deriv-E-wrt-w-hid}\else\autoref{e2:deriv-E-wrt-w-hid}\fi),
el gradiente resulta calculable para todas las neuronas.  Por ello el
nombre `` retropropagación'': se dice que la información del gradiente
de error se propaga hacia atrás a partir de la capa de salida hasta la
primer capa oculta.

El desarrollo del algoritmo de retropropagación a mediados de los años
80 representó un hito en el campo de las redes neuronales, ya
que brindó un método computacionalmente eficiente, usable, para
entrenar el perceptrón multicapa.
