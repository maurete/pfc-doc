\section{Pruebas del clasificador SVM con núcleo RBF}
Los resultados de aplicar el clasificador SVM-RBF para los problemas
definidos se detallan en la \autoref{tbl:rbf-results}.

En líneas generales, la utilización del clasificador SVM-RBF se
traduce en una mejora en las tasas de clasificación respecto de los
otros clasificadores.
El uso de este clasificador con el conjunto de características S-E
resulta en las mejores tasas de clasificación para los tres problemas
definidos, utilizando la estrategia RMB para los problemas
\tripletsvm{} y \micropred{}, así como el criterio del error empírico
para el problema \mipred{}.

Se destaca sin embargo que la estrategia de selección trivial obtiene
resultados en general inferiores a aquellos obtenidos con la
estrategia trivial para el clasificador SVM-lineal. Asimismo, se
observa que la estrategia RMB diverge en algunos casos para el
problema \mipred{} al utilizar únicamente las características de
secuencia.