%
%
\subsection{Aprendizaje}
%
%% En el contexto de las redes neuronales artificiales, el aprendizaje
%% puede ser visto como el problema de efectuar actualizaciones de los
%% pesos en las conexiones de la red de modo que ésta efectúe una tarea
%% deseada.

%% La red debe ``aprender'' los pesos en sus conexiones a partir de los
%% patrones de entrenamiento. El rendimiento en la clasificación mejora
%% progresivamente a medida que se van actualizando los pesos.

%% La habilidad de las redes neuronales de aprender de manera autoomática
%% a partir de ejemplos es quizá la caracteríßtica que hace a su
%% éxito. En lugar de seguir un conjunto de reglas especificadas por
%% expertos humanos, las redes neuronales artificiales ...

El objetivo del entrenamiento en el perceptrón multicapa es ajustar
los pesos de la red de forma que ésta efectúe una transformación de
las entradas a las salidas deseadas. Esta transformación viene dada
por el conjunto de entrenamiento $D$: cada ejemplo $(\xx_j,\yy_j)$ en
$D$ consta de un vector de activación $\xx_j$ y una salida deseada
$\yy_j$.
%% La distancia entre la salida deseada y la salida efectiva de
%% la red se mide con la función de energía (o de costo) $E$:
%% %
%% \begin{align}\label{e2:energy-function}
%%   E = \frac{1}{2}\sum_{j=1}^{\ell}\|\yy_j-\hat{\yy}_j\|,
%% \end{align}
%% %
%% en donde el vector $\hat{\yy}_j$ es el vector compuesto por las
%% salidas de cada neurona en la capa de salida. La función $E$ puede
%% interpretarse como una medida de \e{aptitud} de los pesos de la red al
%% conjunto $D$, de modo que satisfacer el objetivo del aprendizaje es
%% equivalente a encontrar un mínimo global de $E$.

Considerando una neurona $i$ en la capa de salida, sea $s_i(t)$ la
salida de la misma en respuesta al estímulo $\xx(t)$ aplicado en la
capa de entrada.  Para simplicidad en la exposición, se asume que los
índices $i$ de las neuronas en la capa de salida coinciden con los del
vector de salida $\hat{\yy}$, de modo que $s_i=\hat{y}_i$.  La
\e{señal de error} producida en la salida de esta neurona se define
según
%
\begin{align}\label{e2:error-signal-neuron} %4.2
  e_i(t)=s_{i}(t)-y_{i}
\end{align}
%
en donde $y_{i}$ es el $i$-ésimo elemento del vector de respuesta
deseada $\yy$. La \e{energía de error instantáneo} de la neurona
$i$ se define según
%
\begin{align}\label{e2:error-energy-neuron} %4.3
  \C{E}_i(t)=\frac{1}{2}e^2_i(t).
\end{align}
%
Sumando las contribuciones error-energía de todas las neuronas en la
capa de salida, se expresa la \e{energía total de error instantáneo}
de la red entera como
%
\begin{align}\label{e2:error-energy-net} %4.4
  \C{E}(n)&=\sum_{j\in C}\C{E}_j(n) \\
  &=\frac{1}{2}\sum_{j\in C}e^2_j(n).
\end{align}
%
en donde el conjunto $C$ incluye los subíndices para todas las
neuronas en la capa de salida. La energía de error promedio sobre el
conjunto de entrenamiento, denominada también \e{riesgo empírico}, se
define según
%
\begin{align}\label{e2:average-energy-net} %4.5
  \C{E}_{\T{av}}(\ell)&=\frac{1}{\ell}\sum_{n=1}^\ell\C{E}(n) \\
  &=\frac{1}{2\ell} \sum_{n=1}^\ell \sum_{j\in C}e^2_j(n).
\end{align}
%
Naturalmente, tanto la energía de error instantáneo como la energía de
error promedio son funciones de todos los pesos sinápticos del
perceptrón multicapa. Esta dependencia funcional no fue incluida de
forma explícita en la definición de $\C{E}(n)$ y
$\C{E}_{\T{av}}(\ell)$ en pos de simplificar la notación.

Conforme avance el proceso de entrenamiento, se espera que cada
actualización de los pesos de la red traiga consigo una reducción de
los valores de estas funciones de energía del error.
%
\subsubsection{Aprendizaje en línea y aprendizaje por época}
%
El aprendizaje en línea y el aprendizaje por época hacen referencia
al momento en que se efectúa la actualización de los pesos de la
red.

En el aprendizaje en línea, o aprendizaje por patrón, se efectúa una
actualización de los pesos de la red luego de calcular la salida y el
error instantáneo de cada ejemplo del conjunto de aprendizaje.  Este
método se denomina también aprendizaje estocástico, y busca
minimizar la energía de error promedio minimizando el error
instantáneo de cada patrón.  Este método funciona especialmente bien
para grandes conjuntos de entrenamiento que contengan información
redundante.

En el aprendizaje por época, la actualización de los pesos de la red
se efectúa una vez que han sido calculadas las salidas de todos los
patrones en el conjunto de entrenamiento, minimizando la función de
error promedio $\C{E}_{\T{av}}(\ell)$. En el presente trabajo,
se considera en todos los casos el aprendizaje por época.

%% Los procedimientos adaptativos descritos en adelante
%% utilizan este tipo de aprendizaje, ya que la información de gradiente
%% sumada contiene información más fiable acerca de la forma de la
%% función de error como un todo \cite{riedmiller}.  Algunos autores, sin
%% embargo, sugieren que la mejor técnica es la del aprendizaje en línea
%% \cite{haykin}.

%% %
%% \subsubsection{Técnicas adaptativas}
%% %
%% A la fecha, se han propuesto muchas técnicas para afrontar los
%% problemas inherentes del método de descenso por gradiente. La mayoría
%% tienen sus bases de la disciplina de la teoría de
%% optimización.

%% Estas técnicas se pueden dividir en dos
%% categorías. Aquellos algoritmos que utilizan un conocimiento global
%% del estado de la red como un todo, tal como la dirección del vector de
%% actualización de los pesos ``entero'', son llamadas técnicas globales.
%% Existen muchos ejemplos donde los algoritmos de aprendizaje utilizan el
%% conocimiento global \cite{salomon,moeller}.

%% Por contraste, las estrategias de adaptación locales se basan
%% únicamente en información específica a los pesos, tal como el
%% comportamiento temporal de la derivada parcial de este peso. El
%% enfoque local se relaciona más naturalmente con el concepro de
%% procesamiento distribuido en el que los cálculos pueden efectuarse en
%% paralelo.  Más aún, en muchos casos se encuentra que las estrategias
%% locales funcionan mucho mejor que las globales, a pesar del hecho que
%% utilizan menos información y de que normalmente son mucho más rápidas
%% y fáciles de calcular \cite{schiffmann}.
