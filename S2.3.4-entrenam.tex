%
%
\subsection{Aprendizaje}
%
%% En el contexto de las redes neuronales artificiales, el aprendizaje
%% puede ser visto como el problema de efectuar actualizaciones de los
%% pesos en las conexiones de la red de modo que ésta efectúe una tarea
%% deseada.

%% La red debe ``aprender'' los pesos en sus conexiones a partir de los
%% patrones de entrenamiento. El rendimiento en la clasificación mejora
%% progresivamente a medida que se van actualizando los pesos.

%% La habilidad de las redes neuronales de aprender de manera autoomática
%% a partir de ejemplos es quizá la caracteríßtica que hace a su
%% éxito. En lugar de seguir un conjunto de reglas especificadas por
%% expertos humanos, las redes neuronales artificiales ...

El objetivo del entrenamiento en el perceptrón multicapa es ajustar
los pesos de la red de forma que ésta efectúe una transformación de
las entradas a las salidas deseadas. Esta transformación viene dada
por el conjunto de entrenamiento $D$, compuesto por $\ell$ ejemplos
denotados por $(\xx(n),\yy(n))$, $n=1,\ldots,\ell$.
La discusión presentada a continuación sigue de cerca aquella
presentada en \cite[Capítulo~4]{haykin}.

%% La distancia entre la salida deseada y la salida efectiva de
%% la red se mide con la función de energía (o de costo) $E$:
%% %
%% \begin{align}\label{e2:energy-function}
%%   E = \frac{1}{2}\sum_{j=1}^{\ell}\|\yy_j-\hat{\yy}_j\|,
%% \end{align}
%% %
%% en donde el vector $\hat{\yy}_j$ es el vector compuesto por las
%% salidas de cada neurona en la capa de salida. La función $E$ puede
%% interpretarse como una medida de \e{aptitud} de los pesos de la red al
%% conjunto $D$, de modo que satisfacer el objetivo del aprendizaje es
%% equivalente a encontrar un mínimo global de $E$.

Considerando una neurona $i$ en la capa de salida, sea $s_i(n)$ la
salida de la misma en respuesta al estímulo $\xx(n)$ aplicado en la
capa de entrada.
La
\e{señal de error} producida en la salida de esta neurona viene dada
por
%
\begin{align}\label{e2:error-signal-neuron} %4.2
  e_i(n)=y_{k}(n)-s_{i}(n)
\end{align}
%
en donde $y_k(n)$ es la componente del vector de respuesta deseada
$\yy(n)$ correspondiente a la posición de la neurona $i$ en la capa de
salida.
%% en donde $y_{i}$ es el $i$-ésimo elemento del vector de respuesta
%% deseada $\yy$.
La \e{energía de error instantáneo} de la neurona $i$ se define según
%
\begin{align}\label{e2:error-energy-neuron} %4.3
  \C{E}_i(n)=\frac{1}{2}e^2_i(n).
\end{align}
%
Sumando las contribuciones error-energía de todas las neuronas en la
capa de salida, se expresa la \e{energía total de error instantáneo}
de la red como
%
\begin{align}\label{e2:error-energy-net} %4.4
  \C{E}(n)&=\sum_{i\in C}\C{E}_i(n) \\
  &=\frac{1}{2}\sum_{i\in C}e^2_i(n).
\end{align}
%
Aquí, el conjunto $C$ contiene los subíndices de las
neuronas en la capa de salida. La energía de error promedio sobre el
conjunto de entrenamiento
viene dada por
%
\begin{align}\label{e2:average-energy-net} %4.5
  \C{E}_{\T{av}}&=\frac{1}{\ell}\sum_{n=1}^\ell\C{E}(n) \\
  \tab=\frac{1}{2\ell} \sum_{n=1}^\ell \sum_{i\in C}e^2_i(n).
\end{align}
%
Naturalmente, tanto la energía de error instantáneo como la energía de
error promedio son funciones de todos los pesos sinápticos del
perceptrón multicapa. Esta dependencia funcional no fue incluida de
forma explícita en la definición de $\C{E}(n)$ y
$\C{E}_{\T{av}}$ en pos de simplificar la notación.

En general, el entrenamiento del perceptrón multicapa consiste
en un procedimiento gradual, en el cual se actualizan los
pesos de la red buscando minimizar alguna función de energía del error.
