\subsection{Costo computacional}
Al utilizar el núcleo de base radial (RBF) con el clasificador SVM se
incorpora un hiperparámetro a optimizar $\gamma$ a optimizar, lo que
resulta en un incremento del tiempo requerido por los métodos de
optimización de hiperparámetros.

Este incremento se hace patente en la estrategia de búsqueda
exhaustiva, en la que el número de entrenammientos efectuados aumenta
en casi un orden de magnitud.  Afortunadamente, este incremento en el
número de entrenamientos no se refleja proporcionalmente en el tiempo
requerido de ejecución, probablemente debido al hecho que el
clasificador es más ``flexible'' y se adapta más rápidamente (en menos
iteraciones) al problema en cuestión.

Al utilizar el criterio del error empírico se observa un costo computacional
(en tiempo y número de entrenamientos) que está en línea con
aquel del clasificador SVM-lineal.

Mención aparte merece la estrategia RMB, el cual logra conseguir tasas
de clasificación satisfactorias requiriendo un número de
entrenamientos soreprendentemente bajo y un tiempo de ejecución que
para el problema \tripletsvm{} resulta prácticamente indistinguible de
la estrategia trivial.