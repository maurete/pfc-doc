%
%
\subsection{Conjuntos de prueba de miPred}
%
En \cite{ng} se efectuaron pruebas complementarias del método
\work\mipred{} con dos conjuntos denominados \dset{IE-NH} e
\dset{IE-NC}:
%
\begin{description}
\item{\dset{IE-NC}}\\
  Contiene $12387$ ejemplos de \ncrna{s} funcionales extraídos de la
  versión $7$.$0$ de la base de datos \dset{Rfam} \cite{rfam},
  excluyendo aquellos ejemplos que son \premirna{s}.
\item{\dset{IE-NH}}\\
  Comparable al conjunto \dset{cross-species} de \cite{xue}, incluye
  $1918$ \premirna{s} de especies no humanas publicados en la versión
  $8$.$2$ de \work{\mirbase}.
\end{description}
%
Las pruebas consistieron en entrenar el clasificador con el conjunto
de entrenamiento definido en el problema \prob\mipred{}, clasificando
los conjuntos de prueba suplementarios con el modelo obtenido.
Resulta necesario aclarar que, a diferencia de lo que ocurre con los
conjuntos de prueba de \work\tripletsvm{}, los resultados de estas
pruebas no son directamente comparables, ya que no se garantiza que
los conjuntos de entrenamiento y prueba contengan los mismos elementos
que aquellos utilizados por los autores en el trabajo original.

A partir de los resultados que se muestran en la
Tabla~\ref{tbl:suppl-ng} se observa que el clasificador \sbs{SVM-RR}
obtuvo una mayor especificidad para la especie humana en comparación
al resultado reportado por los autores, mientras que logró una menor
capacidad de generalización a otras especies.
Asimismo, los clasificadores \sbs{MLP-B}, \sbs{SVM-LE} y \sbs{SVM-RE}
obtuvieron una tasa de clasificación en línea con la reportada en
\cite{ng} para el conjunto \sbs{IE-NH}.
