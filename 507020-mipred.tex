%
%
\subsection{Conjuntos de prueba de miPred}
%
En el trabajo de \citeauthor{ng} \cite{ng} se prueba el método
\mipred{} con dos conjuntos de prueba suplementarios denominados
\ssf{IE-NH} e \ssf{IE-NC}.
\paragraph{IE-NH}
Comparable al conjunto \ssf{cross-species} de \cite{xue}, incluye
1918 pre-miRNAs de especies no humanas publicadas en la versión 8.2 de
miRBase.
\paragraph{IE-NC}
Contiene 12.387 ejemplos de ncRNAs funcionales obtenidos de la versión
7.0 de la base de datos Rfam \cite{rfam}, eliminando los ejemplos
correspondientes a pre-miRNAs. En un clasificador ideal este conjunto
de datos debería ser clasificado como de clase negativa.

Las pruebas efectuadas consistieron en entrenar el clasificador con el
conjunto de entrenamiento definido en el problema \mipred{},
clasificando luego los conjuntos de prueba enumerados.  Se debe
destacar que, a diferencia del caso \tripletsvm{}, los resultados no
son directamente comparables, ya que no existe forma de garantizar que
los conjuntos de entrenamiento y prueba contengan los mismos elementos
que aquellos utilizados por los autores en el trabajo original.

A partir de los resultados obtenidos, se puede observar que la estrategia
RMB resulta en un clasificador con mayor especificidad, aunque menor
sensibilidad que las demás estrategias de selección de hiperparámetros.
