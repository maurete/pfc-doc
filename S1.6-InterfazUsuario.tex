\section{Interfaz de usuario}
La interfaz de usuario de línea de comandos consiste en
tres funciones principalesque permiten al usuario cargar los datos del disco,
generar el modelo del clasificador y clasificar nuevos datos.

\begin{enumerate}
\item La función \func{problem\_gen} carga del disco los datos de
  entrenamiento y prueba según la especificación del usuario, y
  retorna una estructura en memoria que representa un ``problema''
  sobre el cual trabajar.
\item La función \func{select\_model} permite al usuario elegira la
  estrategia de selección de hiperparámetros a seguir. Recibe como
  argumento el problema (que incluya un datos de entrenamiento)
  generado por \func{problem\_gen} y retorna un clasificador entrenado
  con los hiperparámetros óptimos encontrados.
\item La función \func{problem\_classify} permite clasificar problemas
  que especifiquen un conjunto de prueba.  Recibe como argumento el
  modelo entrenado y el problema que se desea clasificar, y retorna
  una estructura en memoria con las predicciones para todos los
  elementos incluidos en el conjunto de prueba del problema.
\end{enumerate}

\subsection{Interfaz de usuario web}
Para la creación de la interfaz de usuario web se optó por utilizar la
herramienta \eng{Web-demo builder}\footnote{Página web:
  \url{https://bitbucket.org/sinc-lab/webdemobuilder/}}, desarrollado
en el laboratorio \eng{sinc(i)} de la Facultad de Ingeniería y
Ciencias Hídricas.

Se codificó una función específica \func{webif} para la utilización
del software a través de una interfaz web, y un archivo en formato
{\mono Makefile} que permite la generación del archivo empaquetado en
formato zip para utilizar con el software Web-demo builder.

Finalmente, se procedió a la puesta en servicio de una interfaz web de
demostración en el servidor de pruebas provisto por los creadores de
Web-demo builder, la cual está disponible en la siguiente URL:

\url{http://ec2-52-5-194-68.compute-1.amazonaws.com/scripts/57769864/webif/}