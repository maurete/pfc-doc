
%
%
%
\section{Preprocesamiento}
%
Los datos de entrada al sistema vienen dados en forma de archivos en
un formato estándar denominado FASTA, un formato de texto que contiene
secuencias ``anotadas'' (con información complementaria).  El
preprocesamiento es el conjunto de tareas que convierten estos datos a
vectores numéricos ``etiquetados'' tal como son utilizados por la
máquina de aprendizaje. Estas tareas son:
%
\begin{description}
\item[Análisis sintáctico (\e{parsing}):] se trata de interpretar el
  texto de entrada identificando las líneas de descripción
  (anotaciones), de secuencia, y de estructura secundaria si las
  hubiera.
\item[Plegado:] consiste en calcular la información de estructura
  secundaria para todas las secuencias, mediante la invocación de
  software externo.
\item[Extracción de \caract{s}:] es el proceso de generar vectores
  numéricos de longitud fija que representan cada ejemplo a partir de
  la información de secuencia y de estructura secundaria.
\item[Normalización:] ajusta el rango de las componentes del vector de
  características de modo que se ajusten a un intervalo
  preestablecido.
\item[Generación de conjuntos de datos y de validación cruzada:]
  agrupa los ejemplos en conjuntos de entrenamiento y prueba, y genera
  particiones de validación cruzada, según sea el propósito con el que
  se utilizarán los datos: generar un modelo de clasificador o
  predicción de pertenencia clase.
\end{description}
%
De este modo, con el preprocesamiento se obtiene a la salida una
representación de los ejemplos en forma de vectores numéricos
normalizados, utilizables directamente por la máquina de aprendizaje.
