%
%
\subsubsection{Características de tripletes}
%
Las \caract{s} de tripletes, propuestas en el método \work{Triplet-SVM}
\cite{xue}, se basan en la idea de que la propiedad distintiva de los
\premirna{s} es una estructura secundaria en forma de horquilla
con una alta complementariedad entre las bases que forman el \e{tallo}.
%% En el trabajo original, los autores proponen un método para clasificar
%% secuencias con estructura secundaria en forma de horquilla entrenando
%% con \premirna{s} ``reales'' y otras secuencias con la misma estructura
%% secundaria denominadas ``pseudo \premirna{s}''.
Esta idea surge a partir de observar que la distribución
(frecuencia de aparición) de unas sub-estructuras locales llamadas
``tripletes'' difiere significativamente entre los ejemplos que
representan \premirna{s} y los ejemplos de otro tipo.

%% A partir de estas observaciones, proponen generar un conjunto de
%% \caract{s} que representa la distribución de las sub-estructuras
%% locales dentro del ``tallo'' de la horquilla.

Un ``triplete'' es una cadena de caracteres que relaciona la base de
un nucleótido (\ntA, \ntC, \ntG, o \ntU) en la posición $i$ con su
estructura secundaria local en $i-1,i,i+1$, representada en forma
binaria como ``acoplado'' \pairL (paréntesis) y ``no acoplado''
\noPair (punto).
El conjunto de \caract{s} de tripletes incluye medidas que cuentan
el número de ocurrencias de las $32$ combinaciones de tripletes posibles
($4$ bases $\cdot$ $2^3$ combinaciones de estado)
en la región del tallo, junto con $4$ medidas auxiliares que surgen
de cálculos intermedios:
%
\begin{itemize}
\item longitud del tallo $L_3$,
\item número de pares de bases $P$,
\item complementariedad de ambos brazos de la horquilla,
\item proporción de bases \ntG y \ntC en la zona del tallo.
\end{itemize}
%
En la \iflatexml{}Figura~\ref{triplet}\else\autoref{triplet}\fi{} se
representa en forma gráfica el proceso de extracción de las \caract{s}
de tripletes.
