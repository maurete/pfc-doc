%
%
%
\section{Clasificación de \premirna{s}}
%
Los métodos computacionales para clasificación de \premirna{s} se
basan en técnicas de \e{aprendizaje supervisado}, una disciplina de la
Inteligencia Computacional.
En general, estas técnicas permiten generar un \e{modelo} o
representación interna que caracteriza los \premirna{s}, extrayendo
información a partir de un conjunto de datos de ejemplo denominado
\e{conjunto de entrenamiento}.
Se dice que el modelo generado posee capacidad de \e{generalización},
ya que puede ser aplicado sobre nuevos ejemplos que no han sido
presentados con aterioridad.
El modelo funciona como un \e{clasificador}, discriminando los
ejemplos en dos clases: \e{positiva}, asociada a \premirna{s}
``reales'', y \e{negativa}, asociada a los ejemplos que no se
reconocen como \premirna{s}.

En el presente trabajo se utilizan dos técnicas de aprendizaje
supervisado para la generación de clasificadores de \premirna{s}: el
\e{Perceptrón Multicapa} \cite{mlp1,mlp2} y la \e{Máquina de
  Vectores de Soporte} \cite{svm}.

El Preceptrón Multicapa (\e{MLP}, del inglés \eng{Multilayer
  Perceptron}) es un tipo de red neuronal artificial con propagación
hacia adelante, en la que se disponen nodos computadores (neuronas)
organizados en \e{capas}.
La salida de cada neurona se determina al aplicar una \e{función de
  activación}, de tipo sigmoidea, a una suma ponderada de las salidas
en la capa anterior.
Durante el entrenamiento del perceptrón multicapa, se ajustan
progresivamente los pesos (ponderaciones) de cada neurona mediante un
algoritmo de aprendizaje basado en la \e{propagación hacia atrás} del
error de clasificación, hasta satisfacer un criterio de corte
\cite{jain}.

La Máquina de Vectores de Soporte (\e{SVM}, de su nombre en inglés
\eng{Support Vector Machine}) es un algoritmo de clasificación que se
basa en transformar, mediante una función llamada \e{núcleo}, el
espacio $N$-dimensional de los datos de entrada en otro espacio de
dimensión $M: M\gg N$, donde se espera que los datos sean linealmente
separables mediante un hiperplano.
El entrenamiento consiste en encontrar el hiperplano óptimo de
separación para el conjunto de datos presentado \cite{bottou}.
