%
%
\subsection{Búsqueda en la grilla}
%
La \e{búsqueda en la grilla} propuesta en \cite{hsu}, permite
optimizar los hiperparámetros de regularización $C$ y $\gamma$ cuando
se utiliza un núcleo RBF.
La idea general de esta estrategia es considerar cada combinación de
hiperparámetros $(C,\gamma)$ como puntos en el plano $C\gamma$.
Comenzando por una serie de puntos espaciados regularmente
en espacio logarítmico $(\log C,\log\gamma)$, que determinan la
``grilla'' inicial, se entrena y evalúa el clasificador sobre cada
punto (cada combinación de hiperparámetros), para luego interpolar
(``refinar la grilla'') en las cercanías de los puntos donde se
obtiene la mayor tasa de clacificación $G_m$.
La búsqueda continúa repitiendo el procedimiento sobre los puntos no
evaluados hasta satisfacer un criterio de corte.

La grilla inicial viene dada por combinaciones de los siguientes
puntos de muestreo para los hiperparámetros
%
\begin{align}
  \label{initial-grid}
  \log_2 C     \tab= -5, -3, -1, 1, \ldots, 15, \tabs
  \log_2\gamma \tab= -15,-13, -11, \ldots, 3.
\end{align}
%
Para cada punto $(C_i,\gamma_j)$, se entrena y prueba el clasificador
SVM, obteniendo un ${G_m}_{ij}$ promedio de validación cruzada.

En etapas sucesivas, se interpola la grilla alrededor de los puntos
donde se obtuvieron los mejores valores $G_m$. Se definen tres
algoritmos heurísticos que determinan cuáles puntos interpolar:
%
\begin{itemize}
\item
  \e{Zoom}: En un primer paso, se convoluciona la grilla con una
  ventana cuadrada uniforme de valor unitario, obteniendo una versión
  ``suavizada'' de la grilla. Se interpolan puntos en una región
  cuadrada centrada en el punto con mayor valor $G_m$ ``suavizado''.
\item
  \e{Umbral}: A partir de un umbral $G_m$ definido por el usuario,
  por defecto el percentil 90, se interpola en cada dimensión
  alrededor de los puntos por encima del umbral.
\item
  \e{$n$-mejores}: Similar al umbral, se seleccionan $n$ puntos
  con los mayores valores de $G_m$. Se interpola la grilla en ambas
  dimensiones alrededor de estos puntos.
\end{itemize}
%
Una vez interpolada la grilla, se calcula el valor $G_m$ promedio de
validación cruzada para los puntos interpolados. El procedimiento de
refinamiento se repite hasta un máximo de $N$ iteraciones o hasta
satisfacer un criterio de corte.

La búsqueda en la grilla tiene la ventaja de ser conceptualmente
simple, sin embargo, se torna inviable cuando el vector de parámetros
$\B{\theta}$ contiene más de 2 elementos.
Dado que se trata de un método de búsqueda exhaustiva, resulta
generalmente lento.
Cuando se trabaja con un clasificador SVM con núcleo lineal, la grilla
tiene una única dimensión: la del hiperparámetro $\log C$.
