%
%
\subsection{Búsqueda en la grilla}
%
La \e{búsqueda en la grilla} es una estrategia que optimiza los
\hparam{s} de regularización ($C$) y de amplitud del núcleo gaussiano
($\gamma$), basada en la estrategia propuesta en \cite{hsu,libsvm}.
La idea básica es considerar las combinaciones de $(C,\gamma)$ como
coordenadas en el ``espacio de los \hparam{s}''.
En cada punto $(C,\gamma)$, se entrena el clasificador con los
\hparam{s} correspondientes y se evalúa el resultado de validación
cruzada.
La búsqueda comienza con una ``grilla'' de puntos espaciados a
intervalos regulares y continúa interpolando en las cercanías de
aquellos puntos donde se obtuvo las mayores tasas de clasificación de
validación cruzada.
La grilla inicial viene dada por las coordenadas $C$-$\gamma$
%
\begin{align}
  \label{initial-grid}
  \log_2 C     \tab= -5, -3, -1, 1, \ldots, 15, \tabs
  \log_2\gamma \tab= -15,-13, -11, \ldots, 3.
\end{align}
%
Para cada punto $(C_i,\gamma_j)$, se entrena un modelo con los
\hparam{s} correspondientes y se estima la capacidad de generalización
mediante la medida ${\GM}_{ij}$ promedio obtenida aplicando validación
cruzada.
La interpolación de la grilla se lleva a cabo mediante alguno de los
siguientes algoritmos heurísticos:
%
\begin{itemize}
\item
  \e{Zoom}: interpola una región centrada en el punto con mayor tasa
  de clasificación, reduciendo el área de búsqueda a $1/2$ de la
  amplitud anterior en cada dimensión.
\item
  \e{Umbral}: interpola puntos alrededor de las coordenadas cuyo
  resultado se ubica por encima de un valor ``umbral'', por defecto el
  percentil 90 de toda la grilla.  Este algoritmo es el utilizado por
  defecto.
\item
  \e{$n$-mejores}: similar al umbral, interpola la grilla alrededor de
  los $n$ puntos con mejores resultados de validación cruzada.
\end{itemize}
%
Luego de efectuar la interpolación, se entrena y prueba el
clasificador sobre los nuevos puntos generados.
El procedimiento se repite $N$ iteraciones (por defecto, $N=3$).

La búsqueda en la grilla tiene la ventaja de ser conceptualmente
simple, sin embargo, no resulta escalable a más de dos dimensiones.
Dado que se trata de un método de búsqueda exhaustiva, resulta
comparativamente lento.
Cuando se trabaja con un clasificador SVM con núcleo lineal, se
establece la coordenada $\gamma=0$, y la grilla resultante tiene una
única dimensión, la del hiperparámetro $\log C$.
