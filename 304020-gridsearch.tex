%
%
\subsection{Búsqueda en la grilla}
%
La \e{búsqueda en la grilla} se basa en la estrategia propuesta por
\cite{hsu} y permite optimizar el hiperparámetro de regularización $C$
así como el valor $\gamma$ del núcleo RBF.
La idea general de esta estrategia es considerar cada combinación de
hiperparámetros $(C,\gamma)$ como puntos en el plano.
Comenzando por una serie de puntos espaciados regularmente
en espacio logarítmico $(\log C,\log\gamma)$, que determina la
``grilla'' inicial, se entrena y evalúa el clasificador sobre cada
punto (cada combinación de hiperparámetros).
Luego se interpola (``refina'') la grilla en las cercanías de los
puntos donde se obtienen los mejores resultados según una métrica tal
como \GM.
La búsqueda continúa repitiendo el procedimiento sobre los puntos no
evaluados hasta satisfacer un criterio de corte.

La grilla inicial viene dada por combinaciones de los siguientes
puntos de muestreo para los hiperparámetros
%
\begin{align}
  \label{initial-grid}
  \log_2 C     \tab= -5, -3, -1, 1, \ldots, 15, \tabs
  \log_2\gamma \tab= -15,-13, -11, \ldots, 3.
\end{align}
%
Para cada punto $(C_i,\gamma_j)$, se entrena y prueba el clasificador
SVM, obteniendo un valor $P_{ij}$ promedio de validación cruzada.
Típicamente, el valor $P_{ij}$ se establece a la medida $\GM_{ij}$.
En etapas sucesivas, se interpola la grilla alrededor de los puntos
donde se obtuvieron los valores máximos de $P_{ij}$.
Los tres siguientes algoritmos heurísticos determinan los puntos a
interpolar:
%
\begin{itemize}
\item
  \e{Zoom}: En un primer paso, se convoluciona la grilla con una
  ventana cuadrada uniforme de valor unitario, obteniendo una versión
  ``suavizada'' de la grilla.  Se interpolan puntos en una región
  cuadrada centrada en el punto con mayor valor $P_{ij}$
  ``suavizado''.
\item
  \e{Umbral}: A partir de un umbral del valor $P$ definido por el
  usuario, por defecto el percentil 90, se interpola en cada dimensión
  alrededor de los puntos por encima del umbral.
\item
  \e{$n$-mejores}: Similar al umbral, se seleccionan $n$ puntos
  con los mayores valores de $P$.
  Se interpola la grilla en ambas dimensiones alrededor de estos
  puntos.
\end{itemize}
%
Una vez interpolada la grilla, se calcula el valor $P$ promedio de
validación cruzada para los nuevos puntos interpolados.
El procedimiento de refinamiento se repite $N$ iteraciones (por
defecto, $N=3$).

La búsqueda en la grilla tiene la ventaja de ser conceptualmente
simple, sin embargo, se torna inviable cuando el vector de parámetros
$\B{\theta}$ contiene más de 2 elementos.
Dado que se trata de un método de búsqueda exhaustiva, resulta
comparativamente lento.
Cuando se trabaja con un clasificador SVM con núcleo lineal, la grilla
tiene una única dimensión: la del hiperparámetro $\log C$.
