%
%
%
\chapter{Introducción}
\setcounter{page}{1}
%
La Inteligencia Computacional es la disciplina que estudia la
aplicación de técnicas inspiradas en la naturaleza para resolver
problemas computacionales.
Una de las aplicaciones más difundidas de las técnicas de Inteligencia
Computacional es la clasificación automática de datos, una forma de
aprendizaje supervisado que genera clasificadores capaces de
discriminar entre diferentes tipos de datos a partir de ejemplos que
se ``enseñan'' a la máquina.
La utilización de medios automáticos de clasificación es una cuestión
que importa especialmente a quienes trabajan con grandes volúmenes de
datos sobre los cuales una estrategia de ``escaneo manual'' resulta
ineficiente.
La identificación de secuencias de ácidos ribonucleicos es una de
estas tareas.

En este Trabajo se presenta un método para la clasificación de
\premirna{s} mediante técnicas de Inteligencia Computacional.
