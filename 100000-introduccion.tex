%
%
%
\chapter{Introducción}
\setcounter{page}{1}
%
En este Informe se describe el desarrollo de un método para la
clasificación de secuencias de \premirna{}, un tipo específico de
ácido ribonucleico (\e{RNA}, del inglés \eng{ribonucleic acid}),
mediante la utilización de Perceptrones Multicapa (MLPs) y Máquinas de
Vectores de Soporte (SVMs), técnicas de la Inteligencia Computacional
para el aprendizaje automático.

La Inteligencia Computacional es la disciplina que estudia la
aplicación de técnicas inspiradas en la naturaleza para resolver
problemas computacionales.
Dentro de su área de interés, la clasificación automática
es una de las aplicaciones más difundidas.
Las técnicas MLP y SVM utilizadas en el presente Trabajo implementan
el aprendizaje supervisado, generando ``modelos'' capaces de
discriminar entre diferentes tipos de datos, a partir de ejemplos que
se ``enseñan'' a la máquina.
La utilización de medios automáticos de clasificación resulta de
interés para tareas que involucran grandes volúmenes de datos, sobre
los cuales resulta imposible un tratamiento manual.
La identificación de secuencias de \premirna{s} --y de
cadenas de ácidos nucleicos en general-- es una tarea que
presenta estas \caract{s}.
%% Los ácidos ribonucleicos (ARN o RNA, del inglés \e{ribonucleic acid})
%% son un tipo de ácido nucleico formado por una cadena de nucleótidos,
%% cada uno de los cuales contiene una de las cuatro bases nitrogenadas:
%% adenina, guanina, citosina y uracilo.
%%
%% La naturaleza secuencial de los RNA hace que su representación
%% computacional sea sencilla, en forma de cadenas de caracteres que
%% representan la base de cada nucleótido.
