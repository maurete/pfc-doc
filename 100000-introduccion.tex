\chapter{Introducción}

Internet ha transformado la vida de las personas y el funcionamiento
de las organizaciones. Hoy en día el software ya no se limita a
funcionar como soporte del esquema de negocios en las empresas, sino
que se ha convertido en un componente integral de cada aspecto del
funcionamiento de las mismas. Más aún, el software está presente en
innumerables aspectos de la vida cotidiana de las personas. Los
usuarios confían en el buen funcionamiento de los servicios de
software para llevar adelante su rutina diaria. Las organizaciones no
escapan a esta realidad: cada vez más, el éxito de un proyecto viene
dado por la capacidad de ofrecer servicios de software que satisfagan
las exigencias de los usuarios.

El principal desafío que enfrentan las organizaciones productoras de
software es poder responder en tiempo y forma a las demandas de los
usuarios. En muchas organizaciones, la entrega de una nueva
funcionalidad se mide en meses o, en el mejor de los casos,
semanas. En una visión clásica, el problema de la \e{entrega del
  software} se responde aplicando \e{ingeniería del software}. Se
consideran aspectos tales como la ingeniería de requerimientos, la
planificación, el diseño, la codificación, el testing y la validación
de requerimientos. En muchas organizaciones, incluso se utilizan
metodologías ágiles muy productivas para administrar el
desarrollo. Sin embargo, el software no puede considerarse
\e{entregado} hasta no estar publicado y funcionando sobre una
infraestructura bien configurada que le da soporte.

Las prácticas DevOps surgieron en la industria del software como
respuesta a la necesidad de entregar el software con mayor rapidez. En
su aspecto más fundamental, DevOps intenta resolver el problema de la
\e{entrega} del software desde un punto de vista global, desde
que ingresa el requerimiento hasta que el software se encuentra
operativo a los ojos del cliente. Aplicando una visión analítica de
todas las etapas de la producción del software, con prácticas de
mejora continua, automatización y una gestión en base a criterios
cuantificables, la incorporación de prácticas DevOps permite alcanzar
entregas de software confiables, rápidas y de bajo riesgo
\cite{humblefarley}.

\section{Motivación}

El presente proyecto se origina dentro del área de infraestructura de
la Dirección de Informatización y Planificación Tecnológica (DIPT) de
la Universidad Nacional del Litoral (UNL). La Dirección se encarga de
brindar servicios de software a la comunidad universitaria. El área de
infraestructura está conformada por un equipo de cinco administradores
de sistemas que, entre otras tareas, se encargan de la operación de
los servicios de software ofrecidos por la Dirección.

Desde hace un tiempo, el equipo de infraestructura ha emprendido la
tarea de automatizar su trabajo escribiendo piezas de código que le
permitieran reducir el tiempo invertido en tareas repetitivas. En
forma simultánea, a partir de las demandas de los equipos de
desarrollo se implementaron formas de lograr entornos reproducibles
donde probar el software. Otro hito relevante para este proyecto puede
identificarse en el momento que el equipo de infraestructura codificó
scripts para permitir a los desarrolladores aplicar modificaciones en
las bases de datos y actualizar las versiones de los servicios. De
este modo, se logró ofrecer a los desarrolladores el ``autoservicio'' de
dos de las operaciones más solicitadas, sin requerir intervención de
los administradores.

Estas acciones pusieron en evidencia los beneficios de fomentar la
colaboración entre los equipos de desarrollo e infraestructura, ya que
las mismas derivaron en menos frustraciones a la espera de la
resolución de tickets, menos trabajo tedioso y canales de comunicación
productivos entre ambos equipos. El trabajo realizado en este proyecto
es la continuación lógica de estas acciones, y busca dar formalidad a
la colaboración y la búsqueda de una cultura en común entre los
diversos equipos de trabajo que forman parte de la Dirección.

La principal motivación para el desarrollo de este proyecto fue la
necesidad de aumentar la productividad de los equipos, particularmente
en las áreas de desarrollo y de infraestructura. Se necesitaba reducir
el nivel de ocupación de los individuos, ya que se había alcanzado un
punto de saturación, lo que limitaba la capacidad de respuesta a los
requerimientos y la implementación de mejoras.

Desde la gerencia se determinaron además una serie de necesidades
concretas a basadas en la política de gestión de la Dirección:

\begin{itemize}
\item Contar con un servicio de integración y entrega continuas.
\item Generar una cultura en común entre los equipos, con especial
  énfasis en la gestión del código fuente, los artefactos y las
  operaciones.
\item Disponer de herramientas que agilicen la comunicación interna y
  canalicen las alertas.
\end{itemize}
\section{Objetivos}

El objetivo general del proyecto es implementar en la DIPT prácticas y
herramientas acordes a la filosofía DevOps para mejorar la gestión de
los servicios de software. Esta implementación estará orientada hacia
la automatización de procesos, la observabilidad en los servicios y la
generación de una cultura en común entre los equipos de desarrollo e
infraestructura.

\subsection{Objetivos específicos}

\begin{itemize}
\item Identificar y solucionar deficiencias en la infraestructura en
  general.
\item Aplicar automatización en la infraestructura para eliminar
  tareas repetitivas y ofrecer ``autoservicio'' de la infraestructura
  que da soporte directo a los servicios de software
  (aprovisionamiento de instancias y de capacidad de hardware).
\item Implementar un servicio de integración y entrega continuas
  (CI/CD).
\item Analizar los servicios de software y generar una propuesta de
  mejora que incluya la implementación de CI/CD, trabajando en
  conjunto con los equipos de desarrollo involucrados.
\item Implementar las herramientas necesarias para:
\begin{itemize}
\item \e{Aprovisionamiento}: gestionar la infraestructura.
\item \e{Observabilidad}: visualizar métricas y logs (registros)
  de los servicios.
\item \e{Chat}: comunicación interna para los equipos.
\end{itemize}
\item Integrar el trabajo de los equipos de desarrollo y de
  infraestructura sobre los servicios, unificando en un mismo
  repositorio el código fuente del servicio y la configuración de la
  infraestructura.
\item Escribir documentación de buenas prácticas y capacitar a los
  equipos involucrados para optimizar el flujo de trabajo en las
  diferentes áreas.
\end{itemize}
\subsection{Beneficios de la implementación de DevOps en la DIPT}

Se pretende que la implementación de prácticas DevOps en la Dirección
permita alcanzar una gestión más ágil de los servicios de software,
incrementar la comunicación entre los equipos y automatizar las
operaciones permitiendo en general lograr una entrega más rápida de
valor a los usuarios. En particular, se espera alcanzar mejoras en los
siguientes aspectos de la gestión del software:

\begin{itemize}
\item Reducción del tiempo de entrega de las nuevas funcionalidades.
\item Reducción del nivel de estrés de los equipos.
\item Incremento de la calidad del software entregado.
\item Generación de una cultura en común entre los diferentes equipos
  de desarrollo y de infraestructura.
\item Mejoras en la comunicación entre los equipos, con herramientas
  específicas y procedimientos bien definidos.
\item Cultura de responsabilidad compartida del funcionamiento de los
  servicios entre los desarrolladores y el equipo de operaciones.
\item Simplificar y agilizar las tareas de operaciones, avanzando
  hacia un esquema de ``autoservicio'' de la infraestructura para los
  equipos de desarrollo y testing.
\end{itemize}
Además la implementación de DevOps sentará las bases para, en un
futuro:

\begin{itemize}
\item Desarrollar aplicaciones escalables según demanda y
  requerimientos.
\item Independizar los servicios de la infraestructura subyacente, en
  particular de las diferentes tecnologías de virtualización y de los
  proveedores de cómputo y almacenamiento.
\end{itemize}
\section{Alcance}

El desarrollo de este proyecto es el puntapié inicial para formalizar
la adopción de la metodología DevOps en la DIPT. Dada la diversidad
encontrada entre los diferentes equipos y servicios dentro de la
Dirección, no se pretende efectuar una transformación radical: esto
iría en contra de los principios de DevOps. En cambio, se busca
alcanzar una implementación progresiva sobre los servicios conforme
aumente el grado de adopción de las nuevas herramientas y se
fortalezca la cultura en común.

\section{Exclusiones}

En el desarrollo del proyecto no se pretende modificar las
metodologías actuales de ingeniería de requerimientos o la ingeniería
del software más allá de recomendaciones de la gestión del código
fuente. El proyecto se centra en la implementación de herramientas
tecnológicas para agilizar los procesos y la capacitación en el uso de
las mismas. El proyecto no desarrolla en forma explícita los cambios
culturales propuestos por DevOps, ya que los mismos surgirán con el
paso del tiempo, conforme avance el grado de adopción de las nuevas
prácticas y de las herramientas a implementar.

\section{Restricciones}

\begin{itemize}
\item Software libre: la reglamentación de la Organización impide la
  contratación y/o despliegue de software considerado restrictivo, por
  lo que las herramientas a utilizar deberán ser de licencia libre
  \cite{unllibre}.
\item En premisas: las soluciones propuestas deberán estar alojadas en
  premisas de la Organización.
\item Se requiere utilizar Git como herramienta de control de
  versiones del código fuente.
\item Se mantendrá la herramienta actual de gestión de proyectos
  Redmine.
\item Dado el grado de adopción actual, se requiere utilizar Ansible
  como herramienta de gestión automatizada de la configuración.
\end{itemize}
