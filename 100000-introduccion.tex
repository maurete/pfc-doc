%
%
%
\chapter{Introducción}
\setcounter{page}{1}
%
La Inteligencia Computacional es la disciplina que estudia la
aplicación de técnicas inspiradas en la naturaleza para resolver
problemas computacionales.
Una de las aplicaciones más difundidas de las técnicas de Inteligencia
Computacional es la clasificación automática de datos, una forma de
aprendizaje supervisado que genera clasificadores capaces de
discriminar entre diferentes tipos de datos a partir de ejemplos que
se ``enseñan'' a la máquina.
La utilización de medios automáticos de clasificación es una cuestión
que importa especialmente a quienes trabajan con grandes volúmenes de
datos sobre los cuales una estrategia de ``escaneo manual'' resulta
ineficiente.
La identificación de secuencias de ácidos ribonucleicos es una de
estas tareas.
Los ácidos ribonucleicos (ARN o RNA, del inglés \e{ribonucleic acid})
son un tipo de ácido nucleico formado por una cadena de nucleótidos,
cada uno de los cuales contiene una de las cuatro bases nitrogenadas:
adenina, guanina, citosina y uracilo.

%% La naturaleza secuencial de los RNA hace que su representación
%% computacional sea sencilla, en forma de cadenas de caracteres que
%% representan la base de cada nucleótido.

En el presente Trabajo se desarrolló un método para la clasificación
de \premirna{s}, un tipo de ARN, haciendo uso de las técnicas de
aprendizaje supervisado de Perceptrón Multicapa (MLP) y Máquina de
Vectores de Soporte (SVM).
