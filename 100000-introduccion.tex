%
%
%
\chapter{Introducción}
\setcounter{page}{1}
%
En el presente Informe se describe el desarrollo de un método para la
clasificación de secuencias de \premirna{}, un tipo específico de
cadenas ácido ribonucleico (\e{RNA}), mediante la utilización de
perceptrones multicapa (\e{MLPs}) y máquinas de vectores de soporte
(\e{SVMs}), técnicas desarrolladas en el ámbito de la Inteligencia
Computacional para el aprendizaje automático (\e{machine learning}).

La Inteligencia Computacional es la disciplina que estudia la
aplicación de técnicas inspiradas en la naturaleza para resolver
problemas computacionales.
Dentro de su área de interés, la clasificación automática es una de
las aplicaciones más difundidas.
Las técnicas MLP y SVM utilizadas en el presente trabajo funcionan en
un esquema de \e{aprendizaje supervisado}, generando ``modelos''
capaces de discriminar entre diferentes tipos de datos a partir de
ejemplos que se ``enseñan'' a la máquina.
La identificación de secuencias de \premirna{s} --y de cadenas de
ácidos nucleicos en general-- es una tarea que involucra grandes
volúmenes de datos, imposibles de procesar manualmente, por lo que se
recurre a las herramientas ofrecidas por el aprendizaje automático.
