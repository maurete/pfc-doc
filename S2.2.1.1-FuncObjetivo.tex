%
\subsubsection{Función objetivo}
%
Una función objetivo sirve de criterio para determinar la mejor
solución a un problema de optimización, asignando a cada caso de
prueba un valor numérico que refleja la idoneidad del modelo obtenido
con determinados hiperparámetros $\Btheta$.
Intuitivamente, se puede decir que una función objetivo estima el
error de generalización del clasificador, y sus variables de entrada
son, directa o indirectamente, el conjunto de datos de entrenamiento y
los hiperparámetros del clasificador.

Una función objetivo ideal es una función continua, derivable, y con
un mínimo global coincidente con el mínimo de la función del error de
generalización. La función objetivo más utilizada es el error
de clasificación sobre algún conjunto de validación, la cual es una
función discontinua y no derivable, ya que se construye a partir de la
experiencia de clasificación de un conjunto de datos finito.
Sin embargo, como se verá más adelante, resulta posible derivar
funciones objetivo derivables que permiten una solución eficiente al
problema de la búsqueda de hiperparámetros óptimos.
