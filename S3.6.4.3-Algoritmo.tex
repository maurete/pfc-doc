\subsubsection{Algoritmo de optimización}
Partiendo de un punto inicial $(C^0,\gamma^0)$ la búsqueda procede
en cada punto $(C^k,\gamma^k)$ evaluando $\rho^k$ y determinando
un nuevo punto $(C^{k+1},\gamma^{k+1})$ en la dirección del
negativo del gradiente $\nabla\rho^k$ tal que $\rho^{k+1}<\rho^k$
La optimización de la función $\rho$ se efectúa mediante el algoritmo
BFGS \cite{bfgs} en el espacio logarítmico de los hiperparámetros
$(\ln(C),\ln(\gamma))$. Este cambio de coordenadas se traduce en un
incremento de la estabilidad numérica y evita tener que verificar en
cada iteración la no-negatividad de $C$ y $\gamma$.

\begin{align}
  \dpar{\rho}{\ln C}{}= C \dpar{\rho}{C}{}, &&
  \dpar{\rho}{\ln \gamma}{}= \gamma \dpar{\rho}{\gamma}{}
\end{align}
Para un conjunto de entrenamiento $D=((\xx_i,y_i),i=1,\ldots,\ell)$
cada evaluación de $\rho(\Btheta_k)$ consta de los siguientes pasos

\begin{enumerate}
\item Entrenar una máquina de vectores de soporte con núcleo RBF
  con hiperparámetros $\Btheta_k=(C_k,\gamma_k)$ sobre el conjunto
  de entrenamiento completo $D$
\item Calcular $\Bbeta_*$ óptimo para el problema (\ref{svm-oneclass})
\item Calcular el valor de $\rho$ como el producto de
  $\rho_M$ (\ref{rho_m}) y $\rho_r$ según (\ref{rho_r})
\item Calcular el gradiente $\nabla\rho$.
\end{enumerate}
