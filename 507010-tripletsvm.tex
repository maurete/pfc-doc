%
%
\subsection{Conjuntos de prueba de Triplet-SVM}
%
En el trabajo \cite{xue} los autores definen tres conjuntos de prueba
suplementarios para el método \work\tripletsvm{} denominados
\dset{updated}, \dset{conserved-hairpin} y \dset{cross-species}.
El conjunto de entrenamiento es el mismo que el utilizado en el
problema principal \prob\tripletsvm{}, con $163$ ejemplos de clase
positiva tomados de \dset{mirbase50} y $168$ ejemplos de clase
negativa provenientes del conjunto \dset{coding}, todos pertenecientes
a la especie humana.
Con el modelo obtenido, se clasifican los tres conjuntos de prueba:
%
\begin{description}
  \item{\dset{updated}}\\
  Consiste en $39$ pre-miRNAs experimentalmente validados
  pertenecientes a la especie humana, que fueron publicados durante el
  desarrollo del método \work\tripletsvm{}.
%
  \item{\dset{conserved-hairpin}}\\
  Este conjunto de prueba consta de $2444$ secuencias con estructura
  secundaria en forma de horquilla, extraídas del cromosoma humano
  $19$. En su mayoría, estas secuencias son ``pseudo'' \premirna{s}
  (de clase negativa), aunque se conoce que algunas de ellas son
  \premirna{s} verdaderos.
%
  \item{\dset{cross-species}}\\
  Contiene $581$ secuencias de \premirna{s} experimentalmente
  validadas pertenecientes a diversas especies no humanas, tales como
  animales, plantas y virus, extraídas de la versión $5$.$0$ de
  \dset\mirbase{}.
\end{description}
%
Los resultados de estas pruebas se muestran en la
\iflatexml{}Tabla~\ref{tbl:suppl-xue}\else\autoref{tbl:suppl-xue}\fi.
En general, se obtuvieron mejores tasas que las presentadas en
\cite{xue}, con la excepción del conjunto \dset{cross-species}, en
donde los clasificadores \ssf{SVM-RR} y \ssf{MLP-B} resultaron
inferiores al método \work\tripletsvm{}.
