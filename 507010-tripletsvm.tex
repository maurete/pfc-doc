%
%
\subsection{Conjuntos de prueba de Triplet-SVM}
%
En el trabajo \cite{xue} los autores definen tres conjuntos de prueba
suplementarios para el método \work\tripletsvm{} denominados
\dset{updated}, \dset{conserved-hairpin} y \dset{cross-species}.
Tal como en el problema \prob\tripletsvm{}, el entrenamiento se
efectúa con 163 ejemplos de clase positiva tomados de \dset{mirbase50}
y 168 ejemplos de clase negativa provenientes del conjunto
\dset{coding}.
Con el modelo obtenido, se clasifican los tres conjuntos de prueba:
%
\begin{description}
  \item{\dset{updated}}\\
  Consiste en 39 pre-miRNAs experimentalmente validados pertenecientes a
  la especie humana, que fueron publicados durante el desarrollo del
  método Triplet-SVM.
%
  \item{\dset{conserved-hairpin}}\\
  Este conjunto de prueba consta de 2444 secuencias conservadas con
  estructura secundaria de horquilla, extraídas del cromosoma humano
  19. En su mayoría, estas secuencias son pseudo pre-miRNAs (clase
  negativa), aunque se conoce que algunas de ellas son
  pre-miRNAs verdaderos.
%
  \item{\dset{cross-species}}\\
  Contiene 581 secuencias de pre-miRNAs experimentalmente validadas
  pertenecientes a diversas especies no humanas, tales como animales,
  plantas y virus, extraídas de la versión 5.0 de miRBase.
\end{description}
%
Los resultados de estas pruebas se muestran en la
\iflatexml{}Tabla~\ref{tbl:suppl-xue}\else\autoref{tbl:suppl-xue}\fi.
En general, se encontró que el método propio obtuvo mejores tasas de
clasificación que el método \work\tripletsvm{}, a excepción del caso
\dset{cross-species}, donde el método propio en sus variantes
\ssf{SVM-RR} y \ssf{MLP-B} resultó inferior al resultado reportado en
\cite{xue}.
