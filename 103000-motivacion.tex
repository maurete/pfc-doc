%
%
%
\section{Motivación}
%
El debate acerca del número e identidad de los miRNAs presentes en
diferentes genomas es una cuestión abierta.
A modo de ejemplo, para el caso del genoma humano se había estimado
inicialmente que se encontrarían unos pocos cientos de genes miRNA;
posteriormente este número ha sido revisado a unos 1000
\cite{sewer,chang}.
Actualmente, se encuentra que el número de miRNAs descubiertos en
humanos duplica esta cifra \cite{gomes}.
Asimismo, este número podría ser mucho mayor si se considera que
existen miRNAs de evolución más reciente, específicos a la especie
humana \cite{sewer}.

Este proyecto se origina en el Centro de Investigación en Señales, 
Sistemas e Inteligencia Computacional \e{sinc(i)} ante la necesidad de
contar con  una herramienta de clasificación de \premirna{s}, que sea
de fácil utilización por parte del usuario no experto en Inteligencia
Computacional.

%Con este desarrollo se pretende que los especialistas \hl{especificar}
%cuenten con una herramienta sencilla para la aplicación de técnicas
%avanzadas de clasificación, e integrada a su flujo de trabajo
%habitual.
