%
%
%
\section{Pruebas del clasificador SVM con núcleo lineal}
%
Se probó el clasificador SVM con núcleo lineal sobre los tres
problemas definidos, aplicando las estrategias de selección trivial,
de búsqueda exhaustiva y del criterio del error empírico para
determinar el valor óptimo del hiperparámetro $C$.
En la \autoref{tbl:linear-results} se muestran los resultados de
obtenidos de clasificar los conjuntos de prueba de cada problema
caracterizados según el valor medio $E$ y la desviación estándar
$\sigma$ para las 5 repeticiones de cada prueba con semillas
aleatorias diferentes.

En líneas generales, los resultados muestran que para las tres
estrategias de selección de hiperparámetros se obtienen tasas de
clasificación similares.
Para el problema \tripletsvm{}, las tasas obtenidas con el
clasificador SVM-lineal resultan similares a aquellas conseguidas
aplicando el clasificador MLP.
Para los problemas \mipred{} y \micropred{}, en cambio, se observa una
leve mejora en los resultados respecto de aquellos opbtenidos para el
clasificador MLP.
En particular, se observa una reducción de la diferencia SP-SE, lo que
indica que este clasificador es menos sensible al desbalance de
clases.

Tal como en el caso del clasificador MLP, la utilización del conjunto
de características S-E por sobre E trae consigo una leve mejora en los
resultados otenidos.
Similarmente, la utilización de las caractarísticas de secuencia (en
forma única) resulta en tasas de clasificación subóptimas.
