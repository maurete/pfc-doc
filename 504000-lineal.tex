%
%
%
\section{Pruebas del clasificador SVM con núcleo lineal}
%
Se probó el clasificador SVM con núcleo lineal sobre los tres
problemas definidos, aplicando las estrategias de selección trivial,
de búsqueda exhaustiva y de minimización del criterio del error
empírico para determinar el valor óptimo del hiperparámetro $C$.
En la \iflatexml{}Tabla~\ref{tbl:linear-results}\else\autoref{tbl:linear-results}\fi{}
se presentan los resultados de clasificar el conjunto de prueba de
cada problema con un modelo generado a partir del conjunto de
entrenamiento del mismo problema.

Los resultados obtenidos con las tres estrategias de selección del
hiperparámetro $C$ resultaron similares en todos los casos.
\hl{Interpretar porque pasa esto?}

Los conjuntos de \caract{s} que incluyen la estructura secundaria
(\dset{E} y \dset{S-E}) obtuvieron las mejores tasas de clasificación,
tal como se observó en el caso del clasificador MLP.

La desviación estándar resultó nula en el problema \prob{\tripletsvm}
utilizando la estrategia de selección trivial.
Esto es esperable dado que el algoritmo SMO, utilizado para el
entrenamiento de la SVM, es determinístico.
Puede decirse entonces que la variabilidad observada en la estrategia
de selección trivial es atribuible exclusivamente a la variación en la
composición de las particiones de entrenamiento y prueba según la
semilla aleatoria.

Considerando el caso del problema \prob\tripletsvm{} y \caract{s} de
tripletes (\dset{T}), se obtuvieron resultados similares a los
reportados por los autores, aunque con una \SP{} levemente inferior.
Tal como con el clasificador MLP, la utilización de los conjuntos de
\caract{s} \dset{E} y \dset{S-E} mejoró significativamente los
resultados obtenidos con las \caract{s} de tripletes.

En el caso de los problemas \prob\mipred{} y \prob\micropred{}, se
observó que la diferencia entre la especificidad (\SP) y la
sensibilidad (\SE) fue menor a la obtenida con el clasificador MLP.
Esto es un indicio de que el clasificador SVM es menos sensible al
``desbalance de clases'' en los datos de entrenamiento.
Al comparar las tasas obtenidas por el clasificador SVM-lineal contra
aquellas del clasificador MLP, se observa una leve mejora en los
resultados obtenidos sobre los problemas \prob\mipred{} y
\prob\micropred{}.
