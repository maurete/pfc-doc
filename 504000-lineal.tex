\section{Pruebas del clasificador SVM con núcleo lineal}
Se probaron las tres estrategias de selección de hiperparámetros
disponibles para el clasificador SVM con núcleo lineal: selección
trivial, búsqueda exhaustiva y criterio del error empírico. En este
clasificador, el hiperparámetro a ajustar es la variable de
regularización $C$.  En la \autoref{tbl:linear-results} se muestran
los resultados de clasificación obtenidos para los conjuntos de prueba
de cada problema caracterizados según el valor medio $E$ y la
desviación estándar $\sigma$ para las 5 repeticiones de cada prueba
con semillas aleatorias diferentes.

En líneas generales, los resultados muestran que para las tres
estrategias de selección de hiperparámetros se obtienen tasas de
clasificación similares.  Para el problema \tripletsvm{}, las tasas
obtenidas con el clasificador SVM-lineal resultan similares a aquellas
conseguidas aplicando el clasificador MLP. Para los problemas
\mipred{} y \micropred{}, en cambio, se observa una leve mejora en los
resultados. En particular, se observa una reducción de la diferencia
SP-SE, lo que indica que este clasificador es menos sensible a un
desbalance de clases.

Tal como en el caso del clasificador MLP, la utilización del conjunto
de características S-E por sobre E trae consigo una leve mejora en los
resultados otenidos. Similarmente, la utilización de las
caractarísticas de secuencia (en forma única) resulta en tasas de
clasificación subóptimas.