%
%
%
\section{Pruebas del perceptrón multicapa}
%
Se probó el funcionamiento del perceptrón multicapa sobre los tres
problemas de clasificación definidos, aplicando las estrategias
trivial y de búsqueda exhaustiva para determinar el número de neuronas
en la capa oculta.
%% Se probó el desempeño del Perceptrón Multicapa para los tres problemas
%% de clasificación y las dos estrategias de selección del número de
%% neuronas en la capa oculta: trivial y de búsqueda exhaustiva.
En la \iflatexml{}Tabla~\ref{tbl:mlp-results}\else\autoref{tbl:mlp-results}\fi{}
se presentan los resultados obtenidos luego de generar el modelo
de clasificador con el conjunto de entrenamiento definido en cada
problema y clasificar el respectivo conjunto de prueba.
Los resultados se presentan discriminados según el conjunto de
\caract{s} y la estrategia de selección del número de neuronas en la
capa oculta.

Resulta necesario remarcar que, debido a la inicialización aleatoria
de los pesos de la red, al intentar replicar estas pruebas, los
resultados presentarán ligeras variaciones .

