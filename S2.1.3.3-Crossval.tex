%
\subsubsection{Validación cruzada de $k$ iteraciones}
%
En un procedimiento de \e{validación cruzada de $k$ iteraciones}
\cite{crossval}, el conjunto de entrenamiento $D$ se subdivide en $k$
conjuntos disjuntos $V_1,\ldots,V_k$ de tamaños similares denominados
\e{particiones de validación}.
Para cada $i=1,\ldots,k$, se efectúa el entrenamiento sobre un
conjunto $D_i=D\setminus{}V_i$ excluyendo a $V_i$, y se evalúa el
modelo obtenido con los datos de $V_i$.  Una vez completado el
proceso, se dispondrá de $k$ medidas de desempeño que pueden ser
promediadas para obtener una estimación del error de generalización
del modelo. Este resultado se utiliza comúnmente para ajustar los
hiperparámetros que regulan el proceso de entrenamiento, antes de
entrenar un modelo definitivo sobre el conjunto completo $D$.

%% El error de validación cruzada presenta una menor varianza que aquel
%% del método de retención, sin embargo, al no ser los subconjuntos $D_i$
%% independientes e idénticamente distribuidos entre sí, se introduce un
%% sesgo en la estimación del error de generalización.  Tal como en el
%% caso del método de retención, en conjuntos de datos pequeños el método
%% de validación cruzada presenta una alta varianza respecto de la
%% elección de diferentes $D_i$.
