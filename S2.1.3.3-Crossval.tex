%
\subsubsection{Validación cruzada de $k$ iteraciones}
%
En un procedimiento de \emph{validación cruzada de $k$ iteraciones}
\cite{crossval}, el conjunto $D$ se subdivide en $k$ conjuntos
disjuntos $D_1,\ldots,D_k$ de tamaños similares denominados
\e{particiones}.

El entrenamiento de la máquina de aprendizaje se efectúa sobre $k-1$
particiones, y se clasifican aquellos datos de la partición restante
(llamada \e{de validación}) para obtener una medida de desempeño de
clasificación de la partición actual. Se repite el procedimiento $k$
veces, seleccionando en cada iteración una partición diferente para
validación.  Una vez completado el proceso, se dispondrá de $k$
medidas de desempeño de la máquina de aprendizaje. Estas medidas
pueden ser promediadas para obtener un estimador del desempeño de la
máquina de aprendizaje. Este estimador es una medida más acertada que
el error de entrenamiento para estimar el error de generalización del
modelo.

El error de validación cruzada presenta una menor varianza que aquel
del método de retención, sin embargo, al no ser los subconjuntos $D_i$
independientes e idénticamente distribuidos entre sí, se introduce un
sesgo en la estimación del error de generalización.  Tal como en el
caso del método de retención, en conjuntos de datos pequeños el método
de validación cruzada presenta una alta varianza respecto de la
elección de diferentes $D_i$.
