%
%
\subsection{El clasificador lineal}
%
Las máquinas de vectores de soporte se basan en un clasificador simple
llamado clasificador lineal \cite{nilsson}.
Dado un vector de entrada $\xx\in{}X$, el clasificador lineal
determina la clase de $\xx$ según
%
\begin{align*}
  y = \T{signo}(\langle\vv,\xx\rangle+b)
\end{align*}
%
en donde $\vv$ y $b$ son los parámetros del modelo obtenidos mediante
un algoritmo de entrenamiento.
%
\subsubsection{Espacio imagen}
%
La primera adición de la SVM al clasificador lineal básico es una
transformación de los vectores de entrada a un \e{espacio imagen}.
Esto se logra aplicando una función $\BPhi:X\rightarrow{}Z$, que
transforma el \e{espacio de entrada} $X$ a otro espacio vectorial
inducido $Z$, llamado \e{espacio imagen}.
La elección de la función $\BPhi$ es tal que los datos transformados
sean separables en el espacio $Z$.
Dada una entrada $\xx\in{}X$, la máquina de vectores de soporte
calcula un vector imagen $\zz=\BPhi(\xx)$ y le asigna una clase de
salida $\hat{y}=\pm{}1$ según
%
\begin{align*}
  y = \T{signo}(\ww^T \BPhi(\xx)+b)
\end{align*}
%
Los valores de $\ww$ y $b$ se obtienen del algoritmo de entrenamiento,
que se explicará más adelante.
La ecuación $\ww^T\BPhi(\xx)+b=0$ define un hiperplano en el espacio
imagen $Z$, que actúa como frontera de decisión entre las clases
$\hat{\yy}=+1$ e $\hat{\yy}=-1$.
Para visualizarlo, obsérvese que, desde el punto de vista geométrico,
el signo de $\ww^T\BPhi(\xx)+b$ indica de qué lado del plano se
encuentra el punto $\zz$.
%
\begin{quote}
  {\bfseries Notación.}\quad{}En la literatura, resulta común
  encontrar que el vector $\BPhi(\xx)$ se denomina \e{vector de
    características} (\e{feature vector}) y el espacio vectorial $Z$
  como \e{espacio de las características} (\e{feature space}).  En
  este trabajo se prefiere denominarlos con los nombres alternativos
  \e{vector imagen} y \e{espacio vectorial imagen} para evitar
  confusión con el proceso previo de extracción de características.
\end{quote}
%
