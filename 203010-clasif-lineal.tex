%
%
\subsection{El clasificador lineal}
%
Las máquinas de vectores de soporte se basan en un clasificador simple
llamado clasificador lineal \cite{nilsson}.
Dado un vector de entrada $\xx\in{}X$, el clasificador lineal
determina la clase de $\xx$ según
%
\begin{align*}
  \hat{y} = \T{signo}(\pint{\vv}{\xx}+b)
\end{align*}
%
en donde $\vv$ y $b$ son los parámetros del modelo obtenidos mediante
un algoritmo de entrenamiento.
La ecuación $\pint{\vv}{\xx}+b=0$ define un hiperplano en el
espacio lineal $X$, que actúa como frontera de decisión entre las
clases $\hat{y}=+1$ e $\hat{y}=-1$.
Para visualizarlo obsérvese que, desde el punto de vista geométrico,
el signo de $\pint{\vv}{\xx}+b$ indica de qué lado del plano se
encuentra el punto $\xx$.
