%
%
%
\chapter*{Glosario}
%
{\relscale{0.9}
\begin{description}
  \item[Servicio, servicio de software.] Producto de software en un
  estado productivo, con capacidad de ser utilizado por parte del
  usuario.
%
  \item[Usuario, cliente.] Persona, conjunto de personas u
  organización que se benefician con la utilización de un servicio de
  software.
%
  \item[Organización, Dirección (como nombre propio).] El ámbito de
  aplicación del Proyecto, la Dirección de Informatización y
  Planificación Tecnológica de la Universidad Nacional del Litoral
  (DIPT-UNL).
%
  \item[Equipo, equipo de X.] Conjunto de personas que trabajan en un
  área determinada X, tal como desarrollo o infraestructura.
%
  \item[Desarrollo.] Refiere a todas las actividades necesarias para
  la producción del software, hasta la publicación de un
  \textit{artefacto}. Incluye la ingeniería, planificación,
  codificación, verificación y publicación del software.
%
  \item[Artefacto.] Archivo o conjunto de archivos que puede ser
  ejecutado o desplegado en una computadora.
%
  \item[Operación.] Cualquier tarea relativa a la puesta en
  funcionamiento o el mantenimiento del software, tales como el
  despliegue, la actualización, la modificación de datos y la
  configuración.
%
  \item[Despliegue, entrega.] Operación de puesta en funcionamiento
  del software en una instancia que forma parte de la
  infraestructura. Cuando esta puesta en funcionamiento alcanza al
  usuario final, se denomina entrega.
%
  \item[Infraestructura.] Conjunto de recursos físicos o virtuales que
  dan soporte al funcionamiento del software. Incluye servidores,
  componentes de red, reglas de firewall, servicios de virtualización e
  instancias virtualizadas.
%
  \item[Equipo de infraestructura, equipo de operaciones.] En el
  ámbito de la DIPT, refiere al grupo de personas que trabajan en el
  área de infraestructura, quienes además están a cargo de realizar las
  operaciones sobre los servicios de software. En este ámbito, ambos
  conceptos se utilizan de forma intercambiable.
%
  \item[Instancia.] Un servidor virtual sobre el cual se ejecutan los
  servicios de software. En el ámbito de la DIPT en general se dispone
  de una instancia por cada servicio ofrecido.
\end{description}
}
