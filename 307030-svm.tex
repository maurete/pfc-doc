%
%
\subsection{Generación del modelo SVM}
%
Las estrategias de selección de \hparam{s} se codificaron en las
funciones \func{select\_model\_trivial},
\func{select\_model\_gridsearch}, \func{select\_model\_empirical},
\func{select\_model\_rmb}, que en general reciben como argumentos de
entrada los datos de entrenamiento y el tipo de núcleo a utilizar, y
retornan a la salida los \hparam{s} y un modelo final parametrizado
entrenado sobre el conjunto de entrenamiento.
%
\funcentry
    {select\_model\_trivial}
    {Datos de entrenamiento, tipo de núcleo a utilizar}
    {Hiperparámetros $C=1$, $\gamma=\frac{1}{2F}$, modelo entrenado}
    {Retorna los valores $C=1$ y $\gamma=\frac{1}{2F}$
      para un clasificador con núcleo RBF, donde $F$ es la longitud
      del vector de características, junto a un modelo entrenado con
      estos hiperparámetros.}
%
\funcentry
    {select\_model\_gridsearch}
    {Datos de entrenamiento, núcleo a utilizar, número de iteraciones de
      refinamiento (por defecto: 3)}
    {Hierparámetros $C$, $\gamma$ óptimos, modelo entrenado}
    {Implementa la estrategia de búsqueda en la grilla.
      La ``grilla'' es una estructura en memoria con múltiples
      matrices de igual dimensión, en las que se guardan las
      coordenadas $C$, $\gamma$, y los resultados de la validación
      cruzada.
      Las estrategias de ``refinamiento'' codificadas en las funciones
      \func{grid\_nbest}, \func{grid\_zoom}, \func{grid\_threshold}
      interpolan de estas matrices insertando nuevas coordenadas $i,j$
      a validar en la siguiente iteración, repitiendo el procedimiento
      hasta alcanzar el número máximo de iteraciones.
      La función soporta el cálculo paralelo de múltiples coordenadas
      simultáneas para aumentar la velocidad de cómputo.}
%
%% Las funciones \func{select\_model\_rmb} y
%% \func{select\_model\_empirical} consisten básicamente en una
%% invocación a la función de optimización por descenso de gradiente
%% \func{opt\_bfgs} con la función objetivo correspondiente.
%
\funcentry
    {select\_model\_empirical}
    {Datos de entrenamiento, núcleo a utilizar}
    {Hiperparámetros $C$, $\gamma$ óptimos, modelo entrenado}
    {Implementa la estrategia de selección de hiperparámetros
      minimizando la función error empírico.
      Consiste principalmente en una invocación a la función de
      optimización \func{opt\_bfgs} parametrizada con la función
      objetivo que calcula el error empírico
      \func{error\_empirical\_cv}.
      Una vez encontrados los \hparam{s} óptimos, se entrena un modelo
      SVM invocando a \func{mysvm\_train}.}
%
\funcentry
    {select\_model\_rmb}
    {Datos de entrenamiento}
    {Hiperparámetros óptimos, modelo SVM-RBF entrenado}
    {Implementa la selección de hiperparámetros minimizando la cota
      radio-margen.
      Invoca a la función de optimización \func{opt\_bfgs} con la
      función objetivo \func{error\_rmb\_csvm}.
      Una vez encontrado el punto $C,\gamma$ óptimo, entrena una SVM
      que es retornada como salida de la función.}
%
