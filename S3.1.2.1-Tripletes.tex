%
%
\subsubsection{Características de tripletes}
%
Una propiedad distintiva de los \premirna{s} es su estructura
secundaria en forma de horquilla. En \cite{xue}, \citeauthor{xue}
proponen un método para diferenciar ejemplos de \premirna{s}
``reales'' de otras secuencias con la misma estructura secundaria,
denominadas ``pseudo \premirna{s}'', observando que la distribución de
sub-estructuras locales difiere significativamente entre unos y otros.
A partir de estas observaciones, proponen generar un conjunto
de \caract{s} que representa la distribución de las sub-estructuras
locales dentro del ``tallo'' de la horquilla.

Los tripletes son pequeñas cadenas de caracteres que relacionan la
base de cada nucleótido y su estructura secundaria local.  Para un
nucleótido en la posición $i$, el triplete correspondiente es una
cadena de 4 caracteres que contiene la base (\ntA, \ntC, \ntG, o \ntU)
seguida de la sub-estructura secundaria local en $i-1,i,i+1$,
representada en forma binaria como ``acoplado'' \pairL (paréntesis) y
``no acoplado'' \noPair (punto).

En la \iflatexml{}Figura~\ref{triplet}\else\autoref{triplet}\fi{} se
muestra un \premirna{} con estructura secundaria en forma de horquilla
y se describen sus diferentes partes según la estructura secundaria.
Por ejemplo, el ``tallo'' de la horquilla es la región resultante
luego de descartar el bucle central y los extremos no acoplados.

Contando el número de ocurrencias de cada una de las 32 combinaciones
posibles de tripletes en la región del tallo, se obtiene un vector que
caracteriza la distribución de las sub-estructuras locales.  Además,
se incorporan 4 medidas auxiliares que resultan del cálculo de los
tripletes:
%
\begin{itemize}
\item longitud del tallo,
\item número de pares de bases,
\item complementariedad de ambos brazos de la horquilla,
\item proporción de bases \ntG y \ntC en la zona del tallo.
\end{itemize}
%

La principal limitación de las características de tripletes es que su
cálculo resulta posible únicamente para ejemplos en forma de horquilla
con un bucle central único, ya que de otro modo la definición del
``tallo'' pierde sentido. Por ello, estas \caract{s} no se calculan
cuando el ejemplo contiene bucles múltiples en su estrucutra secundaria.

%% En la implementación, se diferencian las \caract{s} de distribución de
%% los tripletes (llamadas ``triplet'') de las \caract{s} auxiliares
%% (llamadas ``triplet-extra'').
A continuación se tabulan las 36 características de tripletes y su
posición en el vector de características.
