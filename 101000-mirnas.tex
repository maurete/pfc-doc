%
%
%
\section{Acerca de los pre-miRNAs}
%
Los \premirna{s} o \e{miRNAs precursores} son pequeñas moléculas
de RNA de unos $70\,[\nt]$ de longitud, que incorporan uno o más
miRNAs dentro de su secuencia.

Los miRNAs o microRNAs son cadenas de RNA de unos $22\,[\nt{}]$ de
longitud que están involucradas en múltiples procesos celulares, tales
como la reproducción y la diferenciación en distintos tipos de
tejidos, y ejercen una función reguladora de la expresión génica
celular, interviniendo en la transcripción de mRNAs (\e{RNA
  mensajeros}) y la síntesis de proteínas \cite{lee-mammal}
\cite{bartel116} \cite{lili}.
Este efecto regulador puede tener gran implicancia en el desarrollo y
evolución de la enfermedad celular.
Se ha demostrado que los miRNAs juegan un rol importante en la
carcinogénesis \cite{aurora,lili}, regulando la proliferación y muerte
celular, entre otras funciones tales como el metabolismo de las grasas
en moscas, los procesos de infección viral, y el control del
desarrollo de flores y hojas en plantas \cite{bartel116,lecellier}.

En su estado natural, las moléculas de RNA se ``pliegan'' sobre sí
mismas, logrando una mayor estabilidad molecular.
La representación bidimensional de esta forma ``plegada'' se denomina
\e{estructura secundaria}.
DEntro del reino animal, la mayoría de los \premirna{s} presenta una
estructura secundaria característica en forma de ``horquilla''
\cite{bartel116,sewer}.
En la \iflatexml{}Figura~\ref{horquilla}\else\autoref{horquilla}\fi{}
se representa la secuencia de una cadena de \premirna{} y
su correspondiente estructura secundaria.
