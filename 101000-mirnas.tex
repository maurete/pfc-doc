%
%
%
\section{Acerca de los pre-miRNAs}
%
Los \e\premirna{s} o \e{miRNA precursores} son pequeñas moléculas de
ácido ribonucleico (\e{RNA}, de \e{ribonucleic acid}) compuestas por
una cadena de %% alrededor de $70$
nucleótidos $[\nt]$, que en los organismos cumplen la función de
``contenedores'' de secuencias más pequeñas, denominadas \e{microRNAs}
o \e{miRNAs}.
Se ha demostrado que los miRNAs están involucrados en múltiples
procesos dentro de las células, ejerciendo una función reguladora de
la expresión génica celular mediante la interferencia de la
transcripción de mRNAs (\e{RNA mensajeros}) y de la síntesis de
proteínas \cite{lee-mammal} \cite{bartel116} \cite{lili}.
Este efecto regulador puede tener gran implicancia en el desarrollo y
evolución de la enfermedad celular:
se ha demostrado que los miRNAs regulan los procesos de
diferenciación, reproducción y muerte celular, por lo que tienen un
rol importante en la carcinogénesis \cite{aurora,lili}.
Asimismo, intervienen en otras funciones esenciales tales como el
metabolismo de las grasas, los procesos de infección viral, y el
control del desarrollo de flores y hojas en plantas
\cite{bartel116,lecellier}.

La identificación de \premirna{s} resulta de interés como un paso
previo a la identificación de los miRNAs, dado que poseen
particularidades que los diferencian de otros tipos de secuencias de
RNA.
Entre estas particularidades se puede mencionar la \e{estructura
  secundaria}, que es la \e{forma} bidimensional que las cadenas de
RNA adquieren ``plegándose'' sobre sí mismas para alcanzar la
estabilidad molecular.
En el reino animal, la estructura secundaria de los \premirna{s}
presenta una forma de ``horquilla'' característica, mientras que en
las plantas suelen mantener una estructura de \e{tallos} y \e{bucles}
con ramificaciones \cite{bartel116,sewer}.
En la \iflatexml{}Figura~\ref{horquilla}\else\autoref{horquilla}\fi{}
se representa una cadena de \premirna{} típica de la especie humana
junto con su correspondiente estructura secundaria.
