%
%
%
\section{Acerca de los (pre-)miRNAs}
%
Los miRNAs, también conocidos como microRNAs, son pequeñas moléculas
de ácido ribonucleico (\eng{RNA}, del inglés \e{ribonucleic acid})
compuestas por una cadena de unos $22$ nucleótidos $[\nt{}]$ de
longitud.  Hace aproximadamente una década se propuso que estas
moléculas previamente ignoradas, aunque presentes en grandes
cantidades en la célula, jugarían un papel decisivo en la reproducción
celular, promoviendo la diferenciación en distintos tipos de tejidos
y/o su permanencia en un estado particular de diferenciación
\cite{lee-mammal}.  Estudios posteriores han demostrado que los miRNAs
ejercen una función reguladora de la expresión génica celular
\cite{bartel116} y están involucrados en varios procesos genéticos
dentro de la célula, como la transcripción de mRNA (\e{RNA
  mensajeros}) y la síntesis de proteínas \cite{lili}.  Este efecto
regulador puede tener gran implicancia en el desarrollo y evolución de
la enfermedad celular. Se ha demostrado que los miRNAs juegan un rol
importante en la carcinogénesis \cite{aurora,lili}, regulando la
proliferación y muerte celular, entre otras funciones tales como el
metabolismo de las grasas en moscas, los procesos de infección viral,
y el control del desarrollo de flores y hojas en plantas
\cite{bartel116,lecellier}.

El debate acerca del número e identidad de los miRNAs presentes en
diferentes genomas es una cuestión abierta: a modo de ejemplo, para el
caso del genoma humano se había estimado inicialmente que se
encontrarían unos pocos cientos de genes miRNA; posteriormente este
número ha sido revisado a unos 1000 \cite{sewer,chang}.
Actualmente, se encuentra que el número de miRNAs descubiertos en
humanos duplica esta cifra \cite{gomes}. Asimismo, este número podría
ser mucho mayor si se considera que existen miRNAs de evolución más
reciente, específicos a la especie humana \cite{sewer}.

Los miRNAs se presentan naturalmente dentro de una molécula denominada
pre-miRNA o \e{miRNA precursor}, de unos $70\,\nt$ de longitud, la
cual contiene uno o más miRNAs (llamados \e{maduros}) en su
secuencia. En la \refer{horquilla} se representa en forma
esquemática la secuencia de una cadena de pre-miRNA y su
correspondiente \e{estructura secundaria}. Esta estructura secundaria
viene dada por la forma en que la cadena de nucleótidos se \e{pliega}
sobre sí misma logrando una mayor estabilidad molecular. Tal como en
el ejemplo de la \refer{horquilla}, en el reino animal la estructura
secundaria típica de un pre-miRNA conforma una especie de
\e{horquilla} \cite{bartel116,sewer}.
