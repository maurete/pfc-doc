\section{Consola de operaciones}

Para la implementación de una consola de operaciones, se evaluaron las
herramientas Rundeck\footnote{
  \href{https://www.rundeck.com/open-source}{https://www.rundeck.com/open-source}
} y StackStorm\footnote{
  \href{https://stackstorm.com/}{https://stackstorm.com/} }.

El funcionamiento de Rundeck se organiza en torno a la ``Ejecución'' de
``Trabajos'' en ``Nodos'', respetando un esquema de permisos y guardando
registro de las acciones ejecutadas por cada usuario. Los ``Trabajos''
se pueden parametrizar con variables, lo cual permite ejecutar scripts
de forma segura y controlada. La interfaz de Rundeck es amigable e
intuitiva, pero también tiene grandes falencias. Por ejemplo, para la
definición de un nuevo nodo se debe generar un archivo XML y cargarlo
mediante la interfaz gráfica. Esta falencia fue determinante para
decidir que Rundeck no se ajusta a las necesidades de la Organización.

En cuanto a StackStorm, su funcionalidad se orienta a ejecutar
``Acciones'' como respuesta ante ``Eventos'' de diversos tipos, tales como
un mensaje en un canal de chat o un disparador/alerta del sistema de
monitoreo. Si bien permite ejecutar comandos como una acción manual,
ésta no es su funcionalidad principal.

En las pruebas efectuadas se ha determinado que tanto Rundeck como
StackStorm son herramientas con mucho potencial para avanzar en el
camino de la automatización de las operaciones. Sin embargo, su
integración y utilización en el entorno actual de la Organización
resultó ser demasiado compleja en ambos casos. Este riesgo estaba
contemplado en la Propuesta del Proyecto presentada oportunamente.

Ante esta situación, se ha determinado que, por el momento, resulta
conveniente mantener la forma actual de efectuar las operaciones sobre
los servicios, dejando como un trabajo a futuro la implementación de
estas herramientas.
