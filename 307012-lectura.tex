%
\subsubsection{Lectura de archivos}
%
La lectura de los archivos de texto se implementa en la función
\func{load\_fasta}.
Cuando la información de estructura secundaria no se encuentra
disponible, \func{load\_fasta} invoca a la función auxiliar
\func{myrnafold}, encargada del cálculo de esta información.
%
\funcentry{load\_fasta}
          {Nombre de archivo a leer}
          {Representacion de los datos leídos como un
            arreglo de datos en memoria}
          {Lee el archivo indicado como argumento, y aplicando reglas
            de correspondencia de expresiones regulares genera un
            arreglo de ejemplos con las siguientes propiedades:
            %
            \begin{itemize}
            \item Número de línea de comienzo y de fin del ejemplo
            \item Línea de descripción
            \item Secuencia
            \item Estructura secundaria (si está presente)
            \item Mínima energía libre (número al final de la
              estructura secundaria)
            \item Identificador (primera ``palabra'' de la línea de
              descripción)
            \end{itemize}
            %
            Para los ejemplos donde no haya información de
            estructura secundaria, se intenta efectuar el ``plegado''
            invocando la función \func{myrnafold}.
          }
%
\funcentry{myrnafold}
          {Secuencia en formato textual}
          {Estructura secundaria y mínima energía libre}
          {Calcula la información de estructura secundaria y de mínima
            energía libre para la secuencia dada mediante un llamado
            al comando de sistema \func{RNAfold}. Si esta invocación
            falla, recurre a la función de Matlab \func{rnafold},
            provista por el paquete ``Bioinformatics Toolbox''.}
%
