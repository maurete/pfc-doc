%
%
\subsection{Medidas de evaluación de un clasificador binario}
%
Los clasificadores binarios generan modelos cuya salida alterna
entre dos valores posibles, que se interpretan en forma genérica
como ``clase positiva'' y ``clase negativa''.
Para este tipo especialmente
importante de clasificadores, se utilizan una serie de métricas que
relacionan la pertenencia de clase de los datos con el
error de clasificación.

Dado un modelo $h$ de salida binaria, se definen las siguientes
medidas que caracterizan el resultado de clasificar un conjunto $T$:
%
\iflatexml\begin{itemize}\else\begin{itemize}[style=nextline]\fi
  \item\e{Verdaderos positivos ($\VP$)}: número de elementos de clase
    positiva correctamente clasificados como positivos.
  \item\e{Verdaderos negativos ($\VN$)}: número de elementos de clase
    negativa correctamente clasificados como negativos.
  \item\e{Falsos positivos ($\FP$)}: número de elementos de clase
    negativa erróneamente clasificados como positivos.
  \item\e{Falsos negativos ($\FN$)}: número de elementos de clase
    positiva erróneamente clasificados como negativos.
\end{itemize}
%
Estas cuatro medidas conforman la llamada \e{matriz de confusión}
característica del clasificador, y dependen en todos los casos del
número de elementos positivos y negativos presentes en el conjunto
$T$.
