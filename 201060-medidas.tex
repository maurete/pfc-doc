%
%
\subsection{Medidas de evaluación de un clasificador binario}
%
En el análisis del comportamiento de los clasificadores binarios se
utiliza una serie de métricas que relacionan la pertenencia de clase
de los datos con el error de clasificación.
Dado un modelo $h$ de salida binaria, entendida como una
discriminación entre una ``clase positiva'' y una ``clase negativa'',
las siguientes medidas caracterizan el resultado de clasificar un
conjunto $T$:
%
\iflatexml\begin{itemize}\else\begin{itemize}[style=nextline]\fi
  \item\e{Verdaderos positivos ($\VP$)}: número de elementos de clase
    positiva correctamente clasificados como positivos.
  \item\e{Verdaderos negativos ($\VN$)}: número de elementos de clase
    negativa correctamente clasificados como negativos.
  \item\e{Falsos positivos ($\FP$)}: número de elementos de clase
    negativa erróneamente clasificados como positivos.
  \item\e{Falsos negativos ($\FN$)}: número de elementos de clase
    positiva erróneamente clasificados como negativos.
\end{itemize}
%
Estas cuatro medidas conforman la llamada \e{matriz de confusión}
característica del clasificador, y dependen en todos los casos del
número de elementos positivos y negativos presentes en el conjunto
$T$.
