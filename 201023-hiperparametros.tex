%
\subsubsection{Hiperparámetros}
%
Se denomina \e{hiperparámetros} a aquellos parámetros que regulan el 
proceso de entrenamiento, que sin formar parte del modelo influyen en
la generación del mismo.
En general, los \hparam{s} determinan el comportamiento de ``alto nivel''
del algoritmo de entrenamiento, tal como la cantidad de ajuste aplicado al
modelo en cada iteración, el grado de tolerancia a los errores, o la
complejidad de la solución.

Al actuar como reguladores del proceso de entrenamiento, los \hparam{s}
deben ser establecidos antes de comenzar el mismo.
En general, la selección de los hiperparámetros se efectúa mediante prueba
y error sobre modelos desechables, ya sea en forma manual o mediante un
algoritmo automático.
Para ciertos tipos específicos de máquina de aprendizaje, existen técnicas
automáticas que configuran los \hparam{s} de manera eficiente.
