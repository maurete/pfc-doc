%
\subsubsection{Hiperparámetros}
%
Los \e{hiperparámetros} son aquellos parámetros que regulan el proceso de
aprendizaje, y que no pueden determinarse directamente a partir de
los datos.
Se denominan de este modo porque en general se relacionan
con conceptos de más alto nivel que los parámetros del modelo, tales
como la complejidad de la solución o la velocidad de aprendizaje.
Los \hparam{s} no pueden ser determinados durante el entrenamiento, sino
que deben ser elegidos antes de efectuar el mismo.
En general, se seleccionan de manera manual en una estrategia de
prueba y error.
Sin embargo, para ciertos tipos de máquinas de aprendizaje
existen técnicas automáticas para su determinación.
