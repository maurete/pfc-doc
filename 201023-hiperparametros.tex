%
\subsubsection{Hiperparámetros}
%
Los \e{\hparam{s}} son aquellos parámetros que no pueden ser determinados
mediante el entrenamiento.
En general su función es regular el comportamiento del algoritmo de
entrenamiento, y se relacionan con propiedades de ``alto nivel'' tales
como la cantidad de ajuste aplicado al modelo en cada iteración,
el grado de tolerancia a los errores, o la complejidad de la solución.
Los valores de los \hparam{s} deben ser establecidos antes de comenzar el
entrenamiento, y en general se aplica una estrategia de prueba y error
sobre modelos desechables, ya sea en forma manual o mediante un
algoritmo automático.
Para ciertos tipos específicos de máquina de aprendizaje, existen técnicas
automáticas que configuran los \hparam{s} de manera eficiente.
