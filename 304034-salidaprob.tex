%
\subsubsection{Salida probabilística del modelo SVM}
%
La salida del modelo $h$ de la máquina de vectores de soporte
(\iflatexml{}Ecuación~\ref{eq:svm-model-hard}\else\autoref{eq:svm-model-hard}\fi)
puede entenderse como el resultado de aplicar la operación signo sobre
una salida real $f$:
%
\begin{align}
\label{e3:svm-model-as-sign-f}
  h(\xx) \tab= \T{signo}(f),\tabs
  f\tab=\langle\ww,\BPhi(\xx)\rangle+b.
\end{align}
%
Aplicando el método propuesto por {Platt} en \cite{platt}, se
calcula a partir de la salida $f$ un estimador
$\hat{p}$ para la probabilidad real $p=P(h=+1|\xx)$:
%
\begin{align}
\label{e3:p-hat}
  \hat{p}\tab=\frac{1}{1+e^{Af+B}} \tabs\approx\,{}p.%P(h=+1|\xx).
\end{align}
%
Los valores $A$ y $B$ se ajustan a partir del conjunto de
validación $V$ resolviendo el problema de
optimización
%
\begin{align}
  \arg\min_{A,B} \tabs -\sum_{j=1}^{\ell^V} t_j\log(\hat{p}_j)+(1-t_j)\log(1-\hat{p}_j),
  \label{abproblem}
\end{align}
%
en donde $\hat{p}_j$ viene dado por la
\iflatexml{}Ecuación~\ref{e3:p-hat}\else\autoref{e3:p-hat}\fi{}, y
$t_j=\frac{y_j+1}{2}$.
%

Las derivadas de $\hat{p}_j$ respecto de $A$ y $B$ vienen dadas por
%
\begin{align}
  \begin{split}
    \dpar{\hat{p}_j}{A}{}&=-f_j e^{Af_j+B}\frac{1}{(1+e^{Af_j+B})^2}
    =-f_j\hat{p}_j(1-\hat{p}_j),\\
    \dpar{\hat{p}_j}{B}{}&=    -e^{Af_j+B}\frac{1}{(1+e^{Af_j+B})^2}
    =-   \hat{p}_j(1-\hat{p}_j).
  \end{split}
\label{e3:deriv-pk-wrt-AB}
\end{align}
%

%% La elección de una función logística como $\hat{p}_j$ se basa en la
%% presunción de que las salidas $f_j$ para las entradas $\{\xx_j\}$ de
%% clase positiva tienen una distribución gaussiana en $f$.

%% Una interpretación intuitiva de $\hat{p}_j$ es que, para una salida
%% $f_j$ de gran magnitud, que se ubica lejos del hiperplano de separación
%% en el espacio inducido por el núcleo, el clasificador SVM tiene amplia
%% certeza de su decisión.  Asimismo, se tendrá que cuando el valor de
%% $f_j=0$, $x_j$ recae exactamente en el plano de separación y se
%% obtiene la peor predicción posible $\hat{p}_j=0.5$ y, similarmente,
%% cuanto más ancho sea el margen de separación, más suave deberá ser la
%% pendiente de $\hat{p}_j$.
