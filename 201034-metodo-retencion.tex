%
\subsubsection{Método de retención}
\label{retencion}
%
En un problema de aprendizaje típico, toda la información que se
conoce del fenómeno a modelar es una muestra $U\in\C{D}$ que contiene
todos los ejemplos conocidos.
El \e{método de retención} consiste en dividir el conjunto $U$ en dos
conjuntos disjuntos $D$ y $T$, que son usados respectivamente como
conjuntos de entrenamiento y de prueba.
La máquina de aprendizaje se entrena con el conjunto $D$, y el error
de generalización se estima midiendo el error del modelo sobre el
conjunto de prueba $T$.

En ocasiones, el conjunto de entrenamiento $D$ se subdivide siguiendo
el mismo procedimiento en conjuntos de \e{estimación} $E$ y
de \e{monitoreo} $M$.
Estos conjuntos son utilizados antes del entrenamiento para ajustar
los hiperparámetros del algoritmo de aprendizaje, generando modelos
de estimación que se evalúan con el conjunto de monitoreo $M$.
Una vez determinados los hiperparámetros, el modelo final se 
entrena con el conjunto de entrenamiento completo $D$.
