%
\subsubsection{Método de retención}
\label{retencion}
%
En un problema de aprendizaje típico, toda la información que se
conoce del fenómeno a modelar es una muestra $U\in\C{D}$ con todos los
ejemplos conocidos.
El \e{método de retención} consiste en seleccionar
un porcentaje de elementos de $U$ para armar el conjunto de prueba
$T$, y utilizar los elementos restantes de $U$ para generar el
conjunto de entrenamiento $D$.
De este modo, la máquina de aprendizaje
se evalúa sobre un conjunto datos independiente del conjunto de
entrenamiento.

En ocasiones, el conjunto de entrenamiento $D$ se subdivide siguiendo
el mismo procedimiento en conjuntos de \e{estimación} $E$ y
\e{validación} $V$.
Estos conjuntos son utilizados durante el
entrenamiento para ajustar los hiperparámetros del algoritmo de
aprendizaje, antes de entrenar el modelo final sobre el conjunto $D$.
