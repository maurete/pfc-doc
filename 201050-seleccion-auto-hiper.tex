%%
%%
%\subsection{Selección automática de hiperparámetros}
%%
%La selección automática de hiperparámetros se efectúa antes del
%entrenamiento del modelo, mediante estrategias específicas para el
%tipo de clasificador y el tipo de hiperparámetro que se desea
%determinar.
%
%En general, esta selección se efectúa mediante técnicas de prueba y
%error, que entrenan un
%modelo preliminar sobre un conjunto de estimación y lo prueban contra
%un conjunto de validación, seleccionando los hiperparámetros que
%obtienen los mejores resultados según alguna métrica estadística.
%
%Para algunos clasificadores y tipos de hiperparámetros específicos,
%resulta posible la utilización de estrategias más ``informadas'' de
%selección de hiperparámetros, que aprovechan alguna propiedad teórica
%del tipo de modelo considerado y efectúan una búsqueda mediante
%descenso de gradiente minimizando una función derivable llamada
%``función objetivo''.
