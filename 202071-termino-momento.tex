%
\subsubsection{Término de momento}
%
Una estrategia propuesta para incrementar la velocidad de aprendizaje
y mejorar la estabilidad del algoritmo de retropropagación consiste en
modificar la regla delta (\ref{e2:delta-rule}) añadiendo un \e{término
  de momento}
%
\begin{align}\label{e2:momentum-term}
  \Delta w_{ij}(t)\tab=-\eta\dpar{\C{E}(t)}{w_{ij}}{} +\mu\Delta
  w_{ij}(t-1).
\end{align}
%
El término de momento simplemente añade una fracción $\mu$ del paso
anterior $t-1$ al paso actual $t$: si el gradiente mantiene su
orientación, el efecto del término de momento es aumentar la velocidad
de aprendizaje; si en cambio el gradiente cambia de dirección, el
momento ``suaviza'' las oscilaciones.
El \hparam{} de momento $0<\mu<1$ se determina mediante prueba y error
para cada problema.
