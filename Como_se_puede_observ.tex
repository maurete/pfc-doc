Como se puede observar en la \autoref{testresults}, las tasas de
clasificación para los distintos conjuntos resultan satisfactorias, e
incluso sensiblemente mejores a aquellas del trabajo original.  Sin
embargo, estos números deberán tomarse con cuidado, ya que han sido
obtenidos entrenando y validando con particiones estáticas, y podrían
ser resultado de un sobreentrenamiento para estos datos en
particular. Se observa también que el perceptrón multicapa presenta
una buena tasa de clasificación incluso cuando se trata de una única
neurona (sin capas ocultas), siempre para este mismo conjunto de
datos.

Se ha implementado el script \mono{triplet\_libsvm.sh} para las
pruebas con \emph{libsvm}, y los scripts \mono{triplet\_svm.m} y
\mono{triplet\_mlp.m} para las pruebas en Matlab de los clasificadores
SVM y MLP respectivamente. Con el software apropiado, estos scripts
se pueden ejecutar directamente para reproducir los resultados de las
pruebas.

\chapter{Análisis/Proyecto}