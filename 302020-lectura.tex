%
%
\subsection{Lectura de los archivos de entrada}
%
El primer paso del preprocesamiento es la interpretación de los
archivos de entrada.
La lectura se efectúa en modo secuencial, línea por línea, y aplicando
análisis sintáctico se identifican las líneas de descripción,
secuencia y estructura secundaria.
El procesamiento en esta etapa genera un arreglo en memoria con
información estructurada de los ejemplos, que contiene la secuencia,
estructura secundaria, y descripción de cada ejemplo.
Este arreglo es utilizado en la etapa de extracción de características
para continuar el procesamiento.

El formato aceptado por el sistema es una extensión del formato FASTA,
un formato originado en un software de igual nombre \cite{fasta} muy
difundido dentro de la disciplina.
El software RNAfold \cite{vienna}, utilizado para el cálculo de la
estructura secundaria, produce archivos con formato similar a FASTA
incorporando información de estructura secundaria.
La especificación del formato aceptado por el sistema es compatible
con el formato FASTA estándar y la variante RNAfold.
