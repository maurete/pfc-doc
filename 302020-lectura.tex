%
%
\subsection{Lectura de los archivos de entrada}
%
La lectura de los archivos de entrada se efectúa en modo secuencial,
línea por línea.
Aplicando análisis sintáctico se identifican líneas de descripción,
secuencia y estructura secundaria, generando un arreglo en memoria con
información estructurada de los datos leídos, que identifica la
secuencia, estructura secundaria, y descripción de cada ejemplo.
%Este arreglo es utilizado en la tarea posterior de extracción de
%\caract{s} para la generación de los vectores de \caract{s}.

El formato aceptado para los archivos de entrada es compatible con el
formato FASTA y con la variante generada por el software RNAfold.
El formato FASTA es muy difundido dentro de la disciplina, y tiene su
origen en el software de igual nombre \cite{fasta}.
El formato generado por el software RNAfold \cite{vienna} es una
extensión de FASTA que incorpora información de la estructura
secundaria.
