%
%
\subsection{Lectura de los archivos de entrada}
%
El primer paso del preprocesamiento es la interpretación de los
archivos de entrada.
La lectura se efectúa en modo secuencial, línea por línea, y aplicando
análisis sintáctico se identifican las líneas de descripción,
secuencia y estructura secundaria.
En esta etapa se genera un arreglo en memoria con información
estructurada de los ejemplos, que contiene la secuencia, estructura
secundaria, y descripción leídas para cada ejemplo.
Este arreglo es utilizado en la tarea posterior de extracción de
\caract{s} para la generación de los vectores de \caract{s}.

El formato aceptado por el sistema es una extensión del formato FASTA,
un formato originado en un software de igual nombre \cite{fasta}, muy
difundido dentro de la disciplina.
El software RNAfold \cite{vienna}, utilizado para el cálculo de la
estructura secundaria, produce archivos con formato similar a FASTA
incorporando información de estructura secundaria.
La especificación del formato aceptado por el sistema es compatible
tanto con el formato FASTA estándar así como la variante RNAfold.
