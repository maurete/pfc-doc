%
%
\subsection{Lectura de los archivos de entrada}
%
La lectura de los archivos de entrada se efectúa en modo secuencial,
línea por línea.
Aplicando análisis sintáctico se identifican líneas de descripción,
secuencia y estructura secundaria, generando un arreglo en memoria con
información estructurada de los datos leídos.
Este arreglo contiene, para cada ejemplo, la descripción, secuencia
completa, y estructura secundaria en campos separados.
%Este arreglo es utilizado en la tarea posterior de extracción de
%\caract{s} para la generación de los vectores de \caract{s}.

El formato aceptado para los archivos de entrada es compatible con el
formato FASTA y con la variante de este formato producida por el
software RNAfold.
FASTA es un formato muy difundido dentro de la disciplina, y tiene su
origen en el software de igual nombre \cite{fasta}.
El formato RNAfold \cite{vienna} es una extensión de FASTA que
incorpora información de la estructura secundaria.
La especificación a continuación define el formato de los archivos de
entrada tal como son interpretados por el método.
