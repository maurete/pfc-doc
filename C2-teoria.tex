%
%
%
%
\chapter{Fundamentos teóricos}
%
La Inteligencia Computacional es una disciplina centrada en la
utilización de metodologías y técnicas inspiradas en la naturaleza
para resolver problemas complejos \hl{del mundo real}.

Los métodos y herramientas utilizados son \hl{parecidos} al
razonamiento humano, ``generalizando'' el conocimiento inexacto o
incompleto, y es capaz de tomar decisiones en forma adaptativa.

Las técnicas utilizadas en el presente trabajo pertenecen al campo del
aprendizaje supervisado, un área de la IC.

En el presente capítulo se introducen las nociones fundamentales de la
disciplina y de las técnicas específicas, el Perceptrón Multicapa y la
Máquina de Vectores de Soporte, a fin de adentrar al lector en los
conceptos a desarrollar más adelante.
