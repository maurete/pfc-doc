%
%
%
%
\chapter{Fundamentos teóricos}
%
La Inteligencia Computacional (\e{IC}) es la disciplina que estudia la
aplicación de técnicas y metodologías inspiradas en la naturaleza para
resolver problemas complejos.
En general, las técnicas de IC presentan
cualidades típicas del razonamiento humano, ``generalizando'' el
conocimiento inexacto o incompleto, y son capaces de tomar decisiones
en forma adaptativa.

En este capítulo se introducen conceptos relevantes de la disciplina
en general así como de las técnicas específicas utilizadas en el
presente trabajo: el Perceptrón Multicapa y la Máquina de Vectores de
Soporte.
