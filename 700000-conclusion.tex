\chapter{7 Conclusión}

En este Proyecto se ha trabajado en la implementación de prácticas
DevOps en el ámbito de la Dirección de Informatización y Planificación
Tecnológica (DIPT). Se integraron múltiples aspectos del trabajo de
los equipos de desarrollo y de operaciones aportando a la generación
de una cultura en común. Por otro lado, se pudo observar un incremento
de la productividad y la simplificación de procesos aplicando
automatización y herramientas adecuadas.

La unificación de repositorios de código compartidos para los equipos
de desarrollo y de operaciones demostró ser una herramienta de cambio
cultural muy poderosa. Los desarrolladores obtienen una visión
práctica de la manera en que se despliega la infraestructura para la
puesta en servicio del software. El equipo de operaciones, por su
parte, comprende que los equipos de desarrollo llevan adelante las
obligaciones de la organización para con los clientes, y ajusta sus
procedimientos en concordancia, para minimizar disrupciones provocadas
por los cambios de infraestructura. Se observó el surgimiento de
instancias de coordinación entre ambos equipos para la aplicación de
cambios, lo que aporta a una conciencia global del estado de los
servicios y la infraestructura.

La automatización tuvo un impacto directo en la carga de trabajo del
equipo de infraestructura. El tiempo liberado generó un incremento de
la productividad y la capacidad de implementación de mejoras, lo cual
incidió incluso en el desarrollo posterior del Proyecto. Las
implementaciones complejas como la automatización del servicio de
frontend se lograron en parte debido a la mayor disponibilidad de
tiempo.

Por otro lado, se concluye que el Proyecto no tuvo éxito en la
implementación de tuberías de integración y entrega continuas. La
implementación realizada nunca fue adoptada por completo por parte de
los equipos de desarrollo. Las tuberías resultaron quizá demasiado
complejas, en parte, debido a la intención de mantener la forma de
trabajo de los equipos y no impactar su productividad. Por su parte,
los equipos de desarrollo no perciben un beneficio en automatizar
estas tareas.

Se toma como aprendizaje que una futura implementación de CI/CD deberá
ser un proceso gradual, con una implementación progresiva. Deberá ser
parte de un proceso de mejora en el seno de los equipos de desarrollo,
optimizando su flujo de trabajo y procedimientos conforme se avance en
la automatización.

Otro aprendizaje obtenido con el desarrollo del Proyecto fue que los
cambios culturales llevan su tiempo. La estructura rígida de tareas a
realizar, fechas límite y entregables -- completamente justificada
dado el contexto y los objetivos del desarrollo del Proyecto --
impactó negativamente en la capacidad de generar y consolidar una
cultura, ya que se debía ``avanzar rápido'' para cumplir con las metas
del Proyecto. Esto generó que en ciertos casos las mejoras fueran
percibidas como un requerimiento o incluso como una imposición.

Finalmente, las herramientas de observabilidad y comunicación han
demostrado sus beneficios. Utilizando el servicio de métricas se ha
logrado visibilizar y resolver problemas en los entornos productivos
que de otro modo hubieran pasado inadvertidos. Por su parte, el
servicio de chat resultó fundamental para la coordinación y el trabajo
en equipo ante la imposición del trabajo remoto debido a la pandemia
de coronavirus.

A modo de revisión se consideró oportuno efectuar un repaso del estado
inicial de la Dirección y compararlo con la situación actual. Al
comienzo del Proyecto:

\begin{itemize}
\item El equipo de infraestructura se ocupaba de crear instancias y
  aprovisionar la configuración ejecutando Ansible en la línea de
  comandos.
\item La configuración de los servicios de frontend, backup, monitoreo
  y registros eran tareas manuales, facilitadas mediante scripts
  específicos.
\item El código fuente de los servicios se guardaba en repositorios
  SVN separado de la configuración de la infraestructura.
\item Cualquier modificación en la configuración de las instancias
  debía solicitarse mediante tickets al equipo de infraestructura.
\item Los desarrolladores contaban con un servicio de agregado de logs
  al que debían acceder mediante una interfaz de línea de comandos.
\end{itemize}
En la actualidad,

\begin{itemize}
\item Los equipos de desarrollo cuentan con herramientas para la
  creación y modificación de la infraestructura sin requerir la
  intervención del equipo de infraestructura.
\item El código y la configuración de los servicios se encuentran en
  el mismo repositorio Git. Se han establecido para el trabajo
  conjunto sobre los repositorios.
\item La configuración de los servicios de frontend, backup, monitoreo
  y registros han sido automatizadas por completo.
\item Se cuenta con un servicio de integración y entrega continuas. Si
  bien la adopción es baja, las implementaciones de referencia sirven
  como ejemplos prácticos y facilitan la implementación cuando los
  equipos de desarrollo así lo dispongan.
\item Se cuenta con herramientas web para la visualización de
  registros y métricas, facilitando la investigación de incidentes y
  la medición del impacto de los cambios aplicados.
\item Se cuenta con un servicio de chat que simplifica la comunicación
  interna y canaliza las alertas y notificaciones ante eventos de
  monitoreo o de gestión del servicio relevantes.
\end{itemize}
\section{Trabajos futuros}

El trabajo realizado en este Proyecto marca sin lugar a dudas el
comienzo hacia una Dirección más dinámica, adaptable y
productiva. Resta aún mucho por hacer, y en este respecto se proponen
los siguientes trabajos a futuro:

\begin{itemize}
\item Adoptar las tuberías de CI/CD en el desarrollo de los
  servicios. Esta mejora deberá surgir dentro de los equipos de
  desarrollo, con un entendimiento de las implicancias sobre la
  metodología de trabajo y los beneficios esperables.
\item Implementar un servicio para efectuar operaciones. Este servicio
  deberá ofrecer la posibilidad de realizar aquellas operaciones no
  relacionadas con la infraestructura, tales como la puesta en
  mantenimiento de un servicio y la posibilidad de generar y restaurar
  dumps de las bases de datos.
\item Implementar servicios con alta disponibilidad. La
  infraestructura actual es suficientemente dinámica y permite
  gestionar servicios altamente disponibles.
\item Adoptar la experimentación continua. Una vez que se cuente con
  servicios altamente disponibles y tuberías de CI/CD, resultará
  posible definir objetivos de nivel de servicio (SLOs) y trabajar con
  despliegues verde azul, posibilitando experimentar con cambios en
  los entornos productivos.
\item Adoptar contenedores y microservicios. Los desarrollos
  tecnológicos actuales tienden hacia la infraestructura definida por
  software y la gestión orquestada de los servicios. Como primer paso
  en este sentido se propone implementar la propuesta de ``doce
  factores''\footnote{ https://12factor.net/es/}.
\end{itemize}
