%
%
\subsubsection{Generación de particiones para validación cruzada}
%
La generación de las particiones de validación cruzada se lleva a cabo
una vez armado el conjunto de entrenamiento con $\ell^E$ elementos.
En lugar de replicar los datos del conjunto de entrenamiento, las
particiones de validación cruzada se generan simplemente como vectores
de índices que apuntan a los ejemplos correspondientes del conjunto de
entrenamiento $D^E$. Partiendo de un vector de índices en orden
aleatorio entre 1 y $\ell^E$, se separan los primeros $p\cdot\ell^E$
elementos para la partición de validación, utilizando los restantes
para la partición de estimación correspondiente. Esta operación se
repite para cada $j=1,\ldots,k$, efectuando un desplazamiento circular
del vector de índices en $\ell^E/k$ elementos entre cada iteración.

El parámetro $p$ regula la proporción de elementos de entrenamiento a
utilizar para validación, mientras que $k$ determina el número de
iteraciones de la validación cruzada. Cuando $p=1/k$, cada ejemplo en
el conjunto de entrenamiento será utilizado exactamente una vez para
validación.
