%
%
\subsubsection{Generación de particiones para validación cruzada}
%
%La generación de las particiones de validación cruzada se lleva a cabo
%una vez armado el conjunto de entrenamiento con $\ell^E$ elementos.
Las particiones de validación cruzada se generan como un conjunto de
índices sobre el conjunto de entrenamiento.
En primer lugar, se genera un vector de índices en orden aleatorio
entre $1$ y $\ell^E$, el número de ejemplos en el conjunto de
entrenamiento.
Luego se seleccionan los primeros $p\cdot\ell^E$ elementos para la
partición de validación, y los elementos restantes para la partición
de estimación correspondiente.
La operación se repite para cada $j=1,\ldots,k$, efectuando un
desplazamiento circular del vector de índices en $\ell^E/k$ elementos
entre iteraciones.
El parámetro $p$ regula la proporción de elementos de entrenamiento a
utilizar para validación, mientras que $k$ determina el número de
iteraciones de la validación cruzada.
Ambos parámetros pueden ser especificados por el usuario, y por
defecto se utilizan los valores $k=10$ $p=1/$.
Cuando $p=1/k$, cada ejemplo del conjunto de entrenamiento será
utilizado exactamente una vez para validación.
