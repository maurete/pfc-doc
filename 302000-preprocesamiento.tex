%
%
%
\section{Preprocesamiento}
%
El preprocesamiento es el conjunto de pasos efectuados para
transformar los datos de entrada, que vienen dados en forma de
archivos de texto, a un formato numérico apto para la máquina de
aprendizaje.
La secuencia de pasos efectuados en la etapa de preprocesamiento son:
%
\begin{enumerate}
\item
  \e{Análisis sintáctico (\e{parsing})}: se trata de interpretar el texto
  en los archivos de entrada identificando las líneas de descripción,
  de secuencia, y de estructura secundaria.
\item
  \e{Plegado}: consiste en calcular la información de estructura
  secundaria invocando herramientas de software externas.
\item
  \e{Extracción de \caract{s}}: es el proceso de generar vectores
  numéricos de longitud fija que representan cada ejemplo, a partir de
  información extraída de la secuencia y de la estructura secundaria.
\item
  \e{Normalización}: modifica el rango de los vectores de
  características de modo que sus elementos se ajusten a un intervalo
  preestablecido.
\item
  \e{Generación de conjuntos de datos}: reagrupa los ejemplos en
  conjuntos de entrenamiento y de prueba, y genera particiones de
  validación cruzada, a partir de la especificación por parte del
  usuario del uso que se dará a los datos: generar un modelo de
  clasificador o predicción de pertenencia de clase.
\end{enumerate}
%
Aplicando estos pasos, se obtiene una representación de los datos en
un formato numérico estandarizado, que puede utilizars como entrada a
las funciones de entrenamiento y clasificación de la máquina de
aprendizaje.
En adelante, se presenta una descripción detallada de cada una de estas
sub-tareas.
