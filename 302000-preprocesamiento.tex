%
%
%
\section{Preprocesamiento}
%
El preprocesamiento se encarga de transformar los datos de entrada,
que vienen dados en forma de archivos de texto, a un formato numérico
apto para la máquina de aprendizaje.
Los pasos efectuados en la etapa de preprocesamiento, tal como se muestra en la
\iflatexml{}Figura~\ref{fig:preprocesamiento}\else\autoref{fig:preprocesamiento}\fi{},
son:
%
\begin{enumerate}
\item
  \e{Análisis sintáctico}: se trata de interpretar el texto en los
  archivos de entrada, identificando las líneas de descripción, de
  secuencia, y de estructura secundaria.
\item
  \e{Plegado}: consiste en calcular la información de estructura
  secundaria incompleta invocando herramientas de software externas.
\item
  \e{Extracción de \caract{s}}: es el proceso de generar vectores
  numéricos de longitud fija que representan cada ejemplo, a partir de
  información extraída de la secuencia y de la estructura secundaria.
\item
  \e{Normalización}: modifica el rango de los vectores de
  características de modo que sus elementos se ajusten a un intervalo
  preestablecido.
\item
  \e{Generación de conjuntos de datos}: reagrupa los ejemplos en
  conjuntos de entrenamiento y de prueba, y genera particiones de
  validación cruzada, a partir de la especificación del usuario y del
  uso pretendido de los datos.
\end{enumerate}
%
De este modo se obtiene una representación de los datos en un formato
numérico estandarizado, que puede utilizarse como entrada a las
funciones de entrenamiento y clasificación de la máquina de
aprendizaje.
