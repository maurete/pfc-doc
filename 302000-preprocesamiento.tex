%
%
%
\section{Preprocesamiento}
%
El preprocesamiento es el conjunto de pasos efectuados para
transformar los datos de entrada a un formato tratable por la máquina
de aprendizaje.  Los datos de entrada al sistema vienen dados como
archivos en formato FASTA, un formato de texto que contiene secuencias
``anotadas'' con información complementaria.
Los distintos pasos llevados a cabo en el preprocesamiento son:
%
\begin{enumerate}
\item
  Análisis sintáctico (\e{parsing}): se trata de interpretar el
  texto de entrada identificando las líneas de descripción
  (anotaciones), de secuencia, y de estructura secundaria si las
  hubiera.
\item
  Plegado: consiste en calcular la información de estructura
  secundaria para todas las secuencias, mediante la invocación de
  software externo.
\item
  Extracción de \caract{s}: es el proceso de generar vectores
  numéricos de longitud fija que representan cada ejemplo a partir de
  la información de secuencia y de estructura secundaria.
\item
  Normalización: ajusta el rango de las componentes del vector de
  características de modo que se ajusten a un intervalo
  preestablecido.
\item
  Generación de conjuntos de datos y de validación cruzada:
  agrupa los ejemplos en conjuntos de entrenamiento y prueba, y genera
  particiones de validación cruzada, según sea el propósito con el que
  se utilizarán los datos: generar un modelo de clasificador o
  predicción de pertenencia clase.
\end{enumerate}
%
En adelante se presenta una descripción de cada uno de estos pasos,
que generan una representación de los ejemplos en forma de
vectores numéricos normalizados utilizables directamente por la
máquina de aprendizaje.
