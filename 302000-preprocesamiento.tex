%
%
%
\section{Preprocesamiento}
%
El preprocesamiento es el conjunto de tareas efectuadas para
transformar los datos de entrada, que viene dados en forma de archivos
de texto, a un formato tratable por la máquina de aprendizaje.
%% Los datos de entrada al sistema vienen dados como archivos en formato
%% FASTA, un formato de texto que contiene secuencias \e{anotadas} con
%% información complementaria de identificación.
Las tareas efectuadas en la etapa de preprocesamiento son:
%
\begin{enumerate}
\item
  \e{Análisis sintáctico (\e{parsing})}: se trata de interpretar el texto
  en los archivos de entrada identificando las líneas de descripción,
  de secuencia, y de estructura secundaria.
\item
  \e{Plegado}: consiste en calcular la información de estructura
  secundaria invocando herramientas de software externas.
\item
  \e{Extracción de \caract{s}}: es el proceso de generar vectores
  numéricos de longitud fija que representan cada ejemplo, a partir de
  información extraída de la secuencia y de la estructura secundaria.
\item
  \e{Normalización}: modifica el rango de los vectores de
  características de modo que sus elementos se ajusten a un intervalo
  preestablecido.
\item
  \e{Generación de conjuntos de datos}: reagrupa los ejemplos en
  conjuntos de entrenamiento y de prueba, y genera particiones de
  validación cruzada, a partir de la especificación por parte del
  usuario del uso que se dará a los datos: generar un modelo de
  clasificador o predicción de pertenencia de clase.
\end{enumerate}
%
Aplicando estos pasos, se obtiene una representación de los datos en
forma matricial, agrupados en conjuntos de ``entrenamiento'' y de
``prueba''. Estos conjuntos son utilizables directamente por la
máquina de aprendizaje.
