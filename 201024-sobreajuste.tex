%
\subsubsection{Sobreajuste}
%
El objetivo teórico del entrenamiento de una máquina de aprendizaje es
encontrar el modelo con mayor capacidad de generalización, que
describa con el menor error posible la distribución $\nu$.
Sin embargo, toda la información disponible es el conjunto de
entrenamiento $D$, que no es más que una realización del fenómeno
descripto por $\nu$.
Esta distinción entre error de entrenamiento y
error de generalización debe ser tenida en cuenta al efectuar el
entrenamiento, ya que el objetivo debe ser obtener el mínimo error de
generalización, evitando que el modelo \e{sobreajuste} los datos.

Se dice que el modelo presenta \e{sobreajuste} cuando éste describe
el conjunto de datos de entrenamiento $D$ en lugar de la relación
subyacente $\nu$ entre las variables de entrada y de salida.
Intuitivamente, el modelo sobreajustado
\e{memoriza ejemplos} del conjunto de aprendizaje en lugar de
\e{aprender las propiedades} de la distribución $\nu$.

El problema del
sobreajuste existe debido a que la máquina de aprendizaje no recibe
información acerca de la \e{probabilidad} de cada ejemplo observado,
ignorando el ruido y los efectos aleatorios en el conjunto de datos de
entrenamiento.
