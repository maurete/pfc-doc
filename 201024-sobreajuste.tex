%
\subsubsection{Sobreajuste}
%
El objetivo teórico del entrenamiento debe ser
encontrar el modelo con mayor capacidad de generalización, que
describa con el menor error posible la distribución $\nu$.
Sin embargo, toda la información disponible es el conjunto $D$, que no
es más que una muestra de probabilidad desconocida del fenómeno
descripto por $\nu$.
Esta distinción entre error de entrenamiento y
error de generalización debe ser tenida en cuenta al efectuar el
entrenamiento, para evitar que el modelo \e{sobreajuste} los datos.

Se dice que un modelo presenta \e{sobreajuste} cuando éste describe
el conjunto de entrenamiento $D$ en lugar de la relación
subyacente $\nu$ entre las variables de entrada y de salida.
Intuitivamente, el modelo sobreajustado
\e{memoriza ejemplos} del conjunto de entrenamiento en lugar de
\e{describir las propiedades} de la distribución $\nu$.
El problema del sobreajuste existe debido a que la máquina de
aprendizaje no recibe información acerca de la \e{probabilidad}
de cada ejemplo observado, ignorando el ruido y los efectos
aleatorios en el conjunto de datos de entrenamiento.
