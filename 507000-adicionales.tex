%
%
%
\section{Pruebas adicionales}
%
Se efectuaron una serie de pruebas complementarias con el propósito de
obtener una idea del comportamiento del método ante otros tipos de
tipos de problemas de clasificación, con conjuntos de prueba con
ejemplos de otras especies o con ejemplos de \premirna{s} más
recientes.

A partir de los resultados observados en las pruebas principales,
en las pruebas complementarias se trabajó únicamente con el conjunto
de características \dset{S-E} y se probaron cuatro combinaciones de
clasificadores y estrategias de selección de \hparam{s}:
%
\begin{itemize}
\item
  \sbs{MLP-B}: Clasificador MLP, estrategia de búsqueda exhaustiva.
\item
  \sbs{SVM-LE}: Clasificador SVM con núcleo lineal, estrategia de
  selección de hiperparámetros mediante el criterio del error
  empírico.
\item
  \sbs{SVM-RE}: Clasificador SVM, núcleo de base radial (RBF),
  selección de hiperparámetros mediante el criterio del error
  empírico.
\item
  \sbs{SVM-RR}: Clasificador SVM con núcleo RBF y estrategia de
  selección de hiperparámetros minimizando la cota radio-margen.
\end{itemize}
%

Algunos de los problemas complementarios se basan en las pruebas
suplementarias descriptas en los trabajos de referencia \cite{xue} y
\cite{ng}.
Por otra parte, se generaron problemas que clasifican los ejemplos de
la especie humana disponibles en la última versión de \dset\mirbase{}
($21$.$0$, junio de 2014) entrenando con los conjuntos de entrenamiento
de los problemas principales \prob\tripletsvm{}, \prob\mipred{} y
\prob\micropred{}.
Por último, el problema llamado \prob\deltamirbase{} prueba la
predicción de los nuevos ejemplos de la especie humana entre dos
versiones sucesivas de la base de datos \dset\mirbase{}.
En adelante, se describe el armado de cada problema junto con los
resultados obtenidos en cada caso.
