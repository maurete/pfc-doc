%
%
%
\section{Pruebas adicionales}
%
En la presente sección se describen las pruebas complementarias
efectuadas con el propósito de obtener una idea general del
comportamiento del método codificado ante distintos tipos de problemas
de clasificación.

Algunos de los problemas generados se basan en las pruebas
suplementarias de los trabajos de referencia \cite{xue,ng}.
Por otra parte, se generaron problemas que clasifican los ejemplos de
la especie humana disponibles en la última versión de miRBase (21.0,
\hl{FECHA}) entrenando con los conjuntos de entrenamiento de los
problemas principales \tripletsvm{}, \mipred{} y \micropred{}.
El problema llamado \deltamirbase{} prueba la capacidad del método de
predecir correctamente los nuevos ejemplos de la especie humana
incorporados a la base de datos miRBase en la última actualización.

Considerando los resultados obtenidos en las pruebas principales, se
utilizó en todos los casos el conjunto de características S-E
aplicando las cuatro variantes del método:
%
\begin{itemize}
\item
  \sbs{MLP-B}: Clasificador MLP, estrategia de búsqueda
  exhaustiva
\item
  \sbs{SVM-LE}: Clasificador SVM, núcleo lineal, estrategia de
  selección de hiperparámetros mediante minimización del criterio del error
  empírico
\item
  \sbs{SVM-RE}: Clasificador SVM, núcleo de base radial (RBF), selección de
  hiperparámetros mediante minimización del criterio del error empírico,
\item
  \sbs{SVM-RR}: Clasificador SVM con núcleo RBF y estrategia de
  selección de hiperparámetros minimizando la cota radio-margen.
\end{itemize}
%
