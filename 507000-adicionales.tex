%
%
%
\section{Pruebas adicionales}
%
Además de las pruebas sobre los tres problemas de clasificación
principales, se realizaron pruebas complementarias con el propósito de
obtener una idea general del comportamiento del método ante distintos
tipos de problemas de clasificación.

En las pruebas complementarias se trabajó únicamente con el conjunto
de características S-E y se probaron cuatro combinaciones de
clasificador--estrategia de selección de \hparam{s}, basadas en los
resultados obtenidos en las pruebas principales:
%
\begin{itemize}
\item
  \sbs{MLP-B}: Clasificador MLP, estrategia de búsqueda
  exhaustiva
\item
  \sbs{SVM-LE}: Clasificador SVM, núcleo lineal, estrategia de
  selección de hiperparámetros mediante minimización del criterio del error
  empírico
\item
  \sbs{SVM-RE}: Clasificador SVM, núcleo de base radial (RBF), selección de
  hiperparámetros mediante minimización del criterio del error empírico,
\item
  \sbs{SVM-RR}: Clasificador SVM con núcleo RBF y estrategia de
  selección de hiperparámetros minimizando la cota radio-margen.
\end{itemize}
%

Algunos de los problemas generados se basan en las pruebas
suplementarias de los trabajos de referencia \cite{xue,ng}.
Por otra parte, se generaron problemas que clasifican los ejemplos de
la especie humana disponibles en la última versión de \dset\mirbase
(21.0, {junio de 2014}) entrenando con los conjuntos de entrenamiento
de los problemas principales \prob\tripletsvm{}, \prob\mipred{} y
\prob\micropred{}.
Por úlrimo, el problema llamado \prob\deltamirbase{} prueba la
capacidad del método de predecir los nuevos ejemplos de la especie
humana incorporados a la base de datos \dset\mirbase en la última
actualización.
