%
%
\subsection{SVM de margen duro}
%
La máquina de vectores de soporte más simple es aquella llamada ``de
margen duro'', y consiste en un clasificador lineal de máximo margen
que incorpora el ``truco del núcleo'': dado un núcleo
$k:X\times{}X\rightarrow\RR$, se asume que el conjunto de datos
transformado $\left((\zz_1,y_1)),\ldots,(\zz_\ell,y_\ell)\right)$,
$\zz_i=\BPhi(\xx_i)$, es linealmente separable en el espacio inducido
$Z$. El entrenamiento de la SVM de margen duro consiste en resolver el
problema de optimización (\ref{e2:svm-problem-kernel}), que en
notación vectorial se escribe
%
\begin{align}
  \label{e2:svm-hard-margin}
  \begin{split}
    \max_{\Balpha}\quad\tabs
      \B{1}^T\Balpha-\frac{1}{2}\Balpha^T\QQ\Balpha\\
    \T{sujeto a} \quad\tabs
      \yy^T\Balpha = 0, \\
      \tabs \alpha_i\geq 0,  i\in {1,\ldots,\ell }.
  \end{split}
\end{align}
%
La matriz $\QQ$ es semidefinida positiva con elementos por
$Q_{ij}=y_iy_jk(\xx_i,\xx_j)$, y $\B{1}$ es un vector columna de
$\ell$ elementos iguales a $1$.  El entrenamiento de este problema se
efectúa comúnmente mediante el algoritmo denominado SMO
(\eng{Sequential Minimal Optimization}, Optimización Mínima
Secuencial) \cite{smo} específicamente diseñado para la SVM, aunque
puede ser resuelto mediante cualquier algoritmo capaz de resolver un
problema de optimización cuadrático.

El modelo de la máquina de vectores de soporte de margen
duro viene dado por
%
\begin{align}
  \begin{split}
    h(\xx) &= \T{signo}\left(\langle\ww^*,\BPhi(\xx)\rangle+b^*\right),
  \end{split}
\label{e2:svm-model-hard0}
\end{align}
%
donde $(\ww^*,b^*)$ (solución al problema primal) viene dada por las
Ecuaciones \ref{e2:w-from-alpha} y \ref{e2:b-from-alpha}.
Incorporando el truco del núcleo, se tiene
%
\begin{align}
  \langle\ww^*,\BPhi(\xx)\rangle \tab=
  \sum_{i=1}^\ell{}y_i\alpha^*_ik(\xx_i,\xx),\\
  b^* \tab= y_j - \sum_{i=1}^\ell{}y_i\alpha^*_ik(\xx_i,\xx_j),
\end{align}
%
donde el subíndice $j$ es aquel para el cual se cumple $\alpha_j>0$.
Incorporando estos resultados al modelo $h(\xx)$, se tiene
%
\begin{align}
  h(\xx) &= \T{signo}\left(\sum_{i=1}^\ell{}y_i\alpha^*_ik(\xx_i,\xx)+b^*\right).
\label{e2:svm-model-hard}
\end{align}
%
