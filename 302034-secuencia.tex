%
%
\subsubsection{Características de la secuencia}
%
Las \caract{s} de la secuencia se proponen en los trabajos \e{miPred}
\cite{ng} y \e{microPred} \cite{batuwita}.
Se calculan directamente a partir de la cadena de texto que representa
la secuencia, ignorando las propiedades de la estructura secundaria.
Se incluyen medidas que cuentan la longitud de la secuencia, el número
de ocurrencias de las bases \ntA{}, \ntG{}, \ntC{} y \ntU{} en forma individual
y en combinaciones de dos posiciones consecutivas (dinucleótidos).
En la Tabla a continuación se enumeran las $23$ \caract{s} de la
secuencia \cite{ng,batuwita} según la posición que ocupan en el vector
de \caract{s}.
