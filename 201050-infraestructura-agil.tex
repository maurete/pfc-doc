%
\subsection{Tuberías y entrega continua}
%
Basándose en las técnicas de compilación, testing e integración
continuas, \citeauthor{humblefarley} extendieron en 2006 estas ideas a
la \e{entrega continua} \cite{humblefarley}. La misma planteaba la
idea de una \e{tubería de entrega} encargada de asegurar que todo el
código y la infraestructura estuvieran siempre en estado productivo,
de modo que pudieran ser desplegados de manera automática con
seguridad.
%
\subsection{Mejora continua}
%
En su libro de \citeyear{toyotakata} \cite{toyotakata},
\citeauthor{toyotakata} observó que la comunidad \e{lean} ignoraba la
enseñanza fundamental del sistema de producción de Toyota, a la cual
denominó el \e{kata de la mejora}\footnote{ \e{Kata} es una palabra
  japonesa originada en las artes marciales, y se refiere a los
  ejercicios que se practican constantemente en búsqueda de
  perfeccionar los movimientos.}. Cada organización tiene sus propias
rutinas de trabajo, y el \e{kata de la mejora} requiere crear una
estructura para practicar diariamente el trabajo de mejora de la
organización, ya que la práctica diaria es la que ``hace a la
perfección''. Establecer un ciclo constante de definir estados futuros
de resultados esperados para la semana, y la práctica diaria de
trabajos de mejora, deben ser la base para alcanzar una mejora a nivel
de la organización.
%
\subsection{La actualidad}
%
Hoy en día sabemos que muchas de las empresas de software más grandes
han incorporado las ideas de DevOps, alcanzando una velocidad sin
precedentes en la entrega del software y manteniendo altos estándares
de calidad y fiabilidad. Incluso son capaces de realizar experimentos
aplicando cambios en los entornos productivos y evaluar la respuesta
de los usuarios. Empresas tales como Netflix, Amazon y Facebook son
ejemplos claros del éxito de DevOps
\cite{handbook}. En paralelo a
estos desarrollos, Google desarrolló la disciplina de
\e{ingeniería de fiabilidad} (SRE), la cual puede entenderse como
una implementación de DevOps
\cite{sre}.
%
\section{Definición de DevOps}
%
DevOps propone prácticas, principios y una cultura buscando eliminar
los ``silos'' en las áreas de desarrollo, operaciones, redes y
seguridad de las TI. Podemos decir que DevOps, el desarrollo ágil y el
movimiento lean son ejemplos de una comprensión de cómo administrar
empresas de TI en el mundo moderno. Quizá la definición más concisa es
aquella elaborada por Gene Kim\footnote{ Gene Kim es uno de los
  promotores más conocidos del movimiento DevOps. Es coautor de The
  Phoenix Project \cite{phoenix} y del libro The DevOps Handbook
  \cite{handbook}, publicaciones consideradas como de referencia
  básica en la disciplina.} \cite{gruver}:
%
\begin{quote}
  \itshape
  DevOps es un conjunto de normas culturales y prácticas
  tecnológicas que permiten un flujo rápido del trabajo planificado
  desde el desarrollo, pasando por el testing hasta las operaciones,
  siempre manteniendo una alta fiabilidad, funcionamiento y
  seguridad. DevOps no se define en torno a las acciones tomadas, sino
  a partir de sus resultados: DevOps es un concepto amplio que abarca
  creencias y prácticas tales como la comunicación entre equipos y una
  cultura en común.
\end{quote}
%
Se identifican una serie de principios rectores que deberían guiar
cualquier implementación de DevOps en una organización. Por un lado,
Gene Kim ofrece un marco de ``las 3 vías'' \cite{3ways} para orientar
las decisiones a la hora de implementar DevOps. Desde otra
perspectiva, en el libro \e{The Site Reliability Workbook}
\cite{workbook} se proponen ``5 principios'' a seguir para una
adopción exitosa de DevOps. Finalmente, se describen los principios de
la ingeniería de fiabilidad de Google, una propuesta más práctica que
se ha popularizado gracias al libro {\eng{\citetitle{sre}}}
\cite{sre}.
%
\subsection{Las tres vías de DevOps}
%
Las ``tres vías'' describen los valores y filosofías que orientan los
procesos, procedimientos y prácticas de DevOps. En estas propuestas se
entiende a la producción del software como un sistema en el que el
trabajo fluye desde la izquierda, donde ingresan requerimientos, hacia
la derecha, donde se entrega el producto al cliente.
%
\subsubsection{Primera vía: Acelerar el flujo de valor }
%
La primera vía pone énfasis en el rendimiento del sistema (de
producción del software) desde una perspectiva global. Se debe
perseguir el objetivo de alcanzar un \e{flujo de valor} continuo,
valiéndose de la automatización y de un pensamiento sistémico para
detectar y eliminar tareas redundantes, burocráticas o repetitivas que
no agregan valor al servicio de software.

La idea de flujo de valor proviene de la filosofía lean y se refiere a
todo el proceso por el que pasa el software, desde el requerimiento
hasta la entrega. Se denomina valor a cualquier acción que se traduce
en un beneficio explícito para el cliente.

Las implicancias de poner en práctica la primera vía incluyen:
%
\begin{itemize}
\item Nunca entregar un producto con defectos conocidos a los
  trabajadores que están más adelante en la línea de producción
\item No permitir una optimización local dentro de un área o un equipo
  de trabajo que implique una degradación global en el sistema de
  producción
\item Siempre buscar incrementar el flujo, por ejemplo evitando
  acumular tareas por resolver, y evitar acumular trabajo ``en curso''
  resolviéndolo tan pronto como sea posible.
\item Generar una comprensión del sistema (de producción del software)
  desde una perspectiva global.
\end{itemize}
%
\subsubsection{Segunda vía: Acelerar la retroalimentación}
%
Se trata de maximizar el flujo de información desde los sistemas hacia
los desarrolladores y responsables del servicios. Esto se logra
generando una cultura de solución de problemas, utilizando
herramientas de telemetría e incorporando pruebas automáticas del
software y de la infraestructura para detectar errores tan pronto como
sea posible.

El resultado principal de aplicar la segunda vía es la capacidad de
comprender y dar respuesta a todos los clientes, sean éstos internos
(otra área de la organización) o externos (el usuario final).
%
\subsubsection{Tercera vía: Cultura de aprendizaje y experimentación continuos}
%
Aplicar el método científico en todos los aspectos del funcionamiento
de la organización, tratando las ideas como hipótesis a ser validadas
y promoviendo una cultura sin culpables (\e{blameless}), que
entiende una falla como un problema a solucionar mejorando el sistema
en lugar de señalar culpables.

Se trata de crear una cultura que promueva la experimentación
continua, incluso tomando riesgos y aprendiendo de los errores; y, por
otro lado, entender que la repetición y práctica continuas son los
prerrequisitos para alcanzar la perfección.

Las implicancias de esta tercera vía son: asignar tiempo explícito
para la mejora del trabajo diario, crear incentivos para que los
equipos asuman riesgos, e introducir fallas a propósito para aumentar
la resiliencia del software y la infraestructura.
%
\subsection{Los 5 principios de DevOps}
%
Los siguientes 5 principios de DevOps \cite{workbook} ofrecen una
descripción alternativa a las ideas expresadas en la propuesta de las
3 vías.
%
\subsubsection{Eliminar silos}
%
Cuando un área de la organización funciona de manera aislada de las
demás, se dice que el conocimiento que ésta maneja queda contenido
dentro de un ``silo''. La aparición de silos de conocimiento incentiva
una optimización local, sin perspectiva del objetivo global de la
organización. En particular, se rechaza la idea de mantener los
equipos de desarrollo y operaciones aislados entre sí.
%
\subsubsection{Normalizar los errores}
%
Los accidentes no son sólo atribuibles a las acciones aisladas de un
individuo, sino que más bien son el resultado de la falta de
salvaguardas para cuando las cosas salen mal, algo que es
inevitable. Por ejemplo, cuando una interfaz de usuario sugiere
aplicar por defecto la acción incorrecta y el operador se encuentra
bajo presión.

Las organizaciones tradicionales suelen tener una cultura de señalar a
la persona que comete un error y aplicar un castigo. Esto incentiva a
las personas a confundir los hechos, ocultar la verdad y culpar a
otros, lo que no son más que distracciones que no agregan valor a la
organización. En su lugar , es mucho más eficiente invertir este
esfuerzo en mejorar los procesos para reducir el impacto y/o para
acelerar la recuperación cuando se vuelva a cometer el mismo error.
%
\subsubsection{El cambio debe ser gradual}
%
La tercer idea clave es que los cambios son mejores cuando son
pequeños y frecuentes. Esto puede sonar radical en aquellas
organizaciones donde los cambios se planifican en detalle con mucha
antelación, pero no se trata de una idea nueva. La noción de que todos
los cambios deben ser considerados por personas experimentadas y
agrupados en lotes para efectuarlos de manera eficiente resulta ser
más o menos lo opuesto a las mejores prácticas.

Si bien es cierto que los cambios son riesgosos, la forma correcta de
aplicarlos es subdividirlos (siempre que sea posible) en componentes
más pequeños de menor riesgo. De este modo se genera una \e{tubería}
con un flujo constante de cambios de bajo riesgo. Acoplando esto a una
estrategia de tests automáticos para los cambios menores y a un
proceso confiable para revertir aquellos cambios que introducen
errores, se da origen a prácticas tales como la integración continua y
la entrega continua.
%
\subsubsection{Las herramientas y la cultura están interrelacionadas}
%
Las herramientas tecnológicas utilizadas son un componente importante
de la cultura de la organización. Resulta fundamental poner énfasis en
que los diferentes equipos de trabajo utilicen las mismas
herramientas, de modo que haya un buen entendimiento entre los
diferentes equipos. Por último, se debe tener presente que una buena
cultura puede sobrellevar falencias de las herramientas, pero no a la
inversa.
%
\subsubsection{La medición es fundamental}
%
La medición resulta fundamental a la hora de romper silos y la
resolución de incidentes. En cada caso, se trata de entender los
acontecimientos utilizando mediciones objetivas, verificar que la
situación mejora tal como se espera, y fundamentar con métricas
objetivas las conversaciones entre los diferentes roles de la
organización.
%
\subsection{La ingeniería de fiabilidad (SRE) de Google}
%
Una de las formas más difundidas de implementar DevOps es la
ingeniería de fiabilidad (\e{site reliability engineering}, \e{SRE})
propuesta por Google \cite{sre}. En comparación, DevOps es un concepto
más amplio y basado en disciplinas previamente establecidas, mientras
que SRE se originó en base a los aspectos prácticos de gestionar los
servicios dentro de Google \cite{workbook}.

SRE se define a partir de los lineamientos detallados a
continuación. Como se verá, tanto DevOps como SRE son similares.
%
\subsubsection{Las operaciones son un problema de software}
%
La idea básica de SRE es que ``hacer bien las operaciones es un
problema de software''. Esto implica que para resolver este problema,
se deberían aplicar los principios de la ingeniería de software. Esta
idea atraviesa todos los aspectos de la empresa, desde los cambios en
el negocio y los procesos hasta los problemas de software más
tradicionales, tales como la reescritura de un componente para
eliminar puntos de falla únicos dentro de la lógica de negocios.
%
\subsubsection{Administrar servicios utilizando niveles de servicio}
%
SRE no pretende alcanzar un 100\% de disponibilidad en ningún
servicio. En su lugar, los equipos del producto (desarrolladores) y de
SRE seleccionan un objetivo de disponibilidad (SLO) acorde al servicio
y al número de usuarios, y la administración del servicio se realiza
de acuerdo a ese objetivo. La determinación de la SLO a aplicar
requiere una gran colaboración entre todos los equipos. Las SLO tienen
también implicaciones culturales: al ser una decisión alcanzada entre
todos los interesados (\eng{stakeholders}), una violación de la SLO
hace que los equipos deban reunirse para decidir ajustes o cambios,
sin intención de buscar culpables.
%
\subsubsection{Trabajar para minimizar el trabajo manual}
%
SRE considera que cualquier trabajo manual es detestable: si una
máquina puede efectuar una operación, entonces debería hacerlo. Esta
es una política que no se encuentra en las organizaciones
tradicionales, en las cuales el ``trabajo'' significa trabajo
manual. Para SRE, el tiempo invertido en tareas operacionales
significa tiempo no utilizado para trabajar en proyectos.

Se deben identificar los lugares de donde provienen los requerimientos
de trabajo manual, para minimizarlos o eliminarlos por completo. Las
operaciones, sin embargo, aportan conocimiento vital a las
decisiones. Proveen una visión real del funcionamiento de los
sistemas.
%
\begin{quote}
  \sfbf{La sabiduría de producción}\smallskip\newline
  La ``sabiduría de producción'' refiere al conocimiento concreto del
  comportamiento de un sistema en producción, orientando las
  decisiones de diseño más allá de la visión ``de pizarra'', aislada
  de la realidad. Los tickets, llamados, y otros tipos de solicitudes
  que reciben los equipos aportan una conexión con la dimensión real
  del comportamiento y diseño de los sistemas.
\end{quote}
%
\subsubsection{Automatizar el trabajo del año entero}
%
El desafío real de la automatización es determinar qué automatizar,
bajo qué condiciones, y cómo automatizarlo. SRE define una cuota
máxima de 50\% para la cantidad de trabajo manual, en contraposición
al trabajo de ingeniería que produce valor duradero. En lugar de
entenderse como un tope, este límite se considera como una garantía
que permite a los ingenieros disponer de un tiempo mínimo del 50\%
para diseñar mejoras duraderas. Con el tiempo, el equipo de SRE
termina automatizando todo lo que puede respecto a un servicio. Esto
le permite incorporar nuevos servicios, o incluso dedicarse a tareas
completamente diferentes.
%
\subsubsection{Moverse rápido reduciendo el costo de las fallas}
%
Uno de los principales beneficios de SRE es una mejora en la velocidad
de la producción del software. Al encargar al equipo de SRE la
solución de problemas complejos, se logran reducir los tiempos de
reparación (MTTR) de las fallas comunes, liberando el tiempo de los
ingenieros desarrolladores.
%
\subsubsection{Propiedad compartida con los desarrolladores}
%
La distinción rígida entre ``desarrollo de aplicaciones'' y
``producción'' se considera contraria a los objetivos de
productividad. Más aún cuando se segregan las responsabilidades y se
considera la importancia de un rol por encima del otro.

Si bien el equipo de SRE se inclina más hacia los problemas de
producción antes que hacia problemas de la lógica del negocio,
comparten sus habilidades y herramientas de desarrollo de software con
el área de desarrollo. En general, un ingeniero SRE tiene más
experiencia en lo referente a la disponibilidad, latencia,
rendimiento, eficiencia, gestión del cambio, monitoreo, respuesta a
emergencias, y planificación de recursos de los servicios que tienen a
cargo. Estas habilidades definen el rol del ingeniero SRE \e{dentro}
del equipo de desarrollo. La experiencia demuestra que la
productividad aumenta cuando se permite, por ejemplo, a los
desarrolladores ajustar configuraciones y a los ingenieros SRE
trabajar sobre el código de la aplicación.
%
\subsubsection{Utilizar las mismas herramientas}
%
En muchos sentidos, las herramientas utilizadas determinan la forma de
trabajar de las personas. Todos los equipos a cargo de un servicio (ya
sea el equipo de desarrollo, de SRE, o cualquier otro) deben utilizar
las mismas herramientas, sin importar cuál sea su rol dentro de la
organización. No es posible administrar de forma fiable un servicio
cuando cada equipo utiliza herramientas con comportamientos diferentes
(que pueden ser potencialmente catastróficos) en situaciones
diferentes. Cuanto mayor es la divergencia, menores son los beneficios
obtenidos al mejorar cada herramienta.
%
%
\section{DevOps en la práctica}
%
DevOps por definición se limita a proponer principios y una cultura en
común, sin especificar prácticas ni herramientas tecnológicas
específicas, ya que su aplicación dependerá de las particularidades y
necesidades de cada organización. Dicho esto, pueden identificarse
algunas prácticas generales comúnmente asociadas en una implementación
de DevOps \cite{awsdevops}, las cuales se describen a continuación.
%
\subsection{Integración y entrega continua}
%
La integración continua es una práctica mediante la cual los
desarrolladores publican en forma periódica los cambios del código al
repositorio central, lo cual dispara la ejecución de pruebas
automáticas. Esto permite encontrar y arreglar los errores con mayor
rapidez, mejorando la calidad del software. Y en el caso de que no
haya errores en los tests, el desarrollador se ahorra tiempo ya que no
tuvo que ejecutarlos manualmente. La entrega continua va un paso más
allá y, en los casos en que la integración continua haya sido exitosa,
se encarga de preparar el código y entregar los artefactos resultantes
a la fase de producción, listos para ser desplegado. En conjunto, este
esquema se conoce como \e{integración y entrega continuas} (CI/CD,
\eng{Continuous Integration/Continuous Delivery})
\cite{devopscicd}. Una implementación exitosa de la integración y
entrega continuas permite al desarrollador enfocarse en el proceso
creativo de la escritura de código y reduce el tiempo requerido para
la entrega del software \cite{humblefarley}.
%
\subsection{Infraestructura como código}
%
La infraestructura como código es una práctica mediante la cual se
gestiona la infraestructura con prácticas de desarrollo de software,
como el control de versiones y la integración continua. Los
desarrolladores y administradores de sistemas interactúan con la
infraestructura modificando el código que la define, y luego aplican
esos cambios utilizando herramientas específicas que llevan a cabo las
acciones requeridas para ajustar los recursos a la especificación del
código. Incluso se pueden aplicar los cambios de manera automática
aprovechando las herramientas de integración continua. Al estar
definida por código, la infraestructura se puede implementar con
rapidez y de forma reproducible en múltiples entornos.

\subsection{Observabilidad}

La observabilidad hace referencia a las prácticas de recopilar y
analizar métricas y registros para ver cómo el desempeño de las
aplicaciones y la infraestructura afectan la experiencia del usuario
final. Es necesario efectuar un monitoreo activo de los servicios y
generar alertas para cuando se detecten anomalías. Las herramientas de
visualización y análisis en tiempo real permiten visualizar los
cambios y evaluar el impacto de éstos sobre los servicios de manera
objetiva y cuantificable, lo que también permite diseñar estrategias
de forma proactiva.

\subsection{Comunicación y colaboración}

El incremento en la comunicación y la colaboración en una organización
es uno de los aspectos culturales clave de DevOps. El uso de las
herramientas de DevOps y la automatización del proceso de entrega de
software establece la colaboración al reunir físicamente los flujos de
trabajo y las responsabilidades de los equipos de desarrollo y
operaciones. Además, estos equipos establecen normas culturales
sólidas que giran en torno a compartir información y facilitar la
comunicación mediante el uso de aplicaciones de chat, sistemas de
seguimiento de proyectos o problemas y wikis. De este modo, se acelera
la comunicación entre los equipos de desarrollo y operaciones e
incluso con otros equipos, como marketing y ventas, lo que permite que
todos los departamentos de la organización se alineen mejor con los
objetivos y proyectos.
