%
\subsubsection{Error de generalización}
%
El error de generalización
$\mathcal{R}_\nu:\C{H}\rightarrow\RR^{\geq0}$ es el error esperado de
un modelo $h$ al clasificar un ``nuevo'' ejemplo independiente
del conjunto $D$, y se define a partir de la función de pérdida $L$ según
%
\begin{align}
  \mathcal{R}_\nu=\mathds{E}_\nu\left[L(\xx,\yy,h(\xx))\right]
  =\int_{X\times Y} L(\xx,\yy,h(\xx))\, d\nu(\xx,\yy).
\end{align}
%
El cálculo de este error requiere conocer la distribución generadora
de datos $\nu$.
Sin embargo, toda la información que se tiene de $\nu$
es la realización $D$, para la cual no se conoce su probabilidad.
Por ello, la aplicación del concepto de error de generalización es
teórica, y se utilizan en la práctica medidas que lo aproximan.
