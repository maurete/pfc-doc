\chapter{Automatización de la infraestructura}

El primer aspecto trabajado en el proyecto fue la automatización de la
infraestructura, entendida en general como la implementación de
servicios que reemplacen las operaciones manuales. Para alcanzar la
automatización fue necesario el despliegue de nuevos servicios, la
codificación de nuevos ``roles'' (módulos o código reutilizable) en
Ansible, y la adaptación de la infraestructura para dar soporte a la
funcionalidad requerida.

La implementación se llevó a cabo en dos etapas. Al principio del
proyecto se puso en servicio la interfaz web de Ansible y se
codificaron los roles que reemplazan las operaciones manuales de
infraestructura. Asimismo, hacia el final del desarrollo del proyecto
se trabajó la automatización de los servidores de frontend y la
actualización de los repositorios de código y de artefactos (GitLab y
Nexus).

\section{Interfaz web para Ansible}

Ya se ha expresado que el equipo de infraestructura utiliza la
herramienta Ansible para la configuración y el mantenimiento de la
infraestructura de varios servicios. Las configuraciones de las
instancias virtuales de los servicios se escriben como código un
repositorio Git interno. A la hora de aplicar la configuración de un
servicio, se ejecuta Ansible desde la línea de comandos y el software
se conecta a las instancias vía SSH y aplica la configuración definida
en el código.

El objetivo principal de la implementación de una interfaz Web para
Ansible es permitir la ejecución del código Ansible por parte de todos
los integrantes de la DIPT y no sólo de parte del equipo de
Infraestructura. La interfaz web permite además transparentar la
gestión de la configuración de los servicios, con una estructura de
permisos y conservando registro de los cambios aplicados sobre cada
servicio.

El software elegido como interfaz web para Ansible fue AWX\footnote{
  https://github.com/ansible/awx}. Se trata de la versión libre del
producto comercial Ansible Tower, ofrecido por los mismos
desarrolladores de Ansible. La implementación se realizó a partir del
código Ansible provisto por los autores de AWX, el cual fue expandido
para incluir la configuración de las contraseñas y claves SSH
utilizadas internamente en la DIPT, la autenticación y autorización
integrada al servicio LDAP, y la configuración automática de los
proyectos a partir del código especificado en el repositorio de AWX.

Además del despliegue del servicio AWX, fue necesario configurar un
usuario SSH con permisos específicos en todas las instancias
``controladas'' por AWX. Se modificaron los \e{templates} de
creación de instancias virtuales para que esta configuración sea
incluida automáticamente en todos los nuevos servicios. También fue
necesaria la modificación de reglas de firewall para que AWX pueda
conectarse a las instancias vía SSH.

En cuanto a la interfaz web de AWX, el lanzamiento de un nuevo trabajo
resulta suficientemente intuitivo. Por ejemplo, si se desea aplicar un
cambio en el servicio \e{hermes} en el entorno de
\e{producción}, el responsable del servicio puede resolver esta
tarea siguiendo la secuencia de pasos:

\begin{enumerate}
\item Editar la configuración en el repositorio Git del servicio.
\item Acceder a AWX e ingresar a la sección \e{Plantillas}.
\item Ubicar la plantilla \e{hermes\_playbook} y presionar el botón de
  lanzamiento.
\item Especificar los datos solicitados en el diálogo de lanzamiento
  (\iflatexml{}Figura~\ref{fig:awxlaunch}\else\autoref{fig:awxlaunch}\fi),
  en particular la \e{rama} de Git del código donde se han efectuado
  los cambios, y el \e{límite} que determina el entorno sobre el cual
  se aplicará la configuración.
\item Continuar y esperar a que se complete la configuración.
\end{enumerate}
%
