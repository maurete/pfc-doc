\chapter{4 Automatización de la infraestructura}

El primer aspecto trabajado en el proyecto fue la automatización de la
infraestructura, entendida en general como la implementación de
servicios que reemplacen las operaciones manuales. Para alcanzar la
automatización fue necesario el despliegue de nuevos servicios, la
codificación de nuevos ``roles'' (módulos o código reutilizable) en
Ansible, y la adaptación de la infraestructura para dar soporte a la
funcionalidad requerida.

La implementación se llevó a cabo en dos etapas. Al principio del
proyecto se puso en servicio la interfaz web de Ansible y se
codificaron los roles que reemplazan las operaciones manuales de
infraestructura. Asimismo, hacia el final del desarrollo del proyecto
se trabajó la automatización de los servidores de frontend y la
actualización de los repositorios de código y de artefactos (GitLab y
Nexus).

\section{Interfaz web para Ansible}

Ya se ha expresado que el equipo de infraestructura utiliza la
herramienta Ansible para la configuración y el mantenimiento de la
infraestructura de varios servicios. Las configuraciones de las
instancias virtuales de los servicios se escriben como código un
repositorio Git interno. A la hora de aplicar la configuración de un
servicio, se ejecuta Ansible desde la línea de comandos y el software
se conecta a las instancias vía SSH y aplica la configuración definida
en el código.

El objetivo principal de la implementación de una interfaz Web para
Ansible es permitir la ejecución del código Ansible por parte de todos
los integrantes de la DIPT y no sólo de parte del equipo de
Infraestructura. La interfaz web permite además transparentar la
gestión de la configuración de los servicios, con una estructura de
permisos y conservando registro de los cambios aplicados sobre cada
servicio.

El software elegido como interfaz web para Ansible fue AWX\footnote{
  https://github.com/ansible/awx}. Se trata de la versión libre del
producto comercial Ansible Tower, ofrecido por los mismos
desarrolladores de Ansible. La implementación se realizó a partir del
código Ansible provisto por los autores de AWX, el cual fue expandido
para incluir la configuración de las contraseñas y claves SSH
utilizadas internamente en la DIPT, la autenticación y autorización
integrada al servicio LDAP, y la configuración automática de los
proyectos a partir del código especificado en el repositorio de AWX.

Además del despliegue del servicio AWX, fue necesario configurar un
usuario SSH con permisos específicos en todas las instancias
``controladas'' por AWX. Se modificaron los \e{templates} de
creación de instancias virtuales para que esta configuración sea
incluida automáticamente en todos los nuevos servicios. También fue
necesaria la modificación de reglas de firewall para que AWX pueda
conectarse a las instancias vía SSH.

En cuanto a la interfaz web de AWX, el lanzamiento de un nuevo trabajo
resulta suficientemente intuitivo. Por ejemplo, si se desea aplicar un
cambio en el servicio \e{hermes} en el entorno de
\e{producción}, el responsable del servicio puede resolver esta
tarea siguiendo la secuencia de pasos:

\begin{enumerate}
\item Editar la configuración en el repositorio Git del servicio.
\item Acceder a AWX e ingresar a la sección \e{Plantillas}.
\item Ubicar la plantilla \e{hermes\_playbook} y presionar el
  botón de lanzamiento.
\item Especificar los datos solicitados en el diálogo de lanzamiento
  (Figura 4.1), en particular la \e{rama} de Git del código donde
  se han efectuado los cambios, y el \e{límite} que determina el
  entorno sobre el cual se aplicará la configuración.
\item Continuar y esperar a que se complete la configuración.
\end{enumerate}
\includegraphics[width=4.77in,height=2.88in]{img_1.png}


\e{Figura 4.1 Diálogo de lanzamiento de la configuración de un
  servicio en AWX.}

En la Figura 4.2 se puede ver la pantalla principal de AWX con un
vistazo general que incluye las ``plantillas'' y los ``trabajos''
aplicados recientemente. En la terminología AWX, una plantilla refiere
al código Ansible que define la configuración para un servicio, y un
trabajo es el registro generado al momento de ejecutar esta
configuración sobre un entorno específico de un servicio.

En la Figura 4.3 se observa la vista con el registro de un
trabajo. Allí se detalla, entre otros datos, el resultado de la
ejecución, la persona que lanzó el trabajo, el número de revisión del
código y el entorno sobre el cual fue aplicado.

\includegraphics[width=6.5in,height=4.09in]{img_2.png}


\e{Figura 4.2 Pantalla principal de AWX.}

\includegraphics[width=6.5in,height=4.09in]{img_3.png}


\e{Figura 4.3 Registro de un trabajo de AWX.}

\section{Automatización de operaciones}

En el análisis del flujo de valor del proceso de creación de un nuevo
servicio se puso en evidencia la necesidad de automatizar ciertas
operaciones de la infraestructura. Para resolver este problema se
escribió código Ansible en forma de dos \e{roles}, denominados
\e{bootstrap} y \e{zabbix-host}. Estos roles, al ser
incorporados al código que gestiona la configuración de los servicios
(actuales o nuevos), permiten efectuar las operaciones que antes se
realizaban en forma manual.

Además de la codificación de los roles, se efectuaron los siguientes
cambios para dar soporte a la automatización en la infraestructura:

\begin{itemize}
\item En GitLab, se otorgaron permisos para poder crear proyectos a
  todos los miembros de la Dirección, y se otorgó en general acceso a
  todos los repositorios internos.
\item Se agregó soporte para la ejecución de los scripts de
  administración en modo no interactivo.
\item Se crearon usuarios específicos en los distintos servicios para
  autorizar las interacciones automáticas.
\item Se habilitaron reglas de firewall para permitir la comunicación
  automatizada.
\end{itemize}
El rol \e{bootstrap} está diseñado para ser invocado al principio
de la ejecución y se encarga del registro de entradas en los
servidores DNS y DHCP, de la creación/modificación de la instancia
virtualizada y asegura la configuración del servicio de backup. Al ser
ejecutado, el rol realiza internamente las siguientes tareas:

\begin{itemize}
\item Se conecta al servidor utilizado para la administración de la
  infraestructura (\e{cerbero}).
\item En este servidor, ejecuta comandos que verifican la existencia
  de las entradas DNS y DHCP requeridas para la instancia.
\item Si las entradas no existen, invoca el comando para crearlas y
  luego ejecuta otro script que propaga los cambios hacia los
  servidores DNS Y DHCP.
\item Se conecta vía API al servidor que administra las instancias
  virtualizadas (\e{copernico} o \e{flanders}).
\item Verifica en forma programática que la instancia existe y que sus
  recursos reflejan lo especificado en el código.
\item Si es necesario, mediante llamadas a la API crea la instancia o
  modifica los recursos asignados a la misma.
\item Se conecta al servidor GitLab para descargar el inventario del
  servicio de backup.
\item Si es necesario, agrega la instancia al servicio de backup y
  dispara una reconfiguración del mismo.
\end{itemize}
A diferencia de \e{bootstrap}, el rol \e{zabbix-host} está
diseñado para ser ejecutado al final del \e{playbook} (el guión
de configuración aplicado por Ansible), lo cual le permite detectar
qué servicios han sido instalados en la instancia, y configura el
servicio de monitoreo con opciones preestablecidas para cada tipo de
servicio. La interacción con el servicio de monitoreo se realiza
mediante llamadas a la API del mismo.

Para ilustrar la ganancia en productividad de la implementación de
estos roles, antes era necesario efectuar las siguientes tareas
manuales para crear un nuevo servicio:

\begin{itemize}
\item Conectarse al servidor de administración y actualizar el código
  SVN.
\item Ejecutar el script para registrar las entradas DNS y DHCP.
\item Ejecutar el script para recargar el estado de los servicios DNS
  y DHCP.
\item Conectarse al servidor principal del cluster de virtualización.
\item Ejecutar comandos de creación de la instancia, revisando las
  opciones correctas según los recursos solicitados para el servicio.
\item En la PC de trabajo actualizar el código Git donde se administra
  el servicio de backup.
\item Agregar entrada en el inventario de backup para la instancia
  recién creada.
\item Aplicar la configuración del servicio de backup ejecutando
  Ansible mediante la línea de comandos.
\item En el servidor de administración, ejecutar el script para la
  configuración del monitoreo de la instancia.
\end{itemize}
Luego de la implementación de los roles de automatización, todas las
acciones anteriores se reduce a escribir algunas líneas en el código
Ansible que gestiona el servicio:

\begin{verbatim*}
bootstrap_host_network: sdmz1
bootstrap_ganeti_cluster: copernico.intranet
bootstrap_instance_disk_size: 25600
bootstrap_instance_memory: 8192
bootstrap_instance_vcpus: 8
bootstrap_instance_os_type: stretch+default
bootstrap_instance_pnode: saturno.intranet
bootstrap_instance_snode: triton.intranet

zabbix_host_macros:
  URL: https://servicios.unl.edu.ar/mesadeentradas/
\end{verbatim*}

Este código reemplaza las acciones manuales previas, eliminando
incluso la necesidad de intervención del equipo de
infraestructura. Junto con la implementación de la interfaz web para
Ansible, esta automatización brinda una plataforma de ``autoservicio''
para el aprovisionamiento de la infraestructura para todos los
miembros de la DIPT. Con todo esto se evitan errores de las
operaciones manuales y se eliminan tiempos de espera.

\section{Servidores de frontend}

Los servidores de frontend son los encargados de recibir las
peticiones HTTP(S) externas y direccionar el tráfico al servicio
interno correspondiente, siguiendo una serie de reglas. La necesidad
de cambiar el proceso de configuración se puso en evidencia en el
análisis del flujo de valor para la creación de un nuevo servicio. El
mismo resultaba tedioso y fundamentalmente no era escalable, lo cual
derivó en la búsqueda de una solución alternativa.

Una forma de atacar el problema hubiera sido configurando una tubería
de CI/CD para publicar automáticamente los cambios en la
configuración. Esto hubiera permitido automatizar la tarea de
\e{aplicar} los cambios, aunque no se hubiera resuelto el
problema de escalabilidad y mantenibilidad del servicio, ya que aún
resultaría necesario escribir la configuración de los servidores en su
propio repositorio y en su lenguaje específico.

La solución implementada consistió en un cambio radical en la
arquitectura, creando un nuevo servicio de frontend capaz de
configurarse a sí mismo a partir del estado actual de los servicios de
software ofrecidos en la Dirección. El seguimiento de este
\e{estado actual}, necesario para la auto-configuración de los
frontends, se realiza mediante otro servicio denominado Consul.

Con esta implementación se eliminaron los pasos de editar la
configuración del frontend, conectarse al servidor y efectuar un
reinicio manual, tareas que requerían de los permisos de un
administrador de sistemas. Tal como ocurrió con la creación del rol
\e{bootstrap}, ahora basta con editar algunas líneas en el código
Ansible que define la configuración del servicio.

Toda la nueva configuración se escribió como código y se aprovisionó
mediante Ansible.

\subsection{Servicio Consul}

Consul\footnote{ https://www.consul.io/} es un sistema distribuido que
provee interconexión entre servicios de manera independiente de la
infraestructura subyacente. Cada nodo ejecuta un \e{agente} que
publica los servicios ofrecidos por el mismo y se encarga de verificar
el estado de los servicios.

Los agentes se comunican con los \e{servidores} de Consul, los
cuales funcionan en un esquema de alta disponibilidad que asegura un
estado consistente incluso cuando hay fallas en algún servidor. Los
servidores mantienen un \e{catálogo}, el cual agrega la
información suministrada por los agentes. La información contenida en
el catálogo incluye los servicios disponibles, los nodos que ofrecen
estos servicios y el estado actual de los mismos. Otros componentes de
la infraestructura (tal es el caso de los servidores de frontend)
pueden consultar el catálogo mediante una API, lo que les permite
descubrir los servicios y nodos disponibles.

La implementación de Consul abarcó dos aspectos. Por un lado, se creó
el \e{servicio Consul}, que incluye la configuración de las
instancias que contienen los servidores de Consul. Por el otro, fue
necesario modificar el código Ansible que sirve de base para todos los
servicios, para que éstos incorporen el \e{agente} de Consul como
parte de su configuración básica.

La implementación del servicio (los servidores Consul) se llevó a cabo
siguiendo las recomendaciones de la documentación oficial. Se creó un
grupo de cinco servidores distribuidos en las diferentes subredes de
la infraestructura para asegurar mayor disponibilidad del servicio
ante fallas de conectividad.

\subsection{Servicio de frontend}

Para el nuevo servicio de frontend se utilizó el software
Traefik\footnote{ https://containo.us/traefik/}, un servidor de proxy
reverso que tiene la capacidad inspeccionar el estado de los servicios
en Consul y ajustar su propia configuración en consecuencia. Traefik
también se encarga de la generación y renovación automática de los
certificados TLS de Let’s Encrypt, una tarea que antes era gestionada
en forma semi-automática por el equipo de infraestructura.

Dado que Traefik no ofrece soporte para funcionar como servidor de
contenido HTTP(S) básico, en las instancias de frontend se instaló
también un servidor Nginx, que se encarga de servir contenido estático
tal como las páginas de error.

El código Ansible que configura el servicio de frontend fue diseñado
de modo que los cambios en la infraestructura tengan el menor impacto
posible sobre la disponibilidad. Esto se logró mediante la
configuración de un servicio Traefik secundario, el cual atiende las
peticiones mientras se migra la configuración del servicio principal.

\subsection{Proceso de migración}

El servicio de frontend consta de tres instancias, cada una de las
cuales atiende el tráfico en una dirección IP diferente. Esto es así
debido a las particularidades de la infraestructura utilizada por la
Dirección. Al tratarse de un servicio crítico, durante la migración se
conservó en todo momento la posibilidad de revertir los cambios de
manera inmediata. Esto se conoce como despliegue azul verde
(\e{blue green deployment})
\href{https://www.zotero.org/google-docs/?WMXTmt}{[18]}.

Antes de comenzar con la migración de los frontends, se configuró el
servicio Consul y se verificó el correcto funcionamiento del mismo. En
términos generales, la migración de cada servicio de frontend se
realizó siguiendo la secuencia de pasos explicada a continuación. En
todos los casos, el entorno \e{verde} hace referencia al nuevo
servicio (con Traefik+nginx), y el entorno azul al servicio de
frontend anterior (nginx con configuración estática). La configuración
de un frontend en el entorno azul se muestra en la Figura 4.4.

\begin{enumerate}
\item Se creó una nueva instancia de frontend (entorno verde), con una
  regla por defecto para delegar todo el tráfico al frontend anterior
  (entorno azul).
\item Se modificaron las reglas de firewall para direccionar el
  tráfico del puerto 80 (HTTP) al entorno verde. Esto permitió la
  generación automática de nuevos certificados HTTPS para el entorno
  verde.
\item Se validaron los certificados generados desde la estación de
  trabajo, forzando la conexión a través del entorno verde. Una vez
  generados los certificados HTTPS, se configuró el firewall para
  direccionar el tráfico del puerto 443 (HTTPS, por donde se publica
  la gran mayoría de los servicios) al entorno verde.
\item Para cada servicio correspondiente al frontend, de a uno por
  vez, se instaló y configuró el agente Consul. De este modo se logró
  publicar el servicio en forma directa a través del entorno verde. Se
  verificó el acceso correcto de cada servicio (Figura 4.5).
\item Una vez migrados todos los servicios, se eliminó el entorno azul
  (Figura 4.6).
\end{enumerate}
\begin{tabular}{|l|}
\hline \includegraphics[width=4.35in,height=3.46in]{img_4.png}


\e{Figura 4.4. Configuración inicial de un frontend. La
  información de ruteo del tráfico HTTP(S) es estática y se almacena
  en un repositorio SVN. Cada actualización de reglas requiere la
  edición de los archivos de configuración y una recarga manual.}
\\ \hline
\end{tabular}
\begin{tabular}{|l|}
\hline \includegraphics[width=4.5in,height=3.54in]{img_5.png}


\e{Figura 4.5. Proceso de migración mediante una estrategia de
  despliegue azul verde. El nuevo frontend se encuentra configurado y
  rutea el tráfico en forma directa a los servicios registrados en
  Consul.} \\ \hline
\end{tabular}
\includegraphics[width=4.5in,height=3.54in]{img_6.png}


\e{Figura 4.6. Servicio de frontend de configuración
  dinámica. Los servicios se identifican a sí mismos ante los
  servidores Consul. Los cambios en los servicios se propagan de
  manera automática y los frontends actualizan sus reglas de ruteo sin
  reiniciarse.}

\section{Actualización de los repositorios}

Como parte de las tareas de adecuación de la infraestructura, se
actualizaron el repositorio de código GitLab y el repositorio de
artefactos Nexus. Ninguno de los dos servicios contaba con la
configuración escrita como código Ansible, ya que su implementación
fue anterior a la adopción de esta herramienta.

\subsection{Repositorio de código GitLab}

Se decidió actualizar el repositorio de código GitLab para corregir
vulnerabilidades y aprovechar las nuevas características ofrecidas por
sus desarrolladores. Como parte de estas actualización se implementó
además el código Ansible necesario para gestionar la infraestructura
del servicio. Este código permitió la configuración del servicio de
\e{runners} que es utilizado para la ejecución de los procesos
automáticos de integración y entrega continuas.

Se actualizó el software desde la versión 9.4 hasta la 13.1. Para
aplicar las actualizaciones se utilizaron las herramientas provistas
por el fabricante. El proceso se realizó de manera incremental,
migrando entre versiones mayores consecutivas hasta alcanzar la
versión actual. Durante las actualizaciones el servicio estuvo fuera
de línea.

\subsection{Repositorio de artefactos Nexus}

El repositorio de artefactos Sonatype Nexus funcionaba en su versión
2, la cual no recibe más soporte por parte del desarrollador. El
proceso fue más complejo que en el caso de GitLab, ya que la migración
de datos entre las versiones 2 y 3 no es automática. La secuencia de
pasos a continuación describen el proceso efectuado.

\begin{itemize}
\item El primer paso fue actualizar el servicio a la última versión de
  la serie 2. Para ello se escribió código Ansible, el cual fue
  probado en un entorno de test. Una vez ajustado el procedimiento, se
  aplicaron los cambios en el entorno de producción.
\item El segundo paso fue escribir código Ansible para la
  configuración de la versión 3. Se aplicó este código de modo que
  ambas versiones funcionaran en la instancia en forma simultánea.
\item Se realizó la migración entre versiones siguiendo el proceso
  explicado la documentación del desarrollador.
\item Se eliminó la instalación correspondiente a la versión 2 y se
  ajustó el código Ansible para que utilice la vesión 3.
\end{itemize}
Una vez actualizado el repositorio, se descubrió que la nueva versión
introdujo cambios incompatibles en su interfaz API. Por ello fue
necesario corregir los scripts de operaciones que utilizan
funcionalidades de esta interfaz.
