%
%
\subsection{Búsqueda exhaustiva}
%
La estrategia de búsqueda exhaustiva utiliza como función objetivo el
valor $G_m$ promedio de validación cruzada, y selecciona los
hiperparámetros óptimos mediante prueba y error dentro de un rango
preestablecido. Dado que la naturaleza de los hiperparámetros es
diferente según el clasificador, el algoritmo de búsqueda también
difiere en cada caso.
%
\subsubsection{Perceptrón multicapa}
%
El hiperparámetro del perceptrón multicapa es el número de neuronas en
la capa oculta $h$, una variable discreta no negativa. La
inicialización aleatoria del perceptrón multicapa introduce
fluctuaciones aleatorias en la función objetivo $G_m$ de validación
cruzada, lo que implica que se deben efectuar una gran cantidad de
repeticiones de inicialización--entrenamiento--prueba para obtener
resultados fiables.  Por ello, en lugar de efectuar la búsqueda sobre
todos los valores posibles, se prueban 20 valores de $h$ entre 0 y 200
en una escala aproximadamente logarítmica:
%
\begin{align}
  \label{mlp-hidden-tries}
  h=0,1,2,3,4,5,7,9,11,14,19,24,32,41,54,70,91,118,154,200.
\end{align}
%
Por defecto, se efectúan 5 repeticiones de
inicialización--entrenamiento--prueba para cada valor de $h$. Se
selecciona aquel valor de $h$ para el cual se obtuvo el mayor valor de
$G_m$ de validación cruzada en promedio.
%
\subsubsection{Máquina de vectores de soporte}
%
Los hiperparámetros a optimizar en una máquina de vectores de soporte
son $C$ y $\gamma$ (para núcleo RBF), que son variables reales no
negativas. El algoritmo de búsqueda se basa en la técnica de
\e{búsqueda en la grilla} propuesta por \citeauthor{hsu} en
\cite{hsu}.

La idea general de esta estrategia es considerar cada combinación de
hiperparámetros $(C,\gamma)$ como puntos en el plano
$C\gamma$. Comenzando por una serie de puntos espaciados regularmente
en espacio logarítmico $(\log C,\log\gamma)$, que determinan la
``grilla'' inicial, se entrena y evalúa el clasificador sobre cada
punto (cada combinación de hiperparámetros), para luego interpolar
(``refinar la grilla'') en las cercanías de los puntos donde se
obtiene el mayor rendimiento. La búsqueda continúa repitiendo el
procedimiento sobre los puntos no evaluados hasta satisfacer un
criterio de corte.

En primer lugar, se definen los puntos de muestreo para los
hiperparámetros
%
\begin{align}
  \label{initial-grid}
  \log_2 C     \tab= -5, -3, -1, 1, \ldots, 15, \tabs
  \log_2\gamma \tab= -15,-13, -11, \ldots, 3,
\end{align}
%
que determinan la grilla inicial. Para cada punto $(C_i,\gamma_j)$, se
entrena y prueba el clasificador SVM, obteniendo un ${G_m}_{ij}$
promedio de validación cruzada.

En etapas sucesivas, se interpola la grilla alrededor de los puntos
donde se obtuvieron los mayores valores $G_m$. Se definen tres
algoritmos heurísticos que determinan cuáles puntos interpolar:
%
\begin{itemize}
%
\item \e{Zoom}: En un primer paso, se convoluciona la grilla con una
  ventana cuadrada uniforme de valor unitario, obteniendo una versión
  ``suavizada'' de la grilla. Se interpolan puntos en una región
  cuadrada centrada en el punto con mayor valor $G_m$ ``suavizado''.
%
\item \e{Umbral}: A partir de un umbral $G_m$ definido por el usuario,
  por defecto el percentil 90, se interpola en cada dimensión
  alrededor de los puntos por encima del umbral.
%
\item \e{$n$-mejores}: Similar al umbral, se seleccionan $n$ puntos
  con los mayores valores de $G_m$. Se interpola la grilla en ambas
  dimensiones alrededor de estos puntos.
%
\end{itemize}
%
Una vez interpolada la grilla, se calcula el valor $G_m$ promedio de
validación cruzada para los puntos interpolados. El procedimiento de
refinamiento se repite hasta un máximo de $N$ iteraciones o hasta
satisfacer un criterio de corte.

La búsqueda en la grilla tiene la ventaja de ser conceptualmente
simple, sin embargo, se torna inviable cuando el vector de parámetros
$\B{\theta}$ contiene más de 2 elementos. Dado que se trata de un
método de búsqueda exhaustiva, resulta generalmente lento.  Cuando se
trabaja con un clasificador SVM con núcleo lineal, la grilla
tiene una única dimensión: la del hiperparámetro $\log C$.
