%
%
\subsubsection{Especificación del formato de entrada}
%
La especificación a continuación define el formato de los archivos de
entrada interpretados por el método.
Los archivos de entrada son de tipo texto plano, y cada uno
contiene uno o más ejemplos contiguos.
Cada ejemplo se describe por una línea de descripción, una o más
líneas de secuencia, y opcionalmente una o más líneas de estructura
secundaria, según las reglas siguientes:
%
\begin{itemize}
\item
  La línea de descripción comienza con el carácter ``mayor que''
  (\mono{>}) seguido de un identificador (llamado ``número de
  acceso'') de la secuencia.
  Continúa con una descripción del ejemplo que puede contener, por
  ejemplo, el nombre científico de la especie o la posición de la
  secuencia dentro del genoma de la especie.
  Tanto el nombre identificador como la descripción son opcionales.
  Cada línea de descripción leída indica el comienzo de un nuevo
  ejemplo.
\item
  Las líneas de secuencia siguen a la de descripción, y consisten en
  caracteres contiguos que representan las bases de los nucleótidos.
  En el ácido ribonucleico (RNA), un nucleótido contiene una de las
  bases adenina (\mono{A}), citosina (\mono{C}), guanina (\mono{G}), o
  uracilo (\mono{U}), luego, los caracteres permitidos en la secuencia
  son \mono{ACGU} y sus variantes en minúsculas \mono{acgu}.
  También se permiten caracteres de fin de línea, que son ignorados
  cuando se encuentran en medio de la secuencia.
  La secuencia termina cuando se encuentra el comienzo de una línea de
  descripción o de estructura secundaria.
\item
  Las líneas de estructura secundaria vienen dadas en notación
  ``punto-paréntesis'' compuesta por caracteres \mono{.}, \mono{(}, y
  \mono{)}, indicando que la base correspondiente está suelta
  (no acoplada), o es primera o segunda dentro de un par de bases,
  respectivamente.
  Opcionalmente, la línea finaliza por un número entre paréntesis que
  indica la denominada ``mínima energía libre'' resultante del cálculo
  de la estructura secundaria.
  La estructura secundaria termina cuando se encuentra el comienzo de
  una línea de descripción, que indica el comienzo de un nuevo
  ejemplo.
\end{itemize}
%
En la \iflatexml{}Figura~\ref{fastafmt}\else\autoref{fastafmt}\fi{} se
muestra un fragmento de archivo FASTA con 3 ejemplos de pre-miRNAs de
la especie \e{Caenorhabditis elegans}, y en la
\iflatexml{}Figura~\ref{rnafoldfmt}\else\autoref{rnafoldfmt}\fi{} se
muestran los mismos ejemplos con formato RNAFold, incluyendo la
estructura secundaria.
