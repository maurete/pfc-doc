%
%
\subsection{Extracción de \caract{s}}
%
La extracción de \caract{s} es el proceso de construir un vector
numérico que representa cada ejemplo.
Las componentes de este vector, llamadas \e{\caract{s}}, son
medidas calculadas a partir del dato original.
El objetivo de la extracción de \caract{s} es alcanzar una mejor
representación de los datos desde el punto de vista de la máquina de
aprendizaje, que sea al mismo tiempo más informativa (menos
redundante) y numéricamente tratable.

La construcción del vector de características conforma toda una
disciplina en sí misma, y su estudio excede el alcance del presente
trabajo.
En cambio, se adoptó un enfoque simple que consiste en construir un
vector de $66$ \caract{s} basado en aquellas de los trabajos
\cite{xue,ng,batuwita}.
Las características calculadas fueron seleccionadas en base a los
siguientes criterios:
%
\begin{enumerate}
\item
  La forma de cálculo está bien explicada en la publicación de
  referencia o en sus documentos suplementarios.
\item
  El cálculo de la \caract{} no requiere de herramientas de software
  adicionales.
\item
  Los autores publicaron bases de datos con la representación en forma
  textual de los ejemplos y las \caract{s} calculadas, que permiten
  validar la implementación propia.
\end{enumerate}
%
Según el tipo de dato que representan, las $66$ \caract{s} calculadas
se agrupan en: $36$ \caract{s} de \e{tripletes}, $23$ \caract{s} de la
\e{secuencia}, y $7$ de la \e{estructura secundaria}.
En adelante, se describen estas \caract{s} en detalle siguiendo este
orden.
Para una mejor comprensión de la terminología, en la
\iflatexml{}Figura~\ref{fig:hairpin-parts}\else\autoref{fig:hairpin-parts}\fi{}
se ilustran las diferentes partes que componen una estructura
secundaria típica en forma de horquilla.
