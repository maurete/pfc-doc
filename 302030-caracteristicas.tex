%
%
\subsection{Extracción de \caract{s}}
%
La extracción de \caract{s} es la tarea que construye para cada
ejemplo un vector numérico que lo representa.
Cada elemento del vector es una \e{\caract{}}, un atributo o medida
que se toma sobre el dato original y se trata como variable aleatoria.
El objetivo de la extracción de \caract{s} es alcanzar una mejor
representación de los datos desde el punto de vista de la máquina de
aprendizaje, que sea al mismo tiempo más informativa (menos
redundante) y numéricamente tratable.

La construcción del conjunto de características conforma toda una
disciplina en sí misma, que excede el alcance del presente trabajo.
En cambio se adopta un enfoque simple, que consiste en construir un
vector de 66 \caract{s} tomando como referencia los trabajos
\cite{xue,ng,batuwita}.
La elección de las 66 \caract{s} se justifica en que cada una de ellas
cumple con los siguientes criterios:
%
\begin{enumerate}
\item
  La forma de cálculo está bien explicada en la publicación de
  referencia o en sus documentos suplementarios.
\item
  El cálculo de la \caract{} no requiere de herramientas de software
  adicionales.
\item
  Los autores publicaron bases de datos con la representación en forma
  textual de los ejemplos y las \caract{s} calculadas, que permiten
  validar la implementación propia.
\end{enumerate}
%
Las 66 medidas que componen el vector de \caract{s} se distinguen,
según el tipo de dato que representan, en 36 \caract{s} de tripletes,
23 \caract{s} de la secuencia, y 7 de la estructura secundaria.
A continuación, se describe la composición del vector de \caract{s}
sguiendo este orden.
