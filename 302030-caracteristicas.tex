%
%
\subsection{Extracción de \caract{s}}
%
La extracción de \caract{s} es el proceso de construir un vector
numérico que representa cada ejemplo.
Las componentes de este vector, llamadas \e{\caract{s}}, son
medidas calculadas a partir del dato original.
El objetivo de la extracción de \caract{s} es alcanzar una mejor
representación de los datos desde el punto de vista de la máquina de
aprendizaje, que sea al mismo tiempo más informativa (menos
redundante) y numéricamente tratable.

La construcción del vector de características conforma toda una
disciplina en sí misma, y su estudio excede el alcance del presente
trabajo.
En cambio, se adoptó un enfoque simple que consiste en construir un
vector de $66$ \caract{s} utilizadas en el estado del arte.
Las características calculadas fueron seleccionadas en base a los
siguientes criterios:
%
\begin{enumerate}
\item
  La forma de cálculo está bien explicada en la publicación de
  referencia o en sus documentos suplementarios.
\item
  El cálculo de la \caract{} no requiere de herramientas de software
  adicionales.
\item
  Los autores publicaron bases de datos con la representación en forma
  textual de los ejemplos y las \caract{s} calculadas, que permiten
  validar la implementación propia.
\end{enumerate}
%
Según el tipo de dato que representan, las $66$ \caract{s} calculadas
son $36$ \caract{s} de \e{tripletes}, $23$ \caract{s} de la
\e{secuencia}, y $7$ de la \e{estructura secundaria}.
Siguiendo esta clasificación, dentro del vector de \caract{s} se
diferencian cuatro ``grupos'':
%
\pdfmargincomment[icon=Note,color=Yellow]{Agrego esta descripción de
  los grupos ya que en las pruebas se hace referencia, a ver si queda
  más claro?}
%
\begin{itemize}
\item \dset{T}: \caract{s} de tripletes ``principales'', ubicadas
  en las posiciones $1$ a $32$,
\item \dset{X}: \caract{s} de tripletes ``adicionales'' en las
  posiciones $33$--$36$,
\item \dset{S}: \caract{s} de la secuencia, con índices $37$--$59$, y
\item \dset{E}: \caract{s} de la estructura secundaria, con
  índices $60$ a $66$.
\end{itemize}
%
El método permite trabajar con cualquier combinación de estos cuatro
grupos, ya sea para efectuar comparaciones entre distintas \caract{s}
o bien porque algunas de ellas no pueden ser calculadas.
El modelo generado en este caso contendrá los índices necesarios
indicando cuáles son las \caract{s} a calcular.

En adelante se describe el vector de \caract{s} en detalle.
Como ayuda para la comprensión de algunos términos utilizados en la
descripción, se muestra en la
\iflatexml{}Figura~\ref{fig:hairpin-parts}\else\autoref{fig:hairpin-parts}\fi{}
una estructura secundaria típica con forma de horquilla, identificando
las diferentes partes que la componen.
