%
%
\subsection{Arquitectura de una red neuronal}
%
Una red neuronal es
una arquitectura paralela en la que la unidad de cómputo
básica es la neurona artifical.
Desde un punto de vista matemático,
la red se representa como un grafo orientado, donde los nodos son
neuronas y las aristas, que tienen un sentido y una ponderación (peso)
asociados, conectan las salidas de una neurona con la entrada de otra.

Según la topología del grafo, se dice que una red neuronal es
\e{acíclica} o \e{con propagación hacia adelante} cuando no existen
bucles dentro de la red. Similarmente, se dice que una red es
\e{recurrente} cuando contiene bucles de {retroalimentación} en su
topología.
En una red recurrente, resulta posible encontrar al menos
un camino que conduce al mismo punto de partida.
%
%
\subsection{El perceptrón multicapa}
%
El perceptrón multicapa tiene una arquitectura acíclica organizada en
\e{capas}. La primer capa se denomina \e{capa de entrada} y contiene,
en lugar de neuronas, nodos sensores que ``leen'' el vector presentado
como entrada a la red. Los valores de las salidas de esta capa equivalen
a los componentes del vector. Las capas subsiguientes contienen
neuronas, cada una de las cuales recibe como entrada \e{todas} las
salidas de la capa anterior.  Para el cálculo de las salidas de cada
capa se requiere conocer los valores de las salidas de la capa anterior,
por lo que se dice que el vector de entrada se \e{propaga hacia
  adelante} a través de la red.

%% Dentro de una misma capa, las neuronas no se conectan entre sí.

La salida global de la red se compone por las salidas de las neuronas en la
última capa, denominada simplemente \e{capa de salida}. Las capas que no son ni de
entrada ni de salida se denominan \e{capas ocultas}.

En la \iflatexml{}Figura~\ref{fig:mlp}\else\autoref{fig:mlp}\fi{} se
representa un perceptrón multicapa de 2 capas, que lee a su entrada un vector de 3
elementos. Este vector es propagado a través de una capa oculta de 4
neuronas hasta alcanzar las 2 neuronas en la capa de
salida, que determinan la salida de la red.
