%
\subsubsection{Error de prueba}
%
Una situación óptima se da cuando se dispone de una muestra $T$,
independiente e idénticamente distribuida a $\nu$, denominada
\e{conjunto de prueba}:
%
\begin{align}
  T=((\tilde{x}_1,\tilde{y}_1),\ldots,(\tilde{x}_N,\tilde{y}_N))\sim\nu^N.
  \label{eq:conj-prueba}
\end{align}
%
A partir del conjunto $T$ se define el \e{error de prueba}
(o \e{de test}) como la pérdida media
%
\begin{align}
  E_{test}=\frac{1}{N}\sum_{n=1}^N
  L(\tilde{x}_N,\tilde{y}_N,h(\tilde{x}_N)).
  \label{eq:error-prueba}
\end{align}
%
Este error de prueba es una estimación no sesgada del error de
generalización verdadero $\C{R}_\nu(h)$, y puede ser utilzado para
comparar modelos.
