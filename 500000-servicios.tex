\chapter{5 DevOps en los servicios}

Ninguna adopción de DevOps puede llamarse como tal si no pretende
generar una cultura en común entre los equipos de desarrollo y
operaciones. En el presente Capítulo se presenta una revisión del
trabajo realizado en este sentido.

Como primer paso se configuraron las herramientas necesarias para la
implementación de CI/CD en los servicios. Luego, en una etapa
exploratoria, se trabajó junto a los desarrolladores sobre dos
servicios de baja complejidad. El objetivo de este trabajo inicial fue
generar una comprensión de las implicancias de una implementación de
CI/CD sobre la forma de trabajo con el código, así como entender el
funcionamiento de las herramientas utilizadas.

Finalmente, se trabajó con tres servicios más complejos en una
implementación más formal, proponiendo buenas prácticas que acercaran
el trabajo de los desarrolladores y operadores hacia una filosofía
DevOps. Esta implementación permitió integrar el código fuente y los
flujos de trabajo de ambos equipos e incorporó automatización de las
tareas posteriores al desarrollo utilizando una tubería de integración
y entrega continuas.

\section{Herramientas CI/CD}

Las herramientas de integración y entrega continuas (CI/CD) permiten
automatizar prácticamente todas las tareas posteriores a la
publicación del código, tales como la compilación, el testing, el
empaquetado y la publicación de artefactos.

Se configuraron los servicios Jenkins y el módulo de CI/CD de GitLab,
para comparar su funcionamiento y determinar la conveniencia de
utilización de cada uno. La elección se justificó en el hecho de que
Jenkins es el servicio más popular
\href{https://www.zotero.org/google-docs/?VQPh0a}{[19]}, mientras que
GitLab es la herramienta adoptada por la Dirección para la gestión del
código fuente, lo que hace de la configuración de su servicio de CI/CD
una tarea relativamente sencilla.

\subsection{Jenkins}

Jenkins ofrece un ecosistema muy completo que ofrece una gran cantidad
de \e{plugins}, lo cual lo convierte en una herramienta muy
adaptable y flexible, capaz de integrar el trabajo con las
herramientas utilizadas por los desarrolladores (Gradle, Nexus) para
cada tipo de proyecto. Soporta múltiples tipos de agentes para la
ejecución de los trabajos (entornos Linux, Docker, Windows, etc.) y
ofrece una herramienta gráfica para crear y editar la tubería de
integración y entrega continuas.

El mayor impedimento encontrado con Jenkins fue la dificultad de
configurarlo con una herramienta de configuración como código tal como
Ansible. En muchos casos fue necesario escribir código Groovy para
lograr la configuración requerida, tomando como referencia
directamente el código fuente de Jenkins y/o de los plugins, ya que la
documentación en general sólo explica cómo efectuar la configuración
mediante la interfaz gráfica. Incluso, resultó imposible de configurar
completamente mediante Ansible la funcionalidad de edición gráfica de
la pipeline.

\subsection{GitLab CI}

El software GitLab es utilizado en la Dirección para gestionar el
código fuente. El mismo incorpora soporte para la ejecución de tareas
de integración y entrega continua. Para habilitar esta funcionalidad,
fue necesario generar una instancia de prueba, sobre la que se instaló
el software denominado \e{gitlab-runner}. Este software se
encarga de ejecutar los trabajos de las tuberías de CI/CD, los cuales
son gestionados desde la interfaz principal de GitLab. La tarea de
implementación definitiva del servicio \e{gitlab-runner} se
realizó en una etapa posterior y formó parte de las tareas de
actualización de GitLab.

\section{Pruebas iniciales de CI/CD}

Se seleccionaron dos servicios de baja complejidad y se trabajó en
conjunto con sus desarrolladores en el modelado de tuberías de CI/CD
para automatizar las tareas que se realizan una vez concluida la
escritura del código. Los servicios seleccionados fueron ``Wordpress
externos'' y ``Portal de firma digital''.

\subsection{Servicio ``Wordpress externos''}

Este servicio consiste en una instalación básica de
WordPress\footnote{ https://wordpress.org/} con algunas
personalizaciones sobre los plugins y temas. El servicio no cuenta con
tests unitarios. Una vez registrados los cambios en el código, las
únicas acciones realizadas son referidas al despliegue, primero en un
entorno de \e{test} donde se validan los requerimientos, para
luego pasar al entorno de producción.

Para automatizar las tareas de despliegue, se agregaron al repositorio
los archivos necesarios que especifican la tubería de CI/CD a
ejecutar. El archivo correspondiente a Jenkins se denomina
\e{Jenkinsfile} y el que modela la tubería de GitLab se denomina
\e{.gitlab-ci.yml}. Ambas tuberías efectúan la siguiente
secuencia de tareas\footnote{ En el Informe de Avance 1 se explicó que
  las tuberías utilizan la herramienta Ansible para efectuar los
  despliegues. Este detalle se omitió a propósito en el presente
  Informe a fin de evitar confusiones: mientras que a lo largo del
  Informe se describe a Ansible como una herramienta de configuración
  como código, en este caso se le dio otro uso experimental para
  efectuar operaciones de despliegue.}:

\begin{enumerate}
\item Validación de la sintaxis: se verifica la sintaxis del código
  para asegurar su integridad.
\item Despliegue en el entorno de test: se especificó como una tarea
  manual para conservar el flujo de trabajo sobre los entornos.
\item Despliegue en el entorno de producción: esta es una tarea manual
  permitida sólo para la rama de desarrollo principal
  (\e{master}).
\end{enumerate}
Resulta oportuno aclarar que las tareas manuales de esta tubería
consisten en un simple click de un botón en las interfaces de
Jenkins/GitLab. Asimismo, cuando una tarea falla, no se permiten
ejecutar las acciones posteriores.

\subsection{Servicio ``Portal de firma digital''}

El servicio es una aplicación Java que permite efectuar y validar
firmas digitales de documentos PDF. Utiliza Gradle como herramienta de
\e{build}. Las tareas realizadas por el desarrollador una vez
concluida la codificación son la compilación y ejecución de tests
(\e{gradle build}), el empaquetado y publicación al repositorio
de artefactos (\e{gradle publish}) y el despliegue a los entornos
de \e{integración}, \e{test}, y producción.

Para las pruebas de CI/CD se modeló una tubería simplificada, que
omite la publicación al repositorio de artefactos. Se agregaron los
archivos \e{Jenkinsfile} y \e{.gitlab-ci.yml} al repositorio
de código. La tubería efectúa las siguientes tareas:

\begin{enumerate}
\item Compilación y ejecución de tests (\e{gradle build}).
\item Generación del artefacto para el despliegue (\e{gradle
  assemble}).
\item Despliegue (automático) en el entorno de \e{integración}.
\item Despliegue en el entorno de test (tarea manual).
\item Despliegue en el entorno de producción: tarea manual habilitada
  sólo para la rama de desarrollo principal (\e{maste}r).
\end{enumerate}
\subsection{Resultados de las pruebas}

La conclusión principal obtenida de estas pruebas es que,
independientemente de las herramientas utilizadas, la implementación
de un esquema de CI/CD en los proyectos viene atada a la forma de
gestionar el código en Git. Esto no es precisamente una novedad, pero
puso en evidencia la falta de una cultura en común en cuanto a las
buenas prácticas de la gestión del código. Se observaron incluso
falencias en el conocimiento de algunos aspectos de Git.

En cuanto a la comparación de las herramientas de CI/CD, se realizaron
las siguientes observaciones:

\begin{itemize}
\item \e{Sintaxis}: el archivo \e{.gitlab-ci.yml} de GitLab
  posee una sintaxis mucho más clara que la del archivo
  \e{Jenkinsfile} utilizado por Jenkins.
\item \e{Simplicidad}: la configuración de CI/CD en GitLab es
  automática, sólo se requiere registrar un archivo
  \e{.gitlab-ci.yml} válido. En el caso de Jenkins, la
  configuración debe efectuarse de forma explícita, aunque podría
  automatizarse con una implementación adecuada.
\item \e{Integración}: La herramienta CI/CD de GitLab se integra
  completamente en flujos de trabajo que utilicen \e{merge
    requests}. La implementación de una funcionalidad similar con
  Jenkins requiere adquirir una versión comercial (privativa) del
  software GitLab.
\item \e{Flexibilidad}: Jenkins ofrece un modelo de pipelines muy
  flexible y altamente configurable, mientras que GitLab impone una
  estructura rígida separada en etapas bien delimitadas.
\item \e{Compatibilidad}: Jenkins permite ejecutar pipelines en
  múltiples entornos tales como Linux, Windows y Docker. En cambio,
  las pipelines de GitLab deben ejecutarse dentro de contenedores
  Docker.
\item \e{Interfaz gráfica}: Jenkins ofrece una interfaz muy
  potente para la generación de pipelines en modo gráfico. GitLab, en
  cambio, requiere escribir la tubería como código.
\item \e{Configuración}: La administración de Jenkins resulta una
  tarea compleja y difícil de automatizar, mientras que GitLab no
  requiere efectuar configuración alguna.
\item \e{Accesibilidad}: En GitLab las operaciones manuales de la
  tubería pueden lanzarse con un solo click desde los \e{merge
    requests}. En el caso de Jenkins resulta necesario cambiar de
  contexto para verificar el estado de CI/CD.
\end{itemize}
Teniendo en cuenta todas estas consideraciones, la herramienta elegida
para las implementaciones de CI/CD fue GitLab.

\section{Análisis de los servicios}

Una vez concluidas las pruebas iniciales de CI/CD se trabajó en la
implementación de prácticas DevOps sobre tres de los servicios
principales ofrecidos por la Dirección. A partir de las necesidades
institucionales, los servicios seleccionados fueron los siguientes:

\begin{itemize}
\item Ilitía: sistema de gestión de servicios a terceros.
\item Mercurio: sistema de cobranzas electrónicas.
\item Hera: sistema de gestión de trámites digitales.
\end{itemize}
Como primer paso, se realizaron reuniones con los equipos de
desarrollo para comprender cómo organizan su trabajo en torno a estos
servicios. Luego se analizaron el sistema de gestión de proyectos, la
documentación interna, el repositorio de código fuente y la
configuración actual de los servidores. Los resultados de este
análisis se presentan a continuación.

\subsection{Gestión de los servicios}

En términos generales, se encontró que la organización del trabajo
sobre los servicios estaba en línea con el proceso descrito
previamente en el análisis del flujo de valor para la entrega de un
nuevo requerimiento (Tabla 1.2). Con algunas diferencias tales como la
frecuencia de los sprints, la gestión de las ramas en el código fuente
o la publicación de artefactos, la metodología de trabajo de los
equipos de desarrollo resultó similar en los tres casos.

\subsection{Gestión del código fuente}

La gestión del código fuente se realizaba utilizando la herramienta
SVN, siguiendo una la estructura de directorios
\e{trunk/tags/branches} estándar para este tipo de repositorios.

En el servicio Mercurio, los tags de SVN se utilizaban para guardar
artefactos compilados de la aplicación, en lugar del código
fuente. Esta práctica había sido implementada en su momento como una
solución temporal hasta tanto el equipo de desarrollo contara con
recursos para migrar al repositorio de artefactos Nexus, tarea que
nunca se concretó. Esta particularidad debe ser considerada al momento
de efectuar la migración a Git.

\subsection{Documentación}

Al revisar la documentación de los servicios se encontraron manuales
de usuario, documentos de requerimientos y descripciones de alto nivel
sobre las integraciones con sistemas externos. Sin embargo, la
documentación técnica (arquitectura, soporte tecnológico requerido,
configuración, compilación, testing, etc.) resultó prácticamente
inexistente. En el sistema de gestión de proyectos se encontró un
proyecto denominado ``Desarrollo de software UNL'', con documentación
genérica para todos los servicios.

\subsection{Gestión de la configuración}

Los tres servicios contaban con configuración escrita como código
Ansible en el repositorio Git propio del equipo de
infraestructura. Este aspecto simplificó en cierto modo la
incorporación de la configuración como código en los nuevos
repositorios Git para los servicios.

\subsection{Soporte tecnológico}

Dentro de la Dirección, el equipo de infraestructura es el encargado
de proponer los entornos de software sobre el que funcionan las
aplicaciones. Por ello, el soporte tecnológico de los tres servicios
resultó similar:

\begin{itemize}
\item Sistema operativo Debian GNU/Linux, versión 9
\item Entorno Java OpenJDK, versión 8
\item Servidor de base de datos PostgreSQL versión 9.6
\item Servidor de aplicación WildFly versiones 12 (Ilitía) y 15
  (Mercurio, Hera)
\item Herramientas de build Gradle (Mercurio, Hera) y Maven (Ilitía)
\item Entornos de integración, test, y producción
\item Esquemas de monitoreo, backup y servicio de dumps estándar de la
  DIPT
\end{itemize}
\section{Integración del código fuente de los equipos de desarrollo y operaciones}

Ya se ha expresado que ninguna transformación DevOps puede llamarse
como tal si no se propone generar una cultura en común entre los
equipos de desarrollo y operaciones. En este sentido es que se decidió
integrar, para cada servicio, el código fuente del equipo de
desarrollo (código de la aplicación) y del equipo de
operaciones/infraestructura (configuración como código).

La unificación del código y la configuración de los servicios es el
primer paso para generar una cultura DevOps en común. Por un lado,
obliga a los equipos a coordinar sus flujos de trabajo, en especial
sobre los cambios que son aplicados en los entornos productivos. Por
otra parte, brinda transparencia y visibilidad del trabajo realizado
por ambos equipos. Finalmente, permite intercambiar roles a los
miembros de cada equipo permitiendo, por ejemplo, la configuración de
la infraestructura por parte de un desarrollador, y la proposición de
mejoras en el proceso de build por parte del operador. Todos estos
aspectos derivan en un incremento de la confianza mutua entre los
equipos y la generación de una cultura en común.

\subsection{Procedimiento}

Para cada servicio se realizó el procedimiento descrito a
continuación. En todos los casos, la migración se efectuó de manera
coordinada con los equipos de desarrollo para no impactar en la
productividad de los mismos.

En primer lugar, se creó en GitLab un nuevo repositorio Git para el
servicio. Se incorporó a este repositorio el código Ansible que
especifica la configuración, tal como estaba configurado en el
repositorio de infraestructura.

Fue necesario adaptar la configuración de Ansible para su
funcionamiento dentro de un repositorio independiente. Para ello, se
generaron nuevos repositorios específicos con roles reutilizables por
parte de todos los servicios. Se realizó además una segunda adaptación
al código Ansible con el propósito de hacerla compatible con el
servicio AWX.

Una vez consolidada y probada la configuración en el repositorio, se
procedió a importar el código fuente de la aplicación desde el
repositorio SVN, utilizando la herramienta Git-SVN\footnote{
  https://git-scm.com/docs/git-svn}. En el caso del servicio Mercurio,
fue necesario filtrar los artefactos binarios para evitar
incorporarlos al repositorio Git. Se trabajó asimismo en conjunto con
el equipo de desarrollo para integrar en su flujo de trabajo el
repositorio Nexus para la publicación de los artefactos.

\subsection{Documento de recomendaciones}

Se escribió un documento de recomendaciones que condensa las lecciones
aprendidas del trabajo realizado sobre los repositorios y las
inquietudes planteadas por los desarrolladores a la hora de integrar
los flujos de trabajo de los equipos. El objetivo del documento es
contribuir a la generación de una cultura en común. Este documento
incluye consideraciones relacionadas a las tuberías de integración y
entrega continuas, así como también una guía básica del uso del
sistema de versionado Git.

\section{Integración y entrega continuas (CI/CD)}

Se implementaron herramientas de integración y entrega continuas
(CI/CD) en los en los servicios con el propósito de automatizar las
tareas efectuadas por el desarrollador una vez finalizada la escritura
del código. Estas tareas se modelan en el servicio de CI/CD con una
\e{tubería} (o \e{pipeline}), consistente en un proceso
automático que se dispara con cada registro de cambios (\e{push})
al repositorio central de código fuente GitLab.

El diseño tomado como base para la implementación de las tuberías fue
aquel propuesto por Humble y Farley en
\href{https://www.zotero.org/google-docs/?9SSytI}{[1]}. Se adaptó el
diseño a las particularidades de la Organización, integrando el flujo
de trabajo de ramas de vida corta especificado en la propuesta de
mejora para el proceso de entrega de un nuevo requerimiento, tal como
se explica en el Capítulo 3. El resultado principal de la
implementación fueron los archivos de definición de la tubería para el
servicio CI/CD de GitLab. Estos archivos fueron incorporados al
repositorio de cada servicio en forma de un \e{merge request}.

La implementación de las tuberías resultó una tarea muy compleja desde
el punto de vista técnico. Fue necesaria la creación de entornos
específicos en forma de contenedores Docker para la ejecución de
tareas. Para lograr compilar y ejecutar los tests de las aplicaciones
dentro de la tubería fue necesario realizar ingeniería inversa y
depuración de los trabajos, en consulta permanente con los
desarrolladores de los servicios.

En adelante, se describe el proceso de implementación de las
tuberías. Se generaron dos diseños: el primero, acorde con la
bibliografía, funciona como una meta a implementar a mediano plazo, ya
que requiere la codificación de nuevos tipos de tests y la adopción
generalizada de la tubería en todos los servicios relacionados.

El segundo diseño es la implementación concreta efectuada, considerada
un compromiso entre las posibilidades actuales de la Organización y la
estructura básica requerida para generar una cultura DevOps adecuada.

\subsection{Tubería objetivo (teórica)}

Este diseño, muy similar a la propuesta presentada en
\href{https://www.zotero.org/google-docs/?lBK1S4}{[1]}, especifica una
tubería completa que ejecuta diferentes tipos de tests, incluyendo
pruebas de humo\footnote{ Una ``prueba de humo'' es un tipo de test que
  valida la funcionalidad principal del software. Es una prueba rápida
  que asegura la integridad a grandes rasgos.} (\e{smoke tests})
y pruebas de capacidad. Se considera una meta a alcanzar a futuro, ya
que requiere la codificación de implementación de nuevos tipos de
tests y la creación de un nuevo entorno para ejecutar las pruebas de
capacidad, tareas que conllevan un cierto grado de complejidad y se
consideran fuera de alcance del presente Proyecto. En la Figura 5.1 se
muestra el diseño de esta tubería.

\begin{tabular}{|l|}
\hline \includegraphics[width=6.34in,height=5.23in]{img_7.png}


\e{Figura 5.1. Tubería teórica} \\ \hline
\end{tabular}
El disparador de la ejecución de la tubería es un push del código al
servidor GitLab. Las tareas se ejecutan en forma secuencial. Una falla
en una tarea aborta la ejecución de la tubería completa, la cual pasa
a un estado fallido.

\subsubsection{1 Etapa de construcción del código (\e{build})}

En esta primera etapa se llevan a cabo las tareas que permiten la
generación de los archivos compilados y los artefactos de la
aplicación. Las tareas se ejecutan en forma secuencial dentro de un
contenedor Docker del servicio de CI/CD.

\begin{enumerate}
\item \textbf{Verificar que la versión no ha sido publicada}. Una
  falla en este paso indica al desarrollador que debe incrementar el
  número de versión para poder ejecutar la tubería.
\item \textbf{Compilar la aplicación}. Se ejecuta la herramienta
  encargada de la compilación tal como Maven o Gradle.
\item \textbf{Ejecutar tests unitarios}. Se ejecuta la suite de tests
  unitarios utilizando la herramienta apropiada. Los tests unitarios
  se ejecutan primero ya que son los más rápidos.
\item \textbf{Ejecutar tests de integración}. Se ejecutan los tests
  que validan la funcionalidad integrada de diferentes componentes del
  software. No se debe confundir estos tests con los de integración de
  sistemas.
\item \textbf{Empaquetar la aplicación}. Se genera el artefacto que
  puede ser desplegado en el servidor de aplicación. Este artefacto se
  guarda en una ubicación temporal hasta el fin de la ejecución de la
  tubería.
\end{enumerate}
\subsubsection{2 Etapa de verificación}

En esta etapa se efectúan pruebas de validación de la aplicación
completa en el entorno controlado de contenedor Docker del servicio
CI/CD que incorpora un servidor de aplicación y brinda acceso a una
base de datos.

\begin{enumerate}
\item \textbf{Configurar el entorno de testing local}. Se configuran
  los archivos y recursos necesarios en el servidor de aplicación para
  poder efectuar el despliegue de la aplicación.
\item \textbf{Desplegar la aplicación}. Se despliega el paquete de la
  aplicación generado en la etapa anterior.
\item \textbf{Pruebas de humo}. Se ejecutan pruebas de humo para
  verificar la integridad global de la aplicación.
\item \textbf{Tests de aceptación automáticos}. Este tipo de tests
  verifican la integridad de los distintos componentes de la
  aplicación durante la ejecución, incluyendo el acceso a todos los
  recursos locales.
\item \textbf{Tests de integración de sistemas}. Estos tests acceden a
  recursos externos, en general servicios web de otros sistemas, y
  verifican la interacción con los mismos.
\end{enumerate}
\subsubsection{3 Publicación}

En el caso que el disparador de la tubería haya sido un cambio sobre
la rama principal de desarrollo (\e{master}), se publica el
paquete de la aplicación al repositorio de artefactos y se genera la
etiqueta de versión correspondiente en el repositorio de código.

\subsubsection{4 Tests de capacidad}

Se deja como tarea a futuro la especificación y diseño de los tests de
capacidad/estrés y el entorno de ejecución para los mismos. La
secuencia de tareas a ejecutar en este entorno es similar a las demás:

\begin{enumerate}
\item \textbf{Configurar entorno}.
\item \textbf{Desplegar la aplicación}.
\item \textbf{Ejecutar pruebas de humo}.
\item \textbf{Ejecutar tests de capacidad}.
\end{enumerate}
\subsubsection{5 Despliegue en el entorno de aceptación de usuario}

Ésta es una tarea que requiere aprobación manual, típicamente de algún
miembro de los equipos de desarrollo o testing. El objetivo es poder
probar manualmente la aplicación en el entorno antes mencionado.

\begin{enumerate}
\item \textbf{Configurar entorno}.
\item \textbf{Despliegue de aplicación}.
\item \textbf{Pruebas de humo}.
\end{enumerate}
\subsubsection{6 Despliegue en el entorno de producción}

Una vez que el software ha pasado por todas las etapas anteriores, se
considera que está listo para su publicación en el entorno productivo,
alcanzando al usuario final. Esta tarea requiere aprobación manual,
típicamente del responsable del proyecto, y sólo está habilitada para
los cambios efectuados sobre la rama principal de desarrollo
(\e{master}).

\begin{enumerate}
\item \textbf{Configurar entorno}.
\item \textbf{Despliegue de aplicación}.
\item \textbf{Pruebas de humo}.
\end{enumerate}
\subsection{Tubería implementada (real) }

La arquitectura implementada es un compromiso entre la tubería teórica
y las posibilidades actuales de la Dirección. Permite a los equipos
organizar su trabajo en torno a la tubería, lo cual permitirá
acostumbrarse a su uso y evolucionar en forma paulatina hacia la
tubería teórica. El diseño de la misma se muestra en la Figura 5.2.

\begin{tabular}{|l|}
\hline \includegraphics[width=6.34in,height=5.23in]{img_8.png}


\e{Figura 5.2. Tubería implementada} \\ \hline
\end{tabular}
La tubería implementada es similar a la tubería teórica, aunque con
menos tareas de testing. Al igual que aquella, un push del código al
servidor GitLab dispara la ejecución de la tubería. Una tarea fallida
aborta la ejecución de la tubería completa.

\subsubsection{1 Etapa de construcción del código}

Entorno de ejecución: contenedor Docker del servicio CI/CD.

\begin{enumerate}
\item \textbf{Verificar que la versión no ha sido publicada}.
\item \textbf{Compilar la aplicación}.
\item \textbf{Ejecutar tests unitarios}.
\item \textbf{Ejecutar tests de integración}.
\item \textbf{Empaquetar la aplicación}. Se genera el artefacto para
  el despliegue y se conserva el mismo en una ubicación temporal.
\end{enumerate}
\subsubsection{2 Etapa de verificación}

Entorno de ejecución: contenedor Docker del servicio CI/CD, con
servidor de aplicación integrado y acceso a la base de datos.

\begin{enumerate}
\item \textbf{Configurar el entorno de testing local}.
\item \textbf{Desplegar la aplicación}.
\item \textbf{Tests de integración de sistemas}.
\end{enumerate}
\subsubsection{3 Publicación}

En el caso que el disparador de la tubería haya sido un cambio sobre
la rama principal (\e{master}), se publica el paquete de la
aplicación al repositorio de artefactos y se genera la etiqueta de
versión correspondiente en el repositorio Git.

\subsubsection{4 Despliegue en el entorno para tests de integración de sistemas}

Esta acción requiere aprobación manual, típicamente de un
desarrollador. Permite actualizar la aplicación en el entorno de
\e{integración} para que otros sistemas puedan validar sus
integraciones.

\begin{enumerate}
\item \textbf{Configurar entorno}.
\item \textbf{Despliegue de aplicación}.
\end{enumerate}
\subsubsection{5 Despliegue en el entorno para tests de aceptación de usuario}

Esta acción requiere aprobación manual. Es idéntica a la de la tubería
teórica, pero excluye las pruebas de humo.

\begin{enumerate}
\item \textbf{Configurar entorno}.
\item \textbf{Despliegue de aplicación}.
\end{enumerate}
\subsubsection{6 Despliegue en el entorno de producción}

También se trata de una acción de aprobación manual. Aplican las
mismas consideraciones que en la tubería teórica, siendo la única
diferencia que no se ejecutan pruebas de humo.

\begin{enumerate}
\item \textbf{Configurar entorno}.
\item \textbf{Despliegue de aplicación}.
\end{enumerate}
\section{Dificultades encontradas en la implementación}

La implementación de las tuberías en los servicios resultó una tarea
de una complejidad mucho mayor a la prevista en el plan de
trabajo. Estas dificultades impactaron retrasando el desarrollo del
Proyecto. A continuación se presenta una revisión de los problemas
encontrados a la hora de implementar tuberías de integración y entrega
continuas en los servicios.

\subsection{Generación de imágenes Docker}

Las imágenes de Docker son entornos de ejecución aislados y
reproducibles, generados a partir de una especificación de
configuración como código basada en un lenguaje de dominio
específico. Cuando una imagen de Docker es instanciada para su
ejecución, se crea un contenedor que incluye el contenido de la imagen
asociado a recursos específicos tales como la memoria asignada y los
recursos de red. Las imágenes de Docker pueden construirse en base a
otras imágenes ya existentes, lo cual ofrece una gran flexibilidad de
adaptación a los requerimientos de cada aplicación.

Dicho esto, también resulta cierto que los entornos Docker son
limitados por diseño. Por ejemplo, dentro de un contenedor se permite
la ejecución de un único proceso, a contramano de lo que sucede en las
instancias virtualizadas, que ejecutan un sistema operativo completo.

A la hora de configurar los entornos de Docker para la ejecución de
las tareas de CI/CD, se optó por generar entornos que se adaptaran a
los requerimientos del proceso de construcción del código y las
pruebas, en lugar de adaptar los procesos a un entorno limitado como
Docker. Esta decisión significó realizar un gran esfuerzo en la
generación de imágenes de Docker personalizadas.

La creación de cada imagen de Docker implicó la creación de un
repositorio específico para el código de la misma. Estos repositorios
incorporan a su vez sus propias pipelines de integración y entrega
continuas, encargadas de generar las imágenes y publicarlas en el
repositorio público Docker Hub. Las imágenes de Docker creadas fueron
las siguientes:

\begin{itemize}
\item diptunl/debian: sistema operativo Debian de base con
  herramientas y configuraciones estándares requeridas por los
  servicios de la DIPT. Incluye configuración de locales, hora y
  fecha, y certificados SSL.
\item diptunl/java: basada en diptunl/debian, incorpora la máquina
  virtual OpenJDK configurada con opciones de ejecución adaptadas a
  los servicios de la DIPT e incorpora los certificados SSL al
  depósito de claves de Java.
\item diptunl/gradle: se basa en diptunl/java y contiene la
  herramienta Gradle.
\item diptunl/maven: se basa en diptunl/java y contiene la herramienta
  Maven.
\item diptunl/wildfly: basada en diptunl/java, diptunl/gradle y
  diptunl/maven, posee un servidor Wildfly que incluye un driver para
  conectarse a PostgreSQL y está parametrizado para los servicios Java
  de la DIPT.
\item diptunl/postgres: basada en diptunl/debian, posee un servidor
  PostgreSQL que soporta el esquema de nombres y permisos utilizados
  en la DIPT.
\end{itemize}
Esta arquitectura de imágenes de Docker fue evolucionando conforme se
avanzó con la implementación de las tuberías de cada servicio.

\subsection{Falta de documentación}

Otro de los grandes problemas encontrados a la hora de la
implementación de las tuberías fue la falta de documentación de los
procesos, configuraciones y comandos necesarios para poder compilar
las aplicaciones y ejecutar los tests. En particular los tests de
integración de sistemas, los cuales fueron diseñados únicamente para
soportar el caso de uso de la ejecución local en la PC del
desarrollador. Estos inconvenientes impactaron en la implementación de
las tuberías generando retrasos debidos a las situaciones de prueba y
error, y de ingeniería inversa a partir de la inspección del código
fuente.

Como un aspecto positivo, debido a estos problemas se incorporaron
modificaciones al código fuente de la aplicación, incluyendo
documentación e incorporando soporte de parametrizaciones para que los
procesos de testing fueran más flexibles.
