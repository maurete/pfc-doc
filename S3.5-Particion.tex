%
%
%
\subsection{Partición de los datos}
%
El método requiere la especificación de un conjunto de datos para
entrenamiento toda vez que se pretenda obtener un modelo de
clasificador. Asimismo, aquellos ejemplos para los que se desea
obtener una predicción (dado un modelo entrenado) componen el conjunto
de prueba.

Una vez efectuadas la extracción de características y la
normalización, los archivos de entrada originales se han convertido en
conjuntos $D_1,\ldots,D_N$ que contienen los vectores de
características normalizados y la información de clase para cada
ejemplo. La partición de los datos refiere al armado de los conjuntos
de entrenamiento $D^E$ y prueba $D^P$ con los elementos de estos
conjuntos a partir de la especificación del usuaio.

El conjunto $D_n=((\xx_i,y_i)),\,i=1,\ldots,\ell_n$ se corresponde con
el $n$-ésimo archivo de entrada. El objetivo es armar los conjuntos de
entrenamiento $D^E$ y de prueba $D^P$ con elementos de los
$D_n,\,n=1,\ldots,N$.  Para cada conjunto $D_n$, el usuario especifica
una proporción $p_n^E$ de elementos del conjunto $D_n$ a utilizar como
entrenamiento, y otra proporción $p_n^P$ de elementos de $D_n$ a
utilizar para prueba. Los valores de $p_n^E,p_n^P$ son tales que
%
\begin{align}
  0\ \leq\ p_n^E, p_n^P\ \leq\  p_n^E + p_n^P\ \leq\ 1,
\end{align}
%
esto es, $p_n^E$ y $p_n^P$ son números enre 0 y 1 cuya suma es menor o
igual a 1.

El primer paso consiste en permutar aleatoriamente el orden de los
elementos de los conjuntos de entrada $D_n$
%
\begin{align}
  D^*_n = ((\xx_j,y_j)), \quad j = \sigma(i,\ell_n,s), \quad i=1,\ldots,\ell_n,
\end{align}
%
donde $\sigma$ es una función de permutación pseudoaleatoria que
requiere la especificación de una \e{semilla} $s$. Seguidamente se
particiona cada conjunto $D_n$ en sub-conjuntos de entrenamiento
y prueba
%
\begin{align}
  D^E_n &= ((\xx_u,y_u)\in D^*_n),& u&=1,\ldots,\ell^E_n, \\
  D^P_n &= ((\xx_v,y_v)\in D^*_n),& v&=1,\ldots,\ell^P_n,
\end{align}
%
donde
%
\begin{align}
  \begin{split}
    \ell^E_n &= \left\lfloor\ell_n\cdot p_n^E \right\rceil, \\
    \ell^P_n &= \lfloor\ell_n\cdot p_n^P\rceil.
  \end{split}
\end{align}
%
El operador $\lfloor\cdot\rceil$ es el operador de redondeo.
Finalmente, se ensamblan los conjuntos de entrenamiento y prueba según
%
\begin{align}
  D^E &= \left( (\xx_u,y_u) \in \bigcup_n {D}^E_n \right), &
  u &= \sigma(i,\ell^E,s), &
  i &= 1,\ldots,\ell^E, &
  \ell^E &= \left| D^E \right| = \sum_n \ell^E_n ,\\
  D^P &= \left( (\xx_v,y_v) \in \bigcup_n {D}^P_n \right), &
  v &= \sigma(j,\ell^P,s), &
  j &= 1,\ldots,\ell^P, &
  \ell^P &= \left| D^P \right| = \sum_n \ell^P_n .
\end{align}
%
Esto es, los conjuntos de entrenamiento $D^E$ y prueba $D^P$ se arman
concatenando los respectivos $D^E_n$ y $D^P_n$ y luego permutando
aleatoriamente el orden de los ejemplos en los conjuntos resultantes.
%
%
\subsection{Generación de particiones para validación cruzada}
%
El armado de las particiones de validación cruzada requiere la
especificación de dos parámetros: $k$, el número de particiones a
generar (iteraciones), y $p$, la proporción de elementos de
entrenamiento a utilizar en cada partición de validación. Por defecto,
se utilizan los valores $k=10$ y $p=1/k$.

La $j$-ésima partición de validación ($j=1,\ldots,k$) se genera con
$p\cdot\ell^E$ ejemplos del conjunto de entrenamiento $D^E$, en
secuencia a partir del elemento $(j-1)\frac{\ell^E}{k}+1$:
%
\begin{align}
  D^V_j \tab= \left((\xx_u,y_u)\right), \tabs
  \begin{split}
    u &= \left\lfloor\frac{\ell^E}{k}(j-1)+1\right\rceil^{(*)},\ldots,
    \left\lfloor\frac{\ell^E}{k}(j-1)+p\ell^E\right\rceil^{(*)}.
  \end{split}
\end{align}
%
El operador $(*)$ indica que el índice $u$ es un índice ``circular'':
si su valor excede el tamaño del conjunto $\ell^E$, se resta $\ell^E$, de
modo que el valor de $u$ se mantiene siempre entre $1$ y $\ell^E$.

Los conjuntos de estimación correspondientes se componen con los
ejemplos de entrenamiento no utilizados para validación:
%
\begin{align}
  D^T_k &= D^E \setminus D^V_j, \quad j=1,\ldots,k.
\end{align}
%
