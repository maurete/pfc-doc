\section{Partición de datos}
Una vez aplicada la extracción de características y normalización, los
datos de entrada antes expresados como archivos de texto han sido
convertidos en conjuntos de vectores de características
$D_1,D_2,\ldots,D_n$.  La partición de los datos se trata de la
implementación del \hyperref[retencion]{método de retención}, y
genera, a partir de la especificación del usuario, el conjunto de
datos de entrenamiento $D^E$ y el de prueba $D^P$ a ser utilizados por
la máquina de aprendizaje para generar el modelo (entrenamiento) o
para clasificar nuevos datos (prueba).

El primer paso a efectuar es permutar aleatoriamente el orden de los
elementos de los conjuntos de entrada $D_k,\,k=1,\ldots,n$:

\begin{align*}
  D^*_k = ((\xx_j,y_j)), \quad j = \sigma(i,\ell_k,s), \quad i=1,\ldots,\ell_k,
\end{align*}
en donde $\sigma$ es una función de permutación pseudoaleatoria que
requiere la especificación de una \e{semilla} $s$. Seguidamente se
particiona cada conjunto $D_k$ en sub-conjuntos de entrenamiento
y prueba

\begin{align*}
  D^E_k &= ((\xx_u,y_u)\in D^*_k),& u&=1,\ldots,\ell^E_k, \\
  D^P_k &= ((\xx_v,y_v)\in D^*_k),& v&=1,\ldots,\ell^P_k,
\end{align*}
donde

\begin{align*}
  \begin{split}
    \ell^E_k &= \left\lfloor\ell_k\cdot p_k^E \right\rceil, \\
    \ell^P_k &= \lfloor\ell_k\cdot p_k^P\rceil,
  \end{split}
  &&
  0\ \leq\ p_k^E, p_k^P\ \leq\  p_k^E + p_k^P\ \leq\ 1.
\end{align*}
Aquí, los valores (especificados por el usuario) $p^E_k$ y $p^E_k$ son
la proporción de elementos del conjunto $D_k$ a ser utilizada para
entrenamiento y prueba, respectivamente.

Los conjuntos de entrenamiento y prueba se componen finalmente según

\begin{align*}
  D^E &= \left( (\xx_u,y_u) \in \bigcup_k {D}^E_k \right), &
  u &= \sigma(i,\ell^E,s), &
  i &= 1,\ldots,\ell^E, &
  \ell^E &= \left| D^E \right| = \sum_k \ell^E_k ,\\
  D^P &= \left( (\xx_v,y_v) \in \bigcup_k {D}^P_k \right), &
  v &= \sigma(j,\ell^P,s), &
  j &= 1,\ldots,\ell^P, &
  \ell^P &= \left| D^P \right| = \sum_k \ell^P_k .
\end{align*}
Nuevamente, aparece la función de permutación pseudoaleatoria $\sigma$
la cual recibe como parámetro una ``semilla'' $s$.
En términos coloquiales, los conjuntos de entrenamiento $D^E$ y prueba $D^P$
se arman concatenando los respectivos $D^E_k$ y $D^P_k$ y luego permutando
aleatoriamente el orden de los conjuntos resultantes.

\subsection{Generación de particiones para validación cruzada}
Dados $k$, el número de particiones a generar, y opcionalmente $p$, la
proporción de elementos a utilizar para validación, las particiones de
validación cruzada son generadas a partir del conjunto de
entrenamiento $D^E$ según

\begin{align*}
  D^V_j &= \left( (\xx_u,y_u) \in {D}^E \right), &
  u &= \left\lfloor\ell^E\frac{j-1}{k}+1\right\rceil,\ldots,
  \left\lfloor\ell^E\frac{j-1}{k}+\ell^E p\right\rceil\T{mod}{\ell^E}, &
  j &= 1,\ldots,k,
\end{align*}
en donde $T{mod}$ representa el operador módulo.
Con las particiones de validación $D^V_j$ así definidas, los respectivos
conjuntos de estimación a utilizar se determinan según

\begin{align*}
  D^T_k &= D^E \setminus D^V_j, \quad j=1,\ldots,k.  
\end{align*}