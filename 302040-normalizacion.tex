%
%
%
\subsection{Normalización de los vectores de características}
%
Los elementos que componen el vector de \caract{s} poseen rangos
numéricos diferentes, ya que representan cantidades de naturaleza
diversa.
Esto implica que algunas componentes del vector tendrán magnitudes
mayores que otras, generando una ponderación implícita de algunas
\caract{s} por sobre otras.
La normalización permite evitar este problema modificando el rango de
cada \caract{} a un intervalo preestablecido, típicamente $[0,1]$ o
$[-1,+1]$.
Esto deriva en problemas mejor condicionados desde el punto de vista
numérico e incrementa la velocidad de convergencia en el
entrenamiento \cite{nnfaq2}.

Dado un vector de características $\xx=(x_{1},x_{2},\ldots,x_{F})$, la
versión normalizada $\xx^*$ del mismo vector se calcula según
%
\begin{align}
  \label{e3:norm-op}
  x_j^{*} = ( x_j + d_j ) s_j, \quad j=1,\ldots,F,
\end{align}
%
donde $\B{s}=(s_{1},\ldots,s_{F})$ es un \e{vector de escala} y
$\B{d}=(d_{1},\ldots,d_{F})$ es un \e{vector de desplazamiento} que
permiten transformar las componentes $x_j$ al intervalo especificado.

Cuando el objetivo es la generación de un modelo de clasificador, los
vectores $\B{s}$ y $\B{d}$ se calculan con el primer archivo leído a
la entrada del método.
El primer paso es armar una matriz $M{}_{({N}\times{}F)}$, donde cada
fila $i=1,\ldots,N$ corresponde a un ejemplo y cada columna
$j=1,\ldots,F$ representa una variable (\caract{}).
Para cada columna $j$ de $M$, se calculan los valores $d_j$ y $s_j$
que transforman el rango de la variable $x_j$ al intervalo deseado.
Para llevar las variables al rango $[0,1]$, $d_j$ y $s_j$ se
calculan según
%
\begin{align}
  d_j &= - \min_i m_{ij}, &
  s_j &= \frac{1}{\max_i m_{ij} - \min_i m_{ij}}.
\end{align}
%
Similarmente, para llevar las variables al intervalo simétrico
$[-1,+1]$,
%
\begin{align}
  d_j &= -\frac{1}{2}\left(\max_i m_{ij} + \min_i m_{ij}\right), &
  s_j &= \frac{2}{\max_i m_{ij} - \min_i m_{ij}}.
\end{align}
%
Una vez calculados los vectores $\B{d}$ y $\B{s}$, se aplica la
normalización (\iflatexml{}Ecuación~\ref{e3:norm-op}\else\autoref{e3:norm-op}\fi)
sobre todos los ejemplos leídos en la entrada.

Para un modelo de clasificador entrenado con datos normalizados, es
importante aplicar la misma normalización sobre todos los ejemplos
futuros a clasificar.
Por ello, los vectores $\B{d}$ y $\B{s}$ se guardan como parte del
mismo modelo.
Cuando se proveen nuevos datos para probar el modelo, los ejemplos se
normalizan según la información contenida en el mismo.
