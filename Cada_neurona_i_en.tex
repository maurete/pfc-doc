Cada neurona $i$ en la red es una unidad de procesamiento simple que
calcula su salida $s_i$ aplicando una \e{función de activación} al
llamado \e{campo local inducido} $v_i$, que es la suma ponderada de
las entradas

\begin{align*}
  v_i \tab= \sum_{j\in\C{P}(i)} s_j w_{ij} - b_i .
\end{align*}
Aquí, $\C{P}(i)$ denota los índices de las unidades en la capa
anterior a la de la $i$-ésima unidad, $w_{ij}$ es el peso
(ponderación) correspondiente al enlace que conecta la unidad $j$ a la
unidad $i$, y $b_i$ es el \e{umbral}, que representa el ``grado de
inhibición'' de la neurona. El umbral se representa comúnmente
como una señal de entrada constante $s_0=1$, y el campo local inducido
se escribe

\begin{align*}
  v_i \tab= \sum_{j\in\{0,\C{P}(i)\}} s_j w_{ij}, \tabs b_i\tab=-w_{i0}.
\end{align*}
