%
%
\subsection{Problema miPred}
%
El problema \mipred{} se basa en los datos utilizados en \cite{ng}
para el entrenamiento y prueba del método \work\mipred.

Si bien en \cite{ng} se describen los pasos efectuados para el armado
de los conjuntos de entrenamiento y prueba, no se dispone de
información de los ejemplos que los componen.
Esto significa que si bien se mantiene el origen de los datos y el
número de ejemplos de clase positiva y negativa en cada conjunto,
resulta imposible replicarlos de manera exacta.

Los ejemplos de clase positiva se obtienen de la versión 8.2 (de julio
de 2006) de la base de datos \work{\mirbase}.
Se utilizan los 323 \premirna{s} de la especie humana disponibles en
esta versión, incluyendo aquellos con estructura secundaria tipo
horquilla, con bucle central único, y ramificada, con bucles
múltiples.
Los ejemplos de clase negativa se obtienen de la base de datos
\dset{coding}, al igual que en el problema \prob\tripletsvm.

Respetando el procedimiento descripto en \cite{ng}, para el armado del
conjunto de entrenamiento \dset{TR-H} se seleccionaron al azar 200
ejemplos de clase positiva y 400 de clase negativa.
El conjunto de prueba \dset{TE-H} se armó con los 123 ejemplos
restantes de clase positiva y 246 ejemplos de clase negativa
seleccionados al azar, excluyendo los ya utilizados para
entrenamiento.
La composición de los conjuntos de entrenamiento y prueba se detalla
en \iflatexml{}Tabla~\ref{tbl:pruebasng}\else\autoref{tbl:pruebasng}\fi.
