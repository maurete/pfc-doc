%
%
\subsection{Problema \mipred{}}
%
El problema \prob\mipred{} se basa en los datos utilizados en
\cite{ng} para el entrenamiento y prueba del método \work\mipred{}.
Si bien en el trabajo original se describe el armado de los conjuntos
de entrenamiento y prueba, no se publican los datos particionados.
Esto implica que no es posible replicar de manera exacta las
particiones generadas por los autores.
En cambio, se particionaron los datos en conjuntos de entrenamiento y
prueba mediante selección aleatoria, respetando el origen y el número
de ejemplos de clase positiva y negativa utilizados en cada caso.

Los ejemplos de clase positiva se obtuvieron de la versión $8$.$2$ (de
julio de $2006$) de la base de datos \dset{\mirbase}, mientras que los
ejemplos de clase negativa se obtuvieron de la base de datos
\dset{coding}, la misma que se utilizó en el problema
\prob\tripletsvm{}.
Respetando el procedimiento descripto en \cite{ng}, para el armado del
conjunto de entrenamiento \dset{TR-H} se seleccionaron al azar $200$
de los $323$ ejemplos de \premirna{s} de la especie humana como clase
positiva y $400$ pseudo \premirna{s} de \dset{coding} como ejemplos de
clase negativa.
El conjunto de prueba \dset{TE-H} se compuso con los $123$ ejemplos de
clase positiva no usados para entrenamiento, y con otros $246$
ejemplos de clase negativa seleccionados al azar de la base
\dset{coding}, excluyendo aquellos ya utilizados para entrenamiento.
La composición de los conjuntos de entrenamiento y prueba de este
problema se resume en la
\iflatexml{}Tabla~\ref{tbl:pruebasng}\else\autoref{tbl:pruebasng}\fi.
