%
%
\subsection{Problema miPred}
%
El problema \mipred{} se basa en los datos utilizados para
entrenamiento y prueba del método miPred \cite{ng}.

Los ejemplos de clase positiva se obtienen de la versión 8.2 (de julio
de 2006) de la base de datos \dataset{miRBase}, que contiene 323
\premirna{s} de la especie humana, incluyendo aquellos con estructura
secundaria tipo horquilla (bucle central único) así como ramificada
(bucles múltiples).
Los ejemplos de clase negativa se obtienen de la base de datos
\dataset{coding}, al igual que en el problema \tripletsvm{}.

Para el armado del conjunto de entrenamiento, se seleccionaron al azar
200 ejemplos de clase positiva y 400 del conjunto de datos negativos.
El conjunto de prueba se armó con los 123 ejemplos restantes de clase
positiva y 246 ejemplos de clase negativa seleccionados al azar,
excluyendo los ya utilizados para entrenamiento.

Si bien se mantiene el origen de los datos y el número de ejemplos de
clase positiva y negativa en los conjuntos de entrenamiento y prueba
respecto del trabajo de referencia, resulta imposible replicar la
composición exacta de ambos conjuntos, ya que los autores no brindan
información de las particiones utilzadas.

%% Si bien los conjuntos de datos de entrenamiento y prueba mantienen las
%% mismas cantidades de ejemplos positivos y negativos y utilizan las
%% mismas fuentes de datos que en el trabajo de referencia, resulta
%% imposible replicar la composición exacta de ambos conjuntos de datos,
%% ya que los autores no publicaron los conjuntos de datos por separado.

La composición de los conjuntos de entrenamiento y prueba se detalla
en \autoref{tbl:pruebasng}.
