\chapter{Observabilidad y comunicación}

La propuesta DevOps enfatiza la necesidad de acelerar la
\e{retroalimentación} mediante canales de comunicación eficientes
para los equipos de trabajo y herramientas adecuadas para la medición
del funcionamiento de los servicios.

Si bien la Dirección cuenta con medios de comunicación formales tales
como el e-mail y el sistema de gestión de proyectos, se observó que el
intercambio de información en el trabajo diario se encontraba disperso
entre éstos, la comunicación oral y las herramientas de mensajería
instantánea tales como Google Hangouts y WhatsApp. Esta diversidad de
canales de comunicación dificulta el acceso a la información e
incrementa el nivel de estrés de los individuos. Para atacar esta
problemática se planteó como objetivo explícito del Proyecto la
implementación de un servicio de chat para uso interno.

Se trabajó además en la implementación de herramientas para aumentar
la observabilidad de los servicios. Se diseñó una arquitectura para la
recolección, almacenamiento y visualización de métricas y eventos, y
se configuró una interfaz de visualización de los registros
(\e{logs}) de los servicios.

Por último, se describe el trabajo efectuado para la implementación de
un servicio de \e{consola de operaciones}, una tarea que no
resultó exitosa.

\section{Servicio de chat}

Para la implementación del servicio de chat se consideraron las
herramientas de código abierto Gitter y Mattermost. Gitter\footnote{
  \href{https://gitter.im/}{https://gitter.im/}} es es una herramienta
de chat ofrecida por los desarrolladores de GitLab con un enfoque
orientado a la creación de comunidades. Mattermost\footnote{
  \href{https://mattermost.com/}{https://mattermost.com/} }, por su
parte, se posiciona como una herramienta para la comunicación interna
de los equipos de trabajo (teams) en entornos empresariales, y es la
alternativa de software libre al servicio comercial ofrecido por
Slack. Ambas herramientas ofrecen funcionalidades similares:
salas/canales de chat, mensajes directos, notificaciones, bots y
comandos.

En primera instancia se desplegaron ambas herramientas en entornos de
prueba para su evaluación. Finalmente se decidió implementar
Mattermost dado que su configuración resultó más sencilla y la
interfaz web más amigable que Gitter.

La implementación definitiva de Mattermost se realizó escribiendo el
código Ansible necesario para la configuración del servicio. Esta
configuración incluye la autenticación de los usuarios a través de
GitLab.

Se creó un canal de chat denominado ``Actividad'' en el cual se publican
automáticamente los eventos de lanzamiento de un trabajo de AWX y su
resultado. También se configuraron notificaciones de eventos de GitLab
a modo de prueba en algunos servicios.

\section{Servicio de análisis de registros (logs)}

La Dirección cuenta con un servicio interno de recolección y
almacenamiento de registros (\e{logs}) para todos los
servicios. Cada instancia cuenta con un agente (\e{filebeat}) que
recopila y envía los registros a un servidor centralizado denominado
\e{logserver}. Los desarrolladores y operadores de los servicios
se conectan al servidor central \e{logserver} vía SSH, y
visualizan los registros utilizando herramientas de línea de comandos
pueden visualizar el contenido de los logs. La forma de acceso vía SSH
presenta una serie de inconvenientes tales como la configuración
manual de los usuarios y sus permisos y la necesidad de establecer
múltiples conexiones para visualizar registros en simultáneo.

Con el propósito de dar mayor accesibilidad a los registros es que se
decidió implementar un servicio de análisis de registros, que permita
visualizar y filtrar los registros en tiempo real, así como también
efectuar búsquedas sobre los mismos.

Se evaluaron las herramientas de código abierto Kibana y
Graylog. Ambas ofrecen una interfaz web para visualizar los registros
en tiempo real y efectuar búsquedas eficientes mediante el motor de
búsqueda Elasticsearch.

Luego de una implementación de prueba inicial, se optó por la
implementación de Graylog a partir de los siguientes criterios:

\begin{itemize}
\item La interfaz de Graylog
  (\iflatexml{}Figura~\ref{fig:graylog}\else\autoref{fig:graylog}\fi{})
  está orientada al análisis de registros,
  mientras que la de Kibana se enfoca en el procesamiento de flujos de
  información textual en general.
\item Graylog ofrece una interfaz API para recibir y guardar registros
  de logs, de potencial utilización directa para registrar eventos por
  parte de los servicios desarrollados en la DIPT.
\item Graylog permite utilizar el directorio LDAP ya existente para la
  autenticación y autorización de los usuarios, mientras que Kibana
  sólo ofrece autenticación y autorización básicas.
\end{itemize}
La configuración del servicio se basó en la guía oficial de
instalación\footnote{
  \href{https://docs.graylog.org/en/3.1/pages/installation/docker.html}{https://docs.graylog.org/en/3.1/pages/installation/docker.html}
} y se guardó como código en un repositorio GitLab. Se utilizó la
herramienta Ansible para el despliegue del nuevo servicio.
