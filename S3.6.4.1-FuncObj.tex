\subsubsection{Función objetivo}
La estrategia aquí propuesta utiliza una de estas cotas
alternativas, denotada $\rho$ y definida según

\begin{align}
  \rho = \rho_R \cdot \rho_M,
\end{align}
donde $\rho_R$, $\rho_M$ son los factores ``radio'' y ``margen''

\begin{align}
  \rho_R &= R^2+\frac{1}{C}, \\
  \rho_M &= \|\ww\|^2+2C\sum\xi_i.
\end{align}
A diferencia de $\T{RM}$, la función $\rho$ ajusta el valor
real de $E_{\T{LOO}}$ con demasiada holgura, con lo que su valor
pierde el significado original de representar una tasa de error. Sin
embargo, tiene la importante propiedad de poseer un mínimo global en
las cercanías de los hiperparámetros óptimos $(C,\gamma)$ que
minimizan la tasa de error $E_{\T{LOO}}$.

\subsubsection{Cálculo de $\rho$}
El valor $R^2$, necesario para el cálculo de $\rho_R$, viene dado por

\begin{align}
  R^2 = 1 - \Bbeta_*^T \KK \Bbeta_*,
\end{align}
donde $\Bbeta_*$ es la solución al problema de optimización conocido
como ``SVM de una clase'' \cite{scholkopf}

\begin{align}
\begin{split}
  \arg\min_{\Bbeta}\quad&\Bbeta^T \KK \Bbeta,\\
  \T{sujeto a}    \quad&0\leq\beta_i\leq{}1,\quad{}i=1,\ldots,\ell,\\
                       &\B{1}^T_\ell\,\Bbeta=1.
  \end{split}
  \label{svm-oneclass}
\end{align}
Aquí, $\B{1}_{\ell}$ es un vector columna con $\ell$ elementos iguales
a 1, y $\KK$ es la matriz del núcleo con elementos
$k_{ij}=k(\xx_i,\xx_j)$. La matriz $\KK$ es definida positiva siempre
que se cumpla la condición

\begin{align}
\label{cond-kmatrix-defpos}
  i\neq j \iff \xx_i\neq\xx_j,\quad \forall\,i,j\in{1,\ldots,\ell},
\end{align}
esto es, siempre que no haya ejemplos repetidos en el conjunto de
entrenamiento. Sin perder mucha generalidad, se considera que tal es
el caso, y con ello, se asegura que el problema \cite{svm-oneclass}
tiene solución única \cite{SOL-UNICA-DEFINIDA-POSITIVA}.

El valor ``margen'' $\rho_M$ es dos veces la solución al problema de
optimización de la SVM (\autoref{svm-primal-blando}).  Dada la
equivalencia primal-dual de la solución, si $\Balpha$ es solución a
la forma dual (\autoref{svmprob-dual-soft}), se tiene simplemente

\begin{align}
\label{prieqdual}
  \rho_M &= 2\left(\frac{1}{2}\|\ww\|^2+C\sum\xi_i \right)
  =2\left(\B{e}^T\Balpha-\frac{1}{2}\Balpha^T\B{Q}\Balpha\right),
\end{align}
donde $\QQ$ es la matriz con elementos $Q_{ij}=y_iy_jk(\xx_i,\xx_j)$.
