%
\subsubsection{La función error empírico}
%
El error de validación del modelo $h$ sobre un conjunto
$V=((\xx_k,y_k))\,i=1,\ldots,\ell^V$ puede escribirse
%
\begin{align}
\label{e3:error-test-alt}
  E^V &= \frac{1}{\ell^V}\sum_{k=1}^{\ell^V} H(-{y}_k {h}(\xx_k))
\end{align}
%
donde $H(\cdot)$ es la función escalón de Heaviside.
Esta formulación
del error con la función escalón $H$ en lugar de la pérdida $0-1$
equivalente pone de relieve el hecho que el error $E^V$ es una función
discontinua, al ser suma de discontinuas, y por tanto no derivable.
La función ``error empírico'' \cite{ayat} se construye a partir de una
interpretación bayesiana del concepto de error, y posee las
propiedades deseables de continuidad y derivabilidad que permiten
su utilización como función objetivo en un esquema de minimización por
descenso por gradiente.

Si se conoce la \e{probabilidad a posteriori} $p_k$ de que el ejemplo
$\xx_k$ pertenezca a la clase positiva
%
\begin{align}
  \label{e3:pk}
  p_k = p(\xx_k) = P(h(\xx_k)=+1|\xx_k),
\end{align}
%
se puede caracterizar la \e{probabilidad de error} $E_k$ cometido al
clasificar el ejemplo $\xx_k$ según
%
\begin{align}
\label{e3:Ek}
  E_k = P(h(\xx_k)\neq y_k) = |t_k-{p}_k| =
  \begin{cases}
    {p}_k, & t_k=0\\ 1-{p}_k, & t_k = 1,
  \end{cases}
\end{align}
%
donde $t_k=\frac{y_k+1}{2}$ es un ``valor deseado'' calculado a partir
de la clase conocida $y_k$. La probabilidad de error para el conjunto
completo $V$ puede escribirse
%
\begin{align}
\label{Err1}
  E = \frac{1}{\ell^V}\sum_{j=1}^{\ell^V} E_k.
\end{align}
%
Ésta es la función \e{error empírico}.
Para el cálculo de $E$ se requiere conocer la probabilidad $p_k$
(\iflatexml{}Ecuación~\ref{e3:pk}\else\autoref{e3:pk}\fi), que
puede determinarse a partir de la salida binaria del modelo SVM.
En su lugar, se utiliza un estimador $\hat{p}_k$, que se deriva
a continuación.
