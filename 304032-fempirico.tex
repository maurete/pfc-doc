%
\subsubsection{La función error empírico}
%
La función error empírico es la función objetivo a minimizar, y se
construye a partir de una estimación de la probabilidad de error del
modelo sobre el conjunto de entrenamiento.
Esta función posee características deseables de continuidad y
derivabilidad que permiten su optimización mediante descenso por
gradiente.

Considerando en primer lugar la pérdida $0-1$ promedio del modelo $h$
sobre un conjunto de validación $V=((\xx_k,y_k)),\,i=1,\ldots,\ell^V$,
se observa que este error puede escribirse
%
\begin{align}
\label{e3:error-test-alt}
E^V \tab = \frac{1}{\ell^V}\sum_{k=1}^{\ell^V} H(-{y}_k {h}(\xx_k))
\tabs = \frac{1}{\ell^V}\sum_{k=1}^{\ell^V} L_{0-1}(y_k,h(\xx_k)).
\end{align}
%
El uso de la función escalón unitario de Heaviside $H(\cdot)$ pone de
relieve el hecho que $E^V$ es una suma de funciones discontinuas, y
por lo tanto no es derivable.
Para salvar esta limitación, la función error empírico se construye a
partir de una estimación $\hat{p}_k$ de la \e{probabilidad a
  posteriori} $p_k$ de que el ejemplo $\xx_k$ pertenezca a la clase
positiva
%
\begin{align}
  \label{e3:pk}
  p_k = p(\xx_k) = P(h(\xx_k)=+1|\xx_k).
\end{align}
%
Por el momento, se supone que $p_k$ es una función continua y
derivable.
La \e{probabilidad de error} $E_k$ del modelo al clasificar el ejemplo
$\xx_k$ puede escribirse en términos de la probabilidad $p_k$ según
%
\begin{align}
\label{e3:Ek}
  E_k = P(h(\xx_k)\neq y_k) = |t_k-{p}_k| =
  \begin{cases}
    {p}_k, & t_k=0\\ 1-{p}_k, & t_k = 1,
  \end{cases}
\end{align}
%
en donde $t_k=\frac{1}{2}({y_k+1})$ es un ``valor deseado'' calculado
a partir de la clase conocida $y_k$.
El error empírico es simplemente la probabilidad de error promedio
para el conjunto completo $V$:
%
\begin{align}
\label{Err1}
  E = \frac{1}{\ell^V}\sum_{j=1}^{\ell^V} E_k.
\end{align}
%
Para el cálculo exacto de $E$ se requiere conocer la probabilidad
$p_k$ (\ref{e3:pk}), la cual no puede determinarse a partir de la
salida binaria del modelo de la SVM.
En su lugar se utiliza un estimador $\hat{p}_k$, que se obtiene
ajustando un modelo probabilístico a la salida del modelo, tal como se
explica a continuación.
%% La probabilidad $p_k$ (\ref{e3:pk}), requerida para el cálculo de $E$,
%% no puede determinarse a partir de la salida binaria del modelo de la
%% SVM.
%% En su lugar, se utiliza un estimador $\hat{p}_k$ que surge de ajustar
%% un modelo probabilístico a la salida del modelo.
%% A continuación se describe la obtención de este estimador.
