%
\subsubsection{La función error empírico}
%
El error de validación del modelo $h$ sobre un conjunto
$V=((\xx_j,y_j))\,i=1,\ldots,\ell^V$ puede escribirse
%
\begin{align}
\label{e3:error-test-alt}
  E^V &= \frac{1}{\ell^V}\sum_{j=1}^{\ell^V} H(-{y}_j {h}(\xx_j))
\end{align}
%
donde $H(\cdot)$ es la función escalón de Heaviside. Esta formulación
del error con la función escalón $H$ en lugar de la pérdida $0-1$
equivalente pone de relieve el hecho que el error $E^V$ es una función
discontinua, al ser suma de discontinuas, y por tanto no derivable.
La función ``error empírico'' \cite{ayat} se construye a partir de una
interpretación bayesiana del concepto de error, y posee las
características deseables de continuidad y derivabilidad que permiten
su utilización como función objetivo en un esquema de minimización por
descenso por gradiente.

Si se conoce la \e{probabilidad a posteriori} $p_j$ de que el ejemplo
$\xx_j$ pertenezca a la clase positiva
%
\begin{align}
  \label{e3:pk}
  p_j = p(\xx_j) = P(h(\xx_j)=+1|\xx_j),
\end{align}
%
se puede caracterizar la \e{probabilidad de error} $E_j$ cometido al
clasificar el ejemplo $\xx_j$ según
%
\begin{align}
\label{e3:Ek}
  E_j = P(h(\xx_j)\neq y_j) = |t_j-{p}_j| =
  \begin{cases}
    {p}_j, & t_j=0\\ 1-{p}_j, & t_j = 1,
  \end{cases}
\end{align}
%
donde $t_j=\frac{y_j+1}{2}$ es un ``valor deseado'' calculado a partir
de la clase conocida $y_j$. La probabilidad de error para el conjunto
completo $V$ puede escribirse
%
\begin{align}
\label{Err1}
  E = \frac{1}{\ell^V}\sum_{j=1}^{\ell^V} E_j.
\end{align}
%
Ésta es la función \e{error empírico}. Puede comprobarse que, a
diferencia del error con pérdida 0-1, ésta es una función continua y
derivable.

Para el cálculo de $E$ se requiere conocer la probabilidad $p_j$
(\iflatexml{}Ecuación~\ref{e3:pk}\else\autoref{e3:pk}\fi), la cual no
puede determinarse a partir de la salida binaria del modelo SVM.  En
su lugar, se utiliza un estimador de $\hat{p}_j$.
