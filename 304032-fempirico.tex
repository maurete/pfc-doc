%
\subsubsection{La función error empírico}
%
La función error empírico es la función objetivo a minimizar, y se
basa en una interpretación probabilística del error del modelo.
Para explicarla, considérese en primer lugar que el error de del
modelo $h$ sobre un conjunto de validación
$V=((\xx_k,y_k)),\,i=1,\ldots,\ell^V$ puede escribirse
%
\begin{align}
\label{e3:error-test-alt}
  E^V &= \frac{1}{\ell^V}\sum_{k=1}^{\ell^V} H(-{y}_k {h}(\xx_k)),
\end{align}
%
donde $H(\cdot)$ es la función escalón unitario de Heaviside.
Esta formulación del error con la función $H$, en lugar de la pérdida
$0-1$ equivalente, pone de relieve el hecho que $E^V$ es una función
discontinua (por ser suma de discontinuas), y por tanto no derivable.
La función error empírico, en cambio, se construye a partir de una
interpretación probabilística del error \cite{ayat}, y posee las
propiedades deseables de continuidad y derivabilidad que permiten
su minimización mediante descenso por gradiente.
Si se conoce la \e{probabilidad a posteriori} $p_k$ de que el ejemplo
$\xx_k$ pertenezca a la clase positiva
%
\begin{align}
  \label{e3:pk}
  p_k = p(\xx_k) = P(h(\xx_k)=+1|\xx_k),
\end{align}
%
se puede caracterizar la \e{probabilidad de error} $E_k$ cometido al
clasificar el ejemplo $\xx_k$ según
%
\begin{align}
\label{e3:Ek}
  E_k = P(h(\xx_k)\neq y_k) = |t_k-{p}_k| =
  \begin{cases}
    {p}_k, & t_k=0\\ 1-{p}_k, & t_k = 1,
  \end{cases}
\end{align}
%
donde $t_k=\frac{1}{2}({y_k+1})$ es un ``valor deseado'' calculado a
partir de la clase conocida $y_k$.
La función error empírico es simplemente la probabilidad de error
promedio para el conjunto completo $V$:
%
\begin{align}
\label{Err1}
  E = \frac{1}{\ell^V}\sum_{j=1}^{\ell^V} E_k.
\end{align}
%
Para el cálculo de $E$ se requiere conocer la probabilidad $p_k$
(\iflatexml{}Ecuación~\ref{e3:pk}\else\autoref{e3:pk}\fi), que no
puede determinarse a partir de la salida binaria del modelo SVM.
En su lugar, se utiliza un estimador $\hat{p}_k$, que se obtiene
ajustando un modelo probabilístico a la salida del modelo.
