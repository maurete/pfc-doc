\begin{longtable}{@{}p{0.08\textwidth}%
@{\hspace{0.01\textwidth}}p{0.22\textwidth}%
@{\hspace{0.01\textwidth}}p{0.68\textwidth}@{}}
  \sfbf{\footnotesize Pos.} &
  \sfbf{\footnotesize Característica} &
  \sfbf{\footnotesize Descripción}
  \endhead
  33 & $L_3$ &
  Longitud del tallo: cantidad de nucleótidos en el tallo de
  la estructura de horquilla. \\
  34 & $P$ &
  Número de pares de bases en el pre-miRNA. \\
  35 & $L_3/P$ &
  Grado de complementariedad entre los dos brazos de la estructura de
  horquilla: para una complementariedad perfecta, se da el valor
  mínimo 2. Este valor aumenta conforme aumenta el número de
  bases ``sueltas'' (no acopladas) en el tallo. \\
  36 & $GC=\frac{N_\ntC+N_\ntG}{L_3}$ &
  Proporción de bases \ntG y \ntC en el tallo.  Se calcula contando el
  número de bases \ntC y \ntG en el tallo y dividiendo por $L_3$.
\end{longtable}
