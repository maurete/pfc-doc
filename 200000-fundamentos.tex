%
%
%
%
\iflatexml{}\else\setcounter{chapter}{1}\fi%
%
\chapter{Fundamentos teóricos}
%
%% El propósito principal del presente proyecto es generar una aplicación
%% de técnicas de la Inteligencia Computacional (IC) sobre un tipo de dato
%% específico, las cadenas de \premirna{s}.
%% La Inteligencia Computacional es la disciplina que estudia la
%% aplicación de técnicas y metodologías inspiradas en la naturaleza para
%% resolver problemas complejos.
%% En general, las técnicas de IC presentan cualidades típicas del
%% razonamiento humano, ``generalizando'' el conocimiento inexacto o 
%% incompleto, y son capaces de tomar decisiones en forma adaptativa.
%
En este capítulo se presenta una introducción de los conceptos
teóricos fundamentales del aprendizaje supervisado y de la
Inteligencia Computacional en general.
Más adelante, se describen las técnicas del perceptrón multicapa y de
la máquina de vectores de soporte, utilizadas en el presente trabajo
para la generación de los clasificadores.
