%
%
%
%
\chapter{Fundamentos teóricos}
%
El propósito principal del presente proyecto es generar una aplicación
de técnicas de la Inteligencia Computacional (IC) sobre un tipo de dato
específico, las cadenas de \premirna{s}.
La Inteligencia Computacional es la disciplina que estudia la
aplicación de técnicas y metodologías inspiradas en la naturaleza para
resolver problemas complejos.
En general, las técnicas de IC presentan cualidades típicas del
razonamiento humano, ``generalizando'' el conocimiento inexacto o 
incompleto, y son capaces de tomar decisiones en forma adaptativa.

En el presente Capítulo se presenta una introducción teórica a los
conceptos fundamentales de la IC en general, en particular del área del
\e{aprendizaje supervisado}.
Más adelante, se describen las técnicas específicas aplicadas en el
presente trabajo: el perceptrón multicapa y la máquina de vectores de
soporte.
