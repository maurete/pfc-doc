\chapter{Fundamentos teóricos}

DevOps es un movimiento cultural que busca mejorar el software como
producto así como también la vida de quienes se dedican a su
producción. Su propuesta es aplicar las prácticas y principios más
fiables del mundo de la manufactura de productos físicos a las
Tecnologías de la Información (TI)
\cite{handbook}. Sus
fundamentos provienen de áreas tales como \e{lean}\footnote{ El
  adjetivo inglés \e{lean} (/li:n/) puede traducirse como ligero,
  esbelto, despojado.} \cite{lean}, la teoría de
las limitaciones
\cite{thegoal}, el sistema de
producción de Toyota
\cite{toyota}, y el
aprendizaje organizacional
\cite{fifth}. El resultado
es la producción de software de alta calidad, confiable, estable y
seguro, en un ambiente de trabajo saludable y con un bajo costo
económico.

Entre los beneficios comúnmente atribuidos a una adopción de prácticas
DevOps podemos citar
\cite{awsdevops}:

\begin{itemize}
\item \e{Velocidad}: los equipos de desarrollo gestionan sus
  servicios mediante la automatización de la infraestructura y la
  integración continua.
\item \e{Entrega rápida}: la reducción del tamaño de las
  actualizaciones junto con la integración continua y la entrega
  continua aumentan la velocidad de llegada de los requerimientos a
  los clientes.
\item \e{Confiabilidad}: la integración continua efectúa pruebas
  del software constantemente haciendo que los problemas se detecten
  antes. Las herramientas de monitoreo proveen retroalimentación
  (\e{feedback}) sobre el funcionamiento del software.
\item \e{Colaboración}: la cultura DevOps enfatiza valores como
  la propiedad y la responsabilidad. Los equipos de desarrollo y de
  operaciones colaboran estrechamente, comparten muchas
  responsabilidades y combinan sus flujos de trabajo.
\end{itemize}
\section{Contexto histórico}

Si bien el origen de DevOps se identifica en el movimiento
\e{lean}, muchos lo reconocen como la continuación lógica de la
propuesta del desarrollo ágil que revolucionó la ingeniería de
software allá por 2001
\cite{agilemanifesto}.

A continuación se presenta un breve repaso de los contextos, los
eventos, y las personas de los que surgieron las ideas que dan forma a
la propuesta DevOps.

\subsection{Los comienzos}

En los comienzos, la persona encargada del desarrollo de software se
encargaba también de su operación. Jean Bartik, programadora de la
primera computadora de uso general ENIAC, aprendió a programarla a
partir de sus diagramas de circuitos. Programar la computadora
requería ajustar perillas y cambiar conexiones de cables, algo que hoy
en día asociamos más a la idea de operación.

Con el avance de las capacidades del hardware, éste pasó a dejar de
ser un impedimento, lo cual posibilitó el desarrollo de software de
mayor complejidad. Desde la década de los 60 hasta finales de los 80
tuvo lugar la llamada \e{crisis del software}, en la que se
identificaron los problemas del desarrollo. Esto dio origen a la
disciplina de ingeniería de software, que estableció un marco teórico
y propuso métodos para la producción del software.

En los años 90, el auge de internet puso en evidencia los problemas de
los métodos tradicionales de ingeniería de software: se necesitaba
responder a las demandas de los usuarios con mayor rapidez. La
operación del software en línea pasó de ser una tarea secundaria a un
aspecto crítico.

\subsection{El movimiento \e{lean}}

En los años 80, el sistema de producción de Toyota inventó técnicas
tales como el análisis del flujo de valor y los tableros de Kanban. En
la década siguiente, estas ideas se expandieron a otras disciplinas
gracias al movimiento \e{lean}.

Dos creencias fundamentales en la cultura lean son que el tiempo de
producción desde el inicio hasta el producto terminado es el mejor
predictor de la calidad, la satisfacción del cliente y la felicidad de
los empleados; y que la mejor forma de alcanzar tiempos cortos de
producción es organizar el trabajo en lotes pequeños.

Lean busca crear valor para el cliente aplicando el pensamiento
científico, asegurando la calidad desde el origen, creando un flujo
constante en la cadena de valor, con un liderazgo basado en la
humildad y respetuoso de cada individuo.

\subsection{El manifiesto ágil}

En 2001 un grupo de críticos del desarrollo de software publicaron el
``manifiesto para el desarrollo de software ágil''
\cite{agilemanifesto}, el cual
proponía un marco de trabajo con valores y principios ``livianos'' en
contraposición a las metodologías formales tradicionales tales como el
modelo de desarrollo en cascada. Un principio clave en este documento
era ``realizar entregas funcionales del software con frecuencia,
idealmente en un par de semanas''. Esta propuesta reforzaba el deseo de
las entregas incrementales con ciclos de desarrollo cortos en
contraste con las entregas grandes derivadas del modelo en
cascada. Otros principios ponían énfasis en la necesidad de
organizarse en equipos pequeños, motivados, bajo un modelo gerencial
de alta confianza.

Se reconoce en la adopción de metodologías ágiles un gran incremento
en la productividad de muchas organizaciones desarrolladoras de
software. Asimismo, muchos momentos clave en la historia de DevOps
tuvieron lugar dentro de la comunidad ágil o en conferencias de
desarrollo ágil.

\subsection{El pase a la infraestructura ágil}

El origen de DevOps puede quizá atribuirse a la presentación
``\e{10 Deploys per Day: Dev and Ops Cooperation at Flickr}''
(``\e{Más de 10 despliegues por día: cooperacion entre desarrollo
  y operaciones en Flickr}'') de John Allspaw y Paul Hammond
\cite{flickr} en la
conferencia \e{Velocity 2009}, donde contaron cómo habían logrado
crear objetivos en común entre los equipos de desarrollo
(\e{Dev}) y operaciones (\e{Ops})
(\iflatexml{}Figura~\ref{fig:devheartops}\else\autoref{fig:devheartops}\fi), y cómo
implementaron las prácticas de integración continua de modo que las
tareas de despliegue formaran parte del trabajo diario de todos. Los
asistentes a la presentación cuentan que en ese momento comprendieron
que estaban frente a un cambio histórico en el mundo de la producción
de software. El impacto fue tal, que ese mismo año se organizó en
Gante (Bélgica) la conferencia DevOpsDays. Allí surgió el nombre
``DevOps''.

