%
\subsubsection{El error de generalización}
%
Para un modelo $h$, el error de generalización es el error esperado al
clasificar un ejemplo ``nuevo'', independiente del conjunto de
entrenamiento $D$, y se define según
%
\begin{align}
  \mathcal{R}_\nu=\mathds{E}_\nu\left[L(\xx,\yy,h(\xx))\right]
  =\int_{X\times Y} L(\xx,\yy,h(\xx))\, d\nu(\xx,\yy).
\end{align}
%
La función $L$ es una \e{función de pérdida} que mide la cantidad de error
del modelo $\hat{\yy}=h(\xx)$ para un ejemplo
$(\xx,\yy)\in{}X\times{}Y$ mediante la transformación
$L(\xx,\yy,\hat{\yy})$.
Si la predicción resultante del modelo $\hat{\yy}$ es correcta, el
error es nulo.
En la práctica, la mayoría de las funciones de pérdida no dependen de
la entrada $\xx$, sino sólo de la predicción $\hat{\yy}$ y la clase
$\yy$.
Algunas de las funciones de pérdida más comunes son:
%
\begin{align}
  \T{Pérdida 0-1:} \quad\tab
    L(\yy,\hat{\yy})=\begin{cases}0,\quad{}\yy=\hat{\yy}\\
      1,\quad\T{en otro caso}\end{cases} \\
  \T{Pérdida ``bisagra'':} \quad\tab
    L(\yy,\hat{\yy})=\max\{0,\yy-\hat{\yy}\}\\
  \T{Error absoluto:} \quad\tab
    L(\yy,\hat{\yy})=|\yy-\hat{\yy}|\\
  \T{Error cuadrático:} \quad\tab
    L(\yy,\hat{\yy})=(\yy-\hat{\yy})^2.
\end{align}

Para poder calcular el error de generalización se requiere conocer la
distribución generadora de datos $\nu$.
Sin embargo, toda la información disponible de $\nu$ es la muestra $D$,
de probabilidad desconocida.
Entonces, la aplicación del concepto de error de generalización es
teórica, y en la práctica se utilizarán otras medidas de error para
aproximarlo.

%% El cálculo de este error requiere conocer la distribución generadora
%% de datos $\nu$.
%% Sin embargo, toda la información que se tiene de $\nu$
%% es la realización $D$, para la cual no se conoce su probabilidad.
%% Por ello, la aplicación del concepto de error de generalización es
%% teórica, y se utilizan en la práctica medidas que lo aproximan.
