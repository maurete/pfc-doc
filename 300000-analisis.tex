\section{3 Análisis de la organización}

La Dirección de Informatización y Planificación Tecnológica (DIPT) es el área de la Universidad Nacional del Litoral (UNL) encargada de brindar servicios de software a la Universidad. Bajo su órbita se encuentran los sistemas de gestión de alumnos, bienestar estudiantil, de presupuesto, de gastos, de gestión del personal, de expedientes, de gestión documental, de cargos y contratos, y de cursos y carreras en línea, entre otros.

\subsection{Aspectos generales}

En cuanto a su composición, la Dirección se subdivide en cuatro áreas: desarrollo, infraestructura, testing y soporte técnico. En el área de desarrollo se identifican 6 equipos con distintas áreas de interés, contabilizando un total de 25 personas. El equipo de infraestructura consta de 5 personas. Asimismo, el área de testing está conformada por 4 personas, y otras tres personas integran el equipo de soporte técnico a usuarios.

La Dirección tiene a su cargo más de 50 servicios de software, los cuales funcionan principalmente sobre plataformas PHP y Java. Actualmente unos 25 de estos servicios cuentan con un desarrollo activo.

\subsubsection{Gestión del código fuente}

Los equipos de desarrollo utilizan repositorios SVN y Git para la gestión del código, dependiendo del servicio. En los repositorios SVN utilizan la estructura estándar de carpetas \textit{trunk/tags/branches} propia de SVN. En los repositorios Git, la utilización de ramas no está muy difundida. En general, se escribe el código nuevo sobre la rama \textit{master} y se generan \textit{tags} específicos para cada versión a publicar.

El equipo de infraestructura utiliza también repositorios Git y SVN para la gestión del código propio. A diferencia de los equipos de desarrollo, el equipo de infraestructura considera el código publicado en la rama principal como la versión productiva. Excepto casos excepcionales, no se utilizan \textit{tags} ni \textit{branches} en estos repositorios.

\subsubsection{La herramienta Ansible}

A partir del año 2018 el equipo de infraestructura comenzó a utilizar la herramienta Ansible\footnote{ https://www.ansible.com/} para describir la configuración de la infraestructura de los servicios. Esta herramienta permite escribir código que describe el estado requerido de la configuración. Al ejecutarse, Ansible efectúa las modificaciones necesarias para alcanzar el estado requerido. Esto se conoce como \textit{configuración como código}. La adopción de Ansible también alcanzó a los equipos de desarrollo, que utilizan la herramienta para generar sus entornos de desarrollo locales a partir del código provisto por el área de infraestructura.

\subsection{Estrategia de análisis}

Al momento de implementar DevOps resulta fundamental contar con una visión global que identifique los roles dentro de la organización, las tareas realizadas por cada uno y las interdependencias entre las mismas.

Evidentemente, efectuar un análisis exhaustivo que identifique todas las tareas posibles, todas las personas involucradas, y todas las interdependencias resulta poco práctico. En cambio se optó por realizar reuniones de trabajo con representantes de los diferentes equipos para generar un entendimiento en común y definir aspectos prioritarios a considerar.

Como objetivo se definió trabajar sobre las tareas que involucran mayor interacción entre los equipos de infraestructura y de desarrollo. Para obtener un panorama general se realizaron reuniones con el equipo de infraestructura y los responsables de los servicios. En estas reuniones se plantearon preguntas como la siguientes:

\begin{itemize}
\item ¿Cuáles son las tareas que realizo con frecuencia?
\item ¿Cuáles de estas tareas considero repetitivas o tediosas?
\item ¿Cuáles acciones que no dependen de mí son necesarias para completar estas tareas?
\item Si pudiera pedir cualquier cosa, ¿qué pediría para mejorar esta tarea? (Incluso se puede pedir no hacerla más)
\end{itemize}
Con estas preguntas, inspiradas en la novela The Phoenix Project \href{https://www.zotero.org/google-docs/?eJVs0O}{[13]}, se buscó identificar las tareas más comunes, improductivas y plausibles de automatización. Se seleccionaron dos procesos para analizar:

\begin{itemize}
\item La creación de entornos para un nuevo servicio.
\end{itemize}
Es un proceso que en términos relativos es poco frecuente, pero se llegó a la conclusión de que es muy burocrático y requiere muchas intervenciones manuales. A pesar de que el equipo de infraestructura ha automatizado muchas tareas, resulta común saltarse alguna de ellas, lo que obliga a volver atrás en la secuencia para revisar y corregir.

\begin{itemize}
\item Entrega de un nuevo requerimiento en un servicio.
\end{itemize}
Abarca desde la solicitud de una nueva funcionalidad por parte del cliente hasta la puesta en producción de la misma. Se trata de un proceso a cargo del equipo de desarrollo, aunque a veces requiere la intervención del equipo de infraestructura, por ejemplo cuando se requieren modificaciones en los permisos, los scripts de actualización, o cualquier cambio de infraestructura requerido en la nueva versión. 

Una vez seleccionados estos procesos se realizaron reuniones adicionales para documentar en detalle las tareas, personas y recursos necesarios para cada uno de ellos.

\subsection{Análisis del flujo de valor}

Para analizar los procesos se recurrió a una forma simplificada de los “mapas de flujo de valor” (\textit{value-stream maps}). En el ámbito de la producción fabril, el mapa de flujo de valor es una técnica \textit{lean} que tiene como objetivo “visualizar y comprender el flujo de materiales e información desde el momento que se realiza el pedido hasta el producto terminado” \href{https://www.zotero.org/google-docs/?bOLGEz}{[17]}. En una organización TI el producto es inmaterial, aunque igualmente se puede establecer una secuencia de acciones necesarias para cumplir un objetivo, tal como un cambio en la infraestructura o la entrega de una funcionalidad de software.

La visualización concisa de la secuencia de tareas permite encontrar los puntos de demora y las actividades que no contribuyen al objetivo del proceso (entrega del producto) o bien a la mejora del proceso en sí mismo (planificación de mejora).

En el análisis realizado se elaboraron tablas (en lugar de un diagrama gráfico) que enumeran la secuencia de tareas. Para cada tarea, se especificó una descripción, actor, tiempo de espera, y tiempo de ejecución. 

\subsubsection{Creación de un nuevo servicio}

La creación de un nuevo servicio se gestiona utilizando un servicio interno denominado Pulmo. El responsable del servicio realiza la solicitud completando los datos que incluyen nombre, descripción, entornos requeridos, recursos de software y hardware necesarios para cada entorno, permisos de firewall y permisos de usuario para administración del servicio.

El sistema genera tickets a resolver por parte del equipo de infraestructura. Este equipo efectúa las siguientes tareas para cada uno de los entornos de integración, test y producción:

\begin{enumerate}
\item Escribir código Ansible: el código se escribe una sola vez a partir de la especificación detallada en el sistema Pulmo y se ejecuta en un entorno local. Esta tarea se efectúa una sola vez, el código generado es aplicable sobre los entornos de integración, test y producción.
\item Crear entrada del host en los servicios DNS y DHCP: se lleva a cabo ejecutando un script específico.
\item Crear instancia: se efectúa manualmente  mediante la línea de comandos en el cluster Ganeti.
\item Agregar instancia al servicio de backup (sólo producción): se modifica el código Ansible del servicio de backup y se aplica ejecutando el código en la línea de comandos.
\item Configurar el servidor de frontend: se edita el archivo de configuración correspondiente y se ejecuta luego un script que actualiza la configuración del redirector.
\item Configurar monitoreo: se agrega el host al servicio Zabbix invocando un script específico de línea de comandos.
\end{enumerate}
La Tabla 1.1 detalla el análisis del flujo de valor para la habilitación de un nuevo servicio. Los tiempos de espera son estimados para tareas con prioridad normal. Los tiempos de ejecución son estimados a partir de la experiencia de los actores involucrados.

\begin{tabular}{|l|l|l|l|}
\hline
Actor & Tarea & Espera & Ejecución \\ \hline
Dev & Solicitar servicio en Pulmo &  & 15min \\ \hline
Ops & Escribir código Ansible & 1h & 1h-8h \\ \hline
\textit{- entorno local creado -} &  &  &  \\ \hline
Ops & Crear entradas en DNS y DHCP &  & 10min \\ \hline
Ops & Crear instancias &  & 20min \\ \hline
Ops & Crear claves &  & 20min \\ \hline
Ops & Ejecutar playbook Ansible &  & 30min \\ \hline
Ops & Configurar frontend &  & 10min \\ \hline
Ops & Configurar monitoreo &  & 20min \\ \hline
\textit{- entornos de integración y test creados -} &  &  &  \\ \hline
Dev/Ops & Depuración de la configuración & 30d & 30min-3h \\ \hline
Dev & Solicitud de pase a producción & 1h & 5min \\ \hline
 & Crear/verificar hosts (DNS, DHCP) & 1h & 10min \\ \hline
 & Crear instancia de producción &  & 20min \\ \hline
 & Crear claves &  & 20min \\ \hline
 & Ejecutar playbook Ansible &  & 30min \\ \hline
 & Configurar servicio de backup &  & 15min \\ \hline
 & Configurar frontend &  & 10min \\ \hline
 & Configurar monitoreo &  & 20min \\ \hline
\textit{- entorno de producción creado -} &  &  &  \\ \hline
Total &  & 30d 3h & 3h-15h \\ \hline
Total (excluyendo espera de pase a producción) &  & 3h & 3h-15h \\ \hline
\end{tabular}
\textit{Tabla 1.1 Análisis del flujo de valor para el proceso de creación de un nuevo servicio.}

\subsubsection{Entrega de un nuevo requerimiento}

Para el análisis de este proceso se considera que el equipo de desarrollo se organiza en un ciclo de \textit{sprints} de una semana de duración. El proceso comprende desde el ingreso del requerimiento hasta la publicación de la nueva versión del servicio en producción.

\begin{enumerate}
\item El cliente (un usuario o un miembro del equipo de testing) crea un ticket solicitando una nueva funcionalidad o describiendo un error en el servicio.
\item Cada semana, el equipo de desarrollo realiza una reunión de coordinación que da comienzo a un nuevo \textit{sprint}. En esta reunión se revisan los requerimientos y se asignan prioridades para las tareas a realizar.
\begin{enumerate}
\item Se desagregan los requerimientos complejos en tareas específicas, creando tickets correspondientes para cada una.
\item Se crea en el repositorio SVN una rama de trabajo para la nueva versión.
\end{enumerate}
\item Durante el \textit{sprint} se continúa el trabajo normal escribiendo el código que resuelve cada tarea. El desarrollador ejecuta tests unitarios en forma manual durante el proceso de desarrollo.
\item Una vez concluido el \textit{sprint},  se ejecutan tests unitarios y de integración sobre el código y se publica la nueva versión, dando lugar a un nuevo \textit{sprint} para la versión siguiente. Este proceso se denomina internamente el \textit{cierre de versión}.
\item El responsable del proyecto actualiza la nueva versión en el entorno de integración y solicita a los responsables de los demás servicios que validen que la interacción con otros sistemas funciona correctamente.
\item Una vez confirmadas las integraciones se actualiza la versión en el entorno de test, y se solicita al solicitante que valide que la nueva funcionalidad es correcta.
\item Cuando el usuario confirma la nueva funcionalidad, el responsable del proyecto actualiza el entorno de producción.
\end{enumerate}
En la Tabla 1.2 se describe el análisis del flujo de valor para este proceso. Los tiempos de espera estipulados se refieren a tareas con prioridad normal. Los tiempos de ejecución se estimaron según la experiencia de los desarrolladores.

\begin{tabular}{|l|l|l|l|}
\hline
Actor & Tarea & Espera & Ejecución \\ \hline
Cliente & Ingresa requerimiento &  &  \\ \hline
Dev & Reunión del sprint & 0-5d & 30min \\ \hline
Dev & Crear rama de trabajo &  & 5min \\ \hline
Dev & Codificar solución & 1h-5d & 2h-10h \\ \hline
Dev & Cierre de versión (testing, publicación) & 5d & 40min \\ \hline
Dev & Despliegue en integración &  & 10min \\ \hline
Dev & Confirmación de integraciones & 2h & 1h \\ \hline
Dev & Despliegue en test &  & 10min \\ \hline
Cliente & Test de aceptación & 1h & 20min \\ \hline
Dev & Despliegue en producción & 1h & 20min \\ \hline
Total &  & 6d - 16d & 5h - 13h \\ \hline
\end{tabular}
\textit{Tabla 1.2 Análisis del flujo de valor para el proceso de entrega de un nuevo requerimiento.}

\subsection{Propuestas de mejora}

Las propuestas de mejora de ambos procesos se realizaron generando en primer lugar una forma simple de “mapa de estado futuro”. Estos mapas de estado futuro se basan en la secuencia de tareas identificadas previamente, las cuales son  modificadas según el estado que se desea alcanzar. El “estado futuro” se determinó en base a los requerimientos organizacionales y al consenso alcanzado en las reuniones de trabajo.

Las propuestas de cambio se basaron en los siguientes principios:

\begin{itemize}
\item Utilizar el repositorio de código GitLab para el código fuente y la configuración de cada servicio.
\item Escribir la configuración de la infraestructura como código (Ansible).
\item Contar con herramientas/servicios (web o de otra naturaleza) que permitan efectuar operaciones de manera autónoma por parte de los equipos de desarrollo e infraestructura, sin necesidad de generar solicitudes en el sistema de tickets.
\item Automatizar las operaciones tanto como sea posible.
\item Contar con un sistema de integración y entrega continua para realizar las tareas de compilación, pruebas de humo, testing unitario y de integración de manera automática, sin requerir la intervención de la persona que escribe el código.
\item Todos los cambios introducidos tienen como objetivo último acelerar la entrega del software al usuario final tanto como sea posible.
\end{itemize}
\subsubsection{Creación de un nuevo servicio}

Siguiendo los principios previamente especificados se propone modificar el proceso de creación de un nuevo servicio del siguiente modo:

\begin{enumerate}
\item El responsable del servicio crea un repositorio con el código Ansible necesario para la configuración del servicio. Esta configuración es directamente aplicable al entorno local para desarrollo.
\begin{itemize}
\item \textit{El responsable debe tener permisos suficientes para crear proyectos en GitLab.}
\item \textit{El equipo de infraestructura debe publicar el código reutilizable que aplica actualmente en sus “playbooks”.}
\end{itemize}
\item El responsable del servicio genera/verifica las claves necesarias (de base de datos, de conexión a otros servicios) para ser utilizadas en los entornos de integración, test y producción.
\item El responsable del servicio inicia un proceso de revisión de la configuración con el equipo de infraestructura creando un \textit{merge request} en el repositorio de código.
\begin{itemize}
\item \textit{Esta instancia reemplaza la creación de tickets en Redmine.}
\end{itemize}
\item Como parte de la resolución del merge request el equipo de infraestructura agrega el nuevo servicio a la interfaz web de Ansible.
\begin{itemize}
\item \textit{Se debe implementar la interfaz web para Ansible.}
\end{itemize}
\item Una vez finalizada la revisión se cierra el \textit{merge request} y el responsable del servicio lanza la tarea de configuración desde la interfaz web de Ansible. Esta tarea efectúa de forma automática las siguientes tareas:
\begin{itemize}
\item Creación de entradas DNS y DHCP
\item Creación de instancia
\item Agregado al servicio de backup (si fuera necesario)
\item Configuración del frontend
\item Configuración del servicio de monitoreo
\end{itemize}
\textit{Para dar soporte a esta automatización se deberán modificar varios aspectos de la infraestructura.}
\end{enumerate}


El análisis del estado futuro se presenta en la Tabla 1.3.

\begin{tabular}{|l|l|l|l|}
\hline
Actor & Tarea & Espera & Ejecución \\ \hline
Dev/Ops & Crear receta de Ansible &  & 1h - 8h \\ \hline
\textit{- entorno local creado -} &  &  &  \\ \hline
Dev/Ops & Crear claves &  & 20min \\ \hline
Dev/Ops & Merge request + depuración & 1h & 1h - 3h  \\ \hline
Ops/Dev & Agregar proyecto a interfaz web Ansible &  & 20min \\ \hline
Dev/Ops & Ejecutar playbook Ansible &  & 30min \\ \hline
\textit{- entornos de integración, test y producción creados -} &  &  &  \\ \hline
Total &  & 1h & 3h-12h \\ \hline
\end{tabular}
\textit{Tabla 1.3 Análisis del estado futuro para la creación de un nuevo servicio.}

\subsubsection{Entrega de un nuevo requerimiento}

Para el proceso de desarrollo y entrega de un nuevo requerimiento se  propuso modificar la secuencia de tareas según se enumera a continuación. Esta propuesta es genérica y la implementación se adaptará a las particularidades de cada servicio.

\begin{enumerate}
\item A partir del requerimiento del cliente, el desarrollador crea un “issue” para el seguimiento de la resolución del caso.
\begin{itemize}
\item \textit{Si la tarea es compleja, deberá separarla en varios “issues” con complejidad menor.}
\item \textit{Las reuniones de los sprints pasan a ser reuniones de revisión de las tareas efectuadas antes que de planificación de las tareas a realizar.}
\end{itemize}
\item El desarrollador crea un merge request mediante un solo click en el issue de GitLab. En la rama respectiva codifica la nueva funcionalidad requerida.
\begin{itemize}
\item \textit{El servicio de integración continua deberá efectuar testing unitario y de integración de manera automática sobre el código perteneciente a esta rama.}
\end{itemize}
\item Cuando se considera resuelto el requerimiento, se elimina el estado \textit{“WIP”} (“en curso”) del merge request en GitLab y se asigna el a otro miembro del grupo de desarrollo para revisión.
\item Se revisa en equipo la solución codificada y se asegura que los tests unitarios y de integración hayan sido exitosos. En caso de encontrarse problemas, se sigue escribiendo código sobre la misma rama hasta que hayan sido solucionados.
\item Se cierra el \textit{merge request} incorporando el código a la rama principal.
\begin{itemize}
\item \textit{Esta acción cierra de forma automática el issue, dispara nuevos tests automáticos sobre la rama de desarrollo principal, y genera y publica un artefacto de versión.}
\end{itemize}
\item Se despliega la nueva versión en el entorno de test  y se solicita al cliente que verifique la resolución del problema.
\item El cliente verifica la funcionalidad y confirma la resolución del problema.
\item Se despliega la nueva versión en el entorno de producción y se resuelve el ticket del cliente.
\end{enumerate}
El análisis del estado futuro se describe en la Tabla 1.4.

\begin{tabular}{|l|l|l|l|}
\hline
Actor & Tarea & Espera & Ejecución \\ \hline
Cliente & Ingresar requerimiento &  &  \\ \hline
Dev & Crear issue en GitLab + merge request & 1d & 5min \\ \hline
Dev & Codificar solución & 1h-5d & 2h-10h \\ \hline
\textit{- de forma automática: tests unitarios y de integración -} &  &  &  \\ \hline
Dev & Resolver WIP y solicitar revisión & 15min & 10min \\ \hline
Dev & Revisión de la solución & 1h & 20min \\ \hline
Dev & Efectuar merge & 5min & 5min \\ \hline
\textit{- de forma automática: cierre de issue, generación de artefacto -} &  &  &  \\ \hline
Dev & Despliegue en test & 15min & 20min \\ \hline
Cliente & Test de aceptación & 1h & 15min \\ \hline
Dev & Despliegue en producción & 1h & 10min \\ \hline
Dev & Resolución del ticket &  & 10min \\ \hline
Total &  & 2d - 6d & 3h - 11h \\ \hline
\end{tabular}
\textit{Tabla 1.4 Análisis del estado futuro para el proceso de entrega de un nuevo requerimiento.}

\subsection{Plan de trabajo}

El plan detallado en la Tabla 1.5 enumera las acciones requeridas y sus objetivos correspondientes (medibles) para alcanzar el “estado futuro” de ambas propuestas de mejora. Tal como los análisis de flujo de valor, este plan es una versión simplificada del formato propuesto por Rother en \textit{Learning to See} \href{https://www.zotero.org/google-docs/?3lDhJE}{[17]}.

Como se podrá apreciar, la lista de acciones propuesta se encuentra en línea con las acciones llevadas a cabo durante el desarrollo del Proyecto. Las implementaciones concretas se describen en los capítulos de aquí en adelante.

\begin{tabular}{|l|l|}
\hline
Acción & Objetivo \\ \hline
Configurar permisos en GitLab & Los desarrolladores acceden a los repositorios con código Ansible.

Los desarrolladores pueden crear proyectos para nuevos servicios. \\ \hline
Instalar y configurar una interfaz web para Ansible & Los desarrolladores pueden aprovisionar servicios. \\ \hline
Automatizar proceso de alta de entradas en DNS y DHCP & No se requiere intervención manual para dar de alta un host. \\ \hline
Automatizar la creación de instancias virtuales. & No se requiere intervención manual para crear una instancia virtual. \\ \hline
Automatizar el agregado de un servicio al backup. & No se requiere intervención manual para agregar un servicio al backup. \\ \hline
Automatizar el agregado de monitoreo a un servicio. & No se requiere intervención manual para agregar monitoreo a un servicio. \\ \hline
Automatizar la configuración de los frontends. & No se requiere intervención manual para configurar un servicio en los frontends. \\ \hline
Escribir un documento que especifique el método de trabajo utilizando ramas de vida corta e integración continua. & Se cuenta con un documento de referencia para el trabajo con el código. \\ \hline
Configurar un sistema de integración y entrega continua. & El servicio está disponible para ser utilizado. \\ \hline
Configurar una consola de operaciones & El servicio está disponible. \\ \hline
\end{tabular}
\textit{Tabla 1.5 Plan de trabajo para alcanzar el “estado futuro” de las propuestas de mejora surgidas del análisis del flujo de valor.}


