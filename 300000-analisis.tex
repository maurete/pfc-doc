%
%
%
\chapter{Análisis de la organización}
%
La Dirección de Informatización y Planificación Tecnológica (DIPT) es
el área de la Universidad Nacional del Litoral (UNL) encargada de
brindar servicios de software a la Universidad. Bajo su órbita se
encuentran los sistemas de gestión de alumnos, bienestar estudiantil,
de presupuesto, de gastos, de gestión del personal, de expedientes, de
gestión documental, de cargos y contratos, y de cursos y carreras en
línea, entre otros.
%
%
\section{Aspectos generales}
%
En cuanto a su composición, la Dirección se subdivide en cuatro áreas:
desarrollo, infraestructura, testing y soporte técnico. En el área de
desarrollo se identifican 6 equipos con distintas áreas de interés,
contabilizando un total de 25 personas. El equipo de infraestructura
consta de 5 personas. Asimismo, el área de testing está conformada por
4 personas, y otras tres personas integran el equipo de soporte
técnico a usuarios.

La Dirección tiene a su cargo más de 50 servicios de software, los
cuales funcionan principalmente sobre plataformas PHP y
Java. Actualmente unos 25 de estos servicios cuentan con un desarrollo
activo.
%
\subsection{Gestión del código fuente}
%
Los equipos de desarrollo utilizan repositorios SVN y Git para la
gestión del código, dependiendo del servicio. En los repositorios SVN
utilizan la estructura estándar de carpetas \e{trunk/tags/branches}
propia de SVN. En los repositorios Git, la utilización de ramas no
está muy difundida. En general, se escribe el código nuevo sobre la
rama \e{master} y se generan \e{tags} específicos para cada versión a
publicar.

El equipo de infraestructura utiliza también repositorios Git y SVN
para la gestión del código propio. A diferencia de los equipos de
desarrollo, el equipo de infraestructura considera el código publicado
en la rama principal como la versión productiva. Excepto casos
excepcionales, no se utilizan \e{tags} ni \e{branches} en estos
repositorios.
%
\subsection{La herramienta Ansible}
%
A partir del año 2018 el equipo de infraestructura comenzó a utilizar
la herramienta Ansible\footnote{\url{https://www.ansible.com/}} para
describir la configuración de la infraestructura de los
servicios. Esta herramienta permite escribir código que describe el
estado requerido de la configuración. Al ejecutarse, Ansible efectúa
las modificaciones necesarias para alcanzar el estado requerido. Esto
se conoce como \e{configuración como código}. La adopción de Ansible
también alcanzó a los equipos de desarrollo, que utilizan la
herramienta para generar sus entornos de desarrollo locales a partir
del código provisto por el área de infraestructura.
%
\section{Estrategia de análisis}
%
Al momento de implementar DevOps resulta fundamental contar con una
visión global que identifique los roles dentro de la organización, las
tareas realizadas por cada uno y las interdependencias entre las
mismas.

Evidentemente, efectuar un análisis exhaustivo que identifique todas
las tareas posibles, todas las personas involucradas, y todas las
interdependencias resulta poco práctico. En cambio se optó por
realizar reuniones de trabajo con representantes de los diferentes
equipos para generar un entendimiento en común y definir aspectos
prioritarios a considerar.

Como objetivo se definió trabajar sobre las tareas que involucran
mayor interacción entre los equipos de infraestructura y de
desarrollo. Para obtener un panorama general se realizaron reuniones
con el equipo de infraestructura y los responsables de los
servicios. En estas reuniones se plantearon preguntas como la
siguientes:
%
\begin{itemize}
\item ¿Cuáles son las tareas que realizo con frecuencia?
\item ¿Cuáles de estas tareas considero repetitivas o tediosas?
\item ¿Cuáles acciones que no dependen de mí son necesarias para
  completar estas tareas?
\item Si pudiera pedir cualquier cosa, ¿qué pediría para mejorar esta
  tarea? (Incluso se puede pedir no hacerla más)
\end{itemize}
%
Con estas preguntas, inspiradas en la novela \eng{The Phoenix Project}
\cite{phoenix}, se buscó identificar las tareas más comunes,
improductivas y plausibles de automatización. Se seleccionaron dos
procesos para analizar:
%
\begin{description}[style=nextline]
%
\item[La creación de entornos para un nuevo servicio.]
  Es un proceso que en términos relativos es poco frecuente, pero se
  llegó a la conclusión de que es muy burocrático y requiere muchas
  intervenciones manuales. A pesar de que el equipo de infraestructura
  ha automatizado muchas tareas, resulta común saltarse alguna de ellas,
  lo que obliga a volver atrás en la secuencia para revisar y corregir.
%
\item[Entrega de un nuevo requerimiento en un servicio.]
  Abarca desde la solicitud de una nueva funcionalidad por parte del
  cliente hasta la puesta en producción de la misma. Se trata de un
  proceso a cargo del equipo de desarrollo, aunque a veces requiere la
  intervención del equipo de infraestructura, por ejemplo cuando se
  requieren modificaciones en los permisos, los scripts de
  actualización, o cualquier cambio de infraestructura requerido en la
  nueva versión.
\end{description}
%
Una vez seleccionados estos procesos se realizaron reuniones
adicionales para documentar en detalle las tareas, personas y recursos
necesarios para cada uno de ellos.
%
%
\section{Análisis del flujo de valor}
%
Para analizar los procesos se recurrió a una forma simplificada de los
``mapas de flujo de valor'' (\e{value-stream maps}). En el ámbito de
la producción fabril, el mapa de flujo de valor es una técnica
\e{lean} que tiene como objetivo ``visualizar y comprender el flujo de
materiales e información desde el momento que se realiza el pedido
hasta el producto terminado'' \cite{learningtosee}. En una
organización TI el producto es inmaterial, aunque igualmente se puede
establecer una secuencia de acciones necesarias para cumplir un
objetivo, tal como un cambio en la infraestructura o la entrega de una
funcionalidad de software.

La visualización concisa de la secuencia de tareas permite encontrar
los puntos de demora y las actividades que no contribuyen al objetivo
del proceso (entrega del producto) o bien a la mejora del proceso en
sí mismo (planificación de mejora).

En el análisis realizado se elaboraron tablas (en lugar de un diagrama
gráfico) que enumeran la secuencia de tareas. Para cada tarea, se
especificó una descripción, actor, tiempo de espera, y tiempo de
ejecución.
%
\subsection{Creación de un nuevo servicio}
%
La creación de un nuevo servicio se gestiona utilizando un servicio
interno denominado Pulmo. El responsable del servicio realiza la
solicitud completando los datos que incluyen nombre, descripción,
entornos requeridos, recursos de software y hardware necesarios para
cada entorno, permisos de firewall y permisos de usuario para
administración del servicio.

El sistema genera tickets a resolver por parte del equipo de
infraestructura. Este equipo efectúa las siguientes tareas para cada
uno de los entornos de integración, test y producción:
%
\begin{enumerate}
\item \sbs{Escribir código Ansible}: el código se escribe una sola vez
  a partir de la especificación detallada en el sistema Pulmo y se
  ejecuta en un entorno local. Esta tarea se efectúa una sola vez, el
  código generado es aplicable sobre los entornos de integración, test
  y producción.
\item \sbs{Crear entrada del host en los servicios DNS y DHCP}: se
  lleva a cabo ejecutando un script específico.
\item \sbs{Crear instancia}: se efectúa manualmente mediante la línea de
  comandos en el cluster Ganeti.
\item \sbs{Agregar instancia al servicio de backup} (sólo producción):
  se modifica el código Ansible del servicio de backup y se aplica
  ejecutando el código en la línea de comandos.
\item \sbs{Configurar el servidor de frontend}: se edita el archivo de
  configuración correspondiente y se ejecuta luego un script que
  actualiza la configuración del redirector.
\item \sbs{Configurar monitoreo}: se agrega el host al servicio Zabbix
  invocando un script específico de línea de comandos.
\end{enumerate}
%
La \iflatexml{}Tabla~\ref{tbl:servicio-inicial}\else%
\autoref{tbl:servicio-inicial}\fi{} detalla el análisis del flujo de
valor para la habilitación de un nuevo servicio. Los tiempos de espera
son estimados para tareas con prioridad normal. Los tiempos de
ejecución son estimados a partir de la experiencia de los actores
involucrados.
%
%
\begin{table}[h]
  \tableStyle
  \smaller
  \iflatexml%
  \begin{tabular}{llrr}
  \else%
  \sisetup{
      table-format = 2.1(2),
      table-number-alignment = right,
      separate-uncertainty=true,
      %uncertainty-separator = \,\smaller
  }
  \begin{tabular}{llrr}
  \fi%
  \toprule 
  Actor & Tarea & Espera & Ejecución \\
  \midrule
  Dev & Solicitar servicio en Pulmo & & 15min \\
  Ops & Escribir código Ansible & 1h & 1h-8h \\
  \addlinespace
  \mcol{4}{c}{ \e{entorno local creado} } \\
  \addlinespace
  Ops & Crear entradas en DNS y DHCP & & 10min \\
  Ops & Crear instancias & & 20min \\
  Ops & Crear claves & & 20min \\
  Ops & Ejecutar playbook Ansible & & 30min \\
  Ops & Configurar frontend & & 10min \\
  Ops & Configurar monitoreo & & 20min \\
  \addlinespace
  \mcol{4}{c}{ \e{entornos de integración y test creados} } \\
  \addlinespace
  Dev/Ops & Depuración de la configuración & 30d & 30min-3h \\
  Dev & Solicitud de pase a producción & 1h & 5min \\
  & Crear/verificar hosts (DNS, DHCP) & 1h & 10min \\
  & Crear instancia de producción & & 20min \\
  & Crear claves & & 20min \\
  & Ejecutar playbook Ansible & & 30min \\
  & Configurar servicio de backup & & 15min \\
  & Configurar frontend & & 10min \\
  & Configurar monitoreo & & 20min \\
  \addlinespace
  \mcol{4}{c}{ \e{entorno de producción creado} } \\
  \addlinespace
  \midrule
  Total & & 30d 3h & 3h-15h \\
  \addlinespace
  \mcol{2}{l}{Total (excluyendo espera de pase a producción)} & 3h & 3h-15h \\
  \bottomrule
  \\
  \end{tabular}
  \caption{
    Análisis del flujo de valor para el proceso de creación de un
    nuevo servicio.
  }
  \label{tbl:servicio-inicial}
\end{table}
%
\subsection{Entrega de un nuevo requerimiento}
%
Para el análisis de este proceso se considera que el equipo de
desarrollo se organiza en un ciclo de \e{sprints} de una semana
de duración. El proceso comprende desde el ingreso del requerimiento
hasta la publicación de la nueva versión del servicio en producción.
%
\begin{enumerate}
\item El cliente (un usuario o un miembro del equipo de testing) crea
  un ticket solicitando una nueva funcionalidad o describiendo un
  error en el servicio.
\item Cada semana, el equipo de desarrollo realiza una reunión de
  coordinación que da comienzo a un nuevo \e{sprint}. En esta
  reunión se revisan los requerimientos y se asignan prioridades para
  las tareas a realizar.
\begin{enumerate}
\item Se desagregan los requerimientos complejos en tareas
  específicas, creando tickets correspondientes para cada una.
\item Se crea en el repositorio SVN una rama de trabajo para la nueva
  versión.
\end{enumerate}
\item Durante el \e{sprint} se continúa el trabajo normal
  escribiendo el código que resuelve cada tarea. El desarrollador
  ejecuta tests unitarios en forma manual durante el proceso de
  desarrollo.
\item Una vez concluido el \e{sprint}, se ejecutan tests
  unitarios y de integración sobre el código y se publica la nueva
  versión, dando lugar a un nuevo \e{sprint} para la versión
  siguiente. Este proceso se denomina internamente el \e{cierre
    de versión}.
\item El responsable del proyecto actualiza la nueva versión en el
  entorno de integración y solicita a los responsables de los demás
  servicios que validen que la interacción con otros sistemas funciona
  correctamente.
\item Una vez confirmadas las integraciones se actualiza la versión en
  el entorno de test, y se solicita al solicitante que valide que la
  nueva funcionalidad es correcta.
\item Cuando el usuario confirma la nueva funcionalidad, el
  responsable del proyecto actualiza el entorno de producción.
\end{enumerate}
%
En la \iflatexml{}Tabla~\ref{tbl:entrega-inicial}\else%
\autoref{tbl:entrega-inicial}\fi{} se describe el análisis del flujo
de valor para este proceso. Los tiempos de espera estipulados se
refieren a tareas con prioridad normal. Los tiempos de ejecución se
estimaron según la experiencia de los desarrolladores.
%
%
\begin{table}[h]
  \tableStyle
  \smaller
  \iflatexml%
  \begin{tabular}{llrr}
  \else%
  \sisetup{
      table-format = 2.1(2),
      table-number-alignment = right,
      separate-uncertainty=true,
      %uncertainty-separator = \,\smaller
  }
  \begin{tabular}{llrr}
  \fi%
  \toprule 
  Actor & Tarea & Espera & Ejecución \\
  \midrule
  Cliente & Ingresa requerimiento & & \\
  Dev & Reunión del sprint & 0-5d & 30min \\
  Dev & Crear rama de trabajo & & 5min \\
  Dev & Codificar solución & 1h-5d & 2h-10h \\
  Dev & Cierre de versión (testing, publicación) & 5d & 40min \\
  Dev & Despliegue en integración & & 10min \\
  Dev & Confirmación de integraciones & 2h & 1h \\
  Dev & Despliegue en test & & 10min \\
  Cliente & Test de aceptación & 1h & 20min \\
  Dev & Despliegue en producción & 1h & 20min \\
  \addlinespace\midrule
  Total & & 6d - 16d & 5h - 13h \\
  \bottomrule
  \\
  \end{tabular}
  \caption{
    Análisis del flujo de valor para el proceso de entrega de un nuevo
    requerimiento.
  }
  \label{tbl:entrega-inicial}
\end{table}
%
%
\section{Propuestas de mejora}
%
Las propuestas de mejora de ambos procesos se realizaron generando en
primer lugar una forma simple de ``mapa de estado futuro''. Estos mapas
de estado futuro se basan en la secuencia de tareas identificadas
previamente, las cuales son modificadas según el estado que se desea
alcanzar. El ``estado futuro'' se determinó en base a los requerimientos
organizacionales y al consenso alcanzado en las reuniones de trabajo.

Las propuestas de cambio se basaron en los siguientes principios:
%
\begin{itemize}
\item Utilizar el repositorio de código GitLab para el código fuente y
  la configuración de cada servicio.
\item Escribir la configuración de la infraestructura como código
  (Ansible).
\item Contar con herramientas/servicios (web o de otra naturaleza) que
  permitan efectuar operaciones de manera autónoma por parte de los
  equipos de desarrollo e infraestructura, sin necesidad de generar
  solicitudes en el sistema de tickets.
\item Automatizar las operaciones tanto como sea posible.
\item Contar con un sistema de integración y entrega continua para
  realizar las tareas de compilación, pruebas de humo, testing
  unitario y de integración de manera automática, sin requerir la
  intervención de la persona que escribe el código.
\item Todos los cambios introducidos tienen como objetivo último
  acelerar la entrega del software al usuario final tanto como sea
  posible.
\end{itemize}
%
\subsection{Creación de un nuevo servicio}
%
Siguiendo los principios previamente especificados se propone
modificar el proceso de creación de un nuevo servicio del siguiente
modo:
%
\begin{enumerate}
\item El responsable del servicio crea un repositorio con el código
  Ansible necesario para la configuración del servicio. Esta
  configuración es directamente aplicable al entorno local para
  desarrollo.
\begin{itemize}
\item \e{El responsable debe tener permisos suficientes para
  crear proyectos en GitLab.}
\item \e{El equipo de infraestructura debe publicar el código
  reutilizable que aplica actualmente en sus ``playbooks''.}
\end{itemize}
\item El responsable del servicio genera/verifica las claves
  necesarias (de base de datos, de conexión a otros servicios) para
  ser utilizadas en los entornos de integración, test y producción.
\item El responsable del servicio inicia un proceso de revisión de la
  configuración con el equipo de infraestructura creando un
  \e{merge request} en el repositorio de código.
\begin{itemize}
\item \e{Esta instancia reemplaza la creación de tickets en
  Redmine.}
\end{itemize}
\item Como parte de la resolución del merge request el equipo de
  infraestructura agrega el nuevo servicio a la interfaz web de
  Ansible.
\begin{itemize}
\item \e{Se debe implementar la interfaz web para Ansible.}
\end{itemize}
\item Una vez finalizada la revisión se cierra el \e{merge
  request} y el responsable del servicio lanza la tarea de
  configuración desde la interfaz web de Ansible. Esta tarea efectúa
  de forma automática las siguientes tareas:
\begin{itemize}
\item Creación de entradas DNS y DHCP
\item Creación de instancia
\item Agregado al servicio de backup (si fuera necesario)
\item Configuración del frontend
\item Configuración del servicio de monitoreo
\end{itemize}
\e{Para dar soporte a esta automatización se deberán modificar
  varios aspectos de la infraestructura.}
\end{enumerate}
%
El análisis del estado futuro se presenta en la \iflatexml{}%
Tabla~\ref{tbl:servicio-futuro}\else\autoref{tbl:servicio-futuro}\fi.
%
\begin{table}[h]
  \tableStyle
  \smaller
  \iflatexml%
  \begin{tabular}{llrr}
  \else%
  \sisetup{
      table-format = 2.1(2),
      table-number-alignment = right,
      separate-uncertainty=true,
      %uncertainty-separator = \,\smaller
  }
  \begin{tabular}{llrr}
  \fi%
  \toprule
  Actor & Tarea & Espera & Ejecución \\
  \midrule
  Dev/Ops & Crear receta de Ansible & & 1h - 8h \\
  \addlinespace
  \mcol{4}{c}{ \e{entorno local creado} } \\
  \addlinespace
  Dev/Ops & Crear claves & & 20min \\
  Dev/Ops & Merge request + depuración & 1h & 1h - 3h \\
  Ops/Dev & Agregar proyecto a interfaz web Ansible & & 20min \\
  Dev/Ops & Ejecutar playbook Ansible & & 30min \\
  \addlinespace
  \mcol{4}{c}{\e{entornos de integración, test y producción creados}} \\
  \addlinespace\midrule
  Total & & 1h & 3h-12h \\
  \bottomrule
  \\
  \end{tabular}
  \caption{
    Análisis del estado futuro para la creación de un nuevo servicio.
  }
  \label{tbl:servicio-futuro}
\end{table}
%
%
\subsection{Entrega de un nuevo requerimiento}
%
Para el proceso de desarrollo y entrega de un nuevo requerimiento se
propuso modificar la secuencia de tareas según se enumera a
continuación. Esta propuesta es genérica y la implementación se
adaptará a las particularidades de cada servicio.
%
\begin{enumerate}
\item A partir del requerimiento del cliente, el desarrollador crea un
  ``issue'' para el seguimiento de la resolución del caso.
\begin{itemize}
\item \e{Si la tarea es compleja, deberá separarla en varios
  ``issues'' con complejidad menor.}
\item \e{Las reuniones de los sprints pasan a ser reuniones de
  revisión de las tareas efectuadas antes que de planificación de las
  tareas a realizar.}
\end{itemize}
\item El desarrollador crea un merge request mediante un solo click en
  el issue de GitLab. En la rama respectiva codifica la nueva
  funcionalidad requerida.
\begin{itemize}
\item \e{El servicio de integración continua deberá efectuar
  testing unitario y de integración de manera automática sobre el
  código perteneciente a esta rama.}
\end{itemize}
\item Cuando se considera resuelto el requerimiento, se elimina el
  estado \e{``WIP''} (``en curso'') del merge request en GitLab y se
  asigna el a otro miembro del grupo de desarrollo para revisión.
\item Se revisa en equipo la solución codificada y se asegura que los
  tests unitarios y de integración hayan sido exitosos. En caso de
  encontrarse problemas, se sigue escribiendo código sobre la misma
  rama hasta que hayan sido solucionados.
\item Se cierra el \e{merge request} incorporando el código a la
  rama principal.
\begin{itemize}
\item \e{Esta acción cierra de forma automática el issue, dispara
  nuevos tests automáticos sobre la rama de desarrollo principal, y
  genera y publica un artefacto de versión.}
\end{itemize}
\item Se despliega la nueva versión en el entorno de test y se
  solicita al cliente que verifique la resolución del problema.
\item El cliente verifica la funcionalidad y confirma la resolución
  del problema.
\item Se despliega la nueva versión en el entorno de producción y se
  resuelve el ticket del cliente.
\end{enumerate}
%
El análisis del estado futuro se describe en la
\iflatexml{}Tabla~\ref{tbl:entrega-futuro}\else\autoref{tbl:entrega-futuro}\fi.
%
%
\begin{table}[h]
  \tableStyle
  \smaller
  \iflatexml%
  \begin{tabular}{llrr}
  \else%
  \sisetup{
      table-format = 2.1(2),
      table-number-alignment = right,
      separate-uncertainty=true,
      %uncertainty-separator = \,\smaller
  }
  \begin{tabular}{llrr}
  \fi%
  \toprule
  Actor & Tarea & Espera & Ejecución \\
  \midrule
  Cliente & Ingresar requerimiento & & \\
  Dev & Crear issue en GitLab + merge request & 1d & 5min \\
  Dev & Codificar solución & 1h-5d & 2h-10h \\
  \addlinespace
  \mcol{4}{c}{\e{de forma automática: tests unitarios y de integración}} \\
  \addlinespace
  Dev & Resolver WIP y solicitar revisión & 15min & 10min \\
  Dev & Revisión de la solución & 1h & 20min \\
  Dev & Efectuar merge & 5min & 5min \\
  \addlinespace
  \mcol{4}{c}{\e{de forma automática: cierre de issue, generación de artefacto}} \\
  \addlinespace
  Dev & Despliegue en test & 15min & 20min \\
  Cliente & Test de aceptación & 1h & 15min \\
  Dev & Despliegue en producción & 1h & 10min \\
  Dev & Resolución del ticket & & 10min \\
  \addlinespace\midrule
  Total & & 2d - 6d & 3h - 11h \\
  \bottomrule
  \\
  \end{tabular}
  \caption{
    Análisis del estado futuro para el proceso de entrega de un nuevo
    requerimiento.
  }
  \label{tbl:entrega-futuro}
\end{table}
%
%
section{Plan de trabajo}
%
El plan detallado en la \iflatexml{}Tabla~\ref{tbl:plan-trabajo}%
\else{}\autoref{tbl:plan-trabajo}\fi{} enumera las acciones requeridas
y sus objetivos correspondientes (medibles) para alcanzar el ``estado
futuro'' de ambas propuestas de mejora. Tal como los análisis de flujo
de valor, este plan es una versión simplificada del formato propuesto
por \citeauthor{learningtosee} en \eng{Learning to See}
\cite{learningtosee}.

Como se podrá apreciar, la lista de acciones propuesta se encuentra en
línea con las acciones llevadas a cabo durante el desarrollo del
Proyecto. Las implementaciones concretas se describen en los capítulos
de aquí en adelante.
%
%
\begin{table}[h]
  \tableStyle
  \smaller
  \iflatexml%
  \begin{tabular}{p{.48\textwidth}p{.48\textwidth}}
  \else%
  \sisetup{
      table-format = 2.1(2),
      table-number-alignment = right,
      separate-uncertainty=true,
      %uncertainty-separator = \,\smaller
  }
  \begin{tabular}{p{.45\textwidth}p{.45\textwidth}}
  \fi%
  \toprule
  Acción & Objetivo \\
  \midrule
  \mrow{2}{*}{Configurar permisos en GitLab}
  &
  Los desarrolladores acceden a los repositorios con código Ansible.
  \\  \addlinespace
  &
  Los desarrolladores pueden crear proyectos para nuevos
  servicios.
  \\  \addlinespace
  Instalar y configurar una interfaz web para
  Ansible
  &
  Los desarrolladores pueden aprovisionar servicios.
  \\  \addlinespace
  Automatizar proceso de alta de entradas en DNS y DHCP
  & No se requiere intervención manual para dar de alta un host.
  \\  \addlinespace
  Automatizar la creación de instancias virtuales.
  &
  No se requiere intervención manual para crear una instancia virtual.
  \\  \addlinespace
  Automatizar el agregado de un servicio al backup.
  &
  No se requiere intervención manual para agregar un servicio al backup.
  \\  \addlinespace
  Automatizar el agregado de monitoreo a un servicio.
  &
  No se requiere intervención manual para agregar monitoreo a un servicio.
  \\  \addlinespace
  Automatizar la configuración de los frontends.
  &
  No se requiere intervención manual para configurar un servicio en los frontends.
  \\  \addlinespace
  Escribir un documento que especifique el método de trabajo utilizando ramas de
  vida corta e integración continua.
  &
  Se cuenta con un documento de referencia para el trabajo con el código.
  \\  \addlinespace
  Configurar un sistema de integración y entrega continua.
  &
  El servicio está disponible para ser utilizado.
  \\  \addlinespace
  Configurar una consola de operaciones
  &
  El servicio está disponible.
  \\  \addlinespace
  \bottomrule
  \\
  \end{tabular}
  \caption{
    Plan de trabajo para alcanzar el estado futuro de
    las propuestas de mejora surgidas del análisis del flujo de valor.
  }
  \label{tbl:plan-trabajo}
\end{table}
%
%
