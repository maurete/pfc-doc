La forma clásica de identificación de nuevos \premirna{s} son las
pruebas experimentales de laboratorio.
Sin embargo, este tipo de pruebas sólo pueden identificar con
fiabilidad aquellos \premirna{s} abundantes dentro de la célula.
Las técnicas computacionales son una alternativa a las pruebas de
laboratorio que superan estas dificultades, ya que al extraer la
información directamente a partir del genoma de cada organismo,
permiten encontrar aquellos \premirna{s} que son específicos a
determinados tipos de tejidos o estadíos de desarrollo celular, y
aquellos escasamente expresados \cite{ding,sheng,xu}.
