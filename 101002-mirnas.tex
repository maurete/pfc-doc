La identificación de pre-miRNAs es el paso previo a la identificación
de miRNAs maduros. La primera opción para la identificación de nuevos
pre-miRNAs son las pruebas experimentales de laboratorio. Sin embargo,
sólo aquellos pre-miRNAs abundantes pueden ser detectados mediante
esta técnica de forma fiable. Esto implica que los pre-miRNAs con un
bajo nivel de expresión, que se expresan en tejidos específicos y/o
que se presentan sólo en determinados estadíos de desarrollo celular
pueden ser fácilmente ignorados mediante la técnica experimental
\cite{ding}\cite{xu}. En pos de superar estas dificultades propias del
método experimental es que surgen técnicas computacionales para
encontrar aquellos pre-miRNAs que son específicos a determinados tipos
de tejidos o estadíos de desarrollo celular, y aquellos escasamente
expresados \cite{sheng}\cite{xu}.