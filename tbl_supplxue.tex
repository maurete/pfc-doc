\begin{table}[h]
  \tableStyle
  %% \sisetup{
  %%   table-format = 2.1,
  %%   table-number-alignment = right,
  %% }
  %% \begin{tabular}{lrrrr}\toprule
  %%   Conj. datos & N. Elem. & Método RMB & Err. Emp. & Xue et al. \\\midrule
  %%   updated           & 39 &   \s{94,9\%} & \s{94,9\%} & 92,3\% \\ 
  %%   cross-species     & 581 &  79,2\% & \s{96,0\%} & 90,9\% \\
  %%   conserved-hairpin & 2444 & \s{95,7\%} & 91,7\% & 89,0\% \\\bottomrule
  %% \end{tabular}
  \sisetup{
    table-format = 2.1(2),
    table-number-alignment = right,
    uncertainty-separator = \,\smaller
  }
  \begin{tabular}{lS[table-format=2.0]
      S[table-format=4.0]SSSSS[table-format=2.1]}
    \toprule
    {Problema} & {Clase} & {Elems.} &
    {MLP-B}    & {SVM-LE}   & {SVM-RE}   & {SVM-RR}   & \cite{xue}\\
    \midrule
    updated           & +1 &   39 &
    94.4(11) & 94.9(00) & 93.3(14) & 94.9(00) & 92.3 \\
    cross-species     & +1 &  581 &
    90.7(30) & 96.0(01) & 92.3(33) & 79.2(00) & 90.9 \\
    conserved-hairpin & -1 & 2444 &
    91.8(31) & 91.6(00) & 93.4(02) & 95.7(00) & 89.0 \\
    \bottomrule
    \\
  \end{tabular}
  \caption{\captionStyle Tasa de clasificación obtenida para los
    problemas de prueba adicionales basados en los conjuntos de prueba
    del método \sbs{Triplet-SVM} \cite{xue}. En la última columna, la
    tasa de clasificación obtenida por los autores.}
  \label{tbl:suppl-xue}

\end{table}