%
%
%
\section{Perceptrón multicapa}
%
El perceptrón multicapa (\eng{MultiLayer Perceptron}, MLP)
\cite{mlp2,mlp1} es un tipo de red neuronal artificial que se utiliza
como máquina de aprendizaje de propósito general.

Las redes neuronales artificiales (\eng{ANNs}) son algoritmos de
cómputo paralelo basados en una arquitectura de \e{unidades}
interconectadas denominadas \e{neuronas}.
Una red neuronal se representa como un grafo orientado, donde los
nodos son las neuronas y las aristas conectan las salidas de una
neurona con la entrada de otra, asociándole a cada conexión una
ponderación o peso.
El perceptrón multicapa combina una topología \e{acíclica} o \e{con
  propagación hacia adelante} con un algoritmo de entrenamiento basado
en la \e{retropropagación}, que ajusta los \e{pesos} de las conexiones
entre neuronas reduciendo el error en la salida.

A continuación, se presenta una descripción del perceptrón multicapa,
comenzando por una reseña de su inspiración biológica y continuando
con una descripción del modelo computacional de una neurona.
Más adelante se describe la arquitectura de la red, y finalmente se
explican diferentes algoritmos de entrenamiento mediante
retropropagación.
