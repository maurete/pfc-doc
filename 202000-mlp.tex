%
%
%
\section{Perceptrón multicapa}
%
El perceptrón multicapa (\eng{Multi-Layer Perceptron}, \e{MLP})
\cite{mlp2,mlp1} es un tipo de red neuronal artificial que se utiliza
como máquina de aprendizaje de propósito general.
Las redes neuronales artificiales son algoritmos de cómputo paralelo
basados en una arquitectura de \e{unidades} interconectadas
denominadas \e{neuronas}.
Una red neuronal se representa como un grafo orientado, donde los
nodos son las neuronas y las aristas conectan las salidas de una
neurona con la entrada de otra, asociándole a cada conexión una
\e{ponderación} o \e{peso}.
El perceptrón multicapa combina una topología \e{acíclica} o \e{con
  propagación hacia adelante} con un algoritmo de entrenamiento basado
en la \e{retropropagación}, encargado de ajustar los pesos de las
conexiones entre neuronas con el objetivo de reducir el error en la
salida.

A continuación, se presenta una descripción del perceptrón multicapa,
comenzando por una reseña de su inspiración biológica y continuando
con una descripción del modelo computacional de una neurona.
Más adelante, se describe la arquitectura de la red, y finalmente se
explica el proceso de entrenamiento mediante retropropagación.
