\section{Pruebas adicionales}
Se efecuaron una serie de pruebas adicionales con el objetivo de
evaluar el comportamiento del método propuesto al clasificar conjuntos
de prueba de naturaleza diferente al conjunto utilizado para
entrenamiento.

En primer lugar, se probaron los conjuntos de prueba suplementarios
presentados en los trabajos \cite{xue,ng}.  Por otra parte, se
clasificaron los elementos de la especie humana de la última versión
disponible de miRBase (21.0, FECHA), entrenando con los conjuntos de
entrenamiento de \tripletsvm{}, \mipred{} y \micropred{}.
Finalmente, se generó el problema \deltamirbase{}, el cual entrena con
la versión 20 de miRBase y en su conjunto de prueba contiene
únicamente aquellos pre-miRNAs incorporados en la versión 21.

En todos los casos, se utilizó el conjunto de características S-E,
aplicando las siguientes combinaciones de clasificador/estrategia de
selección de hiperparámetros

\begin{itemize}
\item \sbs{MLP-B}: Clasificador MLP, estrategia de búsqueda
  exhaustiva,
\item \sbs{SVM-LE}: Clasificador SVM, núcleo lineal, estrategia de
  selección de hiperparámetros mediante el criterio del error
  empírico,
\item \sbs{SVM-RE}: Clasificador SVM, núcleo RBF, selección de
  hiperparámetros mediante error empírico,
\item \sbs{SVM-RR}: Clasificador SVM con núcleo RBF y estrategia de
  selección de hiperparámetros RMB.
\end{itemize}

En adelante, se describe cada uno de esos experimentos de
clasificación y se presentan los resultados obtenidos.

\subsection{Conjuntos de prueba de Triplet-SVM}
En el trabajo de \citeauthor{xue} \cite{xue} los autores definen tres
conjuntos de prueba suplementarios para el método Triplet-SVM
denominados \ssf{UPDATED}, \ssf{CONSERVED-HAIRPIN} y
\ssf{CROSS-SPECIES}.
\paragraph{UPDATED}
Consiste en 39 pre-miRNAs experimentalmente validados pertenecientes a
la especie humana, que fueron publicados durante el desarrollo del
método Triplet-SVM.
\paragraph{CONSERVED-HAIRPIN}
Este conjunto de prueba consta de 2444 secuencias conservadas con
estructura secundaria de horquilla, extraídas del cromosoma humano
19. En su mayoría, estas secuencias son pseudo pre-miRNAs (clase
negativa), aunque se conoce que algunas de ellas se corresponden con
pre-miRNAs verdaderos.
\paragraph{Cross-species}
Contiene 581 secuencias de pre-miRNAs experimentalmente validadas
pertenecientes a diversas especies no humanas, tales como animales,
plantas y virus, extraídas de la versión 5.0 de miRBase.

En las pruebas efectuadas, se entrenó con el mismo conjunto de
entrenamiento del problema \tripletsvm{}, y con el modelo obtenido se
clasificaron los tres conjuntos de prueba. Los resultados de
clasificación de presentan en la \autoref{tbl:suppl-xue}.