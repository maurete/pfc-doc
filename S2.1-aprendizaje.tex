%
%
%
\section{La clasificación como forma de aprendizaje supervisado}
%
En el contexto de la Inteligencia Computacional, la clasificación es
un tipo de \e{aprendizaje supervisado} cuyo objetivo es
{inferir} una función que relaciona \e{datos de entrada} con las
respectivas \e{salidas deseadas}, a partir de una serie de ejemplos que
conforman el llamado \e{conjunto de entrenamiento}. Un algoritmo que
implementa la clasificación se conoce como un \e{clasificador}.
La función que transforma los datos de entrada a una categoría
(clase) se llama \e{modelo} (o {hipótesis}).

Un clasificador es un tipo de \e{máquina de aprendizaje}, y consiste
en un algoritmo que analiza los datos del conjunto de entrenamiento
y produce una función de correspondencia (el modelo) que puede ser
usada para relacionar nuevos ejemplos (datos de entrada) a una clase
(valor de salida).
De este modo, el clasificador es capaz de replicar el comportamiento
del sistema o proceso que pretende modelar, y se dice que el modelo
obtenido es capaz de \e{generalizar} a partir de los datos de
entrenamiento.

A continuación, se presentan definiciones de los conceptos
relevantes del aprendizaje supervisado. La discusión se
basa en el
trabajo de \citeauthor{glasmachers} \cite{glasmachers}.
