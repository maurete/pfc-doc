%
%
%
\subsection{Partición de los datos}
%
La partición de los datos implementa el método de retención, separando
los datos de entrada en un conjunto \e{de entrenamiento} y otro
\e{de prueba}, a partir de la especificación del usuario al momento de
la carga de los archivos.

Internamente, los conjuntos de datos de entrenamiento y prueba se
guardan como matrices $M^E$, $M^P$, de 70 columnas y $\ell^E$ y
$\ell^P$ filas, respectivamente.
Las primeras 66 columnas representan los vectores de características
normalizados de cada ejemplo. La columna 67 contiene la clase del
ejemplo cuando es conocida. Las restantes 3 columnas contienen índices
de referencia, para lograr trazabilidad de los ejemplos.

Para cada conjunto de datos $D_n$ de $\ell_n$ elementos,
correspondiente a un archivo leído en la entrada, se arma una matriz
provisoria con los vectores de \caract{s} correspondientes a cada
ejemplo y se agrega la información de clase y los índices en las
últimas 4 columnas. Luego, se intercambian las filas en forma
pseudoaleatoria, para finalmente seleccionar las primeras
$\ell_n\cdot{}p_n^E$ filas de la matriz resultante, agregándolas al
conjunto de entrenamiento, y las $\ell_n\cdot{}p_n^P$ filas
subsiguientes al conjunto de prueba. Los valores
$0\leq{}p_n^E,p_n^P\leq1$ son especificados por el usuario e indican
la proporción de elementos del conjunto a utilizar para entrenamiento
y prueba, respectivamente.

Una vez incorporados todos los archivos en los conjuntos de
entrenamiento y prueba, se permutan sus filas en forma
pseudoaleatoria. De este modo, quedan conformados el conjunto de
entrenamiento, utilizado en la obtención del modelo, y el conjunto de
prueba, utilizado para clasificación de nuevos ejemplos.

%% Una vez efectuadas la extracción de características y la
%% normalización, los archivos de entrada originales se han convertido en
%% conjuntos $D_1,\ldots,D_N$ que contienen los vectores de
%% características normalizados y la información de clase para cada
%% ejemplo. La partición de los datos refiere al armado de los conjuntos
%% de entrenamiento $D^E$ y prueba $D^P$ con los elementos de estos
%% conjuntos a partir de la especificación del usuaio.

%% El conjunto $D_n=((\xx_i,y_i)),\,i=1,\ldots,\ell_n$ se corresponde con
%% el $n$-ésimo archivo de entrada. El objetivo es armar los conjuntos de
%% entrenamiento $D^E$ y de prueba $D^P$ con elementos de los
%% $D_n,\,n=1,\ldots,N$.  Para cada conjunto $D_n$, el usuario especifica
%% una proporción $p_n^E$ de elementos del conjunto $D_n$ a utilizar como
%% entrenamiento, y otra proporción $p_n^P$ de elementos de $D_n$ a
%% utilizar para prueba. Los valores de $p_n^E,p_n^P$ son tales que
%% %
%% \begin{align}
%%   0\ \leq\ p_n^E, p_n^P\ \leq\  p_n^E + p_n^P\ \leq\ 1,
%% \end{align}
%% %
%% esto es, $p_n^E$ y $p_n^P$ son números enre 0 y 1 cuya suma es menor o
%% igual a 1.

%% El primer paso consiste en permutar aleatoriamente el orden de los
%% elementos de los conjuntos de entrada $D_n$
%% %
%% \begin{align}
%%   D^*_n = ((\xx_j,y_j)), \quad j = \sigma(i,\ell_n,s), \quad i=1,\ldots,\ell_n,
%% \end{align}
%% %
%% donde $\sigma$ es una función de permutación pseudoaleatoria que
%% requiere la especificación de una \e{semilla} $s$. Seguidamente se
%% particiona cada conjunto $D_n$ en sub-conjuntos de entrenamiento
%% y prueba
%% %
%% \begin{align}
%%   D^E_n &= ((\xx_u,y_u)\in D^*_n),& u&=1,\ldots,\ell^E_n, \\
%%   D^P_n &= ((\xx_v,y_v)\in D^*_n),& v&=1,\ldots,\ell^P_n,
%% \end{align}
%% %
%% donde
%% %
%% \begin{align}
%%   \begin{split}
%%     \ell^E_n &= \left\lfloor\ell_n\cdot p_n^E \right\rceil, \\
%%     \ell^P_n &= \lfloor\ell_n\cdot p_n^P\rceil.
%%   \end{split}
%% \end{align}
%% %
%% El operador $\lfloor\cdot\rceil$ es el operador de redondeo.
%% Finalmente, se ensamblan los conjuntos de entrenamiento y prueba según
%% %
%% \begin{align}
%%   D^E &= \left( (\xx_u,y_u) \in \bigcup_n {D}^E_n \right), &
%%   u &= \sigma(i,\ell^E,s), &
%%   i &= 1,\ldots,\ell^E, &
%%   \ell^E &= \left| D^E \right| = \sum_n \ell^E_n ,\\
%%   D^P &= \left( (\xx_v,y_v) \in \bigcup_n {D}^P_n \right), &
%%   v &= \sigma(j,\ell^P,s), &
%%   j &= 1,\ldots,\ell^P, &
%%   \ell^P &= \left| D^P \right| = \sum_n \ell^P_n .
%% \end{align}
%% %
%% Esto es, los conjuntos de entrenamiento $D^E$ y prueba $D^P$ se arman
%% concatenando los respectivos $D^E_n$ y $D^P_n$ y luego permutando
%% aleatoriamente el orden de los ejemplos en los conjuntos resultantes.
