%
%
%
\subsection{Particionado de los datos}
%
El particionado separa los datos leídos en conjuntos de entrenamiento,
utilizado para la generación del modelo del clasificador, y de prueba,
que contiene los ejemplos a usar para clasificación.
La composición de estos conjuntos viene determinada por la
especificación del usuario, que debe indicar, para cada archivo, el
número de elementos a utilizar para entrenamiento y para prueba.
Cuando el archivo se utiliza para entrenamiento, se debe indicar
adicionalmente la clase de los ejemplos que contiene.

El primer paso es ``etiquetar'' los vectores normalizados, extendiéndolos
con números de referencia que indican el archivo de origen, la
posición del dato original dentro del archivo, y la clase del ejemplo
en cuestión.
El segundo paso armar los conjuntos de entrenamiento y prueba como
matrices, cuyas filas son vectores ``etiquetados'' seleccionados al
azar de cada archivo, respetando las proporciones indicadas por el
usuario.
Una vez incorporados todos los ejemplos a los conjuntos de
entrenamiento y prueba, se permutan las filas de ambas matrices en
forma pseudoaleatoria.

La partición de los datos se efectúa de forma tal que los conjuntos de
entrenamiento y prueba resultantes sean disjuntos: si un ejemplo se
incorpora al conjunto de prueba, no será parte del conjunto de
entrenamiento, y viceversa.
