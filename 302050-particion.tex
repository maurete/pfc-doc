%
%
%
\subsection{Particionado de los datos}
%
El particionado separa los datos en dos conjuntos disjuntos de
entrenamiento y de prueba.
Los datos del conjunto de entrenamiento se utilizan para la generación
del modelo del clasificador, mientras que el conjunto de prueba
contiene los ejemplos a usar para clasificación.
Para cada archivo de entrada, el usuario debe indicar el número de
ejemplos que desea incorporar a cada conjunto de datos.
Esta especificación determina la composición de los conjuntos de
entrenamiento y de prueba.
Cuando un archivo se utiliza para entrenamiento, se debe indicar
adicionalmente la clase de los ejemplos que contiene.

El primer paso es ``etiquetar'' los vectores normalizados,
extendiéndolos con números de referencia que indican el archivo de
origen, la posición del dato original dentro del archivo, y la clase
del ejemplo en cuestión.
El segundo paso armar los conjuntos de entrenamiento y prueba como
matrices, cuyas filas son vectores ``etiquetados'' seleccionados al
azar de cada archivo, respetando las proporciones indicadas por el
usuario.
Una vez incorporados todos los ejemplos a los conjuntos de
entrenamiento y prueba, se permutan las filas de ambas matrices en
forma pseudoaleatoria.
