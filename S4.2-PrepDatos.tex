\section{Preparación de los conjuntos de datos}
Con los vectores de características calculados, el siguiente paso es
armar una matriz que representa un conjunto de datos.  Se codificaron
funciones que permiten el ``post-procesamiento'' de estas matrices de
datos: \func{scale\_data}, \func{scale\_sym}: permiten escalar el
conjunto de datos de modo que todas las características recaigan
dentro del mismo rango.

\funcheader{scale\_data, scale\_sym}{Matriz de características, datos
  de escalado (opcional)} {Matriz de YYYYYY filas con 4
características}
Las funciones \func{scale\_data} y \func{scale\_sym} presentan dos
modos de funcionamiento: si recibe como argumento únicamente la matriz
de datos, lleva el rango de cada variable (columna) a un intervalo
preestablecido ($[0,1]$ en el caso de \func{scale\_data}, $[0,1]$ en
el caso de \func{scale\_sym}) y retorna además de la matriz
``normalizada'' y la información de escalado aplicada (desfasaje y
factor).  Cuando recibe información de escalado junto a la matriz de
datos, directamente aplica dicha información a la matriz.  La lógica
de este funcionamiento tiene dos implicancias:

\begin{enumerate}
\item las variables columnas de la matriz son consideradas
  independientes entre sí
\item la normalización debe ser la misma para todos los conjuntos de
  datos, manteniendo las diferencias relativas entre distintos
  conjuntos
\end{enumerate}

\funcheader{balance\_dataset}{Matriz con conjunto de datos de
  entrenamiento completo} {Conjunto de datos sobremuestreado evitando
el desbalance de clases}
La función \func{balance\_dataset} efectúa sobremuestreo sobre la
clase minoritaria repitiendo elementos seleccionados
aleatoriamente. De este modo se evita el desbalance de clases que
pudiera provocar un \e{sesgo} en el clasificador MLP.  Debe evitarse
su uso en clasificadores SVM, ya que la repetición de ejemplos deriva
en problemas de optimización mal condicionados.

\funcheader{stpart}{Número de elementos, Número de particiones,
  Proporción validación/total}{Índices para particiones de
validación cruzada}
La función \func{stpart} Genera conjuntos de índices los cuales, al
aplicarse sobre el conjunto de entrenamiento, generan particiones de
estimación y validación según la XXXXXXXX de validación cruzada.