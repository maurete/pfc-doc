%

A partir de los resultados que se muestran en la
\iflatexml{}Tabla~\ref{tbl:rbf-results}\else\autoref{tbl:rbf-results}\fi{},
se efectuaron las siguientes observaciones:
%
%
\begin{itemize}
\item
  \e{Estrategias de selección de \hparam{s}}.
  La estrategia de selección de \hparam{s} mediante minimización de la
  cota radio-margen obtuvo muy buenos resultados, especialmente cuando
  se considera su bajo coste computacional (analizado más adelante).
  Esta estrategia presentó divergencia al utilizar el conjunto de
  \caract{s} de secuencia \dset{S} para el problema \prob\mipred{}.
  La estrategia del criterio del error empírico presentó una leve
  ventaja respecto a la misma estrategia aplicada con una SVM de
  núcleo lineal.
  A diferencia de lo ocurrido con el clasificador SVM-lineal, la
  estrategia de selección trivial obtuvo resultados inferiores a las
  demás estrategias.
\item
  \e{Conjuntos de \caract{s}}.
  La utilización del conjunto de \caract{s} \dset{S-E} resultó en las
  mayores tasas de clasificación, a excepción del prolema
  \prob\mipred{}, para el cual los mejores resultados variaron entre
  los conjuntos \dset{E} y \dset{S-E}, según la estrategia considerada.
\item
  \e{Variabilidad}.
  Tal como el caso de la estrategia trivial, la minimización de la
  cota radio-margen no efetúa validación cruzada, resultando en una
  desviación estándar nula para el caso del problema
  \prob\tripletsvm{}.
\item
  \e{Problema \tripletsvm{}}.
  Para este problema se obtuvo una elevada \SE{}, en especial para el
  caonjunto de \caract{s} \dset{S-E} y para la estrategia RMB.
\item
  \e{Problema \mipred{}}.
  Para este problema, los resultados resultaron generalmente
  inferiores a los obtenidos con el clasificador SVM-lineal.
\item
  \e{Problema \micropred{}}.
  Para este problema se observó una mejora sensible en los resultados
  obtenidos respecto de los demás clasificadores.
\item
  \e{Desbalance de clases}.
  El efecto del desbalance de clases en los problemas de los problemas
  \prob\mipred{} y \prob\micropred{} resultó similar al observado en
  el clasificador SVM con núcleo lineal, con una \SP{} superior a la
  \SE{} en todos los casos.
\item
  \e{Comparación con los otros clasificadores}.
  Mientras que para el problema \prob\tripletsvm{} las tasas de
  clasificación superaron las obtenidas con el núcleo lineal, para los
  problemas \prob\mipred{} y \prob\micropred{} los resultados variaron
  según el caso considerado.
  Al comparar con el clasificador MLP, puede observarse que este
  clasificador presenta mejores resultados, en particular para los
  problemas más complejos.
\end{itemize}
%
