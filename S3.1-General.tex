%
%
%
\section{Descripción general}
%
La arquitectura del sistema se organiza en los tres componentes
principales de preprocesamiento, generación del modelo del
clasificador y clasificación. Desde el punto de vista del usuario, la
funcionalidad del software se describe más simplemente en torno a las
tareas de generación del modelo y de clasificación. El
preprocesamiento de los datos puede verse como una tarea implícita
invocada por los otros componentes.

Para la generación del modelo del clasificador, las entradas al
sistema son archivos con secuencias para efectuar el entrenamiento.
Estos archivos vienen dados en un formato estándar, y cada uno
contiene secuencias de una misma clase. Por ello, para la generación
del modelo se debe especificar al menos dos archivos, uno con
secuencias de clase ``positiva'' y otro con secuencias de clase
``negativa''. En esta instancia, el usuario puede especificar
parámetros de funcionamiento tales como:
%
\begin{itemize}
\item tipo de máquina de aprendizaje a entrenar (MLP o SVM),
\item estrategia de selección de hiperparámetros,
\item parámetros de validación cruzada,
\item conjuntos de \caract{s} sobre los que trabajar.
\end{itemize}
%
El conjunto de pasos efectuado para la obtención del modelo del
clasificador es
%
\begin{enumerate}
\item Preprocesamiento: interpreta los archivos de secuencias y
  genera el conjunto de datos de entrenamiento y las particiones de
  validación cruzada.
\item Selección de hiperparámetros: consiste en encontrar
  hiperparámetros óptimos de la máquina de aprendizaje para el
  conjunto de datos a utilizar. Dependiendo de la estrategia en
  particular, pueden utilizarse los datos de validación cruzada o el
  conjunto de entrenamiento completo.
\item Entrenamiento: obtiene el modelo del clasificador sobre el
  conjunto completo de entrenamiento usando los hiperparámetros
  óptimos encontrados en el paso anterior.
\end{enumerate}
%
La salida del sistema es entonces el modelo del clasificador, que
puede ser utilizado para la predicción de la pertenencia de clase de
nuevas secuencias.

Para la clasificación, las entradas al sistema son archivos con
secuencias a clasificar y un modelo previamente entrenado.  En este
caso, no se requiere la especificación de parámetros adicionales por
parte del usuario, ya que el modelo contiene toda la información
necesaria para la clasificación.  La secuencia de pasos efectuados en
esste caso es
%
\begin{enumerate}
\item Preprocesamiento: interpreta los archivos de secuencias y
  genera el conjunto de datos de ``prueba'', con los ejemplos a
  clasificar. Utilizando la información del modelo recibido en la
  entrada, se aplica la misma normalización que la efectuada al
  momento del entrenamiento.
\item Clasificación: se aplica sobre el conjunto de prueba obtenido
  del preprocesamiento, obteniendo una predicción de pertenencia de
  clase para cada elemento.
\end{enumerate}
%
La salida del sistema en este caso son predicciones de clase para cada
una de las secuencias leídas en la entrada.

\hl{Agregar interacción con software externo: libsvm, rnafold,
  bioinformatics toolbox.}
