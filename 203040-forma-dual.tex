%
%
\subsection{Formulación dual}
%
En el problema (\ref{e2:svm-problem-basic}) se minimiza una función
cuadrática simple con restricciones relativamente complejas
${y}_i\left(\pint{\ww}{\BPhi(\xx_i)}+b\right)\geq{}1$.
Utilizando la técnica de los multiplicadores de Lagrange
\cite{bottou,kuhntucker}, resulta posible obtener una fomulación
alternativa del problema, con restricciones simples.
La conveniencia de esta formulación se hará evidente más adelante con
la introducción del concepto de núcleos.

La técnica de los multiplicadores de Lagrange permite incorporar las
restricciones del problema dentro de la función a optimizar.
En primer lugar, se define el funcional ``lagrangiano''
%
\begin{align}\label{e2:lagrangian}
  \C{L}(\ww,b,\Balpha) = \frac{1}{2} \|\ww\|^2 + \sum_{i=1}^{\ell}
  \alpha_i (1-y_i(\pint{\ww}{\BPhi(\xx)}+b)).
\end{align}
%
El funcional $\C{L}$ contiene la función a optimizar más las
restricciones ${y}_if_i\geq{}1$, expresadas en la forma normalizada
$g(\xx)\leq0$ y multiplicadas por las variables de holgura $\alpha_i$,
denominadas \e{multiplicadores de Lagrange}.
A partir del lagrangiano se plantea la forma \emph{dual} del problema
(\ref{e2:svm-problem-basic}):
%
\begin{equation}\label{e2:svm-problem-dual}
  \begin{aligned}
    \max_\alpha \tabs\quad f(\Balpha) = \min_{\ww,b} \C{L}(\ww,b,\Balpha)\\
    \T{sujeto a} \tabs\quad \alpha_i \geq 0\T{ para todo } i\in\{1,\ldots,\ell\}.
  \end{aligned}
\end{equation}
%
En este problema las restricciones son más sencillas y aplican
únicamente a los multiplicadores de Lagrange.
Dado que $\frac{1}{2}\|\ww\|^2$ es una función continuamente derivable
y convexa, y dado que $\C{L}$ es continuamente derivable respecto de
sus argumentos en las cercanías del punto óptimo
$(\ww^*,b^*,\Balpha^*)$, las condiciones de Karush-Kuhn-Tucker
\cite{kuhntucker} garantizan la existencia de una solución
$(\ww^*,b^*,\Balpha^*)$ al problema dual (\ref{e2:svm-problem-dual})
tal que $(\ww^*,b^*)$ también es solución al problema primal
(\ref{e2:svm-problem-basic}).

Puede demostrarse que la solución $(\ww^*,b^*,\Balpha^*)$ del problema
dual es un punto de ``silla'', en el que las derivadas parciales de
$\C{L}$ se anulan.
Esto implica que
%
\begin{align}\label{e2:lagrangian-w}
  \evalen{\ww=\ww^*}{\dpar{\C{L}(\ww,b,\Balpha)}{\ww}{}}
    &=\ww^*-\sum_{i=1}^\ell y_i\alpha_i\BPhi(\xx_i) = 0
    &\Rightarrow&& \ww^* &= \sum_{i=1}^\ell y_i\alpha_i\BPhi(\xx_i),
  \\
  \label{e2:lagrangian-sum-yalpha}
  \evalen{b=b^*}{\dpar{\C{L}(\ww,b,\Balpha)}{b}{}}
    &=-\sum_{i=1}^\ell y_i\alpha_i = 0
      &\Rightarrow&& \sum_{i=1}^\ell y_i\alpha_i &=  \pint{\yy}{\Balpha} = 0.
\end{align}
%
Incorporando los resultados (\ref{e2:lagrangian-w}) y
(\ref{e2:lagrangian-sum-yalpha}) en (\ref{e2:lagrangian}), el
lagrangiano puede escribirse
%
\begin{align*}
  \C{L}(\ww,b,\Balpha)
  & =
    \frac{1}{2}\sum_{i=1}^\ell\sum_{j=1}^\ell y_iy_j\alpha_i\alpha_j
    \pint{\BPhi(\xx_i)}{\BPhi(\xx_j)} \\
    &\qquad\qquad +
    \sum_{i=1}^{\ell} \alpha_i \left(1-y_i\left(\pint{
    \sum_{j=1}^\ell y_j\alpha_j\BPhi(\xx_j)}{\BPhi(\xx_i)}
    +b\right)\right)\\
  & =
    -\frac{1}{2}\sum_{i=1}^\ell\sum_{j=1}^\ell y_iy_j\alpha_i\alpha_j
    \pint{\BPhi(\xx_i)}{\BPhi(\xx_j)} +
    \sum_{i=1}^{\ell} \alpha_i  - b \sum_{i=1}^\ell y_i\alpha_i \\
 & =
    -\frac{1}{2}\sum_{i=1}^\ell\sum_{j=1}^\ell y_iy_j\alpha_i\alpha_j
    \pint{\BPhi(\xx_i)}{\BPhi(\xx_j)} +
    \sum_{i=1}^{\ell} \alpha_i  .
\end{align*}
%
Con este resultado, se reescribe el problema dual en función de los
multiplicadores de Lagrange $\alpha_i$:
%
\begin{align}
  \begin{split}
    \max_\alpha &\quad
    -\frac{1}{2}\sum_{i=1}^\ell\sum_{j=1}^\ell y_iy_j\alpha_i\alpha_j
    \pint{ \BPhi(\xx_i)}{\BPhi(\xx_j) } +
    \sum_{i=1}^{\ell} \alpha_i \\
    \T{sujeto a} &\quad \pint{\yy}{\Balpha} = 0, \\
    &\quad \alpha_i \geq 0\T{ para todo } i\in\{1,\ldots,\ell\}.
  \end{split}
  \label{e2:svm-problem-dual2}
\end{align}
%
Este problema se resuelve directamente para los multiplicadores
$\alpha_i$, resultando en una solución $\Balpha^*$.
El vector solución $\ww^*$ al problema original
(\ref{e2:svm-problem-basic}) se obtiene por (\ref{e2:lagrangian-w}):
%
\begin{align}\label{e2:w-from-alpha}
  \ww^* = \sum_{i=1}^\ell y_i\alpha^*_i\BPhi(\xx_i).
\end{align}
%
El cálculo de $b^*$ se deriva de las condiciones de complementariedad
de Karush-Kuhn-Tucker \cite{kuhntucker,bottou}, que establecen que
para todo $i\in\{1,\ldots,\ell\}$, se cumple
$\alpha^*_i(1-y_i(\pint{\ww^*}{\BPhi(\xx_i)}+b^*))=0$.  Entonces, para
algún $\alpha^*_i\neq0$, se tiene
%
\begin{align}\label{e2:b-from-alpha}
  b^* = y_i - \pint{\ww^*}{\BPhi(\xx_i)} .
\end{align}
%
La existencia de $\alpha^*_i\neq0$ está garantizada siempre que el
conjunto de entrenamiento tenga elementos de ambas clases
\cite{glasmachers}.
