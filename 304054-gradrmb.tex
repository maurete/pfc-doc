%
\subsubsection{Cáclulo del gradiente $\nabla\rho$}
%
El cálculo de las derivadas de $\rho$ se basa en resultados de
análisis de perturbación de problemas de optimización, ya que tanto
$\rho_R$ como $\rho_M$ son soluciones a problemas de este tipo
\cite{chung}.

Las derivadas de $R^2$ respecto de los hiperparámetros $C$ y $\gamma$
vienen dadas por
%
\begin{align}
  \dpar{R^2}{C}{} &= \dpar{(1-\Bbeta^T\KK\Bbeta)}{C}{}
  = -\Bbeta^T \dpar{\KK}{C}{}\Bbeta = 0, \\
  \dpar{R^2}{\gamma}{} &= \dpar{(1-\Bbeta^T\KK\Bbeta)}{\gamma}{}
  = -\Bbeta^T \dpar{\KK}{\gamma}{}\Bbeta.
\end{align}
%
Luego,
%
\begin{align}
  \dpar{\rho_R}{C}{} &= -\frac{1}{C^2}, \\
  \dpar{\rho_R}{\gamma}{} &= -\Bbeta^T \dpar{\KK}{\gamma}{}\Bbeta.
\end{align}
%
La existencia de estas está garantizada siempre que las restricciones
del problema de optimización no dependan de los hiperparámetros
\cite{chung}.
Ahora bien, en la forma dual del problema SVM
(\iflatexml\autoref{svmprob-dual-soft}\else{}Ecuación~\ref{svmprob-dual-soft}\fi)
esta condición no se cumple, ya que la restricción $\alpha_i\leq{}C$
depende del hiperparámetro $C$.
Para salvar este inconveniente, se observa que el problema
(\ref{svmprob-dual-soft}) es equivalente a
%
\begin{align}
\begin{split}
    \max_{\bar{\Balpha}}\quad&
    f(\bar{\Balpha}) = C^2 \left( \frac{\B{1}^T\bar{\Balpha}}{C}
    -\frac{1}{2}\bar{\Balpha}^T\QQ\bar{\Balpha}\right)\\
    \T{sujeto a}\quad & \yy^T\bar{\Balpha} = 0, \\
    & 0\leq\bar{\alpha}_i\leq 1,
    \T{ para todo } i\in {1,\ldots,\ell }
\end{split}
\end{align}
%
donde se efectuó el cambio de variable $\bar{\Balpha}=\Balpha/C$.
Aplicando el mismo cambio de variables en $\rho_M$ se tiene:
%
\begin{align}
  \label{eq:rmb-alpha-equiv}
  \rho_M=2\left(\B{e}^T\Balpha-\frac{1}{2}\Balpha^T\B{Q}\Balpha\right)
  = 2C^2\left(\frac{\B{1}^T\bar{\Balpha}}{C} -
  \frac{1}{2}\bar{\Balpha}^T\QQ\bar{\Balpha}\right).
\end{align}
%
Entonces, se está en condiciones de calcular las derivadas
de $\rho_M$ respecto a $C$ y $\gamma$:
%
\begin{align}
    \dpar{\rho_M}{C}{}
    &= \dpar{}{C}{}\left( 2C^2\left(\frac{\B{1}^T\bar{\Balpha}}{C} -
    \frac{1}{2}\bar{\Balpha}^T\QQ\bar{\Balpha}\right)
    \right) \\
    &= 4C \left(\frac{\B{1}^T\bar{\Balpha}}{C} -
    \frac{1}{2}\bar{\Balpha}^T\QQ\bar{\Balpha}\right)
    - 2C^2 \left(\frac{\B{1}^T\bar{\Balpha}}{C^2} \right) \\
    &= 2\left(\B{1}^T\bar{\Balpha}
    - C \bar{\Balpha}^T\QQ\bar{\Balpha}\right) \\
    &= \frac{2}{C} \left(\B{1}^T\Balpha - \Balpha^T\QQ\Balpha\right),
\end{align}
%
%
\begin{align}
    \dpar{\rho_M}{\gamma}{}
    &= \dpar{}{\gamma}{}
    2\left(  \B{1}^T\Balpha-\frac{1}{2}\Balpha^T\QQ\Balpha \right) \\
    &= - \Balpha^T \dpar{\QQ}{\gamma}{}\Balpha \\
    & = - \Balpha^T \left(\yy^T \dpar{\KK}{\gamma}{}\yy\right) \Balpha.
    %% \\
    %% & = \sum_{i,j=1}^\ell \alpha_i\alpha_j y_i y_j \dpar{k(\xx_i,\xx_j)}{\gamma}{}, \\[0.2em]
\end{align}
%
Los elementos $\dpar{k_{ij}}{\gamma}{}$ de la matriz $\dpar{\KK}{\gamma}{}$
vienen dados por
%
\begin{align}
  \dpar{k_{ij}}{\gamma}{}
  = \dpar{}{\gamma}{}k(\xx_i,\xx_j)
  = \dpar{}{\gamma}{} \left(e^{-\gamma\|\xx_i-\xx_j\|}\right)
  = -k_{ij}\|\xx_i-\xx_j\|.
\end{align}
%
Con estos resultados, se calculan las derivadas de la función $\rho$
respecto de $C$ y $\gamma$ según
%
\begin{align}
    \dpar{\rho}{C}{} &= \dpar{\rho_M}{C}{} \rho_R + \rho_M \dpar{\rho_R}{C}{} \\
    &= \frac{2}{C} \left(\B{1}^T\Balpha - \Balpha^T\QQ\Balpha\right) \left( R^2 + \frac{1}{C} \right)
    - 2\left(  \B{1}^T\Balpha-\frac{1}{2}\Balpha^T\QQ\Balpha \right)
    \left( \frac{1}{C^2} \right)  \\[2em]
    \dpar{\rho}{\gamma}{} &= \dpar{\rho_M}{\gamma}{} \rho_R + \rho_M \dpar{\rho_R}{\gamma}{}\\
    &= \left( - \Balpha^T \left(\yy^T \dpar{\KK}{\gamma}{}\yy\right) \Balpha \right)
    \left( R^2 + \frac{1}{C} \right)
    - 2\left(  \B{1}^T\Balpha-\frac{1}{2}\Balpha^T\QQ\Balpha \right)
    \left( \Bbeta^T \dpar{\KK}{\gamma}{} \Bbeta \right)
\end{align}
%
donde los elementos de las matrices $\QQ$ y $\dpar{\KK}{\gamma}{}$ son
%
\begin{align}
  q_{ij}&=y_i y_j k(\xx_i,\xx_j)= y_i y_j e^{-\gamma\|\xx_i-\xx_j\|}, \\
  \dpar{k_{ij}}{\gamma}{}&=-k_{ij}\|\xx_i-\xx_j\| = -\|\xx_i-\xx_j\|e^{-\gamma\|\xx_i-\xx_j\|}.
\end{align}
%
