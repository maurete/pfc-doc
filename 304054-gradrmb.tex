%
\subsubsection{Cáclulo del gradiente $\nabla\rho$}
%
El cálculo de las derivadas de $\rho$ se basa en resultados de
análisis de perturbación de problemas de optimización, ya que tanto
$\rho_R$ como $\rho_M$ son soluciones a problemas de este tipo
\cite{chung}.
Las derivadas de $\rho_R$ respecto de los hiperparámetros $C$ y
$\gamma$ vienen dadas por
%
\begin{align}
%
  \dpar{\rho_R}{C}{} &
  = \dpar{}{C}{}\left(1-\Bbeta^T\KK\Bbeta\right)
       + \dpar{}{C}{}\left(\frac{1}{C}\right)
  = -\Bbeta^T \dpar{\KK}{C}{}\Bbeta - \frac{1}{C^2}
  = -\frac{1}{C^2}, \\[1ex]
%
  \dpar{\rho_R}{\gamma}{} &
  = \dpar{}{\gamma}{}\left(1-\Bbeta^T\KK\Bbeta)\right)
       + \dpar{}{\gamma}{}\left(\frac{1}{C}\right)
  = -\Bbeta^T \dpar{\KK}{\gamma}{}\Bbeta.
%
\end{align}
%
Para garantizar la existencia de estas derivadas, resulta necesario
reescribir las restricciones $\alpha_i\leq{}C$ del problema de la SVM
(\ref{svmprob-dual-soft}) de forma tal que las mismas no dependan del
\hparam{} $C$ \cite{chung}.
Esto se logra efectuando el cambio de variable
$\bar{\Balpha}=\Balpha/C$ en (\ref{svmprob-dual-soft}):
%
\begin{align}
\begin{split}
    \max_{\bar{\Balpha}}\quad&
    f(\bar{\Balpha}) = C^2 \left( \frac{\B{1}^T\bar{\Balpha}}{C}
    -\frac{1}{2}\bar{\Balpha}^T\QQ\bar{\Balpha}\right)\\
    \T{sujeto a}\quad & \yy^T\bar{\Balpha} = 0, \\
    & 0\leq\bar{\alpha}_i\leq 1,
    \T{ para todo } i\in {1,\ldots,\ell }.
\end{split}
\end{align}
%
Con este cambio de variable el valor de $\rho_M$ viene dado por:
%
\begin{align}
  \label{eq:rmb-alpha-equiv}
  \rho_M=2\left(\B{e}^T\Balpha-\frac{1}{2}\Balpha^T\B{Q}\Balpha\right)
  = 2C^2\left(\frac{\B{1}^T\bar{\Balpha}}{C} -
  \frac{1}{2}\bar{\Balpha}^T\QQ\bar{\Balpha}\right).
\end{align}
%

Las derivadas de $\rho_M$ respecto a los \hparam{s} $C$ y $\gamma$ se
calculan según:
%
\begin{align}
    \dpar{\rho_M}{C}{}
    &= \dpar{}{C}{}\left( 2C^2\left(\frac{\B{1}^T\bar{\Balpha}}{C} -
    \frac{1}{2}\bar{\Balpha}^T\QQ\bar{\Balpha}\right)
    \right) \nonumber\\
    &= 4C \left(\frac{\B{1}^T\bar{\Balpha}}{C} -
    \frac{1}{2}\bar{\Balpha}^T\QQ\bar{\Balpha}\right)
    - 2C^2 \left(\frac{\B{1}^T\bar{\Balpha}}{C^2} \right) \nonumber\\
    &= 2\left(\B{1}^T\bar{\Balpha}
    - C \bar{\Balpha}^T\QQ\bar{\Balpha}\right) \nonumber\\
    &= \frac{2}{C} \left(\B{1}^T\Balpha - \Balpha^T\QQ\Balpha\right),\\[1ex]
    \dpar{\rho_M}{\gamma}{}
    &= \dpar{}{\gamma}{}
    2\left(  \B{1}^T\Balpha-\frac{1}{2}\Balpha^T\QQ\Balpha \right) \nonumber\\
    &= - \Balpha^T \dpar{\QQ}{\gamma}{}\Balpha \nonumber\\
    & = - \Balpha^T \left(\yy^T \dpar{\KK}{\gamma}{}\yy\right) \Balpha.
    %% \\
    %% & = \sum_{i,j=1}^\ell \alpha_i\alpha_j y_i y_j \dpar{k(\xx_i,\xx_j)}{\gamma}{}, \\[0.2em]
\end{align}
%
Los elementos $\dpar{k_{ij}}{\gamma}{}$ de la matriz
$\dpar{\KK}{\gamma}{}$ vienen dados por
%
\begin{align}
  \dpar{k_{ij}}{\gamma}{}
  = \dpar{}{\gamma}{}k(\xx_i,\xx_j)
  = \dpar{}{\gamma}{} \left(e^{-\gamma\|\xx_i-\xx_j\|}\right)
  = -k_{ij}\|\xx_i-\xx_j\|.
\end{align}
%
Finalmente se calculan las derivadas de $\rho$ respecto de $C$ y
$\gamma$ mediante
%
\begin{align}
    \dpar{\rho}{C}{} &= \dpar{\rho_M}{C}{} \rho_R + \rho_M \dpar{\rho_R}{C}{} \nonumber\\
    &= \frac{2}{C} \left(\B{1}^T\Balpha - \Balpha^T\QQ\Balpha\right) \left( R^2 + \frac{1}{C} \right)
    - 2\left(  \B{1}^T\Balpha-\frac{1}{2}\Balpha^T\QQ\Balpha \right)
    \left( \frac{1}{C^2} \right) \label{drho-dc}, \\[1em]
    \dpar{\rho}{\gamma}{} &= \dpar{\rho_M}{\gamma}{} \rho_R + \rho_M \dpar{\rho_R}{\gamma}{}\nonumber\\
    &= \left( - \Balpha^T \left(\yy^T \dpar{\KK}{\gamma}{}\yy\right) \Balpha \right)
    \left( R^2 + \frac{1}{C} \right)
    - 2\left(  \B{1}^T\Balpha-\frac{1}{2}\Balpha^T\QQ\Balpha \right)
    \left( \Bbeta^T \dpar{\KK}{\gamma}{} \Bbeta \right), \label{drho-dgamma}
\end{align}
%
en donde los elementos de las matrices $\QQ$ y $\dpar{\KK}{\gamma}{}$
son
%
\begin{align}
  q_{ij}&=y_i y_j k(\xx_i,\xx_j)= y_i y_j e^{-\gamma\|\xx_i-\xx_j\|}, \\
  \dpar{k_{ij}}{\gamma}{}&=-k_{ij}\|\xx_i-\xx_j\| = -\|\xx_i-\xx_j\|e^{-\gamma\|\xx_i-\xx_j\|}.
\end{align}
%
