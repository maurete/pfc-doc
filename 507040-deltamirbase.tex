
\subsection{Problema $\mathbf{\mathsf{\Delta}}$miRBase}
La base de datos miRBase se publica periódicamente, con números de
versión incrementales, incorporando en cada nueva versión nuevos
pre-miRNAs experimentalmente validados.  Si se contara con un
clasificador de secuencias de pre-miRNAs ideal, al entrenarlo con la
versión $X$ de miRBase, éste sería capaz de predecir con una
sensibilidad del 100\% todas las nuevas secuencias incorporadas en la
versión $Y>X$.

El razonamiento anterior no es del todo válido, ya que se sabe por
ejemplo, que dentro de miRBase existen secuencias consideradas
erróneamente como pre-miRNAs.  Sin embargo, la propuesta de clasificar
las nuevas secuencias entre dos versiones sucesivas de miRBase resulta
interesante en el sentido que brinda una idea del comportamiento del
método en problemas ``del mundo real''.

Para esta prueba se generó un problema \deltamirbase{} en el cual el conjunto de entrenamiento
contiene como ejemplos positivos \e{todas} las entradas correspondientes
a la especie humana presentes en la versión 20 de miRBase,
y el conjunto de prueba contiene únicamente las nuevas incorporaciones
en la versión 21 de miRBase, también de la especie humana.
Como ejemplos negativos, se utilizaron elementos de la base de datos CODING \cite{xue},
así como de \ssf{human other ncRNAs} presentada en \cite{batuwita}.
La composición de los conjuntos de entrenamiento y prueba se detalla en
la \autoref{tbl:problem-deltamirbase}.

Los resultados de clasificación del conjunto de prueba, consistente en
25 nuevos pre-miRNAs de la especie humana incorporados en la versión
21 de miRBase, son mostrados en la \autoref{tbl:suppl-deltamirbase21}.
Allí se observa que la tasa máxima obtenida es del 88\%, y se da para
el clasificador SVM-RE, seguido por la estrategia RMB con una tasa del
84\%. Los clasificadores MLP y SVM con núcleo lineal obtienen un menor
rendimiento con tasas alredeor del 70\%.

Estos resultados de clasificación coinciden con la idea que el
clasificador SVM con núcleo RBF es capaz de generar funciones de
decisión más complejas, y por ende más ``ajustables'' a problemas
complejos. Asimismo, se confirma el hecho que la estrategia RMB tiende
a resultar en un clasificador con mayor especificidad que
sensibilidad.
