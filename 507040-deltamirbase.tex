%
%
\subsection{Problema $\mathbf{\mathsf{\Delta}}$\mirbase}
%
Una de las tareas más comunes para un clasificador de secuencias de
\premirna{s} es la predicción de nuevos ejemplos de clase positiva,
proponiendo al usuario nuevos {candidatos} para validar en forma
experimental.

La naturaleza periódica de las publicaciones de \work\mirbase{}
permite generar un problema que brinda una idea de la capacidad de
predicción del método sobre los nuevos ejemplos de clase positiva
incorporados entre dos versiones sucesivas.
Tomando como base la versión 20 de \work\mirbase, la versión siguiente
(21) contiene 25 nuevos ejemplos de \premirna{s} experimentalmente
validados de la especie humana.
Las pruebas se basan en entrenar con los ejemplos positivos de la
versión 20 y clasificar los nuevos ejemplos incorporados a la versión
21.

El problema \prob\deltamirbase{} representa este caso de prueba
incorporando en el conjunto de entrenamiento \e{todos} los ejemplos de
la especie humana presentes en \work\mirbase{} 20.
Como ejemplos negativos de entrenamiento, se utilizaron elementos de
las bases de datos \dset{coding} \cite{xue} y \dset{human other ncRNAs}
\cite{batuwita}.
El conjunto de prueba los 25 nuevos ejemplos de la especie humana
publicados en \work\mirbase 21.
La composición de los conjuntos de entrenamiento y prueba se detalla en
la \iflatexml{}Tabla~\ref{tbl:problem-deltamirbase}\else\autoref{tbl:problem-deltamirbase}\fi.

Los resultados de clasificar el conjunto de prueba del problema
\prob\deltamirbase{} son mostrados en la
\iflatexml{}Tabla~\ref{tbl:suppl-deltamirbase21}\else\autoref{tbl:suppl-deltamirbase21}\fi.
La tasa máxima se obtuvo para el clasificador SVM-RE con el 88\%
(22/25) de ejemplos correctamente clasificados, seguido por SVM-RR con
una tasa del 84\% (21/25).
Los clasificadores SVM-LE y MLP-B obtuvieron una menor exactitud que
rondó el 70\% de aciertos.
Estos resultados muestran la mayor generalidad del clasificador SVM
con núcleo RBF ante problemas de clasificación más complejos.
