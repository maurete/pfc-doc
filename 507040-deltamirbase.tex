%
%
\subsection{Problema \deltamirbase{}}
%
Una de las tareas más comunes para un clasificador de secuencias de
\premirna{s} es la predicción de nuevos ejemplos de clase positiva,
proponiendo al usuario nuevos {candidatos} para validar en forma
experimental.

La naturaleza periódica de las publicaciones de \work\mirbase{}
permite generar un problema que brinda una idea de la capacidad de
predicción del método sobre los nuevos ejemplos de clase positiva
incorporados entre dos versiones sucesivas.
Tomando como base la versión $20$ de \work\mirbase{}, la versión siguiente
($21$) contiene $25$ nuevos ejemplos de \premirna{s} experimentalmente
validados de la especie humana.
Las pruebas se basan en entrenar con los ejemplos positivos de la
versión $20$ y clasificar los nuevos ejemplos incorporados a la versión
$21$.

El problema complementario \prob\deltamirbase{} representa este caso
de prueba, incorporando en el conjunto de entrenamiento \e{todos} los
ejemplos de la especie humana presentes en la versión $20$
de\work\mirbase{}.  Como ejemplos negativos de entrenamiento, se
utilizaron elementos de las bases de datos \dset{coding} \cite{xue} y
\dset{human other ncRNAs} \cite{batuwita}.
El conjunto de prueba contiene los $25$ nuevos ejemplos de la especie humana
publicados en la versión $21$ de \work\mirbase{}.
La composición de los conjuntos de entrenamiento y prueba se detalla en
la Tabla~\ref{tbl:problem-deltamirbase}.
