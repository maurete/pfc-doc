%
%
\subsection{Formulación dual}
%
El problema (\ref{e2:svm-problem-basic}) intenta encontrar el mínimo
de una función cuadrática simple con restricciones ``complejas''
${y}_i(\ww^T\BPhi(\xx_i)+b)\geq{}1$.  Utilizando la técnica de los
multiplicadores de Lagrange \cite{bottou,kkt}, resulta posible obtener
una fomulación alternativa del problema, con restricciones simples,
cuya conveniencia será aparente con la introducción del concepto de
núcleos.

La técnica de los multiplicadores de Lagrange permite incorporar las
restricciones del problema dentro de la función a optimizar.  En
primer lugar, se define el funcional lagrangiano
%
\begin{align}\label{e2:lagrangian}
  \C{L}(\ww,b,\Balpha) = \frac{1}{2} \|\ww\|^2 + \sum_{i=1}^{\ell}
  \alpha_i (1-y_i(\langle\ww,\BPhi(\xx)\rangle+b)).
\end{align}
%
Como se puede ver, el funcional $\C{L}$ es la función a optimizar más
las restricciones ${y}_if_i\geq{}1$ en forma normalizada $g(\xx)\leq0$
y multiplicadas por las ``variables de holgura'' $\alpha_i$,
denominadas \e{multiplicadores de Lagrange}. A partir del lagrangiano
se plantea la formulación \emph{dual} del problema
%
\begin{align}\label{e2:svm-problem-dual}
  \begin{split}
    \max_\alpha \tabs\quad f(\Balpha) = \min_{\ww,b} \C{L}(\ww,b,\Balpha)\\
    \T{sujeto a} \tabs\quad \alpha_i \geq 0\T{ para todo } i\in\{1,\ldots,\ell\}.
  \end{split}
\end{align}
%
En este problema las restricciones son más sencillas y
aplican únicamente a los multiplicadores de Lagrange. 
Dado que $\frac{1}{2}\|\ww\|^2$ es una función continuamente derivable
y convexa, y dado que $\C{L}$ es continuamente derivable respecto de
sus argumentos en las cercanías del punto óptimo
$(\ww^*,b^*,\Balpha^*)$, las condiciones de Karush-Kuhn-Tucker
\cite{kkt} garantizan la existencia de una solución
$(\ww^*,b^*,\Balpha^*)$ al problema dual
(\iflatexml{}Ecuación~\ref{e2:svm-problem-dual}\else\autoref{e2:svm-problem-dual}\fi)
tal que $(\ww^*,b^*)$ también es solución al problema primal
(\iflatexml{}Ecuación~\ref{e2:svm-problem-basic}\else\autoref{e2:svm-problem-basic}\fi).

Al ser el punto ``silla'' $(\ww^*,b^*,\Balpha^*)$ solución del
problema dual (\ref{e2:svm-problem-dual}), sus respectivas derivadas
parciales se anulan:
%
\begin{align}\label{e2:lagrangian-w}
  \evalen{\ww=\ww^*}{\dpar{\C{L}(\ww,b,\Balpha)}{\ww}{}}
    &=\ww^*-\sum_{i=1}^\ell y_i\alpha_i\BPhi(\xx_i) = 0
    &\Rightarrow \ww^* = \sum_{i=1}^\ell y_i\alpha_i\BPhi(\xx_i)
  \\
  \label{e2:lagrangian-sum-yalpha}
  \evalen{b=b^*}{\dpar{\C{L}(\ww,b,\Balpha)}{b}{}}
    &=-\sum_{i=1}^\ell y_i\alpha_i = 0
      &\Rightarrow \sum_{i=1}^\ell y_i\alpha_i = \yy^T\Balpha = 0.
\end{align}
%
Reemplazando los resultados (\ref{e2:lagrangian-w}) y
(\ref{e2:lagrangian-sum-yalpha}) en la
\iflatexml{}Ecuación~\ref{e2:lagrangian}\else\autoref{e2:lagrangian}\fi,
el lagrangiano puede escribirse
%
\begin{align*}
  \C{L}(\ww,b,\Balpha)
  & = 
    \frac{1}{2}\sum_{i=1}^\ell\sum_{j=1}^\ell y_iy_j\alpha_i\alpha_j
    \langle\BPhi(\xx_i),\BPhi(\xx_j)\rangle \\
    &\qquad\qquad +
    \sum_{i=1}^{\ell} \alpha_i \left(1-y_i\left(\left\langle
    \sum_{j=1}^\ell y_j\alpha_j\BPhi(\xx_j) ,\BPhi(\xx_i)\right\rangle
    +b\right)\right)\\
  & = 
    -\frac{1}{2}\sum_{i=1}^\ell\sum_{j=1}^\ell y_iy_j\alpha_i\alpha_j
    \langle\BPhi(\xx_i),\BPhi(\xx_j)\rangle +
    \sum_{i=1}^{\ell} \alpha_i  - b \sum_{i=1}^\ell y_i\alpha_i \\
 & = 
    -\frac{1}{2}\sum_{i=1}^\ell\sum_{j=1}^\ell y_iy_j\alpha_i\alpha_j
    \langle\BPhi(\xx_i),\BPhi(\xx_j)\rangle +
    \sum_{i=1}^{\ell} \alpha_i  .
\end{align*}
%
Con este resultado, se reescribe el problema dual en función de los
multiplicadores de Lagrange $\alpha_i$:
%
\begin{align}
  \begin{split}
    \max_\alpha &\quad
    -\frac{1}{2}\sum_{i=1}^\ell\sum_{j=1}^\ell y_iy_j\alpha_i\alpha_j
    \langle \BPhi(\xx_i),\BPhi(\xx_j) \rangle +
    \sum_{i=1}^{\ell} \alpha_i \\
    \T{sujeto a} &\quad \yy^T\Balpha = 0, \\
    &\quad \alpha_i \geq 0\T{ para todo } i\in\{1,\ldots,\ell\}.
  \end{split}
  \label{e2:svm-problem-dual2}
\end{align}
%
Este problema se resuelve directamente, resultando en una solución
$\Balpha^*$.  El vector solución $\ww^*$ al problema original
(\ref{e2:svm-problem-basic}) se obtiene por la
 \iflatexml{}Ecuación~\ref{e2:lagrangian-w}\else\autoref{e2:lagrangian-w}\fi:
%
\begin{align}\label{e2:w-from-alpha}
  \ww^* = \sum_{i=1}^\ell y_i\alpha^*_i\BPhi(\xx_i).
\end{align}
%
El cálculo de $b^*$ óptimo se deriva de las condiciones de
complementariedad de Karush-Kuhn-Tucker \cite{kkt,bottou}, que
establecen que para todo $i\in\{1,\ldots,\ell\}$, se cumple
$\alpha^*_i(1-y_i(\langle\ww^*,\BPhi(\xx_i)\rangle+b^*))=0$. Entonces,
para algún $\alpha^*_i\neq0$, se tiene
%
\begin{align}\label{e2:b-from-alpha}
  b^* = y_i - \langle\ww^*,\BPhi(\xx_i)\rangle .
\end{align}
%
La existencia de $\alpha^*_i\neq0$ está garantizada siempre que el
conjunto de entrenamiento tenga elementos de ambas clases
\cite{glasmachers}.
