%
\subsubsection{Sensibilidad y especificidad}
%
A partir de los valores de la matriz de confusión se definen las
medidas de \e{sensibilidad ($\SE$)} y \e{especificidad ($\SP$)}:
%
\begin{align}
  \SE \tab = \frac{\VP}{\VP+\FN}, \tabs \SP \tab = \frac{\VN}{\VN+\FP}.
\end{align}
%
La sensibilidad, también llamada a veces \e{razón de veraderos
  positivos} (\e{TPR}), indica la \e{tasa de acierto} al clasificar
elementos de clase positiva, y la especificidad (también \e{razón de
  veraderos negativos}) representa la tasa de acierto al clasificar
ejemplos de clase negativa.
Ambas medidas son independientes del
número de elementos presentes en el conjunto de prueba, aunque no de
la proporción entre elementos positivos y negativos.

%% La importancia de las medidas de sensibilidad y especificidad es que
%% pueden utilizarse para elegir, por ejemplo, el clasificador con menor
%% cantidad de falsos negativos ($\SP=1$), una característica deseable en
%% aplicaciones tales como detectores de patógenos en bancos de sangre y
%% detectores de explosivos en aeropuertos.
%
\subsubsection{Media geométrica de la sensibilidad y  la especificidad}
%
En ocasiones, se requiere medir el funcionamiento de un clasificador a
partir de una única medida que caracterice la aptitud del clasificador
en general.
En estos casos, resulta común utilizar la media geométrica
entre la sensibilidad y la especificidad
%
\begin{align}
  \GM \tab = \sqrt{\SE\cdot\SP}.
\end{align}
%
%
\subsubsection{Otras medidas}
%
Otras medidas posibles para evaluar el comportamiento de un clasificador
binario son:
%
\begin{align}
    \T{Precisión} \tab\tabs \PR \tab=\frac{\VP}{\VP+\FP},\\
    \T{Exactitud} \tab\tabs \AC \tab=\frac{\VP+\VN}{\VP+\FP+\VN+\FN},\\
    \T{Valor-F}   \tab\tabs \T{\slshape F} \tab=\frac{\PR\cdot{}\SE}{\PR+\SE}, \\
    \iflatexml\T{Coef. corr. de Matthews}\else
    \parbox{4cm}{Coeficiente de correlación de Matthews}\fi \tab\tabs
    \T{\slshape MCC} \tab = \frac{\VP\cdot{}\VN-\FP\cdot\FN
    }{\sqrt{(\VP+\FP)(\VP+\FN)(\VN+\FP)(\VN+\FN)}}.
\end{align}
%
La precisión, por ejemplo, es la proporción aciertos de entre todos
los resultados positivos, y la exactitud es la proporción de
resultados correctos independientemente de la clase.
