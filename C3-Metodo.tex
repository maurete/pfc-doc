%
%
%
%
\chapter{Descripción del software}
%
El software desarrollado es un sistema de reconocimiento de
\premirna{s} cuya funcionalidad puede describirse en torno a tres
componentes principales: generación del modelo del clasificador,
clasificación y preprocesamiento de los datos. La generación del
modelo consiste en encontrar \hparam{s} óptimos para luego obtener el
modelo del clasificador sobre los datos de entrenamiento.  La
clasificación es la aplicación del modelo sobre datos nuevos para
obtener predicciones de clase. El preprocesamiento abarca el conjunto
de tareas ejecutadas previo a la generación del modelo y/o
clasificación, y su propósito es transformar los archivos de datos a
un formato utilizable por la máquina de aprendizaje.

Como punto de partida se consideraron los trabajos previos
\cite{xue,ng,batuwita,sheng,sewer,ding}. En ellos se presentan
herramientas de software con propósito similar al del método
desarrollado en el presente trabajo. En particular, se utilizaron
conjuntos de datos y técnicas de extracción de \caract{s} de
\cite{xue,ng,batuwita} debido a la disponibilidad de los datos
suplementarios.

La implementación se efectuó en lenguaje Matlab versión R2012b y hace
uso de los módulos adicionales ``Neural Network Toolbox'' y
``Bioinformatics Toolbox'' del mismo software. Además, integra
funcionalidad con las herramientas ``libSVM'' \cite{libsvm} y RNAFold
del paquete ``Vienna RNA'' \cite{vienna}. Se utilizó el software
\cite{webdemobuilder} para la creación de la interfaz web.
