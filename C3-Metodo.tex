%
%
%
%
\chapter{Descripción del software}
%
La funcionalidad del software desarrollado se centra en dos tareas
principales: la generación del modelo del clasificador y la
clasificación de nuevas secuencias con un modelo previamente
generado.

La generación del modelo es la tarea de entrenar un modelo de
clasificador óptimo para un \e{conjunto de entrenamiento} $D$.  En
este caso, la entrada al sistema es una serie de archivos de
secuencias, que en conjunto deberán formar un conjunto de
entrenamiento válido, con ejemplos de clase positiva y negativa.
La secuencia de pasos efectuada para la obtención del modelo óptimo
es:
%
\begin{enumerate}
\item Preprocesamiento de los datos
\item Generación de particiones de validación cruzada
\item Optimización de los hiperparámetros del clasificador
\item Entrenamiento
\end{enumerate}
%
Una vez efectuado este proceso, se obtiene a la salida del método un
modelo de clasificador apto para clasificar nuevas secuencias.

La clasificación de nuevas secuencias consiste en aplicar un modelo
previamente generado sobre los archivos de secuencias provistos como
entrada, obteniendo predicciones de clase para cada una de ellas. En
este caso, el método recibe como entrada un modelo de clasificador
entrenado y uno o más archivos de secuencia.  A la salida, genera
predicciones de clase para cada una se las secuencias procesadas.  La
obtención de predicciones de clase consta de los siguientes pasos
%
\begin{enumerate}
\item Preprocesamiento de los archivos de secuencias
\item Aplicación del modelo sobre las secuencias
\end{enumerate}
%
De este modo se obtienen las predicciones de clase para todas las
secuencias procesadas.

Tanto para la generación del modelo como para la clasificación, el
sistema efectúa una serie de tareas denominadas genéricamente como
``preprocesamiento''. En particular, estas tareas consisten en
%
\begin{enumerate}
\item Plegado de secuencias
\item Extracción de características
\item Normalización
\item Aleatorización
\end{enumerate}
%
Estos pasos de preprocesamiento tienen como objetivo la generación de
un ``problema de clasificación'' mejor condicionado desde el punto de
vista del clasificador.  Los archivos de texto contienen secuencias de
(candidatos a) pre-miRNAs a procesar. A través del ``plegado'' de las
secuencias se calcula la estructura secundaria, incorporando más
información que la directamente disponible de la secuencia.
Seguidamente, se aplica sobe la secuencia junto a la información de
plegado un proceso llamado \e{extracción de características},
construyendo para cada secuencia original un vector de longitud fija
cuyos elementos son más informativos y menos redundantes en
comparación con la información de entrada ``en bruto''.  La
normalización y aleatorización de los datos se efectúan para mejorar
las condiciones en que se entrena el clasificador.

Luego se prosigue con el entrenamiento o clasificación,
``alimentando'' al algoritmo con vectores de características en lugar
de la representación original.

El presente capítulo se organiza en tres secciones: la primera
describe la parte de \e{preprocesamiento} de los datos, convirtiendo
las secuencias de texto a vectores de características. La segunda
parte describe el proceso de generación del modelo del
clasificador. Finalmente, la última parte presenta una descripción de
la tarea de clasificación.
