%
%
%
%
\chapter{Descripción del software}
%
La funcionalidad del software desarrollado se centra en dos tareas
principales: la obtención del modelo del clasificador y la
clasificación de secuencias con este modelo.

La obtención del modelo consiste en la optimización de los
hiperparámetros para el clasificador seleccionado, para luego efectuar
el entrenamiento con los hiperparámetros óptimos, obteniendo un modelo
capaz de clasificar nuevos datos.

La clasificación requiere la especificación de un modelo a utilizar
además de los datos a clasificar, y consiste simplemente en obtener
predicciones de clase para cada secuencia utilizando el modelo
provisto.

Los datos de entrada al método vienen dados en archivos de texto que
contienen directamente las secuencias a procesar.  Sobre estos datos
en formato texto se aplica un proceso llamado \e{extracción de
  características} que construye un vector de longitud fija con
medidas tomadas sobre estos datos y cuyos elementos son más
informativos y no redundantes en comparación con la información de
entrada ``en bruto''.

Luego se prosigue con el entrenamiento o clasificación,
``alimentando'' al algoritmo con vectores de características en lugar
de la representación original.

El presente capítulo se organiza en tres secciones: la primera
describe la parte de \e{preprocesamiento} de los datos, convirtiendo
las secuencias de texto a vectores de características. La segunda
parte describe el proceso de generación del modelo del
clasificador. Finalmente, la última parte presenta una descripción de
la tarea de clasificación.
