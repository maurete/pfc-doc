\iflatexml\setcounter{chapter}{3}\fi
%
%
%
%
\chapter{Método propuesto}
%
El método propuesto en el presente Trabajo consiste en un sistema de
reconocimiento de pre-miRNAs completo, cuya funcionalidad abarca desde
la entrada de datos en un formato estándar hasta la obtención de
predicciones de clase, incluyendo el entrenamiento y obtención del
modelo del clasificador.

Uno de los objetivos principales del diseño es la facilidad de uso
para el usuario no experto. Por ello, el método incorpora un alto
grado de automatización del flujo de trabajo, incluyendo la extracción
de características, la normalización de los conjuntos de datos, la
selección de hiperparámetros del clasificador, entrenamiento y
clasificación.

En el presente capítulo se describen las funcionalidades principales a
modo de especificación, sin entrar en detalles de implementación.
