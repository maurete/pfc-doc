\section{Documentación}
Se generó documentación para el usuario en el mismo código fuente para
referencia en línea, así como mediante la creación de una guía de
usuario que se adjunta al presente Informe.

La documentación del código fuente se realizó siguiendo el estándar de
documentación de Matlab, resultando en la adición de aproximadamente
700 líneas de ayuda que el usuario puede consultar en la línea de
comandos de Matlab mediante la función {\mono help}.

Además, se redactó una guía del usuario con instrucciones de
instalación, requerimientos del sistema, utilización de la interfaz de
línea de comandos, y generación y utilización de la interfaz web.  Se
adjunta a este documento una copia de dicha guía del usuario.

\chapter{Pruebas}
La evaluación del método propuesto se efectuó a través de una serie de
pruebas experimentales, que consisten en entrenar el clasificador
sobre un conjunto de entrenamiento determinado y luego clasificar un
conjunto de prueba correspondiente, obteniendo así las métricas de
clasificación de sensibilidad, especificidad y $G_m$.  Cada prueba se
efectuó para las diferentes variantes de clasificador/núcleo,
estrategia de selección de hiperparámetros y conjunto de
características, permitiendo obtener medidas relativas para un mismo
conjunto de entrenamiento/prueba.  Se definieron tres \e{problemas} de
clasificación a partir de la bibliografía, cada uno de los cuales
especifica los datos a utilizar dentro de los conjuntos de
entrenamiento y de prueba.

La partición de las fuentes de datos en conjuntos de entrenamiento
y prueba dependen de una semilla aleatoria, así como la generación
de las particiones de validación cruzada. En la implementación, esta
separación depende de un número que hace de semilla pseudoaleatoria.
Todas las pruebas se repitieron para 5 semillas pseudoaleatorias
diferentes, para evitar cualquier sesgo
que una partición particular pudiera introducir al problema.
Por ello, en las tablas de resultados se presentan las medidas de
clasificación SE, SP, Gm caracterizadas por su valor medio $E$ y su
desviación estándar $\sigma$ de los 5 resultados
obtenidos.

Para todos los problemas, se efectuaron pruebas para las
características de secuencia (S), de estructura secundaria (E), y de
secuencia y estructura secundaria en conjunto (S-E).  Para aquellos
problemas que contienen únicamente ejemplos con estructura secundaria
tipo horquilla (bucle único), se probaron las características de
tripletes (T), auxiliares de tripletes (X), y de tripletes y
auxiliares en conjunto (T-X).
Para las estrategias de selección de hiperparámetros mediante búsqueda
exhaustiva, la medida de clasificación a optimizar en todos los casos
fue el valor $G_m$. En todos los casos que se aplicó validación
cruzada se utilizaron 10 particiones, el valor por defecto definido en
la implementación del método propuesto. Cualquier otra configuración
utiliza los valores predeterminados en la implementación.