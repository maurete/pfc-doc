%
%
\subsection{Conceptos básicos del aprendizaje automático}
%
El objetivo fundamental del aprendizaje automático es modelar un
sistema (fenómeno) desconocido conociendo únicamente las entradas
(estímulos) y/o salidas (respuestas) del mismo, mediante un
algoritmo computacional llamado \e{máquina de aprendizaje}.
Dentro del aprendizaje automático, se diferencia el aprendizaje
supervisado del aprendizaje no supervisado.
En el aprendizaje supervisado, la máquina ``aprende''
a producir las respuestas deseadas a través de la presentación de
\e{ejemplos} con entradas y salidas reales del sistema a modelar.
En el aprendizaje no supervisado, en cambio, la máquina extrae 
propiedades de las observaciones tratando de inferir el
funcionamiento interno del sistema desconocido.

A continuación se presentan definiciones formales de los conceptos
relevantes en el campo del aprendizaje automático.
