%
\subsubsection{Error de prueba}
%
Dado un \e{conjunto de prueba}
$\,T=((\tilde{\xx}_1,\tilde{\yy}_1),\ldots,(\tilde{\xx}_N,\tilde{\yy}_N))$, el
\e{error de prueba} es la pérdida media del modelo sobre el conjunto $T$:
%
\begin{align}
  E_{prueba}=\frac{1}{N}\sum_{n=1}^N
  L(\tilde{\xx}_N,\tilde{\yy}_N,h(\tilde{\xx}_N)).
  \label{eq:error-prueba}
\end{align}
%
Este error de prueba es una estimación no sesgada del error de
generalización siempre que $T$ sea una muestra independiente e
idénticamente distribuida a $\nu$:
%
\begin{align}
  T\sim\nu^N.
  \label{eq:conj-prueba-sim-nu}
\end{align}
%
En la práctica, esta última condición no es más que un supuesto,
ya que en general no se dispone de información de la probabilidad
de la muestra $T$.
