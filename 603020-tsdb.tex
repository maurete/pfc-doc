\subsection{Base de datos de series de tiempo}

Para la implementación de la base de datos de series de tiempo se
consideraron los ofrecimientos de Prometheus\footnote{
  \href{https://prometheus.io/}{https://prometheus.io/} } e
InfluxDB\footnote{
  \href{https://www.influxdata.com/}{https://www.influxdata.com/}
}. Si bien ambos ofrecen funcionalidades similares, se optó por la
implementación de InfluxDB debido a su mejor capacidad para el manejo
de datos textuales\footnote{
  \url{https://prometheus.io/docs/introduction/comparison/\#prometheus-vs-influxdb}{https://prometheus.io/docs/introduction/comparison/\#prometheus-vs-influxdb}
} así como su simplicidad de instalación y configuración.

La implementación se efectuó escribiendo el código Ansible necesario y
desplegando el servicio en la infraestructura. Al tratarse de un nuevo
tipo de base de datos, fue necesario codificar un script para efectuar
el proceso de backup, de acuerdo a la política de resguardo de datos
de la Dirección.

\subsection{Visualización}

Como servicio de visualización se adoptó el software Grafana\footnote{
  \href{https://grafana.com/}{https://grafana.com/} }, una herramienta
orientada a la creación de tableros (dashboards) que permiten
visualizar datos provenientes de distintas fuentes, entre los que se
incluyen Zabbix, InfluxDB y Graylog\footnote{ Estrictamente, Grafana
  no utiliza Graylog como origen de datos, sino Elasticsearch, el
  motor de indexación y búsqueda utilizado por Graylog como
  \e{backend}.}. La decisión de utilizar Grafana frente a otras
alternativas se basó en el hecho que el equipo de infraestructura
cuenta con experiencia previa en su utilización.

El aprovisionamiento del servicio se efectuó utilizando Ansible y
Docker Compose, guardando el código respectivo en un repositorio de
GitLab.

En la
\iflatexml{}Figura~\ref{fig:grafana}\else\autoref{fig:grafana}\fi{} se
muestra un tablero en utilización que muestra métricas obtenidas de
diversas fuentes.
