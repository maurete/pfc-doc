%
%
%
\section{La clasificación como forma de aprendizaje supervisado}
%
La \e{clasificación} es la tarea de asignar una \e{clase} o
\e{categoría} a una entidad u observación.
En el contexto del aprendizaje supervisado, un \e{clasificador} es un
algoritmo específico --un tipo de \e{máquina de aprendizaje}-- que
determina las reglas de clasificación basándose en \e{ejemplos}.
Estas reglas se representan en una función matemática llamada
\e{modelo}, generada a partir de los ejemplos observados en un proceso
denominado \e{entrenamiento}.
Una vez efectuado el entrenamiento, el modelo obtenido puede ser
utilizado para inferir la clase correspondiente a nuevas
observaciones.
En la presente sección se introducen éstos y otros conceptos del
aprendizaje automático aplicables a la construcción y evaluación de
clasificadores.
