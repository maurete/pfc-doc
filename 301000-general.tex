%
%
%
\section{Descripción general}
%
La arquitectura del software codificado puede describirse en torno a
tres tareas principales:
%
\begin{enumerate}
\item
  \e{Preprocesamiento}:
  transforma los datos de entrada desde su forma original (cadenas de
  texto) a un formato utilizable por la máquina de aprendizaje.
\item
  \e{Generación del modelo del clasificador}:
  aplica estrategias de selección de \hparam{s} y efectúa
  entrenamiento de la máquina de aprendizaje para obtener un modelo
  de clasificador óptimo.
\item
  \e{Clasificación}:
  obtiene predicciones de clase aplicando un modelo generado
  previamente.
\end{enumerate}
%
Las entradas y salidas del sistema son diferentes según sea el
objetivo de la invocación: generar un modelo de clasificador o aplicar
un modelo preexistente para obtener predicciones sobre nuevos datos.

En la generación del modelo de clasificador, las entradas al sistema
son archivos de texto con secuencias de ejemplo, que se utilizan para
armar el conjunto de entrenamiento.
Por cada archivo se especifica una clase, que indica el tipo de
secuencias que contiene: \e{positiva} (+1) cuando se trata de ejemplos
de \premirna{s} verdaderos, o \e{negativa} (-1) cuando el archivo
contiene ``pseudo'' \premirna{s}.
Se requiere la especificación de al menos un archivo de cada clase.
%% Se permite establecer opciones que regulan el comportamiento de la
%% generación del modelo tales como
%% %
%% \begin{itemize}
%% \item Tipo de máquina de aprendizaje a entrenar (MLP o SVM).
%% \item Parámetros de validación cruzada.
%% \item Conjuntos de \caract{s} sobre los que trabajar.
%% \item Estrategia de selección de hiperparámetros.
%% \end{itemize}
%% %
%%
La generación del modelo del clasificador es un proceso de tres pasos:
%
\begin{enumerate}
\item
  El preprocesamiento interpreta los archivos de entrada y
  genera el conjunto de datos de entrenamiento y las particiones de
  validación cruzada.
\item
  La selección de hiperparámetros es el proceso de determinar
  hiperparámetros óptimos de la máquina de aprendizaje para
  el conjunto de entrenamiento dado. A grandes rasgos, consiste en
  generar modelos ``desechables'' probando diferentes configuraciones
  mediante validación cruzada hasta obtener un óptimo.
\item
  El entrenamiento es el paso final, que entrena un modelo del
  clasificador con el conjunto completo y los hiperparámetros
  óptimos encontrados en el paso anterior.
\end{enumerate}
%
La salida del sistema es el modelo del clasificador, que
puede ser utilizado para la predicción de la pertenencia de clase de
nuevas secuencias.

En el caso de la clasificación, el sistema recibe en la entrada un
modelo de clasificador junto a uno o más archivos de secuencias.
A diferencia del modo de generación del modelo, no se
requiere especificar la clase de los archivos. %% , ni tampoco se requiere
%% la especificación de parámetros adicionales por parte del usuario, ya
%% que el modelo contiene toda la información necesaria para la
%% clasificación.
Los pasos efectuados para la obtención de las predicciones de clase
son
%
\begin{enumerate}
\item
  Mediante el preprocesamiento, se interpretan los archivos de
  secuencias y se genera el conjunto de datos de ``prueba'' con los
  ejemplos a clasificar. Utilizando la información del modelo recibido
  en la entrada, se aplica la misma normalización que la efectuada al
  momento del entrenamiento.
\item
  La clasificación se aplica sobre el conjunto de prueba obtenido
  del preprocesamiento, obteniendo una predicción de pertenencia de
  clase para cada ejemplo.
\end{enumerate}
%
La salida del sistema en este caso son predicciones de clase para cada
una de las secuencias leídas en la entrada.
