%
%
%
\section{Descripción general}
%
La arquitectura del software codificado se describe en torno a
tres componentes principales: \e{preprocesamiento}, \e{generación
del modelo del clasificador}, y \e{clasificación}. 

El preprocesamiento transforma los datos de entrada desde su forma
original de cadenas de caracteres a un formato numérico tratable por
la máquina de aprendizaje.
Abarca la lectura e interpretación de los archivos de entrada, la
extracción de \caract{s}, y otras tareas de preparación de los datos,
generando una representación interna de los datos que se utiliza como
entrada a la máquina de aprendizaje, ya sea para la generación del
modelo o para clasificación.

La generación del modelo del clasificador aplica estrategias de
selección de \hparam{s} y efectúa el entrenamiento
para obtener un modelo de clasificador óptimo.
Como entrada, recibe los datos del preprocesamiento y procede en dos
pasos: en primer lugar, aplica alguna estrategia para determinar los
hiperparámetros óptimos, y luego efectúa el entrenamiento propiamente
dicho.
A la salida devuelve un modelo de clasificador que puede ser
utilizado para la predicción de clase de nuevas secuencias.

La clasificación obtiene predicciones de clase sobre nuevos ejemplos
aplicando un modelo generado previamente.
Para la obtención de predicciones de clase, se debe contar con un
modelo de clasificador generado previamente, además de archivos de
secuencias para clasificar.
El primer paso es aplicar el preprocesamiento sobre los archivos de
entrada, generando un conjunto de datos denominado ``de prueba''.
Luego se aplica el modelo sobre este conjunto, obteniendo a la salida
las predicciones de clase correspondientes.

%% El preprocesamiento lee a su entrada archivos en un formato de texto
%% estándar denominado FASTA, y aplicando un proceso denominado
%% extracción de características, convierte los datos leídos a matrices
%% numéricas.
%% Los datos numéricos son luego reagrupados en nuevas matrices que
%% representan los conjuntos de datos utilizados por la máquina de
%% aprendizaje.

%% Cuando el objetivo es la generación de un modelo de clasificador, la
%% entrada al sistema consiste en archivos de texto con secuencias de
%% ejemplo.
%% Por cada archivo, se especifica una clase, indicando que el mismo
%% contiene ejemplos de \premirna{s} validados (clase positiva) o bien
%% ejemplos de ``pseudo'' \premirna{s} (clase negativa).
%% Se requiere la especificación de al menos un archivo de cada clase.
%% Aplicando preprocesamiento, estos archivos se utilizan para el armado
%% del conjunto de entrenamiento.
