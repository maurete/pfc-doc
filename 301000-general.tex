%
%
%
\section{Descripción general}
%
El software presenta dos modos de funcionamiento principales:
%
\begin{enumerate}
\item
  La generación del modelo del clasificador tiene como objetivo
  obtener un modelo óptimo de clasificador para un conjunto de datos
  de entrenamiento particular.
\item
  La clasificación recibe a su entrada datos de prueba junto a un
  modelo de clasificador, y retorna a la salida predicciones de clase
  para cada uno de los ejemplos.
\end{enumerate}
%
%
%
\subsection{Generación del modelo del clasificador}
%
En este modo de funcionamiento, las entradas al sistema son archivos
de texto con secuencias de ejemplo para conformar el conjunto de
entrenamiento. Para cada archivo se especifica una clase, que indica
el tipo de secuencias contenidas: \e{positiva} (+1) cuando se trata de
ejemplos de \premirna{s} verdaderos, o \e{negativa} (-1) cuando el
archivo contiene ``pseudo'' \premirna{s}. Se requiere la
especificación de al menos un archivo de cada clase.
Se permite establecer opciones que regulan el comportamiento
de la generación del modelo tales como
%
\begin{itemize}
\item Tipo de máquina de aprendizaje a entrenar (MLP o SVM).
\item Parámetros de validación cruzada.
\item Conjuntos de \caract{s} sobre los que trabajar.
\item Estrategia de selección de hiperparámetros.
\end{itemize}
%

La generación del modelo del clasificador es un proceso de tres pasos
%
\begin{enumerate}
\item El preprocesamiento interpreta los archivos de entrada y
  genera el conjunto de datos de entrenamiento y las particiones de
  validación cruzada.
\item La selección de hiperparámetros es el proceso de determinar
  hiperparámetros óptimos de la máquina de aprendizaje para
  el conjunto de entrenamiento dado. A grandes rasgos, consiste en
  generar modelos ``desechables'' probando diferentes configuraciones
  mediante validación cruzada hasta obtener un óptimo.
\item El entrenamiento es el paso final, que entrena un modelo del
  clasificador con el conjunto completo y los hiperparámetros
  óptimos encontrados en el paso anterior.
\end{enumerate}
%
La salida del sistema es el modelo del clasificador, que
puede ser utilizado para la predicción de la pertenencia de clase de
nuevas secuencias.
%
%
\subsection{Clasificación}
%
En este modo de funcionamiento, el sistema recibe en la entrada un
modelo de clasificador junto a uno o más archivos de secuencias a
clasificar.  A diferencia del modo de generación del modelo, no se
requiere especificar la clase de los archivos, ni tampoco se requiere
la especificación de parámetros adicionales por parte del usuario, ya
que el modelo contiene toda la información necesaria para la
clasificación. Los pasos efectuados para la obtención de las
predicciones de clase son
%
\begin{enumerate}
\item Mediante el preprocesamiento, se interpretan los archivos de
  secuencias y se genera el conjunto de datos de ``prueba'' con los
  ejemplos a clasificar. Utilizando la información del modelo recibido
  en la entrada, se aplica la misma normalización que la efectuada al
  momento del entrenamiento.
\item La clasificación se aplica sobre el conjunto de prueba obtenido
  del preprocesamiento, obteniendo una predicción de pertenencia de
  clase para cada ejemplo.
\end{enumerate}
%
La salida del sistema en este caso son predicciones de clase para cada
una de las secuencias leídas en la entrada.
