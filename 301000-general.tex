%
%
%
%\section{Descripción general}
%

La arquitectura del clasificador desarrollado se describe en torno a
tres tareas principales:
%
\begin{itemize}
\item
  El \e{preprocesamiento} transforma los datos de entrada a un formato
  numérico tratable por la máquina de aprendizaje.
  Abarca la interpretación de los archivos de entrada y la extracción
  de \caract{s}, junto a otras tareas de preparación de los datos, y
  genera una representación interna que se utiliza para la generación
  del modelo o para clasificación, según se requiera.
\item
  La \e{generación del modelo del clasificador} aplica estrategias de
  selección de \hparam{s} y luego efectúa el entrenamiento de la
  máquina de aprendizaje, generando a la salida un modelo de
  clasificador óptimo que puede ser utilizado para la predicción de
  pertenencia de clase de nuevos ejemplos.
\item
  La \e{clasificación} obtiene predicciones de clase sobre nuevos
  ejemplos, aplicando un modelo generado previamente.
\end{itemize}
%
Estas tareas reflejan el flujo de trabajo necesario para la predicción
de pertenencia de clase de nuevos datos, generando un modelo a partir
de ejemplos de clase conocida.
En este Capítulo se presenta una descripción del clasificador
desarrollado organizada según dichas tareas.
