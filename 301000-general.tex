%
%
%
\section{Descripción general}
%
La arquitectura del software codificado puede describirse en torno a
tres tareas principales:
%
\begin{enumerate}
\item
  \e{Preprocesamiento}:
  transforma los datos de entrada desde su forma original de cadenas
  de caracteres a un formato numérico tratable por la máquina de
  aprendizaje.
\item
  \e{Generación del modelo del clasificador}:
  aplica estrategias de selección de \hparam{s} y efectúa el
  entrenamiento de la máquina de aprendizaje para obtener un modelo
  de clasificador óptimo.
\item
  \e{Clasificación}:
  obtiene predicciones de clase sobre nuevos ejemplos aplicando un
  modelo generado previamente.
\end{enumerate}
%
El preprocesamiento abarca la lectura e interpretación de los archivos
de entrada y las demás tareas efectuadas para la preparación de los
datos previo a la aplicación de la máquina de aprendizaje, ya sea para
la generación del modelo o para clasificación.

%% El preprocesamiento lee a su entrada archivos en un formato de texto
%% estándar denominado FASTA, y aplicando un proceso denominado
%% extracción de características, convierte los datos leídos a matrices
%% numéricas.
%% Los datos numéricos son luego reagrupados en nuevas matrices que
%% representan los conjuntos de datos utilizados por la máquina de
%% aprendizaje.

La generación del modelo del clasificador recibe los datos del
preprocesamiento y procede en dos pasos:
en primer lugar, aplica alguna estrategia para determinar los
hiperparámetros óptimos de entrenamiento de la máquina de aprendizaje
a partir de los datos de entrenamiento.
Una vez fijados los hiperparámetros efectúa el entrenamiento,
generando en la salida un modelo de clasificador que puede ser
utilizado para la predicción de clase de nuevas secuencias.

%% Cuando el objetivo es la generación de un modelo de clasificador, la
%% entrada al sistema consiste en archivos de texto con secuencias de
%% ejemplo.
%% Por cada archivo, se especifica una clase, indicando que el mismo
%% contiene ejemplos de \premirna{s} validados (clase positiva) o bien
%% ejemplos de ``pseudo'' \premirna{s} (clase negativa).
%% Se requiere la especificación de al menos un archivo de cada clase.
%% Aplicando preprocesamiento, estos archivos se utilizan para el armado
%% del conjunto de entrenamiento.

Para la obtención de predicciones de clase, se debe contar con un
modelo de clasificador generado previamente, además de archivos de
secuencias para clasificar.
El primer paso es aplicar el preprocesamiento sobre los archivos de
entrada, generando un conjunto de datos denominado ``de prueba''.
Luego se aplica el modelo sobre este conjunto, obteniendo a la salida
las predicciones de clase correspondientes.
