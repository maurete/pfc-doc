%
%
%
%\section{Descripción general}
%

La arquitectura del método codificado se describe en torno a tres
tareas principales:
%
\begin{itemize}
\item
  El \e{preprocesamiento} transforma los datos de entrada a un formato
  numérico tratable por la máquina de aprendizaje.
  Abarca la interpretación de los archivos de entrada y la extracción
  de \caract{s}, junto a otras tareas de preparación de los datos, y
  genera una representación interna que se utiliza para la generación
  del modelo o para clasificación, según se requiera.
\item
  La \e{generación del modelo del clasificador} aplica estrategias de
  selección de \hparam{s} y luego efectúa el entrenamiento de la
  máquina de aprendizaje, generando a la salida un modelo de
  clasificador óptimo que puede ser utilizado para la predicción de
  pertenencia de clase de nuevos ejemplos.
\item
  La \e{clasificación} obtiene predicciones de clase sobre nuevos
  ejemplos, aplicando un modelo generado previamente.
\end{itemize}
%
Estas tareas modelan el flujo de trabajo necesario para la predicción
de pertenencia de clase de nuevos datos, generando un modelo a partir
de ejemplos de clase conocida.

En el presente Capítulo, se describen estas tareas así como las
interfaces de usuario codificadas para la utilización del software por
parte del usuario final.
