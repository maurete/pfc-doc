%
\subsubsection{Error de entrenamiento}
%
El conjunto de entrenamiento $D$ es una muestra (de probabilidad
desconocida) de la distribución generadora de los datos $\nu$.
Esta muestra determina una distribución \e{empírica}
%
\begin{align}
  \hat{\nu}=\frac{1}{\ell}\sum_{i=1}^{\ell}\delta_{(\xx_i,\yy_i)}(\xx,\yy),
\end{align}
%
en donde $\delta_{(\xx,\yy)}$ es la delta de Dirac centrada en el
punto $(\xx,\yy)$.
La distribución $\hat{\nu}$ conduce a la definición del
\e{error de entrenamiento} como el error del modelo $h$ sobre los
datos de entrenamiento en $D$:
%
\begin{align}
  \hat{\C{R}}_D(h)=E_\T{entrenamiento}=\R{E}_{\hat{\nu}}[L(\xx,\yy,h(\xx))]
  =\frac{1}{\ell}\sum_{i=1}^\ell L(\xx_i, \yy_i, h(\xx_i)),
\end{align}
%
en donde $L$ es una función de pérdida.
La forma de entrenamiento más básica e importante de una máquina
de aprendizaje consiste en seleccionar el modelo que minimiza el
error de entrenamiento
%
\begin{align}
  \hat{h}:=\arg \min \left\{ \hat{\C{R}}_D(h)\middle|h \in H\right\}.
\end{align}
%
Este procedimiento se conoce también como la minimización del riesgo
empírico sobre el conjunto de entrenamiento.
