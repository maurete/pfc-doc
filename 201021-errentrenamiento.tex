%
\subsubsection{Error de entrenamiento}
%
El conjunto de entrenamiento $D$ es una muestra (de probabilidad
desconocida) de la distribución generadora de los datos $\nu$.
Esta muestra determina una distribución \e{empírica}
%
\begin{align}
  \hat{\nu}=\frac{1}{\ell}\sum_{i=1}^{\ell}\delta_{(\xx_i,\yy_i)}(\xx,\yy),
\end{align}
%
en donde $\delta_{(\xx,\yy)}$ es la delta de Dirac centrada en el
punto $(\xx,\yy)$.
La distribución $\hat{\nu}$ conduce a la definición del
\e{error de entrenamiento} como el error del modelo $h$ sobre los
datos de entrenamiento en $D$:
%
\begin{align}
  E_D=\R{E}_{\hat{\nu}}[L(\xx,\yy,h(\xx))]
  =\frac{1}{\ell}\sum_{i=1}^\ell L(\xx_i, \yy_i, h(\xx_i)).
\end{align}
%
La forma de entrenamiento básica de una máquina de aprendizaje
consiste en seleccionar el modelo $h(x)$ que minimiza el error de
entrenamiento
%
\begin{align}
  {h}:=\arg \min_{\hat{h}} \left\{
  E_D\!\left(\hat{h}\right)\,\middle|\,\hat{h} \in \C{H}\right\}.
  \label{e2:minerrorempirico}
\end{align}
%
%en donde $H$ es una familia de funciones utilizables como modelo.
Este procedimiento se conoce también como la minimización del riesgo
empírico sobre el conjunto de entrenamiento.
