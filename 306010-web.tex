%
%
\section{Interfaz web de demostración}
%
Se generó una interfaz web utilizando la herramienta \eng{\webdemo{}}
\cite{webdemobuilder}, que interactúa con el clasificador a través de
una función específica.
La misma recibe un número de argumentos fijos, que son presentados
como campos del formulario web, y genera luego de cada ejecución un
``reporte'' que es presentado al usuario.
Cuando la invocación es exitosa, el reporte contiene el modelo
generado y las predicciones de clase del conjunto de prueba (si
corresponde), y puede descargarse y utilizarse posteriormente como
modelo para clasificar nuevos datos.
Cuando se encuentran errores, el reporte informa al usuario del
problema ocurrido.

%% Los argumentos de entrada en el formulario web son:
%% %
%% \begin{enumerate}
%% \item Tipo de clasificador requerido
%% \item Grupos de \caract{s} a incluir
%% \item Estrategia de selección de \hparam{s}
%% \item Archivo con datos de entrenamiento de clase positiva
%% \item Archivo con datos de entrenamiento de clase negativa
%% \item Modelo generado previamente
%% \item Archivo con datos de prueba
%% \item Clase de los datos de prueba (usado para estadísticas)
%% \end{enumerate}
%% %
Según la tarea a realizar, se deben cargar los siguientes datos en el
formulario web:
%
\begin{itemize}
\item
  \e{Para la generación del modelo del clasificador}
  %
  \begin{itemize}
  \item
    Tipo de clasificador
  \item
    Conjunto de características
  \item
    Estrategia de selección de \hparam{s}
  \item
    Archivo FASTA con datos de entrenamiento de clase positiva
  \item
    Archivo FASTA con datos de entrenamiento de clase negativa
  \end{itemize}
  %
\item
  \e{Para clasificación con un modelo generado previamente}
  %
  \begin{itemize}
  \item
    Archivo con el modelo de clasificador
  \item
    Archivo FASTA con datos de prueba
  \item
    Clase de los datos de prueba: opcional, utilizado para calcular
    una tasa de error sobre el conjunto.
  \end{itemize}
  %
\end{itemize}
%
Luego de la ejecución, se muestra en pantalla un enlace al archivo de
reporte, que ser puede descargado y almacenado para su utilización
posterior.
