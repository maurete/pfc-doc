%
%
\subsection{Interfaz de usuario web}
%
La función \func{webif} permite la utilización del sistema a través
una interfaz web provista por el software \eng{\webdemo{}}
\cite{webdemobuilder}, un desarrollo propio del laboratorio
\eng{sinc(i)} de la Facultad de Ingeniería y Ciencias Hídricas.
Esta función abarca toda la funcionalidad del sistema, aunque presenta
menor flexibilidad que la interfaz de línea de comandos.
Recibe un número fijo de argumentos, que son presentados al usuario en
el formulario web generado por \eng{\webdemo{}}:
%
\begin{itemize}
\item
  Para la generación del modelo del clasificador:
  %
  \begin{itemize}
  \item
    Tipo de clasificador requerido,
  \item
    Conjunto de características consideradas,
  \item
    Estrategia de selección de hiperparámetros,
  \item
    Archivo FASTA con datos de entrenamiento de clase positiva,
  \item
    Archivo FASTA con datos de entrenamiento de clase negativa.
  \end{itemize}
  %
\item
  Para clasificación con un modelo generado previamente:
  %
  \begin{itemize}
  \item
    Archivo con el modelo de clasificador,
  \item
    Archivo FASTA con datos de prueba,
  \item
    Clase de los ejemplos contenidos en el conjunto de prueba (opcional).
  \end{itemize}
  %
\end{itemize}
%
La función está diseñada para ser invocada en modo no interactivo, y
en caso de éxito genera un reporte en formato HTML que incluye el
modelo de clasificador incorporado.
El archivo HTML puede ser guardado y utilizado posteriormente por el
usuario como modelo para clasificar nuevos datos a través de la
interfaz web.
En caso de encontrarse errores, la función \func{webif} escribe un
archivo \e{log} que es presentado al usuario informando del error
encontrado.

\hl{Agregar/publicar web de demostración.}
