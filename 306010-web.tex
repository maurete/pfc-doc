%
%
\subsection{Interfaz de usuario web}
%
Se generó una interfaz web utilizando la herramienta \eng{\webdemo{}}
\cite{webdemobuilder}.
Esta interfaz interactúa con el método a través de una función
codificada para este propósito denominada \e{webif}.
La función está diseñada para ser invocada en una línea de comandos no
interactiva, generando en todos los casos un archivo que es presentado
al usuario a través de la web.
Cuando la invocación es exitosa, genera un reporte en formato HTML que
incorpora el modelo del clasificador generado y/o un listado con
predicciones de clase para los ejemplos en el conjunto de prueba.
El reporte HTML puede utilizarse sucesivamente como modelo para
clasificar nuevos datos a través de la interfaz web.
En caso de encontrarse errores, la función genera un reporte
informando al usuario del error encontrado.

La función \func{webif} recibe 8 argumentos en la entrada:
%
\begin{enumerate}
\item Tipo de clasificador
\item Conjunto de \caract{s}
\item Estrategia de selección de \hparam{s}
\item Datos de entrenamiento de clase positiva
\item Datos de entrenamiento de clase negativa
\item Modelo previamente generado
\item Datos de prueba
\item Clase de los datos de prueba (usado para estadísticas)
\end{enumerate}
%

Según la tarea que deseeefectuar, el usuario debe cargar los
siguientes datos en el formulario web:
%
\begin{itemize}
\item
  \e{Para la generación del modelo del clasificador}
  %
  \begin{itemize}
  \item
    Tipo de clasificador
  \item
    Conjunto de características
  \item
    Estrategia de selección de \hparam{s}
  \item
    Archivo FASTA con datos de entrenamiento de clase positiva
  \item
    Archivo FASTA con datos de entrenamiento de clase negativa
  \end{itemize}
  %
\item
  \e{Para clasificación con un modelo generado previamente}
  %
  \begin{itemize}
  \item
    Archivo con el modelo de clasificador
  \item
    Archivo FASTA con datos de prueba
  \item
    Clase de los datos de prueba: opcional, utilizado para calcular
    una tasa de error sobre el conjunto.
  \end{itemize}
  %
\end{itemize}
%
