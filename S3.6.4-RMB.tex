
\subsection{Minimización de la cota radio-margen}
La cota ``radio-margen'' es una función con la propiedad de ser un
límite superior al error de validación cruzada dejando uno fuera, y se
calcula directamente a partir de la información disponible en el
modelo de una máquina de vectores de soporte.  La función
``radio-margen'' propuesta originalmente por \citeauthor{vapnik}
\cite{vapnik} para un modelo SVM  con término de regularización $L2$
(\autoref{svm-l2}), viene dada por % vapnik sec. 10.7, pag 441

\begin{align}
  \T{RM} = 4R^2 \|w\|^2.
\end{align}
Aquí, $R$ es el radio de la hiperesfera que, en el espacio inducido
por el núcleo, engloba todos los vectores de soporte, y $\|\ww\|$ es
la norma del vector normal al hiperplano de separación, que define el
margen del modelo $h$ de la SVM.  Para esta cota, se cumple
desigualdad

\begin{align}
  E_{\T{LOO}} \leq \T{RM},
\end{align}
donde $E_{\T{LOO}}$ es el error de validación cruzada dejando uno
fuera.

El error de validación cruzada dejando uno fuera $E_{\T{LOO}}$ es uno
de los mejores estimadores del error de generalización disponibles.
Por ello, la idea de minimizar una cota de este error, que es a la vez
es continua, derivable y fácilmente calculable, resulta muy atractiva.
De hecho, en \cite{chapelle} se presenta un algoritmo de minimización
de la cota $\T{RM}$ mediante descenso por gradiente en el espacio de
los hiperparámetros.  Sin embargo, dado que la mayoría de SVMs
utilizadas en la práctica incorporan regularización $L1$ en lugar de
$L2$, la utilidad real de minimizar la cota $\T{RM}$ es limitada.

En \cite{chung} se proponen varias funciones ``alternativas'',
inspiradas en la cota radio-margen $\T{RM}$, y aplicables al caso
específico del núcleo gaussiano (RBF) y formulación $L1$.

