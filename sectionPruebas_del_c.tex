\section{Pruebas del clasificador MLP}
Se probó el clasificador con los conjuntos de datos de enrenamiento y
prueba definidos en los problemas de clasificación detallados
previamente. En la implementación del método propuesto, el
hiperparámetro a optimizar es el número de neuronas en la capa oculta
del MLP, adoptando valores en el rango de los números naturales, o
bien 0 para el caso especial de un MLP sin capa oculta (o perceptrón
simple). La búsqueda exhausiva determina el número óptimo de neuronas
en la capa oculta probando 20 valores ``razonables'' tal como se
explica en \autoref{doc:mlpsearch}.

Los resultados obtenidos de las pruebas efectuadas con el clasificador
MLP se detallan en la \autoref{tbl:mlp-results}. Se debe remarcar que,
debido a la naturaleza aleatoria en la inicialización de un perceptrón
multicapa, los resultados obtenidos al intentar replicar estas pruebas
pueden variar ligeramente.