%
%
\subsection{Selección trivial}
%
La estrategia de selección trivial consiste en seleccionar valores
genéricos para los \hparam{s}, razonables para la mayoría de los
problemas:
%
\begin{itemize}
\item
  Para el \hparam{} de regularización $C$, el valor preestabecido es
  $1$ \cite{libsvm}.
\item
  Para el \hparam{} $\gamma$ de amplitud para el núcleo RBF, se
  establece $\gamma=\frac{1}{2F}$, donde $F$ es el número de elementos
  en el vector de características \cite{glasmachersigel}.
\end{itemize}
%
Es de esperar que el modelo resultante del uso de esta estrategia no
sea óptimo, aunque resulta de utilidad como base para la comparación
frente a las otras estrategias disponibles.
