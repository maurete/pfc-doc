%
%
\subsection{Selección trivial}
%
La estrategia de selección trivial consiste en seleccionar valores
preestablecidos (genéricos) para los hiperparámetros, apropiados para
la mayoría de los problemas.
El valor preestablecido para el hiperparámetro de regularización $C$
es $1$ \cite{libsvm}.
En el caso del hiperparámetro de amplitud del núcleo RBF, se calcula
un valor $\gamma=\frac{1}{2F}$, donde $F$ es el número de elementos en
el vector de características \cite{glasmachersigel}.

Es de esperar que el modelo resultante del uso de esta estrategia no
sea óptimo, aunque sí resulta de utilidad como base para la
comparación contra otros modelos parametrizados con estrategias más
complejas.
