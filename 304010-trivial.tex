%
%
\subsection{Selección trivial}
%
La estrategia de selección trivial consiste en seleccionar valores
preestablecidos (genéricos) para los hiperparámetros, apropiados para
la mayoría de los problemas.

La estrategia selecciona el valor $1$ para el hiperparámetro de
regularización $C$ \cite{libsvm}, y un valor $\gamma=\frac{1}{2F}$
para el hiperparámetro de amplitud del núcleo RBF, donde $F$ es el
número de características consideradas \cite{glasmachersigel}.

Es de esperar que el modelo resultante del uso de esta estrategia no
sea óptimo, aunque sí será de utilidad para la comparación con otros
modelos obtenidos mediante estrategias más avanzadas.
