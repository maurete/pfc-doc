%
%
%
\subsection{Lectura de los archivos de entrada}
%
El primer paso del preprocesamiento es la interpretación de los
archivos de entrada. La lectura se efectúa en modo secuencial, línea
por línea, aplicando análisis sintáctico que identifica las líneas de
descripción, secuencia y estructura secundaria.  Este análisis se
efectúa mediante expresiones regulares y genera un arreglo en memoria
de los ejemplos, conteniendo las propiedades secuencia, estructura
secundaria, descripción, identificador, entre otras.

Este arreglo de salida es utilizado en el paso de extracción de
características para continuar el procesamiento.
%
\subsubsection{Plegado}
%
Cuando el archivo leído no contiene información de estructura
secundaria, ésta se calcula invocando herramientas externas. El
software soporta la utilización de la herramienta ``RNAfold'',
incluida en el paquete de software ``Vienna RNA''
\cite{vienna}. Alternativamente, también permite la utilización de la
función de Matlab \mono{rnafold} distribuida como parte del paquete
``Bioinformatics Toolbox'' de dicho software.
