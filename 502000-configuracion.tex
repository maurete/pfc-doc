%
%
%
\section{Configuración de las pruebas}
%
Cada \e{prueba} refiere a un experimento en particular, que combina un
{problema} de clasificación, un \e{conjunto de \caract{s}}, y una
estrategia de selección de \hparam{s}.
En cada experimento se probaron los tres clasificadores: perceptrón
multicapa, máquina de vectores de soporte con núcleo lineal, y máquina
de vectores de soporte con núcleo RBF, generando un modelo del
clasificador a partir de los datos de entrenamiento definidos en el
problema, y evaluando su desempeño al clasificar el conjunto de
prueba.

\paragraph{Semillas pseudoaleatorias.}
La inicialización de los algoritmos pseudoaleatorios que particionan
los datos influye en la composición de los problemas \prob\mipred{} y
\prob\micropred{}, generando particiones más o menos ``complejas'' de
clasificar, según los ejemplos que contengan.
Para lograr que las pruebas efectuadas fueran reproducibles, se
fijaron cinco números arbitrarios que se utilizaron como ``semillas''
para inicializar los algoritmos pseudoaleatorios, y se repitió cada
prueba con las cinco semillas.

\paragraph{Conjuntos de \caract{s}.}
Se definieron para las pruebas seis conjuntos de \caract{s}:
%
\begin{itemize}
\item\dset{T} ($2$): incluye las $32$ \caract{s} con el número de
  ocurrencias de cada triplete, sin incluir las medidas auxiliares.
\item\dset{X} ($3$): incluye las $4$ \caract{s} auxiliares de
  tripletes.
\item\dset{S} ($4$): contiene $23$ \caract{s} de la secuencia.
\item\dset{E} ($5$): compuesto por las $7$ \caract{s} de la estructura
  secundaria.
\item\dset{T-X} ($6$): contiene las $36$ \caract{s} de tripletes,
  incluyendo número de ocurrencias y medidas auxiliares.
\item\dset{S-E} ($8$): combina $30$ \caract{s} de la secuencia y de la
  estructura secundaria.
\end{itemize}
%
El número entre paréntesis refiere al identificador utilizado para
cada conjunto de \caract{s} en el código.
Los conjuntos de \caract{s} de la secuencia (\dset{S}), de la
estructura secundaria (\dset{E}), y de la secuencia y estructura
secundaria combinadas (\dset{S-E}) se probaron con todos los
problemas.
Los conjuntos de \caract{s} relacionados con los tripletes (\dset{T},
\dset{X}, y \dset{T-X}), en cambio, se probaron sólo con el problema
\prob\tripletsvm{}, ya que es el único que no contiene ejemplos con
estructura secundaria ramificada (con bucles múltiples).

\paragraph{Entorno.}
Las pruebas se efectuaron en un entorno Matlab, versión R2012b, en un
ordenador personal de escritorio con $8$\iflatexml{}Gb\else\si{\giga
  b}\fi{} de RAM, procesador Intel Core i5-4440 con $4$ núcleos de
3.10\iflatexml{}GHz\else\si{\giga Hz}\fi{}, y un sistema operativo
Debian GNU/Linux, versión $8$ (``jessie'').
La configuración del sistema permite el cómputo paralelo con los $4$
núcleos del sistema.
