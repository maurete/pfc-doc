%
%
%
\section{Configuración de las pruebas}
%
Cada ``prueba'' refiere a un experimento consistente en generar un
modelo de clasificador y luego evaluar su desempeño al clasificar
un conjunto de prueba.
Para cada prueba, se establece un {problema} de clasificación, que
determina los datos utilizados como conjuntos de entrenamiento y de
prueba, un \e{conjunto de \caract{s}}, que regula las componentes de
los vectores de \caract{s} a utilizar, y una estrategia de selección
de \hparam{s}.

\paragraph{Semillas pseudoaleatorias.}
La inicialización de los algoritmos pseudoaleatorios que particionan
los datos influye en la composición de los problemas,
generando particiones más o menos ``complejas'' para la máquina
de aprendizaje según los ejemplos que contengan.
Para lograr que las pruebas efectuadas fueran reproducibles, se
fijaron cinco números arbitrarios que se utilizaron como ``semillas''
para inicializar los algoritmos pseudoaleatorios, y se repitió cada
prueba para cada semilla.

\paragraph{Características.}
Para todos los problemas, se efectuaron pruebas para las
características de secuencia (S), de estructura secundaria (E), y de
secuencia y estructura secundaria en conjunto (S-E).
Para aquellos problemas que contienen únicamente ejemplos con
estructura secundaria tipo horquilla (bucle único), se probaron las
características de tripletes (T), auxiliares de tripletes (X), y de
tripletes y auxiliares en conjunto (T-X).

\paragraph{Entorno.}
Las pruebas del software se efectuaron en entorno Matlab versión
R2012b, en un ordenador personal de escritorio con 8\iflatexml{}Gb\else\si{\giga b}\fi{} de
RAM, procesador Intel Core i5-4440 de 4 núcleos de 3.10\iflatexml{}GHz\else\si{\giga Hz}\fi{},
y sistema operativo Debian GNU/Linux versión 8.
La configuración del sistema permite el cómputo paralelo con los 4
núcleos del sistema.

%% \paragraph{Validación cruzada.}
%% En todos los casos que se aplicó validación cruzada se utilizaron 10
%% particiones, el valor por defecto definido en la implementación del
%% método propuesto.
