%
%
%
\section{Configuración de las pruebas}
%
Una prueba es un experimento que consiste en generar un modelo de
clasificador y luego evaluar su desempeño clasificando un conjunto de
prueba.
Cada prueba se parametriza con un \e{problema} de clasificación, que
determina los datos utilizados como conjuntos de entrenamiento y de
prueba, un \e{conjunto de \caract{s}} que establece las componentes de
los vectores de \caract{s}, y una estrategia de selección de \hparam{s}.

\paragraph{Semillas pseudoaleatorias.}
La inicialización de los algoritmos pseudoaleatorios utilizados para
particionar los datos influye en la composición de los problemas,
generando particiones de datos más o menos complejas de modelar y/o
clasificar.
Para lograr que las pruebas efectuadas fueran reproducibles, se
utilizaron cinco ``semillas'' para inicializar los algoritmos
pseudoaleatorios, y se repitió cada prueba para las cinco
inicializaciones.

\paragraph{Características.}
Para todos los problemas, se efectuaron pruebas para las
características de secuencia (S), de estructura secundaria (E), y de
secuencia y estructura secundaria en conjunto (S-E).
Para aquellos problemas que contienen únicamente ejemplos con
estructura secundaria tipo horquilla (bucle único), se probaron las
características de tripletes (T), auxiliares de tripletes (X), y de
tripletes y auxiliares en conjunto (T-X).

\paragraph{Entorno.}
Las pruebas del software se efectuaron en entorno Matlab versión
R2012b, en un ordenador personal de escritorio con 8\si{\giga b} de
RAM, procesador Intel Core i5-4440 de 4 núcleos de 3.10\si{\giga Hz},
y sistema operativo Debian GNU/Linux versión 8.
La configuración del sistema permite el cómputo paralelo con los 4
núcleos del sistema.

%% \paragraph{Validación cruzada.}
%% En todos los casos que se aplicó validación cruzada se utilizaron 10
%% particiones, el valor por defecto definido en la implementación del
%% método propuesto.
