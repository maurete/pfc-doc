%
\subsubsection{Sobreajuste}
%
Se dice que un modelo presenta \e{sobreajuste} cuando éste describe
el conjunto de datos de entrenamiento en lugar de la relación
subyacente $\nu$ entre las variables de entrada y de salida.
Intuitivamente, se puede decir que existe sobreajuste cuando el modelo
\e{memoriza ejemplos} del conjunto de aprendizaje en lugar de
\e{aprender las propiedades} de la relación que existe entre las variables
de entrada y de salida. El problema del
sobreajuste existe debido a que la máquina de aprendizaje no recibe
información acerca de la \e{probabilidad} de cada ejemplo observado,
ignorando el ruido y los efectos aleatorios en el conjunto de datos de
entrenamiento.
