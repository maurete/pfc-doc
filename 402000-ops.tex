\section{Automatización de operaciones}

En el análisis del flujo de valor del proceso de creación de un nuevo
servicio se puso en evidencia la necesidad de automatizar ciertas
operaciones de la infraestructura. Para resolver este problema se
escribió código Ansible en forma de dos \e{roles}, denominados
\e{bootstrap} y \e{zabbix-host}. Estos roles, al ser
incorporados al código que gestiona la configuración de los servicios
(actuales o nuevos), permiten efectuar las operaciones que antes se
realizaban en forma manual.

Además de la codificación de los roles, se efectuaron los siguientes
cambios para dar soporte a la automatización en la infraestructura:

\begin{itemize}
\item En GitLab, se otorgaron permisos para poder crear proyectos a
  todos los miembros de la Dirección, y se otorgó en general acceso a
  todos los repositorios internos.
\item Se agregó soporte para la ejecución de los scripts de
  administración en modo no interactivo.
\item Se crearon usuarios específicos en los distintos servicios para
  autorizar las interacciones automáticas.
\item Se habilitaron reglas de firewall para permitir la comunicación
  automatizada.
\end{itemize}
El rol \e{bootstrap} está diseñado para ser invocado al principio
de la ejecución y se encarga del registro de entradas en los
servidores DNS y DHCP, de la creación/modificación de la instancia
virtualizada y asegura la configuración del servicio de backup. Al ser
ejecutado, el rol realiza internamente las siguientes tareas:

\begin{itemize}
\item Se conecta al servidor utilizado para la administración de la
  infraestructura (\e{cerbero}).
\item En este servidor, ejecuta comandos que verifican la existencia
  de las entradas DNS y DHCP requeridas para la instancia.
\item Si las entradas no existen, invoca el comando para crearlas y
  luego ejecuta otro script que propaga los cambios hacia los
  servidores DNS Y DHCP.
\item Se conecta vía API al servidor que administra las instancias
  virtualizadas (\e{copernico} o \e{flanders}).
\item Verifica en forma programática que la instancia existe y que sus
  recursos reflejan lo especificado en el código.
\item Si es necesario, mediante llamadas a la API crea la instancia o
  modifica los recursos asignados a la misma.
\item Se conecta al servidor GitLab para descargar el inventario del
  servicio de backup.
\item Si es necesario, agrega la instancia al servicio de backup y
  dispara una reconfiguración del mismo.
\end{itemize}
A diferencia de \e{bootstrap}, el rol \e{zabbix-host} está
diseñado para ser ejecutado al final del \e{playbook} (el guión
de configuración aplicado por Ansible), lo cual le permite detectar
qué servicios han sido instalados en la instancia, y configura el
servicio de monitoreo con opciones preestablecidas para cada tipo de
servicio. La interacción con el servicio de monitoreo se realiza
mediante llamadas a la API del mismo.

Para ilustrar la ganancia en productividad de la implementación de
estos roles, antes era necesario efectuar las siguientes tareas
manuales para crear un nuevo servicio:

\begin{itemize}
\item Conectarse al servidor de administración y actualizar el código
  SVN.
\item Ejecutar el script para registrar las entradas DNS y DHCP.
\item Ejecutar el script para recargar el estado de los servicios DNS
  y DHCP.
\item Conectarse al servidor principal del cluster de virtualización.
\item Ejecutar comandos de creación de la instancia, revisando las
  opciones correctas según los recursos solicitados para el servicio.
\item En la PC de trabajo actualizar el código Git donde se administra
  el servicio de backup.
\item Agregar entrada en el inventario de backup para la instancia
  recién creada.
\item Aplicar la configuración del servicio de backup ejecutando
  Ansible mediante la línea de comandos.
\item En el servidor de administración, ejecutar el script para la
  configuración del monitoreo de la instancia.
\end{itemize}
Luego de la implementación de los roles de automatización, todas las
acciones anteriores se reduce a escribir algunas líneas en el código
Ansible que gestiona el servicio:

\iflatexml\else
\fvset{fontsize=\footnotesize,frame=leftline,numbers=left,xleftmargin=1cm}
\fi
\begin{Verbatim}
bootstrap_host_network: sdmz1
bootstrap_ganeti_cluster: copernico.intranet
bootstrap_instance_disk_size: 25600
bootstrap_instance_memory: 8192
bootstrap_instance_vcpus: 8
bootstrap_instance_os_type: stretch+default
bootstrap_instance_pnode: saturno.intranet
bootstrap_instance_snode: triton.intranet

zabbix_host_macros:
  URL: https://servicios.unl.edu.ar/mesadeentradas/
\end{Verbatim}

Este código reemplaza las acciones manuales previas, eliminando
incluso la necesidad de intervención del equipo de
infraestructura. Junto con la implementación de la interfaz web para
Ansible, esta automatización brinda una plataforma de ``autoservicio''
para el aprovisionamiento de la infraestructura para todos los
miembros de la DIPT. Con todo esto se evitan errores de las
operaciones manuales y se eliminan tiempos de espera.

\section{Servidores de frontend}

Los servidores de frontend son los encargados de recibir las
peticiones HTTP(S) externas y direccionar el tráfico al servicio
interno correspondiente, siguiendo una serie de reglas. La necesidad
de cambiar el proceso de configuración se puso en evidencia en el
análisis del flujo de valor para la creación de un nuevo servicio. El
mismo resultaba tedioso y fundamentalmente no era escalable, lo cual
derivó en la búsqueda de una solución alternativa.

Una forma de atacar el problema hubiera sido configurando una tubería
de CI/CD para publicar automáticamente los cambios en la
configuración. Esto hubiera permitido automatizar la tarea de
\e{aplicar} los cambios, aunque no se hubiera resuelto el
problema de escalabilidad y mantenibilidad del servicio, ya que aún
resultaría necesario escribir la configuración de los servidores en su
propio repositorio y en su lenguaje específico.

La solución implementada consistió en un cambio radical en la
arquitectura, creando un nuevo servicio de frontend capaz de
configurarse a sí mismo a partir del estado actual de los servicios de
software ofrecidos en la Dirección. El seguimiento de este
\e{estado actual}, necesario para la auto-configuración de los
frontends, se realiza mediante otro servicio denominado Consul.

Con esta implementación se eliminaron los pasos de editar la
configuración del frontend, conectarse al servidor y efectuar un
reinicio manual, tareas que requerían de los permisos de un
administrador de sistemas. Tal como ocurrió con la creación del rol
\e{bootstrap}, ahora basta con editar algunas líneas en el código
Ansible que define la configuración del servicio.

Toda la nueva configuración se escribió como código y se aprovisionó
mediante Ansible.

\subsection{Servicio Consul}

Consul\footnote{ https://www.consul.io/} es un sistema distribuido que
provee interconexión entre servicios de manera independiente de la
infraestructura subyacente. Cada nodo ejecuta un \e{agente} que
publica los servicios ofrecidos por el mismo y se encarga de verificar
el estado de los servicios.

Los agentes se comunican con los \e{servidores} de Consul, los
cuales funcionan en un esquema de alta disponibilidad que asegura un
estado consistente incluso cuando hay fallas en algún servidor. Los
servidores mantienen un \e{catálogo}, el cual agrega la
información suministrada por los agentes. La información contenida en
el catálogo incluye los servicios disponibles, los nodos que ofrecen
estos servicios y el estado actual de los mismos. Otros componentes de
la infraestructura (tal es el caso de los servidores de frontend)
pueden consultar el catálogo mediante una API, lo que les permite
descubrir los servicios y nodos disponibles.

La implementación de Consul abarcó dos aspectos. Por un lado, se creó
el \e{servicio Consul}, que incluye la configuración de las
instancias que contienen los servidores de Consul. Por el otro, fue
necesario modificar el código Ansible que sirve de base para todos los
servicios, para que éstos incorporen el \e{agente} de Consul como
parte de su configuración básica.

La implementación del servicio (los servidores Consul) se llevó a cabo
siguiendo las recomendaciones de la documentación oficial. Se creó un
grupo de cinco servidores distribuidos en las diferentes subredes de
la infraestructura para asegurar mayor disponibilidad del servicio
ante fallas de conectividad.

\subsection{Servicio de frontend}

Para el nuevo servicio de frontend se utilizó el software
Traefik\footnote{ https://containo.us/traefik/}, un servidor de proxy
reverso que tiene la capacidad inspeccionar el estado de los servicios
en Consul y ajustar su propia configuración en consecuencia. Traefik
también se encarga de la generación y renovación automática de los
certificados TLS de Let’s Encrypt, una tarea que antes era gestionada
en forma semi-automática por el equipo de infraestructura.

Dado que Traefik no ofrece soporte para funcionar como servidor de
contenido HTTP(S) básico, en las instancias de frontend se instaló
también un servidor Nginx, que se encarga de servir contenido estático
tal como las páginas de error.

El código Ansible que configura el servicio de frontend fue diseñado
de modo que los cambios en la infraestructura tengan el menor impacto
posible sobre la disponibilidad. Esto se logró mediante la
configuración de un servicio Traefik secundario, el cual atiende las
peticiones mientras se migra la configuración del servicio principal.

\subsection{Proceso de migración}

El servicio de frontend consta de tres instancias, cada una de las
cuales atiende el tráfico en una dirección IP diferente. Esto es así
debido a las particularidades de la infraestructura utilizada por la
Dirección. Al tratarse de un servicio crítico, durante la migración se
conservó en todo momento la posibilidad de revertir los cambios de
manera inmediata. Esto se conoce como despliegue azul verde
(\e{blue green deployment}) \cite{bluegreen}.

Antes de comenzar con la migración de los frontends, se configuró el
servicio Consul y se verificó el correcto funcionamiento del mismo. En
términos generales, la migración de cada servicio de frontend se
realizó siguiendo la secuencia de pasos explicada a continuación. En
todos los casos, el entorno \e{verde} hace referencia al nuevo
servicio (con Traefik+nginx), y el entorno azul al servicio de
frontend anterior (nginx con configuración estática). La configuración
de un frontend en el entorno azul se muestra en la
\iflatexml{}Figura~\ref{fig:feinicial}\else\autoref{fig:feinicial}\fi.

\begin{enumerate}
\item Se creó una nueva instancia de frontend (entorno verde), con una
  regla por defecto para delegar todo el tráfico al frontend anterior
  (entorno azul).
\item Se modificaron las reglas de firewall para direccionar el
  tráfico del puerto 80 (HTTP) al entorno verde. Esto permitió la
  generación automática de nuevos certificados HTTPS para el entorno
  verde.
\item Se validaron los certificados generados desde la estación de
  trabajo, forzando la conexión a través del entorno verde. Una vez
  generados los certificados HTTPS, se configuró el firewall para
  direccionar el tráfico del puerto 443 (HTTPS, por donde se publica
  la gran mayoría de los servicios) al entorno verde.
\item Para cada servicio correspondiente al frontend, de a uno por
  vez, se instaló y configuró el agente Consul. De este modo se logró
  publicar el servicio en forma directa a través del entorno verde. Se
  verificó el acceso correcto de cada servicio
  (\iflatexml{}Figura~\ref{fig:femigracion}\else\autoref{fig:femigracion}\fi).
\item Una vez migrados todos los servicios, se eliminó el entorno azul
  (\iflatexml{}Figura~\ref{fig:fefinal}\else\autoref{fig:fefinal}\fi).
\end{enumerate}
