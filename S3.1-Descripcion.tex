%
%
%
\section{Preprocesamiento de los datos}
%
El preprocesamiento se trata de convertir los datos originales a
datos tratables por el clasificador.

%%%%%%%%%%%%%%%%%%%%%%%%%%%%%%%%%%%%%%%%%%%%%%%%%%%%%%%%%%%%%%%%%%
%%%%%%%%%%%%%%%%%%%%%%%%%%%%%%%%%%%%%%%%%%%%%%%%%%%%%%%%%%%%%%%%%%
%%%%%%%%%%%%%%%%%%%%%%%%%%%%%%%%%%%%%%%%%%%%%%%%%%%%%%%%%%%%%%%%%%


Se distinguen tres tareas principales que describen la funcionalidad
del sistema: generación del problema de clasificación, obtención del
modelo óptimo del clasificador, y clasificación (predicción) de nuevos
ejemplos.

La generación del problema de clasificación abarca el procesamiento de
los datos de entrada, la extracción de características, la
normalización de las matrices de datos y la partición de en conjuntos
de entrenamiento, prueba, y validación cruzada.

La obtención del modelo del clasificador incluye las tareas de
optimización de los hiperparámetros y el entrenamiento con el conjunto
completo de entrenamiento.  El modelo obtenido en este proceso puede
ser utilizado para clasificar nuevos ejemplos.

La clasificación es la tarea de aplicar el modelo ya entrenado a
nuevos datos, a fin de obtener predicciones de clase.

Desde un punto de vista global, el método recibe en su entrada
archivos con secuencias de (candidatos de) pre-miRNAs a procesar, y
opcionalmente un archivo de modelo producto de un entrenamiento
anterior.  Los pre-miRNAs (candidatos) leídos en la entrada pasan a
integrar los conjuntos de entrenamiento (obtención del modelo) y
prueba (predicción de pertenencias de clase) según especificación del
usuario.  A la salida, según las tareas a efectuar, devuelve un modelo
de clasificador entrenado y/o predicciones de clase para los ejemplos
del conjunto de prueba.
