%
%
%
\section{Interfaz de usuario}
%
Se codificaron funciones de alto nivel que brindan acceso a la
funcionalidad del sistema en una interfaz de usuario unificada.
Estas funciones permiten al usuario la carga y el preprocesamiento de
los datos, la generación de modelos de clasificador, y la utilización
de estos modelos para clasificar nuevos ejemplos.
%
\begin{enumerate}
\item
  La función \func{problem\_gen} efectúa la carga de datos a partir de
  archivos FASTA, y efectúa el preprocesamiento según la
  especificación del usuario.
  Retorna una estructura en memoria denominada ``problema'' con los
  datos preprocesados, que se utiliza como entrada a las funciones de
  generación del modelo y de clasificación.
\item
  La función \func{select\_model} genera el modelo del clasificador a
  partir de los datos de entrenamiento.
  Permite la construcción de modelos MLP y SVM aplicando diferentes
  estrategias de selección de \hparam{s}.
  Recibe como argumento un problema generado con \func{problem\_gen},
  y retorna un modelo de clasificador entrenado según las opciones
  especificadas por el usuario.
\item
  La función \func{problem\_classify} permite clasificar los datos de
  prueba de un problema generado con \func{problem\_gen}.
  Recibe como argumento el modelo entrenado y el problema que se desea
  clasificar, y retorna las predicciones resultantes de aplicar el
  modelo a los elementos del conjunto de prueba.
\end{enumerate}
%

%% TODO: Describir los casos de uso mediante escenarios
Los casos de uso típicos del sistema son la generación de un modelo de
clasificador y la obtención de predicciones de clase.
Haciendo uso de las funciones de interfaz de usuario, estas tareas se
realizan según los siguientes pasos
%
\begin{itemize}
\item
  \e{Para obtener un nuevo modelo de clasificador}:
  %
  \begin{enumerate}
  \item
    Generar un problema de clasificación mediante la función
    \func{problem\_gen} que contenga datos de entrenamiento de clase
    positiva y negativa.
  \item
    Entrenar un clasificador invocando a la función
    \func{select\_model}, pasándole como argumento el problema
    generado.
  \end{enumerate}
  %
\item
  \e{Para obtener predicciones de clase}:
  %
  \begin{enumerate}
  \item
    Generar un problema de clasificación mediante la función
    \func{problem\_gen} con datos de prueba para clasificar.
    Para obtener predicciones válidas, se debe proveer como argumento
    adicional la información de normalización del problema utilizado
    para el entrenamiento o del modelo generado con
    \func{select\_model}.
  \item
    Efectuar la clasificación mediante la función
    \func{problem\_classify}, pasándole como argumentos el problema
    generado en el punto anterior, y un modelo generado previamente
    con la función \func{select\_model}.
  \end{enumerate}
  %
\end{itemize}
%
