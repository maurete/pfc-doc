%
%
%
\section{Línea de comandos}
%
Se codificaron funciones de alto nivel que permiten el uso del método
mediante una interfaz de línea de comandos unificada.
Estas funciones proponen al usuario un flujo de trabajo en tres pasos:
%
\begin{enumerate}
\item
  El primer paso es efectuar el preprocesamiento de los datos mediante
  una función que interpreta en su entrada los archivos FASTA.
  Para cada archivo, se especifica la pertenencia de clase y el uso
  pretendido (entrenamiento o prueba) del mismo.
  A la salida retorna una estructura en memoria, llamada ``problema'',
  que incluye un conjunto de datos de entrenamiento y/o uno de prueba
  generados según la especificación.
\item
  La segunda función efectúa la construcción del modelo del
  clasificador.
  Recibe como entrada un ``problema'' con datos de entrenamiento junto
  a una especificación del tipo de clasificador requerido (MLP,
  SVM-lineal o SVM-RBF) y la estrategia de selección de \hparam{s} a
  aplicar.
  A la salida, retorna el modelo de clasificador obtenido.
\item
  El tercer paso es la clasificación: la función correspondiente
  recibe un ``problema'' de clasificación y un modelo.
  Aplicando este modelo, calcula y retorna las predicciones de clase
  para los ejemplos del conjunto de prueba dentro del problema.
\end{enumerate}
%
%%
%% %% TODO: Describir los casos de uso mediante escenarios
%% Los casos de uso típicos del sistema son la generación de un modelo de
%% clasificador y la obtención de predicciones de clase.
%% Haciendo uso de las funciones de interfaz de usuario, estas tareas se
%% realizan según los siguientes pasos
%% %
%% \begin{itemize}
%% \item
%%   \e{Para obtener un nuevo modelo de clasificador}:
%%   %
%%   \begin{enumerate}
%%   \item
%%     Generar un problema de clasificación mediante la función
%%     \func{problem\_gen} que contenga datos de entrenamiento de clase
%%     positiva y negativa.
%%   \item
%%     Entrenar un clasificador invocando a la función
%%     \func{select\_model}, pasándole como argumento el problema
%%     generado.
%%   \end{enumerate}
%%   %
%% \item
%%   \e{Para obtener predicciones de clase}:
%%   %
%%   \begin{enumerate}
%%   \item
%%     Generar un problema de clasificación mediante la función
%%     \func{problem\_gen} con datos de prueba para clasificar.
%%     Para obtener predicciones válidas, se debe proveer como argumento
%%     adicional la información de normalización del problema utilizado
%%     para el entrenamiento o del modelo generado con
%%     \func{select\_model}.
%%   \item
%%     Efectuar la clasificación mediante la función
%%     \func{problem\_classify}, pasándole como argumentos el problema
%%     generado en el punto anterior, y un modelo generado previamente
%%     con la función \func{select\_model}.
%%   \end{enumerate}
%%   %
%% \end{itemize}
%% %
