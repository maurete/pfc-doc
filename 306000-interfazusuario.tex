%
%
%
\section{Interfaz de usuario}
%
El sistema incluye una interfaz de usuario de línea de comandos de
``alto nivel'', común para los distintos tipos de clasificadores
soportados.
Esta interfaz de usuario consiste en tres funciones específicas que
permiten al usuario cargar los datos del disco, generar el modelo del
clasificador, y clasificar nuevos datos.
%
\begin{enumerate}
\item
  La función \func{problem\_gen} carga datos de entrenamiento y de
  prueba según la especificación del usuario, y retorna una estructura
  en memoria denominada ``problema'', que puede utilizarse como
  entrada a las funciones de obtención del modelo y de clasificación.
  A grandes rasgos, esta función lleva a cabo la etapa de
  preprocesamiento de los datos.
\item
  La función \func{select\_model} permite obtener el modelo del
  clasificador a partir de los datos de entrenamiento.
  Recibe como argumento un problema generado con \func{problem\_gen},
  y retorna un modelo de clasificador óptimo según las opciones
  especificadas como parámetros.
  La función ofrece la funcionalidad de construcción del modelo del
  clasificador tanto MLP como SVM, con sus respectivas estrategias de
  selección de hiperparámetros.
\item
  La función \func{problem\_classify} permite clasificar los datos de
  prueba especificados dentro de un problema generado con
  \func{problem\_gen}.
  Recibe como argumento el modelo entrenado y el problema que se desea
  clasificar, y retorna una estructura en memoria con las predicciones
  para todos los elementos incluidos en el conjunto de prueba
  definidos en el problema.
\end{enumerate}
%
Los casos de uso típicos del sistema son la generación de un modelo de
clasificador y la obtención de predicciones de clase.
Utilizando las funciones de interfaz de usuario, estas tareas pueden
efectuarse según los siguientes pasos
\begin{itemize}
\item
  Obtener un nuevo modelo de clasificador:
  %
  \begin{enumerate}
  \item
    Generar un problema de clasificación mediante la función
    \func{problem\_gen} que contenga datos de entrenamiento de clase
    positiva así como negativa.
  \item
    Entrenar un clasificador invocando a la función
    \func{select\_model}.
  \end{enumerate}
  %
\item
  Obtener predicciones de clase:
  %
  \begin{enumerate}
  \item
    Generar un problema de clasificación mediante la función
    \func{problem\_gen} que contenga datos de prueba para clasificar.
  \item
    Invocar la función \func{problem\_classify}, pasándole como
    argumentos el problema y un modelo obtenido mediante
    \func{select\_model}.
  \end{enumerate}
  %
\end{itemize}
%
