%
%
%
\section{Interfaz de usuario}
%
Se codificaron funciones de alto nivel que funcionan como una interfaz
de usuario de línea de comandos ``amigable''.
Esta interfaz consiste en tres funciones principales que permiten al
usuario cargar los datos del disco, generar el modelo del
clasificador, y clasificar nuevos datos.
%
\begin{enumerate}
\item
  La función \func{problem\_gen} carga del disco los datos de
  entrenamiento y prueba según la especificación del usuario, y
  retorna una estructura en memoria que representa un ``problema''
  sobre el cual trabajar.
\item
  La función \func{select\_model} permite al usuario elegir la
  estrategia de selección de hiperparámetros a seguir.
  Recibe como argumento el problema (con datos de entrenamiento)
  generado por \func{problem\_gen}, y retorna un clasificador
  entrenado con los hiperparámetros óptimos encontrados.
  La función incluye soporte tanto para clasificadores MLP así como SVM.
\item
  La función \func{problem\_classify} permite clasificar los datos de prueba
  especificados dentro de un problema.
  Recibe como argumento el modelo entrenado y el problema que se desea
  clasificar, y retorna una estructura en memoria con las predicciones
  para todos los elementos incluidos en el conjunto de prueba del
  problema.
\end{enumerate}
%
Los pasos a seguir para entrenar un nuevo modelo de clasificador
son:
%
\begin{enumerate}
\item Generar un problema de clasificación mediante la función
  \func{problem\_gen} que contenga datos de entrenamiento de clase
  positiva así como negativa.
\item Entrenar un clasificador invocando a la función
  \func{select\_model}.
\end{enumerate}
%
Cuando se requiere obtener predicciones de clase, el procedimiento
en la línea de comandos es el siguiente:
%
\begin{enumerate}
\item Generar un problema de clasificación mediante la función
  \func{problem\_gen} que contenga datos de prueba para clasificar.
\item Invocar la función \func{problem\_classify} pasándole como argumentos
  el modelo generado previamente y el problema con datos de prueba.
\end{enumerate}
%
